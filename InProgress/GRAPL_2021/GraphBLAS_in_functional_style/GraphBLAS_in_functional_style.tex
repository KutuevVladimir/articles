\documentclass[sigplan,screen]{acmart}
\usepackage{hyperref}
\usepackage{minted}
%\usepackage[table]{xcolor}
\usepackage{multirow}

%\usepackage{cite}
%\usepackage{amsmath,amsfonts}
\usepackage{algorithmic}
\usepackage{graphicx}
\usepackage{textcomp}
\usepackage{xcolor}


\setcopyright{acmcopyright}
\copyrightyear{2018}
\acmYear{2018}
\acmDOI{XXXXXXX.XXXXXXX}

%% These commands are for a PROCEEDINGS abstract or paper.
\acmConference[Conference acronym 'XX]{Make sure to enter the correct
  conference title from your rights confirmation emai}{June 03--05,
  2018}{Woodstock, NY}
%%
%%  Uncomment \acmBooktitle if the title of the proceedings is different
%%  from ``Proceedings of ...''!
%%
%%\acmBooktitle{Woodstock '18: ACM Symposium on Neural Gaze Detection,
%%  June 03--05, 2018, Woodstock, NY}
\acmPrice{15.00}
\acmISBN{978-1-4503-XXXX-X/18/06}


\begin{document}


\title{GraphBLAS-like API Design in Functional Style}

\author{Kirill Garbar}
%\authornote{Both authors contributed equally to this research.}
\email{st087492@student.spbu.ru}
%\orcid{1234-5678-9012}
\author{Igor Erin}
\email{igor.erin.a@gmail.com}
\author{Artem !!!}
\email{!!!}
\author{Dmitriy Panfilyonok}
\email{dmitriy.panfilyonok@gmail.com}
\author{Semyon Grigorev}
\email{s.v.grigoriev@spbu.ru}
%\authornotemark[1]
%\email{webmaster@marysville-ohio.com}
\affiliation{%
  \institution{St Petersburg State University}
  \streetaddress{P.O. Box 1212}
  \city{St Petersburg}
  %\state{Ohio}
  \country{Russia}
  \postcode{43017-6221}
}


%%
%% By default, the full list of authors will be used in the page
%% headers. Often, this list is too long, and will overlap
%% other information printed in the page headers. This command allows
%% the author to define a more concise list
%% of authors' names for this purpose.
\renewcommand{\shortauthors}{dgdgdg}



\begin{abstract}
    GraphBLAS API standard describes linear algebra based blocks to build parallel graph analysis algorithms.
    While it is a promising way to high-performance graph analysis, there are a number of drawbacks such as complicated API, hardness of implementation for GPGPU, and explicit zeroes problem.
    We show that the utilization of techniques from functional programming can help to solve some GraphBLAS design problems.
\end{abstract}

\begin{CCSXML}
<ccs2012>
 <concept>
  <concept_id>10010520.10010553.10010562</concept_id>
  <concept_desc>Computer systems organization~Embedded systems</concept_desc>
  <concept_significance>500</concept_significance>
 </concept>
 <concept>
  <concept_id>10010520.10010575.10010755</concept_id>
  <concept_desc>Computer systems organization~Redundancy</concept_desc>
  <concept_significance>300</concept_significance>
 </concept>
 <concept>
  <concept_id>10010520.10010553.10010554</concept_id>
  <concept_desc>Computer systems organization~Robotics</concept_desc>
  <concept_significance>100</concept_significance>
 </concept>
 <concept>
  <concept_id>10003033.10003083.10003095</concept_id>
  <concept_desc>Networks~Network reliability</concept_desc>
  <concept_significance>100</concept_significance>
 </concept>
</ccs2012>
\end{CCSXML}
  
\ccsdesc[500]{Computer systems organization~Embedded systems}
\ccsdesc[300]{Computer systems organization~Redundancy}
\ccsdesc{Computer systems organization~Robotics}
\ccsdesc[100]{Networks~Network reliability}
 
\keywords{graph analysis, sparse linear algebra, GraphBLAS API, GPGPU, parallel programming, functional programming, .NET, OpenCL, FSharp}

\maketitle

\section{Introduction}

Scalable high-performance graph analysis is an actual challenge.
There is a big number of ways to attack this challenge~\cite{Coimbra2021} and the first promising idea is to utilize general-purpose graphic processing units (GPGPU).
Such existing solutions, as CuSha~\cite{10.1145/2600212.2600227} and Gunrock~\cite{7967137} show that utilization of GPUs can improve the performance of graph analysis, moreover it is shown that solutions may be scaled to multi-GPU systems.
But low flexibility and high complexity of API are problems of these solutions.

The second promising thing which provides a user-friendly API for high-performance graph analysis algorithms creation is a GraphBLAS API~\cite{7761646} which provides linear algebra based building blocks to create graph analysis algorithms.
The idea of GraphBLAS is based on a well-known fact that linear algebra operations can be efficiently implemented on parallel hardware.
Along with that, a graph can be natively represented using matrices: adjacency matrix, incidence matrix, etc.
While reference CPU-based implementation of GraphBLAS, SuiteSparse~\cite{10.1145/3322125}, demonstrates good performance in real-world tasks, GPU-based implementation is challenging.

One of the challenges in this way is that real data are often sparse, thus underlying matrices and vectors are also sparse, and, as a result, classical dense data structures and respective algorithms are inefficient. 
So, it is necessary to use advanced data structures and procedures to implement sparse linear algebra, but the efficient implementation of them on GPU is hard due to the irregularity of workload and data access patterns.
Though such well-known libraries as cuSPARSE show that sparse linear algebra operations can be efficiently implemented for GPU, it is not so trivial to implement GraphBLAS on GPU. 
First of all, it requires \textit{generalized} sparse linear algebra, thus it is impossible just to reuse existing libraries which are almost all specified for operations over floats.
The second problem is specific optimizations, such as masking fusion, which can not be natively implemented on top of existing kernels.
Nevertheless, there is a number of implementations of GraphBLAS on GPU, such as GraphBLAST~\cite{yang2019graphblast}, GBTL~\cite{7529957}, which show that GPUs utilization can improve the performance of GraphBLAS-based graph analysis solutions.
But these solutions are not portable because they are based on Nvidia Cuda stack.

Although GraphBLAS is solid and mature standard with a number of implementation, it has limitations and shortcomings discussed in a talk given by John R. Gilbert~\cite{talk:graphblas_did_wrong}. Some of them are lack of interoperability and introspection, what is an obstacle on the way of GraphBLAS integration into real-world data analysis pipelines. Implicit zeroes mechanism and masking, which uses mix of engineering and math, leads to unpredictable memory usage in some cases, keeping API complex for both implementation and usage.

To provide portable linear algebra based GraphBLAS-inspired GPU graph analysis tool we developed a \textit{Spla} library\footnote{Source code of Spla library available at: \url{https://anonymous.link}}.
This library utilizes OpenCL for GPU computing to be portable across devices of different vendors and aimed to solve some of GraphBLAS limitations.
To sum up, the contribution of this work is the following.
\begin{itemize}
    \item Design of the library with GraphBLAS-inspired API proposed. The proposed solution addresses some GraphBLAS limitations, such as lack of introspection, implicit zeroes mechanism and ambiguous masking. Also, the core of the library is configurable, so GPU acceleration can be plugged in for some operations with a little effort.
    \item The proposed design implemented in C++ using OpenCL to provide GPU acceleration of some operations. Such linear algebra operations as matrix-vector multiplication for both dense and sparse vector, masked matrix-matrix multiplication, implemented on GPU. Totally, Spla provides all operations required to implement GPU-accelerated versions of breadth-first search (BFS), single source shortest path (SSSP), page rank (PR), and triangles counting (TC), that also are implemented.
    \item Evaluation on BFS, SSSP, PR, and TC, and real-world graphs shows portability across different vendors and promising performance: proposed solution consistently outperforms SuiteSparse, reaching up to 25 times speedup on some graphs, and shows performance comparable with GraphBLAST, achieving up to 36 times speedup in some cases. Surprisingly, for some problems, the proposed solution on integrated Intel GPU shows better performance than SuiteSparse on the respective CPU. At the same time, the evaluation shows that further optimizations are required.
\end{itemize}
\section{Related Work}

Linear algebra in FP. 

Graph Analysis in FP. Mokhov, ....

GPU in FP. Futhark, Lift, AnyDSL, ForOcaml, Accelerate, etc....

Linear algebra + graph analysis in FP (Scala~\cite{7965105}, Haskell)
%%\section{Brahma.FSharp}

Runtime code generation: generics, functions.... 
In comparison to templates (GraphBLAST): compiled library, not header-only library. 
Easy compilation, no additional compile-time dependencies, compilers and so on. 
Cold start problem. 
More flexible, more information can be captured by translator.
Kernels can be cashed. 

Basic OpenCL C: basic control flow, local memory, atomic operations, barriers, memory flags,

Supported F\#-related features: DU, pattern matching, structures, ....

Message-based asynchronious API F\#-native \textbf{Mailbox processor}\footnote{Mailbox processor is a standard primitive to organize message-based asynchronious computations. Official documentation: \url{https://fsharp.github.io/fsharp-core-docs/reference/fsharp-control-fsharpmailboxprocessor-1.html}. Access date: 08.01.2023.} primitive.


\section{Design Principles}

Basic principles of proposed design described in this section.
Here we will use .NET-like style for generic types: $\texttt{Type}_1\langle\texttt{Type}_2\rangle$ means that the type $\texttt{Type}_1$ is generic and $\texttt{Type}_2$ is a type parameter.
Also we use F\#-like type notations and syntax in our examples.

\subsection{Types tor Graphs, Matrices, and Operations}

Suppose one have an edge-labelled graph $G$ where labels have type $\texttt{T}_{\texttt{lbl}}$. 
Suppose also one declare a generic type $\texttt{Matrix} \langle \texttt{T} \rangle$ to use this type for graph representation where type parameter \texttt{T} is a type of matrix cell. 
It is obvious that type of cell of adjacency matrix of graph $G$ should a special type which has only two values: some value of type $\texttt{T}_{\texttt{lbl}}$ or special value \texttt{Nothing}.
This idea can be naturally expressed using discriminated unions (or sum types) which actively used not only in functional languages such as F\#, OCalm, or Haskell, but also in TypeScript, Swift and other popular languages. 
Moreover, the described case is widely used and there is a standard type in almost all languages which supports discriminated unions: $\texttt{Option} \langle \texttt{T} \rangle$ in F\# or OCaml, or $\texttt{Maybe} \langle \texttt{T} \rangle$ in Haskell. 
In F\# this type defined as presented in listing~\ref{lst:optionType}.

\begin{listing}[h]
\begin{minted}{fsharp}
type Option<T> = None | Some of T
\end{minted}
\caption{\texttt{Option} type definition}
\label{lst:optionType}
\end{listing}


Thus, to represent the graph $G$ as a matrix one should use an instance of $\texttt{Matrix} \langle \texttt{Option}\langle \texttt{T}_{\texttt{lbl}} \rangle \rangle $  of generic type $\texttt{Matrix}\langle \texttt{T} \rangle$.
This way we can explicitly separate cells which should be stored and which does not: for cells with value \texttt{Some(x)} \texttt{x} should be stored, and for cells with value \texttt{None} should not be stored. Note that there is no storage and data transfer overhead in this solution: one can use any format for sparse matrices and store only $\texttt{T}_{\texttt{lbl}}$ values as usual.

In these settings, natural type for binary operation is $$\texttt{Option}\langle T_1 \rangle \to \texttt{Option}\langle T_2 \rangle \to \texttt{Option}\langle T_3 \rangle.$$ But this type is not restrictive enough: it allows one to define operation which returns some non-zero (\texttt{Some(x)}) value for two zeroes (\texttt{None}-s), while we expect that $$\texttt{None op None = None}$$ for any operation \texttt{op}.   

To solve this problem one can introduce additional type-level constraints, but such constraints can not be expressed in F\#.
Actually, one need more power type system whish supports dependetn types.
An alternative solution is to introduce a type $\texttt{AtLeastOne} \langle T_1, T_2 \rangle$ as presented in listing~\ref{lst:AtLeastOneType}. This type is less flexible (for example it disallows one to apply operation partially) but is explicitly shows that we expect that at least one argument of operation should be non-zero. 

\begin{listing}[h]
    \begin{minted}{fsharp}
    type AtLeastOne<T1,T2> =
    | Both of T1 * T2
    | Left of T1
    | Right of T2
    \end{minted}
    \caption{\texttt{AtLeastOne} type definition}
    \label{lst:AtLeastOneType}
\end{listing}

Finally in this settings binary operations should have the following type: $$\texttt{AtLeastOne} \langle T_1, T_2 \rangle \to \texttt{Option}\langle T_3 \rangle.$$
This type disallows one to build non-zero value from two zeroes, and explicitly shows whether result should be stored or not.
Thus, proposed typing scheme solves the \textit{problem of explicit and implicit zeroes}.
Moreover it allows one to generalize element-wise operations.
For example, binary operations for element-wise addition, element-wise multiplication, and even for masking can be described as presented in listings~\ref{lst:opIntAdd}, \ref{lst:opIntMult}, \ref{lst:opMask} respectively.

\begin{listing}[h]
    \begin{minted}{fsharp}
let op_int_add args =
    match args with
    | Both (x, y) -> 
        let res = x + y 
        if res = 0 then None else Some res 
    | Left x -> Some x
    | Right y -> Some y
    \end{minted}
    \caption{An example of element-wise addition operation definition}
    \label{lst:opIntAdd}
\end{listing}

\begin{listing}[h]
    \begin{minted}{fsharp}
let op_int_mult args =
    match args with
    | Both (x, y) -> 
        let res = x * y 
        if res = 0 then None else Some res 
    | Left x | Right y -> None
    \end{minted}
    \caption{An example of element-wise multiplication operation definition}
    \label{lst:opIntMult}
\end{listing}

\begin{listing}[h]
    \begin{minted}{fsharp}
let op_mask args =
    match args with
    | Both (x, y) -> Some x
    | Left x | Right y -> None
    \end{minted}
    \caption{An example of masking operation definition}
    \label{lst:opMask}
\end{listing}

\subsection{Operations Over Matrices and Vectors}

GraphBLAS API introduces \textbf{monoid} and \textbf{semiring} abstraction to specify element-wise operations for functions over matrices and vectors.
We propose to use binary operations instead as a parameters for functions over matrices and vectors. 
Using proposed types we always know that identity is always \texttt{None}, so we do not need specify identity separately as a part of semiring or monoid.
Additionally, applications often do not require operations actually satisfy semiring or monoid properties, so usage of correctly typed functions should be more clear and less confusing than usage of mathematical object in non-convenient settings.

For example, function for element-wise matrix-matrix operations\footnote{We name it \texttt{map2} to be consistent with similar functions over standard collections.} should has the following type:
\begin{alignat*}{2}
    \texttt{map2} & : & \\ 
        &   & \texttt{ op: AtLeastOne} \langle T_1, T_2 \rangle \to \texttt{Option} \langle T_3 \rangle \\
        & \to & \texttt{m}_1: \texttt{Matrix} \langle \texttt{Option} \langle T_1\rangle \rangle \\
        & \to & \texttt{m}_2: \texttt{Matrix} \langle \texttt{Option} \langle T_2 \rangle \rangle \\
        & \to & \texttt{result}: \texttt{Matrix} \langle \texttt{Option} \langle T_3\rangle \rangle
\end{alignat*}

In this settings we can predefine a set of widely used binary operations, and allow user to specify own ones, and combine all of them freely and safely.
Moreover, this way allows one to introduce monoids and semirings as an additional level of abstraction if necessary. This way we can simplify core API: one should implement a relatively small set of high-order generic functions that unify functions from existing API.  


\subsection{Fusion}

\textit{Kernels fusion} is a way to reduce memory allocation for intermediate data structures and it is an important part of GraphBLAS-inspired way to high-performance graph analysis~\cite{10.1145/3466795}. Runtime metaprogramming allows us to implement runtime fusion for kernels: having an expression over matrices and vectors we can build an optimized F\# function from predefined building blocs, and after that translate this optimized version to OpenCL C.

While it is unlikely possible to implement general fusion, it should be possible to provide fusion for a fixed set of operations. This way it is important to minimize API which can be done using high-level abstractions as shown before. This allows us to define \textbf{mask} as a partial case of generic element-wise function (respective operation is presented in listing~\ref{lst:opMask}), not an optional parameter of other functions.
This makes API more homogeneous and clear. 

\section{Implementation Details}

To evaluate ideas described above we start a development of library named GraphBLAS\#\footnote{Sources of GraphBLAS\# on GiHub: \url{https://github.com/YaccConstructor/GraphBLAS-sharp}. Access date: 08.01.2023.}.

We use a Brahma.FSharp library for running time translation of F\# code to OpenCL C, and for translated kernels execution. 
Brahma.FSharp is based on code quotations, thus utilizes strong typing to provide more static code checks, and polymorphic first class functions for general highly abstract code creation.
Additionally, Brahma.FSharp provides special workflow builder to simplify heterogeneous programming and automate resource management.  

Abstraction layers which hides details of matrix representation and operations implementation.
Currently we are working on COO and CSR formats and respective operations.

\subsection{Implemented Operations}
Matrix-matrix element-wise

Vector-vector

Matrix-vector 

Matrix-matrix multiplication

Transpose

....
\section{Evaluation}

While our project\footnote{GraphBLAS\# project page: \url{https://github.com/YaccConstructor/GraphBLAS-sharp}.} is in a very early stage we implemented and evaluated only generic matrix-matrix element-wise function \texttt{map2} to demonstrate that the proposed solution may provide not only a way to an expressive compact high-level API, but also a way to its high-performance implementation. This function implemented for matrices in CSR format. We use Brahma.FSharp to support GPGPUs.  

We evaluate \texttt{map2} function parameterized by two operations: \texttt{op\_int\_add} (listing~\ref{lst:opIntAdd}) to get element-wise addition in terms of GraphBLAS API, and \texttt{op\_int\_mult} (listing~\ref{lst:opIntMult}) to get element-wise multiplication.

\subsection{Environment}
We perform our experiments on the PC with Ubuntu 20.04 installed and with the following hardware configuration: Intel Core i7--4790 CPU, 3.60GHz, 32GB DDR4 RAM with GeForce GTX 2070, 8GB GDDR6, 1.41GHz.

For comparison we choose a SuiteSparse:GraphBLAS as a reference CPU implementation of GraphBLAS API, and CUSP\footnote{Cusp is a CUDA-based library for sparse linear algebra and graph computations: \url{https://cusplibrary.github.io/}.} as a most stable GPGPU implementation of generic sparse linear algebra.

\begin{table}[H]
    \centering
    \caption{Matrices for evaluation}
    \label{matrices}  
    \begin{tabular}{ | c || c | c | c | }
    \hline
    Matrix & Size & NNZ & Squared matrix NNZ \\ \hline
    \hline
    wing & 62 032 & 243 088 & 714,200 \\ \hline
    luxembourg\_osm & 114 599 & 119 666 & 393 261 \\ \hline
    amazon0312 & 400 727 & 3 200 440 & 14 390 544 \\ \hline
    amazon-2008 & 735 323 & 5 158 388 & 25 366 745 \\ \hline
    web-Google & 916 428 & 5 105 039 & 29 710 164 \\ \hline
    webbase-1M & 1 000 005 & 3 105 536 & 51 111 996 \\ \hline
    cit-Patents & 3 774 768 & 16 518 948 & 469 \\ \hline
    \end{tabular}
\end{table}


\subsection {Dataset}

For evaluation we select a set of matrices from SuiteSparse matrix collection\footnote{SuiteSparse matrix collection: \url{https://sparse.tamu.edu/}.}
To simplify the evaluation of element-wise operations over matrices with different structures we precomputed the square of each matrix.
Characteristics of selected matrices are presented in table~\ref{matrices}.

\subsection{Evaluation Results}

\begin{table*}[h]
    \centering    
    \caption{Evaluation results for element-wise operations, time in ms}    
    \label{perf-comparison}
    \begin{tabular}{|c||c|c|c|c||c|c|}
    \hline
    \multirow{3}{*}{Matrix} & \multicolumn{4}{c||}{Elemint-wise addition} & \multicolumn{2}{c|}{Elemint-wise multiplication}\\    
    \cline{2-7}        
    & \multicolumn{2}{c|}{GraphBLAS-sharp} & \multirow{2}{*}{SuiteSparse} & \multirow{2}{*}{CUSP} & \multirow{2}{*}{GraphBLAS-sharp} & \multirow{2}{*}{SuiteSparse}        \\
    &No \texttt{AtLeastOne}&\texttt{AtLeastOne}& & & &\\ 
    \hline
    \hline
    wing            & $4,3 \pm 0,8$       & $4,3 \pm 0,6$      & $2,7\pm 0,9$   & $1,5\pm 0,0$   & $3,7 \pm 0,5$      & $3,5\pm 0,4$\\
    \hline
    luxembourg\_osm & $4,9 \pm 0,7$       & $4,1 \pm 0,5$      & $3,0\pm 1,1$   & $1,2\pm 0,1$  & $3,8 \pm 0,6$      & $3,0\pm 0,6$ \\
    \hline
    amazon0312      & $22,3 \pm 1,3$       & $22,1 \pm 1,3$      & $33,4\pm 0,8$  & $11,0\pm 1,4$  & $18,7 \pm 0,9$      & $35,7\pm 1,4$ \\
    \hline
    amazon-2008     & $38,7 \pm 0,8$       & $39,0 \pm 1,0$     & $55,9\pm 1,0$  & $19,1\pm 1,4$ & $34,5 \pm 1,0$     & $58,9\pm 1,9$  \\
    \hline
    web-Google      & $43,4 \pm 0,8$       & $43,4 \pm 1,1$     & $67,2\pm 7,5$  & $21,3\pm 1,3$  & $39,0 \pm 1,2$     & $66,2\pm 0,4$ \\
    \hline
    webbase-1M      & $63,6 \pm 1,1$      & $63,7 \pm 1,3$      & $86,5\pm 2,0$  & $24,3\pm 1,3$  & $54,5 \pm 0,7$      & $37,6\pm 5,6$ \\
    \hline
    cit-Patents     & $26,9 \pm 0,7$      & $26,0 \pm 0,7$      & $183,4\pm 5,4$ & $10,8\pm 0,6$   & $24,3 \pm 0,7$      & $162,2\pm 1,7$ \\     
    \hline
    \end{tabular}    
\end{table*}

To benchmark .NET-based implementation we use \textit{BenchmarkDotNet}\footnote{\textit{BenchmarkDotNet}: \url{https://benchmarkdotnet.org/}. Access date: 12.06.2022.} which allows one to automate benchmarking process for .NET platform.
We run each function XXX times, !!!
Time is measured in milliseconds. The time to prepare data and initially transfer it to GPU is not included.

Results of performance evaluation are presented in table~\ref{perf-comparison}.
We can see, that for element-wise addition our implementation slightly slower than SuiteSparse:GraphBLAS for small matrices (\textbf{wing, luxembourg\_osm}) and up to 7 times faster for big matrices (1.5 times median). At the same time our implementation 2.5 times slower than CUSP-based. For element-wise multiplication comparison with SuiteSparse:GraphBLAS shows almost similar results except matrix \textbf{webbase-1M} for which our implementation slower than SuiteSparse:GraphBLAS.

Comparison between original element-wise addition over primitive types, without \texttt{AtLeasOne} and generalized version which uses \texttt{AtLeastOne} type is also presented in table~\ref{perf-comparison}.
We can see, that more complex data types and element-wise operations do not poor performance of matrix-matrix operations because data transfer dominates arithmetic computations for sparse matrices processing, and the proposed abstraction does not increase the memory footprint.

\section{Conclusion}

We presented Spla, generalized sparse linear algebra framework with vendor-agnostic GPUs accelerated computations. Library design addresses some GraphBLAS limitations, such as lack of interoperability, implicit zeroes and inflexible masking. The evaluation of the proposed solutions for some real-world graph data in four different algorithms shows, that OpenCL-based solution has a promising performance, comparable to analogs, has acceptable scalability on devices of different GPU vendors, and, surprisingly, has a speedup in some cases when compared with highly-optimized CPU library on some integrated GPUs. All in all, there are still a plenty of research questions and directions for improvement. Some of them are listed bellow.

\begin{itemize}
    \item \textit{Performance tuning}. There is a still space for optimizations. Better workload balancing must be done. Performance must be improved on AMD and Intel devices. More optimized algorithms must be implemented, such as SpGEMM  algorithm proposed by Nagasaka et al.~\cite{8025284/spgemm/nagasaka} for general \textit{mxm} operation.
    \item \textit{Operations}. Additional linear algebra operations must be implemented as well as useful subroutines for filtering, joining, loading, saving data, and other manipulations involved in typical graphs analysis.
    \item \textit{Graph streaming}. The next important direction of the study is streaming of data from CPU to GPU. CuSha adopt data partitioning techniques for graphs processing which do not fit single GPU. Modern GPUs have a limited VRAM. Even high-end devices allow only a moderate portion of the memory to be addressed by the kernel at the same time. Thus, manual streaming of the data from CPU to GPU is required in order to support analysis of extremely large graphs, which count billions of edges to process.
    \item \textit{Multi-GPU}. Finally, scaling of the library to multiple GPUs must be implemented. Gunrock shows, that such approach can increase overall throughput and speedup processing of really dense graph. In connection with a streaming, it can be an ultimate solution for a large real-world graphs analysis.
\end{itemize}


\begin{acks}
  This research is sponsored by Huawei.
\end{acks}
  
%%
%% The next two lines define the bibliography style to be used, and
%% the bibliography file.
\bibliographystyle{ACM-Reference-Format}
\bibliography{GraphBLAS_in_functional_style}

\end{document}
\endinput