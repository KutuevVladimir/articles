\section{Conclusion and Future Work}


We present a work in progress that demonstrates a way to utilize both the power of high-level languages and performance of GPGPUs to implement GraphBLAS-like API.
Our preliminary evaluation shows that the proposed way is promising: high-level abstractions do not poor performance, allows one to create compact expressive API, and metaprogramming techniques provide a way to utilize GPUs and achieve reasonable performance.

In the future, first of all, we should extend our library to provide an API which is at least as powerful as GraphBLAS API. Known drawbacks of GraphBLAS design\footnote{\href{https://docs.google.com/document/d/1fMmm-Bmew0wpgJRrjyMHy6G-zPq6R6kQlRum560_4S0/edit}{``What did GraphBLAS Get Wrong?''}, John Gilbert, UC Santa Barbara, GraphBLAS BoF at HPEC, September 2022} should be analyzed to find a way to fix them. 

The next step is evaluation of the solution on real-world cases and comparison with other implementations of GraphBLAS API on different devices and different algorithms.

%Moreover, it will be useful for the community to implement an analog of LAGraph\footnote{LAGraph is a collection of algorithms implemented using GraphBLAS. Project sources on GitHub: \url{https://github.com/GraphBLAS/LAGraph}. Access date: 12.01.2021.} algorithms collection for .NET on the top of our library.

Finally, we should implement high-level optimizations, like kernels fusion and specialization to achieve high performance.


