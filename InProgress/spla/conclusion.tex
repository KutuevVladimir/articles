\section{Conclusion}

We presented Spla, generalized sparse linear algebra framework with vendor-agnostic GPUs accelerated computations. The evaluation of the proposed solutions for some real-world graph data in four different algorithms shows, that OpenCL-based solution has a promising performance, comparable to analogs, has acceptable scalability on devices of different GPU vendors, and, surprisingly, has a speedup in some cases when compared with highly-optimized CPU library on some integrated GPUs. All in all, there are still a plenty of research questions and directions for improvement. Some of them are listed bellow.

\begin{itemize}
    \item \textit{Performance tuning}. There is a still space for optimizations. Better workload balancing must be done. Performance must be improved on AMD and Intel devices. More optimized algorithms must be implemented, such as SpGEMM  algorithm proposed by Nagasaka et al.~\cite{8025284/spgemm/nagasaka} for general \textit{mxm} operation.
    \item \textit{Operations}. Additional linear algebra operations must be implemented as well as useful subroutines for filtering, joining, loading, saving data, and other manipulations involved in typical graphs analysis.
    \item \textit{Graph streaming}. The next important direction of the study is streaming of data from CPU to GPU. CuSha adopt data partitioning techniques for graphs processing which do not fit single GPU. Modern GPUs have a limited VRAM. Even high-end devices allow only a moderate portion of the memory to be addressed by the kernel at the same time. Thus, manual streaming of the data from CPU to GPU is required in order to support analysis of extremely large graphs, which count billions of edges to process.
    \item \textit{Multi-GPU}. Finally, scaling of the library to multiple GPUs must be implemented. Gunrock shows, that such approach can increase overall throughput and speedup processing of really dense graph. In connection with a streaming, it can be an ultimate solution for a large real-world graphs analysis.
\end{itemize}