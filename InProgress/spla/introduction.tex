\section{Introduction}

Scalable high-performance graph analysis is an actual challenge.
There is a big number of ways to attack this challenge~\cite{Coimbra2021} and the first promising idea is to utilize general-purpose graphic processing units (GPGPU).
Such existing solutions, as CuSha~\cite{10.1145/2600212.2600227} and Gunrock~\cite{7967137} show that utilization of GPUs can improve the performance of graph analysis, moreover it is shown that solutions may be scaled to multi-GPU systems.
But low flexibility and high complexity of API are problems of these solutions.

The second promising thing which provides a user-friendly API for high-performance graph analysis algorithms creation is a GraphBLAS API~\cite{7761646} which provides linear algebra based building blocks to create graph analysis algorithms.
The idea of GraphBLAS is based on a well-known fact that linear algebra operations can be efficiently implemented on parallel hardware.
Along with that, a graph can be natively represented using matrices: adjacency matrix, incidence matrix, etc.
While reference CPU-based implementation of GraphBLAS, SuiteSparse~\cite{10.1145/3322125}, demonstrates good performance in real-world tasks, GPU-based implementation is challenging.

One of the challenges in this way is that real data are often sparse, thus underlying matrices and vectors are also sparse, and, as a result, classical dense data structures and respective algorithms are inefficient. 
So, it is necessary to use advanced data structures and procedures to implement sparse linear algebra, but the efficient implementation of them on GPU is hard due to the irregularity of workload and data access patterns.
Though such well-known libraries as cuSPARSE show that sparse linear algebra operations can be efficiently implemented for GPGPU, it is not so trivial to implement GraphBLAS on GPGPU. 
First of all, it requires \textit{generalized} sparse linear algebra, thus it is impossible just to reuse existing libraries which are almost all specified for operations over floats.
The second problem is specific optimizations, such as masking fusion, which can not be natively implemented on top of existing kernels.
Nevertheless, there is a number of implementations of GraphBLAS on GPGPU, such as GraphBLAST~\cite{yang2019graphblast}, GBTL~\cite{7529957}, which show that GPGPUs utilization can improve the performance of GraphBLAS-based graph analysis solutions.
But these solutions are not portable because they are based on Nvidia Cuda stack.

\textcolor{red}{Brief GraphBLAS limitation discussion with respect to 2.3}.

To provide portable GPU implementation of GraphBLAS API we developed a \textit{Spla} library\footnote{Source code of Spla library available at: \url{https://anonimus.link}}.
This library utilizes OpenCL for GPGPU computing to be portable across devices of different vendors and aimed to solve some GraphBLAS limitations.
To sum up, the contribution of this work is the following.
\begin{itemize}
    \item Design of GraphBLAS-inspired API proposed. The proposed design solves some GraphBLAS limitations, such as \textcolor{red}{brief summary of the most important limitations solved in Spla}. Additionally, \textcolor{red}{the most important additional properties of the proposed design}.
    \item The proposed design implemented in C++ using OpenCL to provide GPU acceleration of some operations. Such linear algebra operations as matrix-vector multiplication for both dense and sparse vector, masked matrix-matrix multiplication, implemented on GPU. Totally, Spla provides all operations required to implement GPU-accelerated versions of breadth-first search (BFS), single source shortest path (SSSP), page rank (PR), and triangles counting (TC), that also were implemented.
    \item Evaluation on BFS, SSSP, PR, and TC, and real-world graphs shows portability across different vendors and promising performance: \textcolor{red}{our outstanding results}. Surprisingly, for some problems, the proposed solution on embedded Intel graphic card shows better performance than SuiteSparse on the respective CPU. At the same time, the evaluation shows that further optimization is required.
\end{itemize}