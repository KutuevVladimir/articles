\section{GPGPU Implementation}

Library uses OpenCL 1.2 as underlying compute API. 
Boost Compute~\cite{10.1145/2909437.2909454:boost:compute} is utilized as a high-level library on top of the OpenCL functionality. 
It provides thread-safe kernel caching, meta-kernel programming, and a set of basic parallel primitives such as \textit{device vector}, \textit{sort}, \textit{reduce}, \textit{scan}, etc., which is extended further to meet this project requirements.
Taskflow~\cite{Huang2022TaskflowAL} is used as a tasking library. It supports task-dependencies and dynamic tasking, utilized in order to create and execute sub-tasks. 

User-defined \textit{Types} are represented as POD-structures and handled by the library as fixed-size sequences of bytes.
User-defined \textit{Functions} are effectively textual strings with OpenCL code, injected into generalized meta-kernels.
Library has a number of predefined types, such as \textit{signed/unsigned integers}, \textit{floating point} types, and a set of common operations, such as \textit{arithmetic}, \textit{logic}, \textit{first/second}, etc.

For particular SpVSpM implementation ESC algorithm~\cite{10.1145/2699470:esc:algo} is employed. 
Masked SpGEMM is based on Yang et al. work~\cite{yang2019graphblast} solution. 
Tiled GPU merge path~\cite{inproceedings:gpu_merge_path} utilized for element-wise addition and masking implementation.
The code is generalized and written in a form of meta-kernels, so actual functions for elements reduction or multiplication are injected later.
Kernel compilation is done on demand if no previously cached entry is present.