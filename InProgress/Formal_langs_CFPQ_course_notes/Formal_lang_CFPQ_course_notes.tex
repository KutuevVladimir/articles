%\documentclass[a4paper,12pt]{article}  % standard LaTeX, 12 point type
\documentclass[12pt, a4paper, table]{book}

\usepackage{algpseudocode}
\usepackage{algorithm}
\usepackage{algorithmicx}

\usepackage{geometry}
\usepackage{amsfonts,latexsym}
\usepackage{amsthm}
\usepackage{amssymb}
\usepackage[utf8]{inputenc} % Кодировка
\usepackage[english,russian]{babel} % Многоязычность
\usepackage{mathtools}
\usepackage{hyperref}
\usepackage{tikz}
\usepackage{dsfont}
\usepackage{multicol}
\usepackage[bb=boondox]{mathalfa}

\usetikzlibrary{fit,calc,automata,positioning}

\theoremstyle{definition}
\newtheorem{definition}{Определение}[section]
\newtheorem{example}{Пример}[section]
\newtheorem{theorem}{Теорема}[section]
\newtheorem{proposition}[theorem]{Proposition}
\newtheorem{lemma}[theorem]{Лемма}
\newtheorem{corollary}[theorem]{Corollary}
\newtheorem{conjecture}[theorem]{Conjecture}
\newtheorem{note}[theorem]{Утверждение}


% unnumbered environments:

\theoremstyle{remark}
\newtheorem*{remark}{Remark}
%\newtheorem*{notation}{Notation}

\setlength{\parskip}{5pt plus 2pt minus 1pt}
%\setlength{\parindent}{0pt}


\algtext*{EndWhile}% Remove "end while" text
\algtext*{EndIf}% Remove "end if" text
\algtext*{EndFor}% Remove "end for" text
\algtext*{EndFunction}% Remove "end function" text


\usepackage{color}
\usepackage{listings}
\usepackage{caption}
\usepackage{graphicx}
\usepackage{ucs}

\graphicspath{{pics/}}

%\geometry{left=2cm}
%\geometry{right=1.5cm}
%\geometry{top=2cm}
%\geometry{bottom=2cm}




%\lstnewenvironment{algorithm}[1][]
%{
%    \lstset{
%        frame=tB,
%        numbers=left,
%        mathescape=true,
%        numberstyle=\small,
%        basicstyle=\small,
%        inputencoding=utf8,
%        extendedchars=\true,
%        keywordstyle=\color{black}\bfseries,
%        keywords={,function, procedure, return, datatype, function, in, if, else, for, foreach, while, denote, do, and, then, assert,}
%        numbers=left,
%        xleftmargin=.04\textwidth,
%        #1 % this is to add specific settings to an usage of this environment (for instnce, the caption and referable label)
%    }
%}
%{}

\newcommand{\tab}[1][0.3cm]{\ensuremath{\hspace*{#1}}}

\newcommand{\rvline}{\hspace*{-\arraycolsep}\vline\hspace*{-\arraycolsep}}

\newcommand{\derives}[1][*]{\xRightarrow[]{#1}}
\newcommand{\first}[1][1]{\textsc{first}_{#1}}
\newcommand{\follow}[1][1]{\textsc{follow}_{#1}}

\setcounter{MaxMatrixCols}{20}


\tikzset{
%->, % makes the edges directed
%>=stealth’, % makes the arrow heads bold
node distance=4cm, % specifies the minimum distance between two nodes. Change if necessary.
%every state/.style={thick, fill=gray!10}, % sets the properties for each ’state’ node
initial text=$ $, % sets the text that appears on the start arrow
}

\tikzstyle{symbol_node} = [shape=rectangle, rounded corners, draw, align=center]

\tikzstyle{r_state} = [shape=rectangle, draw, minimum size=0.2cm]

\tikzstyle{prod_node} = [shape=rectangle, draw, align=center]

\tikzset{
    between/.style args={#1 and #2}{
         at = ($(#1)!0.5!(#2)$)
    }
}

%every node/.style = {shape=rectangle, rounded corners,
%      draw, align=center,
%      top color=white, bottom color=blue!20}

\newcommand{\bfgray}[1]{\cellcolor{lightgray}\textbf{#1}}

\newenvironment{scaledalign}[4]
  {
    \begingroup
    #1
    \setlength\arraycolsep{#2}
    \renewcommand{\arraystretch}{#3}
    \begin{center}
    \begin{equation}
    \begin{aligned}
    #4
  }
  {
    \end{aligned}
    \end{equation}
    \end{center}
    \endgroup
  }

\title{Приложения теории формальных языков и синтаксического анализа}
\author{Семён Григорьев}
\date{\today}

\begin{document}
\maketitle
\newpage
\tableofcontents
\newpage

\input{List_of_contributors}
\chapter*{Введение}

Теория формальных языков находит применение не только для ставших уже классическими задач синтаксического анализа кода (языков программирования, искусственных языков) и естественных языков, но и в других областях, таких как статический анализ кода, графовые базы данных, биоинформатика, машинное обучение.

Например, в машинном обучении использование формальных грамматик позволяет передать искусственной нейронной сети, предназначенной для генерации цепочек с определёнными свойствами (генеративной нейронной сети), знания о синтаксической структуре этих цепочек, что позволяет существенно упростить процесс обучения и повысить качество результата~\cite{10.5555/3305381.3305582}.
Вместе с этим, развиваются подходы, позволяющие нейронным сетям наоборот извлекать синтаксическую структуру (строить дерево вывода) для входных цепочек~\cite{kasai-etal-2017-tag,kasai-etal-2018-end}.

В биоинформатике формальные грамматики нашли широкое применение для описания особенностей вторичной структуры геномных и белковых последовательностей~\cite{Dyrka2019,WJAnderson2012,zier2013rna}.
Соответствующие алгоритмы синтаксического анализа используются при создании инструментов обработки данных.

Таким образом, теория формальных языков выступает в качестве основы для многих прикладных областей, а алгоритмы синтаксического анлиза применимы не только для обработки естественных языков или языков программирования.
Нас же в данной работе будет интересовать применение теории формальных языков и алгоритмов синтаксического анализа для анализа графовых баз данных и для статического анализа кода.

Одна из классических задач, связанных с анализом графов --- это поиск путей в графе.
Возможны различные формулировки этой задачи.
В некоторых случаях необходимо выяснить, существует ли путь с определёнными свойствами между двумя выбранными вершинами.
В других же ситуациях необходимо найти все пути в графе, удовлетворяющие некоторым свойствам или ограничениям. 
Например, в качестве ограничений можно указать, что искомый путь должен быть простым, кратчайшим, гамильтоновым и так далее.

Один из способов задавать ограничения на пути в графе основан на использовании формальных языков.
Базовое определение языка говорит нам, что язык --- это множество слов над некоторым алфавитом.
Если рассмотреть граф, рёбра которого помечены символами из алфавита, то путь в таком графе будет задавать слово: достаточно соединить последовательно символы, лежащие на рёбрах пути.
Множество же таких путей будет задавать множество слов или язык.
Таким образом, если мы хотим найти некоторое множество путей в графе, то в качестве ограничения можно описать язык, который должно задавать это множество.
Иными словами, задача поиска путей может быть сформулирована следующим образом: необходимо найти такие пути в графе, что слова, получаемые конкатенацей меток их рёбер, принадлежат заданному языку.
Такой класс задач будем называть задачами поиска путей с ограничениям в терминах формальных языков.

Подобный класс задач часто возникает в областях, связанных с анализом граф-структурированных данных и активно исследуется~\cite{doi:10.1137/S0097539798337716,axelsson2011formal,10.1007/978-3-642-22321-1_24,Ward:2010:CRL:1710158.1710234,barrett2007label,doi:10.1137/S0097539798337716}.
Исследуются как классы языков, применяемых для задания ограничений, так и различные постановки задачи.

Граф-структурированные данные встречаются не только в графовых базах данных, но и при статическом анализе кода: по программе можно построить различные графы отображающие её свойства.
Скажем, граф вызовов, граф потока данных и так далее.
Оказывается, что поиск путей в специального вида графах с использованием ограничений в терминах формальных языков позволяет исследовать некоторые свойства программы.
Например проводить межпроцедурный анализ указателей или анализ алиасов~\cite{Zheng,10.1145/2001420.2001440,10.1145/2714064.2660213}, строить срезы программ~\cite{10.1145/193173.195287}, проводить анализ типов~\cite{10.1145/373243.360208}.

В данной работе представлен ряд алгоритмов для поиска путей с ограничениями в терминах формальных языков.
Основной акцент будет сделан на контекстно-свободных языках, однако будут затронуты и другие классы: регулярные, многокомпонентные контекстно-свободные (Multiple Context-Free Languages, MCFL~\cite{SEKI1991191}) и конъюнктивные языки.
Будет показано, что теория формальных языков и алгоритмы синтаксического анализа применимы не только для анализа языков программирования или естественных языков, а также для анализа графовых баз данных и статического анализа кода, что приводит к возникновению новых задач и переосмыслению старых.


Структура данной работы такова.
В начале (в части~\ref{chpt:GraphTheoryIntro}) мы рассмотрим основные понятия из теории графов, необходимые в данной работе. Данные разделы являются подготовительными и не обязательны к прочтению, если такие понятия как \textit{ориентированный граф} и \textit{матрица смежности} уже известны читателю. Более того, они лишь вводят определения, подазумевая, что более детальное изучение соответствующих разделов науки остается за рамками этой работы и скорее всего уже проделано читателем.
Затем, в главе~\ref{chpt:FormalLanguageTheoryIntro} мы введём основные понятия из теории формальных языков.
Далее, в главе~\ref{chpt:FLPQ} рассмотрим различные варианты постановки задачи поиска путей с ограничениями в терминах формальнх языков, обсудим базовые свойства задач, её разрешимость в различных постановках и т.д..
И в итоге зафиксируем постановку, которую будем изучать далее.
После этого, в главах~\ref{chpt:CFPQ_CYK}--\ref{chpt:GLR} мы будем подробно рассматривать различные алгоритмы решения этой задачи, попутно вводя специфичные для рассматриваемого алгоритма структуры данных.
Большинство алгоритмов будут основаны на классических алгоритмах синтаксического анализа, таких как CYK или LR.
Все главы, начиная с~\ref{chpt:GraphTheoryIntro}, снабжены списком вопросов и задач для самостоятельного решения и закрепления материала.
%\chapter[Некоторые понятия линейной алгебры]{Некоторые понятия линейной алгебры\footnote{Неообходимо понимать, что, с одной строны, в данном разделе рассматриваются самые базовыепонятия, которые даются практически в любом учебнике алгебры. С другой же стороны, определения данных понятий оказываются весьма вариативными и часто вызывают дискуссии. Напрмиер, интересный анализ тонкостей определения группы можно найти в первом и втором параграфах первого раздела книги Николая Александровича Вавилова ``Конкретная теория групп''~\cite{VavilovGroups}. Мы же дадим определения, удобные для дальнейшего изложения материала.}}\label{chpt:LinAlIntro}

При изложении ряда алгоритмов будут активно использоваться некоторые понятия и инструмены линейной алгебры, такие как моноид, полукольцо или матрица.
В данном разделе необходимые понятия будут определены и приведены некоторые примеры соответствующих конструкций. Для более глубокого изучения материала рекомендуются соответствующие разделы алгебры.

$$
\oplus
\otimes
\mathbb{1}
\mathbb{0}
$$

\section{Бинарные операции и их свойства}


Введём понятие \textit{бинарной операции} и рассмотрим некоторые её свойства, такие как \textit{коммутативность} и \textit{ассоциативность}.

\begin{definition}[Двухместная функция] Функцию, принимающую два аргумента, $f: S \times K \to Q$ будем называть двухместной или функцией арности два.
Для запси таких функций будем использовать типичную нотацию: $c = f(a,b)$.
\end{definition}


\begin{definition}[Бинарная операция] 
Бинарная операция --- это двухместная функция, от которой дополнительно требуется, чтобы оба аргумента и результат лежали в одном и том же множестве: $f: S \times S \to S$. В таком случае говорят, что бинарная операция определена на некотором множестве $S$. Для обозначения произвольной бинарной операции будем использовать символ $\circ$ и пользоваться инфиксной нотацией для записи: $c = a \circ b$.
\end{definition}




\begin{definition}[Внешняя бинарная операция]
Внешняя бинарная операция --- это бинарная операция, у которой аргументы лежат в разных множествах, при этом результат --- в одном из этих множеств. Иными словами $\circ: K \times S \to S$, где $K$ может быть не равно $S$  --- внешняя бинарная операция.
\end{definition}


Необходимо помнить, что как функции, так и бинарные операции, могут быть частично определёнными (частичные функции, частичные бинарные операции). Типичным примером частично определённой бинарной операции является деление на целых числах: она не определена, если второй аргумент равен нулю.


Бинарные операции могут обладать некоторыми дополнительными свойствами, такими как \textit{коммутативность} или \textit{ассоциативность}, позволяющими преобразовывать выражения, составленные с использованием этих операций.


\begin{definition}[Коммутативность]
Бинарная операция $\circ : S \times S \to S$ называется коммутативной, если для любых  $x_1 \in S, x_2 \in S$ верно, что  $x_1 \circ x_2 = x_2 \circ x_1$.
\end{definition}

\begin{example} Рассмотрим несколько примеров коммутативных и некоммутативных операций.
	\begin{itemize}
		\item Опреация сложения на целых числах $+$ является коммутативной: известный ещё со школы перестановочный закон сложения.
		\item Операция конкатенации на строках $+$ не является коммутативной: $$``ab" + ``c" \ = ``abc" \neq ``c" + ``ab" \ = ``cab".$$
		\item Операция умножения на целых числах является коммутативной: известный ещё со школы перестановочный закон умножения.
		\item Операция умножения матриц (над целыми числами) $\cdot$ не является коммутативной:
		$$\begin{pmatrix} 
		1 & 1 \\ 0 & 0
		\end{pmatrix}
		\cdot
		\begin{pmatrix} 
		0 & 0 \\ 1 & 1
		\end{pmatrix}
		=
		\begin{pmatrix} 
		1 & 1 \\ 0 & 0
		\end{pmatrix}
		\neq
		\begin{pmatrix} 
		0 & 0 \\ 1 & 1
		\end{pmatrix}
		\cdot
		\begin{pmatrix} 
		1 & 1 \\ 0 & 0
		\end{pmatrix}
		=
		\begin{pmatrix} 
		0 & 0 \\ 1 & 1
		\end{pmatrix}
		.$$
	\end{itemize}
\end{example}

\begin{definition}[Ассоциативность]
Бинарная операция $\circ : S \times S \to S$ называется ассоциативной, если для любых  $x_1 \in S, x_2 \in S, x_3 \in S$ верно, что  $(x_1 \circ x_2) \circ x_3 = x_1 \circ (x_2 \circ x_3)$. Иными словами, для ассоциативной операции результат вычислений не зависит от порядка применения операций.
\end{definition}

\begin{example} Рассмотрим несколько примеров ассоциативных и неассоциативных операций.
	\begin{itemize}
		\item Опреация сложения на целых числах $+$ является ассоциативной.
		\item Операция конкатенации на строках $+$ является ассоциативной: $$``ab" + ``c" \ = ``abc" \neq ``c" + ``ab" \ = ``cab".$$
		\item Операция умножения на целых числах является ассоциативной.
		\item Операция возведения в степень (над целыми числами) $\hat{\mkern6mu}$ не является ассоциативной:
		$$(2\hat{\mkern6mu}2)\hat{\mkern6mu}3 = 4 \hat{\mkern6mu} 3 = 64 \neq 2\hat{\mkern6mu}(2\hat{\mkern6mu}3) = 2 \hat{\mkern6mu} 8  = 256.$$
	\end{itemize}
\end{example}


\begin{definition}[дистрибутивность]
!!!
\end{definition}

\begin{definition}[идемпотентность]
!!!
\end{definition}

\begin{definition}[Нейтральный элемент]
Пусть есть коммутативная бинарная операция $\circ$ на множестве $S$. Говорят, что $x\in S$ является нейтарльным элементом по операции $\circ$, если для любого $y\in S$ верно, что $x \circ y = y \circ x = y$. Если бинарная операция не является коммутативной, то можно пределить \textit{нейтральный слева} и \textit{нейтральный справа} элементы по аналогии.
\end{definition}


\section{Полугруппа}


множество с заданной на нём ассоциативной бинарной операцией $(S,\cdot )$ 


Коммутативная полугруппа


The set of positive integers with addition. (With 0 included, this becomes a monoid.)
The set of integers with minimum or maximum. (With positive/negative infinity included, this becomes a monoid.)
Square nonnegative matrices of a given size with matrix multiplication.
Any ideal of a ring with the multiplication of the ring.
The set of all finite strings over a fixed alphabet $\Sigma$ with concatenation of strings as the semigroup operation — the so-called ``free semigroup over $\Sigma$''. With the empty string included, this semigroup becomes the free monoid over $\Sigma$.


\section{Моноид}


Полугруппа с нейтральным элементом.



\section{Группа}


Непустое множество $G$ с заданной на нём бинарной операцией $*$: $ \mathrm {G} \times \mathrm {G} \rightarrow \mathrm {G}$ называется группой $ (\mathrm {G} ,*)$, если выполнены следующие аксиомы:

ассоциативность: $\forall (a,b,c\in G)\colon (a*b)*c=a*(b*c)$;
наличие нейтрального элемента: $ \exists e\in G\quad \forall a\in G\colon (e*a=a*e=a)$;
наличие обратного элемента: $ \forall a\in G\quad \exists a^{-1}\in G\colon (a*a^{-1}=a^{-1}*a=e)$.

Иными словами, группа --- это моноид с дополнительным требованием наличия обратных элементов.

\begin{definition}[Абелева группа] --- операция коммутативна.
\end{definition}


\section{Полукольцо}

A semiring is a set R equipped with two binary operations + and $\otimes$, called addition and multiplication, such that:[3][4][5]

(R, +) is a commutative monoid with identity element 0:
(a + b) + c = a + (b + c)
0 + a = a + 0 = a
a + b = b + a
(R, $\otimes$) is a monoid with identity element 1:
(a$\otimes$b)$\otimes$c = a$\otimes$(b$\otimes$c)
1$\otimes$a = a$\otimes$1 = a
Multiplication left and right distributes over addition:
a$\otimes$(b + c) = (a$\otimes$b) + (a$\otimes$c)
(a + b)$\otimes$c = (a$\otimes$c) + (b$\otimes$c)
Multiplication by 0 annihilates R:
0$\otimes$a = a$\otimes$0 = 0


\section{Кольцо}


A ring is a set R equipped with two binary operations[a] + (addition) and $\otimes$ (multiplication) satisfying the following three sets of axioms, called the ring axioms[1][2][3]

R is an abelian group under addition, meaning that:
(a + b) + c = a + (b + c) for all a, b, c in R   (that is, + is associative).
a + b = b + a for all a, b in R   (that is, + is commutative).
There is an element 0 in R such that a + 0 = a for all a in R   (that is, 0 is the additive identity).
For each a in R there exists $-a$ in R such that $a + (-a) = 0$   (that is, $-a$ is the additive inverse of a).
R is a monoid under multiplication, meaning that:
(a $\otimes$ b) $\otimes$ c = a $\otimes$ (b $\otimes$ c) for all a, b, c in R   (that is, $\otimes$ is associative).
There is an element 1 in R such that a $\otimes$ 1 = a and 1 $\otimes$ a = a for all a in R   (that is, 1 is the multiplicative identity).[b]
Multiplication is distributive with respect to addition, meaning that:
a $\otimes$ (b + c) = (a $\otimes$ b) + (a $\otimes$ c) for all a, b, c in R   (left distributivity).
(b + c) $\otimes$ a = (b $\otimes$ a) + (c $\otimes$ a) for all a, b, c in R   (right distributivity).


\section{Поле}

\section{Матрицы и вектора}

Вектор

Матрица 

Про матричное произведение, тензорное произведение, ещё что-то.

\section{Вопросы и задачи}
\begin{enumerate}
	\item Привидите примеры некоммутативных операций.
	\item Привидите примеры ситуаций, когда наличие у бинарных операций каких-либо дополнитльных свойств (ассоциативности, коммутативности), позволяет строить более эффективные алгоритмы, чем в общем случае.
\end{enumerate}
\chapter{Общие сведения теории графов}\label{chpt:GraphTheoryIntro}

В данном разделе мы дадим определения базовым понятиям из теории графов, рассмотрим несколько классических задач из области анализа графов и алгоритмы их решения.
Всё это понадобится нам при последующей работе.

\section{Основные определения}

\begin{definition}
  \textit{Граф} $\mathcal{G} = \langle V, E, L \rangle$, где $V$ --- конечное множество вершин, $E$ --- конечное множество рёбер, т.ч. $E \subseteq V \times L \times V$, $L$ --- конечное множество меток на рёбрах.
\end{definition}

В дальнейшем речь будет идти о конечных ориентированных помеченных графах.
Мы будем использовать термин \textit{граф} подразумевая именно конечный ориентированный помеченный граф, если только не оговорено противное.

Также мы будем считать, что все вершины занумерованы подряд с нуля.
То есть можно считать, что $V$ --- это отрезок $[0, |V| - 1]$ неотрицательных целых чисел, где $|V|$ --- размер множества $V$.

\begin{example}[Пример графа и его графического представления]
  Пусть дан граф $$\mathcal{G}_1 = \langle \{0,1,2,3\}, \{(0,a,1), (1,a,2), (2,a,0), (2,b,3), (3,b,2)\}, \{a,b\} \rangle.$$
  Графическое представление графа $\mathcal{G}_1$:
  \begin{center}
  \begin{tikzpicture}[on grid, auto]
     \node[state] (q_0)   {$0$};
     \node[state] (q_1) [above right=1.4cm and 1.0cm of q_0] {$1$};
     \node[state] (q_2) [right=2.0cm of q_0] {$2$};
     \node[state] (q_3) [right=2.0cm of q_2] {$3$};
      \path[->]
      (q_0) edge  node {a} (q_1)
      (q_1) edge  node {a} (q_2)
      (q_2) edge  node {a} (q_0)
      (q_2) edge[bend left, above]  node {b} (q_3)
      (q_3) edge[bend left, below]  node {b} (q_2);
  \end{tikzpicture}
  \end{center}
\end{example}

\begin{definition}
  \textit{Ребро} ориентированного помеченного графа $\mathcal{G} = \langle V, E, L \rangle$ это упорядоченная тройка $e = (v_i,l,v_j) \in V \times L \times V$.
\end{definition}

\begin{example}[Пример рёбер графа]
$(0,a,1)$  и $(3,b,2)$ --- это рёбра графа $\mathcal{G}_1$. При этом, $(3,b,2)$ $(2,b,3)$ --- это разные рёбра, что видно из рисунка.
\end{example}

\begin{definition}
  \textit{Путём} $\pi$ в графе $\mathcal{G}$ будем называть последовательность рёбер такую, что для любых двух последовательных рёбер $e_1=(u_1,l_1,v_1)$ и $e_2=(u_2,l_2,v_2)$ в этой последовательности, конечная вершина первого ребра является начальной вершиной второго, то есть $v_1 = u_2$. Будем обозначать путь из вершины $v_0$ в вершину $v_n$ как $$v_0 \pi v_n = e_0,e_1, \dots, e_{n-1} = (v_0, l_0, v_1),(v_1,l_1,v_2),\dots,(v_{n-1},l_n,v_n).$$

\begin{center}
  \begin{tikzpicture}[on grid, auto]
     \node[state] (v_1)   {$v_1$};
     \node[state] (v_n) [right=2.0cm of v_1] {$v_n$};
      \path[->]
      (v_1) edge [out=45] node {$\pi$} (v_n);
  \end{tikzpicture}
  \end{center}
\end{definition}

\begin{example}[Пример путей графа]
$(0,a,1),(1,a,2) = 0\pi_1 2$  --- путь из вершины 0 в вершину 2 в графе $\mathcal{G}_1$.
При этом, $(0,a,1),(1,a,2),(2,b,3),(3,b,2) = 0\pi_2 2$ --- это тоже путь из вершины 0 в вершину 2 в графе $\mathcal{G}_1$, но он не равен $0\pi_1 2$.
\end{example}

Кроме того, нам потребуется отношение, отражающее факт существования пути между двумя вершинами.

\begin{definition}\label{def:reach}
  \textit{Отношение достижимости} в графе:
  $(v_i,v_j) \in P \iff \exists v_i \pi v_j$.
\end{definition}

Отметим, что рефлексивность этого отношения часто зависит от контекста.
В некоторых задачах по-умолчанию $(v_i,v_i) \notin P$, а чтобы это было верно, требуется явное наличие ребра-петли.

Один из способов задать граф --- это задать его \textit{матрицу смежности}.

\begin{definition}
  \textit{Матрица смежности} графа $\mathcal{G}=\langle V,E,L \rangle$ --- это квадратная матрица $M$ размера $n \times n$, где $|V| = n$ и ячейки которой содержат множества.
  При этом $l \in M[i,j] \iff \exists e = (i,l,j) \in E$.
\end{definition}

Заметим, что наше определение матрицы смежности отличается от классического, в котором матрица отражает лишь факт наличия хотя бы одного ребра и, соответственно, является булевой. То есть $M[i,j] = 1 \iff \exists e = (i,\_,j) \in E$.


Также можно встретить матрицы смежности, в ячейках которых всё же хранится некоторая информация, однако, в единственном экземпляре. То есть запрещены параллельные рёбра.
Такой подход часто можно встретить в задачах о кратчайших путях: в этом случае в ячейке хранится расстояние между двумя вершинами.
При этом, так как в качестве весов часто рассматривают произвольные (в том числе отрицательные) числа, то в задачах о кратчайших путях отдельно вводят значение ``бесконечность'' для обозначения ситуации, когда между двумя вершинами нет пути или его длина ещё не известна. 
Всё это приводит к тому, что \textit{матрица смежности} --- это обобщённое понятие, нежели конкретный специальный тип матриц.
Данная конструкция даёт общее представление о том, как в матричном виде ханить различную информацию о смежноти вершин в графе.

\begin{example}[Пример матрицы смежности неориентированного графа]
  Неориентированный граф:
  \begin{center}
  \begin{tikzpicture}[on grid,auto]
     \node[state] (q_0)   {$0$};
     \node[state] (q_1) [above right = 1.4cm and 1cm of q_0] {$1$};
     \node[state] (q_2) [right = 2cm of q_0] {$2$};
     \node[state] (q_3) [right = 2cm of q_2] {$3$};
      \path[-]
      (q_0) edge  node {} (q_1)
      (q_1) edge  node {} (q_2)
      (q_2) edge  node {} (q_0)
      (q_2) edge  node {} (q_3);
  \end{tikzpicture}
  \end{center}

  И его матрица смежности:
  $$
  \begin{pmatrix}
    1 & 1 & 1 & 0 \\
    1 & 1 & 1 & 0 \\
    1 & 1 & 1 & 1 \\
    0 & 0 & 1 & 1
  \end{pmatrix}
  $$
\end{example}

\begin{example}[Пример матрицы смежности ориентированного графа]
  Ориентированный граф:
  \begin{center}
  \begin{tikzpicture}[shorten >=1pt,on grid,auto]
     \node[state] (q_0)   {$0$};
     \node[state] (q_1) [above right = 1.4cm and 1cm of q_0] {$1$};
     \node[state] (q_2) [right = 2cm of q_0] {$2$};
     \node[state] (q_3) [right = 2cm of q_2] {$3$};
      \path[->]
      (q_0) edge  node {} (q_1)
      (q_1) edge  node {} (q_2)
      (q_2) edge  node {} (q_0)
      (q_2) edge[bend left, above]  node {} (q_3)
      (q_3) edge[bend left, below]  node {} (q_2);
  \end{tikzpicture}
  \end{center}

  И его матрица смежности:
  $$
  \begin{pmatrix}
    1 & 1 & 0 & 0 \\
    0 & 1 & 1 & 0 \\
    1 & 0 & 1 & 1 \\
    0 & 0 & 1 & 1
  \end{pmatrix}
  $$
\end{example}

\begin{example}[Пример матрицы смежности помеченного графа]
  Помеченный граф:
  \begin{center}
  \begin{tikzpicture}[shorten >=1pt,on grid,auto]
     \node[state] (q_0)   {$0$};
     \node[state] (q_1) [above right = 1.4cm and 1cm of q_0] {$1$};
     \node[state] (q_2) [right = 2cm of q_0] {$2$};
     \node[state] (q_3) [right = 2cm of q_2] {$3$};
      \path[->]
      (q_0) edge  node {a} (q_1)
      (q_1) edge  node {a} (q_2)
      (q_2) edge  node {a} (q_0)
      (q_2) edge[bend left = 20]  node {a} (q_3)
      (q_2) edge[bend left = 60]  node {b} (q_3)
      (q_3) edge[bend left, below]  node {b} (q_2);
  \end{tikzpicture}
  \end{center}

  И его матрица смежности:
  $$
  \begin{pmatrix}
    \varnothing   & \{a\}       & \varnothing & \varnothing \\
    \varnothing   & \varnothing & \{a\}       & \varnothing \\
    \{a\}         & \varnothing & \varnothing & \{a,b\} \\
    \varnothing   & \varnothing & \{b\}       & \varnothing
  \end{pmatrix}
  $$
\end{example}

\begin{example}[Пример матрицы смежности взвешенного графа]
  Взвешенный граф для задачи о кратчайших путях:
  \begin{center}
  \begin{tikzpicture}[shorten >=1pt,on grid,auto]
     \node[state] (q_0)   {$0$};
     \node[state] (q_1) [above right = 1.4cm and 1cm of q_0] {$1$};
     \node[state] (q_2) [right = 2cm of q_0] {$2$};
     \node[state] (q_3) [right = 2cm of q_2] {$3$};
      \path[->]
      (q_0) edge  node {-1.4} (q_1)
      (q_1) edge  node {2.2} (q_2)
      (q_2) edge  node {0.5} (q_0)
      (q_2) edge[bend left, above]  node {1.85} (q_3)
      (q_3) edge[bend left, below]  node {-0.76} (q_2);
  \end{tikzpicture}
  \end{center}

  И его матрица смежности (для задачи о кратчайших путях):
  $$
  \begin{pmatrix}
    0 & -1.4 & \infty & \infty \\
    \infty & 0 & 2.2 & \infty \\
    0.5 & \infty & 0 & 1.85 \\
    \infty & \infty & -0.76 & 0
  \end{pmatrix}
  $$
\end{example}

Мы ввели лишь общие понятия.
Специальные понятия, необходимые для изложения конкретного материала, будут даны в соответствующих главах.

\section{Задачи поиска путей}

Одна из классических задач анализа графов --- это задача поиска путей между вершинами с различными ограничениями.

При этом, возможны различные постановки задачи.
С одной стороны, по тому, что именно мы хотим получить в качестве результата:
\begin{itemize}
\item Наличие хотя бы одного пути, удовлетворяющего ограничениям, в графе. В данном случае не важно, между какими вершинами существует путь, важно лишь наличие его в графе.
\item Наличие пути, удовлетворяющего ограничениям, между некоторыми вершинами: задача достижимости.
      При данной постановке задачи, нас интересует ответ на вопрос достижимости вершина $v_1$ из вершины $v_2$ по пути, удовлетворяющему ограничениям.
      Такая постановка требует лишь проверить существование пути, но не обязательно его предоставлять в явном виде.
\item Поиск одного пути, удовлетворяющего ограничениям: необходимо не только установить факт наличия пути, но и  предъявить его.
\item Поиск всех путей: необходимо предоставить все пути, удовлетворяющеие заданным ограничениям.
\end{itemize}

С другой стороны, задачи различаются ещё и по тому, как фиксируются множества стартовых и конечных вершин.
Здесь возможны следующие варианты:
\begin{itemize}
\item из одной вершины до всех,
\item между всеми парами вершин,
\item межу фиксированной парой вершин,
\item между двумя множествами вершин.
\end{itemize}

Стоит отметить, что последний вариант является самым общим и сотальные --- лишь его частные случаи. 
Однако этот вариант часто выделяют отдельно, подразумевая, что остальные, выделенные, варианты в него не включаются. В итоге мы можем сформулировать прямое произведение различных постановок задач о поиске путей, перебирая возможные варианты желаемого результата и фиксируя разные стартоыве и финальные множетсва.

Часто при поиске путей на них накладывают дополнительные ограничения. Например, можно потребовать, чтобы пути были простыми или не проходили через определённые вершины.
Ограничение, имеющее важное прикладное значение, --- минимальность длины искомого пути.
Одна из важных задач, имеющих как прикладное, так и теоретическое значение --- \textit{поиск кратчайших путей в графе между всеми парами вершин(англ. APSP --- all-pairs shortest paths)}.



\section{APSP и транзитивное замыкание графа}

Заметим, что отношение достижимости (\ref{def:reach}) является транзитивным.
Действительно, если существует путь из $v_i$ в $v_j$ и путь из $v_j$ в $v_k$, то существует путь из $v_i$ в $v_k$.

\begin{definition}
  \textit{Транзитивным замыканием графа} называется транзитивное замыкание отношения достижимости по всему графу.
\end{definition}

Как несложно заметить, транзитивное замыкание графа --- это тоже граф.
Более того, если решить задачу поиска кратчайших путей между всеми парами вершин, то мы построим транзитивное замыкание.
Поэтому сразу рассмотрим алгоритм Флойда-Уоршелла~\cite{Floyd1962, Bernard1959, Warshall1962}, который является общим алгоритмом поиска кратчайших путей (умеет обрабатывать рёбра с отрицательными весами, чего не может, например, алгоритм Дейкстры). Его сложность: $O(n^3)$, где $n$ --- количество вершин в графе.
При этом, данный алгоритм легко упростить до алгоритма построения транзитивного замыкания.

\begin{algorithm}
  \floatname{algorithm}{Listing}
\begin{algorithmic}[1]
\caption{Алгоритм Флойда-Уоршелла}
\label{lst:algoFloydWarxhall}
\Function{FloydWarshall}{$\mathcal{G}$}
    \State{$M \gets$ матрица смежности $\mathcal{G}$}
    \State{$n \gets$ $|V(\mathcal{G})|$}
    \For{k = 0; k < n; k++}
      \For{i = 0; i < n; i++}
        \For{j = 0; j < n; j++}
          \State{$M[i,j] \gets$ min$(M[i,j], M[i,k] + M[k,j])$}
        \EndFor
      \EndFor
    \EndFor
\State \Return $M$
\EndFunction
\end{algorithmic}
\end{algorithm}


\begin{example}
  Пусть дан следующий граф:
  \begin{center}
  \begin{tikzpicture}[shorten >=1pt,on grid,auto]
     \node[state] (q_0)   {$0$};
     \node[state] (q_1) [above right = 1.4cm and 1cm of q_0] {$1$};
     \node[state] (q_2) [right = 2cm of q_0] {$2$};
     \node[state] (q_3) [right = 2cm of q_2] {$3$};
      \path[->]
      (q_0) edge  node {} (q_1)
      (q_1) edge  node {} (q_2)
      (q_2) edge  node {} (q_0)
      (q_2) edge[bend left, above]  node {} (q_3)
      (q_3) edge[bend left, below]  node {} (q_2);
  \end{tikzpicture}
  \end{center}

  Построим его транзитивное замыкание:
  \begin{center}
  \begin{tikzpicture}[shorten >=1pt,on grid,auto]
     \node[state] (q_0)   {$0$};
     \node[state] (q_1) [above right = 1.4cm and 1cm of q_0] {$1$};
     \node[state] (q_2) [right = 2cm of q_0] {$2$};
     \node[state] (q_3) [right = 2cm of q_2] {$3$};
      \path[->]
      (q_0) edge[loop below] node {} ()
      (q_1) edge[loop above] node {} ()
      (q_2) edge[loop below] node {} ()
      (q_3) edge[loop below] node {} ()

      (q_0) edge  node {} (q_1)
      (q_1) edge[bend right] node {} (q_0)
      (q_1) edge  node {} (q_2)
      (q_2) edge[bend right] node {} (q_1)
      (q_2) edge  node {} (q_0)
      (q_0) edge[bend right] node {} (q_2)
      (q_2) edge[bend left, above]  node {} (q_3)
      (q_3) edge[bend left, below]  node {} (q_2)
      (q_0) edge[bend right = 60]  node {} (q_3)
      (q_1) edge[bend left, above]  node {} (q_3);
  \end{tikzpicture}
  \end{center}
  \begin{remark}
    На самом деле, петли у вершины 3 может и не быть, т.к. это зависит от определения.
  \end{remark}
\end{example}

\begin{remark}
Приведем список интересных работ по APSP для ориентированных графов с вещественными весами (здесь $n$ --- количество вершин в графе):
\begin{itemize}
    \item M.L. Fredman (1976) --- $O(n^3(\log \log n / \log n)^\frac{1}{3})$ --- использование дерева решений~\cite{FredmanAPSP1976}
    \item W. Dobosiewicz (1990) --- $O(n^3 / \sqrt{\log n})$ --- использование операций на Word Random Access Machine~\cite{Dobosiewicz1990}
    \item T. Takaoka (1992) --- $O(n^3 \sqrt{\log \log n / \log n})$ --- использование таблицы поиска~\cite{Takaoka1992}
    \item Y. Han (2004) --- $O(n^3 (\log \log n / \log n)^\frac{5}{7})$~\cite{Han2004}
    \item T. Takoaka (2004) --- $O(n^3 (\log \log n)^2 / \log n)$~\cite{Takaoka2004}
    \item T. Takoaka (2005) --- $O(n^3 \log \log n / \log n)$~\cite{Takaoka2005}
    \item U. Zwick (2004) --- $O(n^3 \sqrt{\log \log n} / \log n)$~\cite{Zwick2004}
    \item T.M. Chan (2006) --- $O(n^3 / \log n)$ --- многомерный принцип ``разделяй и властвуй''~\cite{Chan2008}
    \item и др.
\end{itemize}
\end{remark}

\section{APSP и произведение матриц}
Алгоритм Флойда-Уоршелла можно переформулировать в терминах перемножения матриц. Для этого введём обозначение.


\begin{definition}
Пусть даны матрицы $A$ и $B$ размера $n \times n$. Определим операцию $\star$, которую называют \textit{Min-plus matrix multiplication}:

    $A \star B = C$ --- матрица размера $n \times n$, т.ч.
    $C[i,j] = \min_{k = 1 \dots n} \{ A[i,k] + B[k,j] \}$
\end{definition}

Также, обозначим за $D[i,j](k)$ минимальную длину пути из вершины $i$ в $j$, содержащий максимум $k$ рёбер.
Таким образом, $D(1) = M$, где $M$ --- это матрица смежности, а решением APSP является $D(n-1)$, т.к. чтобы соединить $n$ вершин требуется не больше $n-1$ рёбер.

\begin{center}
    $D(1) = M$ \\
    $D(2) = D(1) \star M = M^2$ \\
    $D(3) = D(2) \star M = M^3$ \\
    $\dots$ \\
    $D(n-1) = D(n-2) \star M = M^{n-1}$ \\
\end{center}

Итак, мы можем решить APSP за $O(n K(n))$, где $K(n)$ --- сложность алгоритма умножения матриц.
Сразу заметим, что, например, $D(3)$ считать не обязательно, т.к. можем посчитать $D(4)$ как $D(2) \star D(2)$.
Поэтому:

\begin{center}
    $D(1) = M$ \\
    $D(2) = M^2 = M \star M$ \\
    $D(4) = M^4 = M^2 \star M^2$ \\
    $D(8) = M^8 = M^4 \star M^4$ \\
    $\dots$ \\
    $D(2^{\log(n-1)}) = M^{2^{\log(n-1)}} = M^{2^{\log(n-1)} - 1} \star M^{2^{\log(n-1)} - 1}$ \\
    $D(n-1) = D(2^{\log(n-1)})$ \\
\end{center}

Теперь вместо $n$ итераций нам нужно $\log{n}$. В итоге, сложность --- $O(\log{n} K(n))$.
Данный алгоритм называется \textit{repeated squaring}\footnote{Пример решения APSP с помощью repeated squaring: \url{http://users.cecs.anu.edu.au/~Alistair.Rendell/Teaching/apac_comp3600/module4/all_pairs_shortest_paths.xhtml}}.

\section{Умножение матриц с субкубической сложностью}
В предыдущем подразделе мы свели проблему APSP к проблеме min-plus matrix multiplication, поэтому взглянем на эффективные алгоритмы матричного умножения.

Сложность наивного произведения двух матриц составляет $O(n^3)$, это приемлемо только для малых матриц, для больших же лучше использовать алгоритмы с субкубической сложностью.
Отметим, что мы называем сложность субкубической, если она равна $O(n^{3-\varepsilon})$, где $\varepsilon > 0$, иначе говоря, меньшей, чем $O(n^3)$.

Первый субкубический алгоритм опубликовал Ф. Штрассен в 1969 году, его сложность --- $O(n^{\log_2 7}) \approx O(n^{2.81})$~\cite{Strassen1969}. Основная идея --- рекурсивное разбиение на блоки $2 \times 2$ и вычисление их произведения с помощью только 7 умножений, а не 8.
Впоследствии алгоритмы усовершенствовались до ${\approx} O(n^{2.5})$~\cite{Pan1978,BiniCapoRoma1979,Schonhage1981,CoppWino1982}. В настоящее время наиболее асимптотически быстрым является алгоритм Копперсмита --- Винограда со сложностью $O(n^{2.376})$~\cite{CoppWino1990}.

Несмотря на то, что у приведенных алгоритмов неплохая алгоритмическая сложность, мы всё же не можем использовать их для вычисления min-plus matrix multiplication, так как в субкубических алгоритмах требуется, чтобы элементы образовывали кольцо. Покажем, что $\mathbb{R} \cup \{+\infty\}$ с операциями min и + являются полукольцом, а не кольцом:
\begin{enumerate}
    \item $min(a, b) = min(b, a)$
    \item $min(min(a, b)), c) = min(a, min(b, c)))$
    \item $min(a, +\infty) = min(+\infty, a) = a$, т.е. $+\infty$ --- нейтральный элемент относительно операции min

    \item $(a + b) + c = a + (b + c)$

    \item $min(a, b) + c = min(a + c, b + c)$
    \item $a + min(b, c) = min(a + b, a + c)$

    \item $a + \infty = \infty + a = \infty$
    \item Но для произвольного элемента $a$: $\nexists d$, т.ч. $min(a, d) = min(d, a) = +\infty$, т.е. для любого элемента нет обратного относительно операции min
\end{enumerate}

Таким образом, вопрос о субкубических алгоритмах решения APSP всё ещё открыт~\cite{Chan2010}.
Кроме того, более простая задача APSP с булевыми матрицами также не решена за субкубическую сложность. Приведем обоснование этого факта.

\begin{definition}
  Матрица называется \textit{булевой}, если она состоит из 0 и 1.
\end{definition}

Булевы матрицы с поэлементными операциями дизъюнкции и конъюнкции являются полукольцом. Покажем это: пусть $A$, $B$ и $C$ --- булевы матрицы, тогда:
\begin{enumerate}
    \item $A \vee B = B \vee A$
    \item $(A \vee B) \vee C = A \vee (B \vee C)$
    \item $A \vee N = N \vee A = A$, где $N$ --- матрица, полностью состоящая из 0

    \item $(A \wedge B) \wedge C = A \wedge (B \wedge C)$

    \item $(A \vee B) \wedge C = (A \wedge C) \vee (B \wedge C)$
    \item $A \wedge (B \vee C) = (A \wedge B) \vee (A \wedge C)$

    \item $A \wedge N = N \wedge A = N$
\end{enumerate}

Булевы матрицы тоже не являются кольцом, т.к. не имеют обратный элемент относительно операции дизъюнкции (т.е. для произвольной булевой матрицы $A$: $\nexists D$, т.ч. $D$ --- булева матрица и $A \vee D = D \vee A = N$). Следовательно, субкубические алгоритмы не подходят для перемножения булевых матриц, т.к. в них используются обратные элементы (например, для разности). В частности, они не применимы к классической матрице смежности, которая ведёт себя как булева матрица.

%\section{Вопросы и задачи}
%\begin{enumerate}
%  \item Реализуйте абстракцию полукльца, позволяющую конструировать полукольца с произвольными операциями.
%  \item Реализуйте алгоритм произведения матриц над произвольным полукольцом. Используйте результат решения предыдущей задачи.
%  \item Используя результаты предыдущих задач, реализуйте алгоритм построения транзитивного замыкания через произведение матриц.
%  \item Используя результаты предыдущих задач, реализуйте алгоритм решения задачи APSP для ориентированного через произведение матриц.
%  \item Используя существующую библиотеку линейной алгебры для CPU, решите задачу построения транзитивного замыкания графа. 
%  \item Используя существующую библиотеку линейной алгебры для CPU, решите задачу APSP для ориентированного графа.
%  \item Используя существующую библиотеку линейной алгебры для GPGPU, решите задачу построения транзитивного замыкания графа. 
%  \item Используя существующую библиотеку линейной алгебры для GPGPU, решите задачу APSP для ориентированного графа.
%  \item Сравните произволительность решений предыдущих задач
%\end{enumerate}

\chapter{Общие сведения теории формальных языков}\label{chpt:FormalLanguageTheoryIntro}

В данной главе мы рассмотрим основные понятия из теории формальных языков, которые пригодятся нам в дальнейшем изложении.

\begin{definition}
\textit{Алфавит} --- это конечное множество.
Элементы этого множества будем называть \textit{символами}.
\end{definition}

\begin{example}
  Примеры алфавитов

  \begin{itemize}
    \item Латинский алфавит $\Sigma = \{ a, b, c, \dots, z\}$
    \item Кириллический алфавит $\Sigma = \{ \text{а, б, в, \dots, я}\}$
    \item Алфавит чисел в шестнадцатеричной записи 
    $$\Sigma = \{0, 1, 2, 3, 4, 5, 6, 7 ,8,9, A, B, C, D, E, F \}$$
  \end{itemize}
\end{example}

Традиционное обозначение для алфавита --- $\Sigma$.
Также мы будем использовать различные прописные буквы латинского алфавита. Для обозначения символов алфавита будем использовать строчные буквы латинского алфавита: $a, b, \dots, x, y, z$.

Будем считать, что над алфавитом $\Sigma$ всегда определена операция конкатенации $(\cdot): \Sigma^* \times \Sigma^* \to \Sigma^*$.
При записи выражений символ точки (обозначение операции конкатенации) часто будем опускать: $a \cdot b = ab$.

\begin{definition}
\textit{Слово} над алфавитом $\Sigma$ --- это конечная конкатенация символов алфавита $\Sigma$: $\omega = a_0 \cdot a_1 \cdot \ldots \cdot a_m$, где $\omega$ --- слово, а для любого $i$ $a_i \in \Sigma$.
\end{definition}

\begin{definition}
Пусть $\omega = a_0 \cdot a_1 \cdot \ldots \cdot a_m$ --- слово над алфавитом $\Sigma$.
Будем называть $m + 1$ \textit{длиной слова} и обозначать как $|\omega|$.
\end{definition}

\begin{definition}
\textit{Язык} над алфавитом $\Sigma$ --- это множество слов над алфавитом $\Sigma$.
\end{definition}

\begin{example}

Примеры языков.

  \begin{itemize}
    \item Язык целых чисел в двоичной записи $\{0, 1, -1, 10, 11, -10, -11, \dots\}.$
    \item Язык всех правильных скобочных последовательностей $$\{(), (()), ()(), (())(), \dots\}.$$
  \end{itemize}
\end{example}

Любой язык над алфавитом $\Sigma$ является подмножеством $\Sigma^*$ --- множества всех слов над алфавитом $\Sigma$.

Заметим, что язык не обязан быть конечным множеством, в то время как алфавит всегда конечен и изучаем мы конечные слова.

%\begin{definition}
\textit{Способы задания языков}
\begin{itemize}
\item Перечислить все элементы. Такой способ работает только для конечных языков. Перечислить бесконечное множество не получится.
\item Задать генератор --- процедуру, которая возвращает очередное слово языка.
\item Задать распознователь --- процедуру, которая по данному слову может определить, принадлежит оно заданному языку или нет.
\end{itemize}


%Теоретико-множественные задачи над языками и их применение. 
%О том, что моногое --- про пересечение, проверку пустоты, вложенность.





%\section{Вопросы и задачи}
%\begin{enumerate}
%  \item !!! 
%  \item !!!
%\end{enumerate}

\chapter{Задача о поиске путей с ограничениями в терминах формальных языков}\label{chpt:FLPQ}



В данной главе сформулируем постановку задачи о поиске путей в графе с ограничениями. 
А также приведём несколько примеров областей, в которых применяются алгоритмы решения этой задачи.

\section{Постановка задачи }


Пусть нам дан конечный ориентированный помеченный граф $\mathcal{G}=\langle V,E,L \rangle$.
Функция $\omega(\pi) = \omega((v_0, l_0, v_1),(v_1,l_1,v_2),\dots,(v_{n-1},l_{n-1},v_n)) = l_0 \cdot l_1 \cdot \ldots \cdot l_{n-1} $ строит слово по пути посредством конкатенации меток рёбер вдоль этого пути.
Очевидно, для пустого пути данная функция будет возвращать пустое слово, а для пути длины $n  > 0$ --- непустое слово длины $n$.

Если теперь рассматривать задачу поиска путей, то окажется, что то множество путей, которое мы хотим найти, задаёт множество слов, то есть язык.
А значит, критерий поиска мы можем сформулировать следующим образом: нас интересуют такие пути, что слова из меток вдоль них принадлежат заданному языку.
\begin{definition} \label{def1}
    \textit{Задача поиска путей с ограничениями в терминах формальных языков} заключается в поиске множества путей $\Pi = \{\pi \mid \omega(\pi) \in \mathcal{L}\}$.
    
\end{definition}

В задаче поиска путей мы можем накладывать дополнительные ограничения на путь (например, чтобы он был простым, кратчайшим или Эйлеровым~\cite{kupferman2016eulerian}), но это уже другая история.

Другим вариантом постановки задачи является задача достижимости.

\begin{definition} \label{def2}
    \textit{Задача достижимости} заключается в поиске множества пар вершин, для которых найдется путь с началом и концом в этих вершинах, что слово, составленное из меток рёбер пути, будет принадлежать заданному языку.
    $\Pi' = \{(v_{i}, v_{j}) \mid \exists v_{i} \pi v_{j}, \omega(\pi) \in \mathcal{L}\}$.
    
\end{definition}

При этом, множество $\Pi$ может являться бесконечным, тогда как $\Pi'$ конечно, по причине конечности графа $\mathcal{G}$.

Язык $\mathcal{L}$ может принадлежать разным классам и быть задан разными способами. Например, он может быть регулярным, или контекстно-свободным, или многокомпонентным контекстно-свободным.

Если $\mathcal{L}$ --- регулярный, $\mathcal{G}$ можно рассматривать как недетерминированный конечный автомат (НКА), в котором все вершины и стартовые, и конечные.
Тогда задача поиска путей, в которой $\mathcal{L}$ --- регулярный, сводится к пересечению двух регулярных языков.

Более подробно мы рассмотрим случай, когда $\mathcal{L}$ --- контекстно-свободный язык.

Путь $G = \langle \Sigma, N, P \rangle$ --- контекстно-свободная граммтика.
Будем считать, что $L \subseteq \Sigma$.
Мы не фиксируем стартовый нетерминал в определении грамматики, поэтому, чтобы описать язык, задаваемый ей, нам необходимо отдельно зафиксировать стартовый нетерминал.
Таким образом, будем говорить, что $L(G,N_i) = \{ w | N_i \xRightarrow[G]{*} w  \}$ --- это язык задаваемый граммтикой $G$ со стартовым нетерминалом $N_i$.

\begin{example}
    Пример задачи поиска путей.
    
    Дана грамматика  $G:$
    \begin{align*}
    S   &\to a b \\ 
    S   &\to a S b
    \end{align*}
    
    Эта грамматика задаёт язык $\mathcal{L} = a^n b^n$.
    
    И дан граф $\mathcal{G}:$
    
    \begin{center}
        \begin{tikzpicture}[node distance=3cm, shorten >=1pt,on grid,auto]
        \node[state] (q_0)   {$0$};
        \node[state] (q_1) [above right=of q_0] {$1$};
        \node[state] (q_2) [right=of q_0] {$2$};
        \node[state] (q_3) [right=of q_2] {$3$};
        \path[->]
        (q_0) edge  node {a} (q_1)
        (q_1) edge  node {a} (q_2)
        (q_2) edge  node {a} (q_0)
        (q_2) edge[bend left, above]  node {b} (q_3)
        (q_3) edge[bend left, below]  node {b} (q_2);
        \end{tikzpicture}
        
    \end{center}
    
    Тогда примерами путей, принадлежащих множеству $\Pi = \{\pi \mid \omega(\pi) \in \mathcal{L}\}$, являются:
    
    \begin{center}
        \begin{tikzpicture}[node distance=2cm, shorten >=1pt,on grid,auto]
        \node[state] (q_1) {$1$};
        \node[state] (q_2) [right=of q_1] {$2$};
        \node[state] (q_3) [right=of q_2] {$3$};
        \path[->]
        (q_1) edge  node {a} (q_2)
        (q_2) edge  node {b} (q_3);
        \end{tikzpicture}
        
    \end{center}
    
    \begin{center}
        \begin{tikzpicture}[node distance=2cm,shorten >=1pt,on grid,auto]
        \node[state] (q_0)   {$0$};
        \node[state] (q_1) [right=of q_0] {$1$};
        \node[state] (q_2) [right=of q_1] {$2$};
        \node[state] (q_3) [right=of q_2] {$3$};
        \node[state] (q_4) [right=of q_3] {$2$};
        \path[->]
        (q_0) edge  node {a} (q_1)
        (q_1) edge  node {a} (q_2)
        (q_2) edge  node {b} (q_3)
        (q_3) edge  node {b} (q_4);
        \end{tikzpicture}
        
    \end{center}
    
\end{example}


\section{О разрешимости задачи}

Задачи из определения \ref{def1} и \ref{def2} сводятся к построению пересечения языка $\mathcal{L}$ и языка, задаваемого путями графа, $R$. 
А мы для обсуждения разрешимости задачи рассмотрим более слабую постановку задачи:

\begin{definition}
    Необходимо проверить, что существует хотя бы один такой путь $\pi$ для данного графа, для данного языка $\mathcal{L}$, что $\omega(\pi) \in \mathcal{L}$.
    
\end{definition}

Эта задача сводится к проверке пустоты пересечения языка $\mathcal{L}$ c $R$ --- регулярным языком, задаваемым графом. От класса языка $\mathcal{L}$ зависит её разрешимость:

\begin{itemize}
    \item Если $\mathcal{L}$ регулярный, то получаем задачу пересечения двух регулярных языков: 
    
    $\mathcal{L} \cap R = R'$.
    $R'$ --- также регулярный язык.
    Проверка регулярного языка на пустоту --- разрешимая проблема.
    
    \item Если $\mathcal{L}$ контекстно-свободный, то получаем задачу
    
    $\mathcal{L} \cap R = CF$ --- контекстно-свободный.
    Проверка контекстно-свободного языка на пустоту --- разрешимая проблема.
    
    \item Помимо иерархии Хомского существуют и другие классификации языков.
    Так например, класс конъюнктивных (Conj)
    языков~\cite{DBLP:journals/jalc/Okhotin01}
    является строгим расширением контекстно-свободных языков и все так же позволяет полиномиальный синтаксический анализ.
    
    Пусть $\mathcal{L}$ --- конъюнктивный. При пересечении конъюнктивного и регулярного языков получается конъюнктивный ($\mathcal{L} \cap R = Conj$), а проблема проверки Conj на пустоту не разрешима~\cite{DBLP:journals/tcs/Okhotin03a}.
    
    \item Ещё один класс языков из альтернативной иерархии, не сравнимой с Иерархией Хомского, --- MCFG (multiple context-free grammars)~\cite{SEKI1991191}.
    Как его частный случай --- TAG (tree adjoining grammar)~\cite{Joshi1997}.
    
    Если $\mathcal{L}$ принадлежит классу MCFG, то $\mathcal{L} \cap R$ также принадлежит MCFG. Проблема проверки пустоты MCFG разрешима~\cite{SEKI1991191}.
    
\end{itemize}

Существует ещё много других классификаций языков, но поиск универсальной иерархии до сих пор продолжается.

Далее, для изучения алгоритмов решения, нас будет интересовать задача $R \cap CF$.

\section{Области применения}

Поиск путей с ограничениями в виде формальных языков широко применяется вразличных областях. Ниже даны ключевые работы по применению поиска путей с контекстно-свободными ограничениями и сылки на них для более детального ознакомления.

\begin{itemize}
    \item Межпроцедурный Статанализ кода. 
    Идея начала активо разрабатываться Томасом Репсом~\cite{Reps}. Далее последовал ряд, в том числе инженерных работ, применяющих достижимость с контекстно-свободными ограничениями для анализа указателей, анализа алиасов и других прикладных задач~\cite{LabelFlowCFLReachability,specificationCFLReachability,Zheng}.    
    \item Графовые БД. Впервые задача сформулирована Михалисом Яннакакисом~\cite{Yannakakis}. Запросы с контекстно-свободными ограничениями нашли своё применение различных областях.
    \begin{itemize}
        \item Социальные сети~\cite{Hellings2015PathRF}.
        \item RDF обработка~\cite{10.1007/978-3-319-46523-4_38}.
        \item Биоинформатика~\cite{cfpqBio}.    
    \end{itemize}
    
\end{itemize}

%\begin{itemize}
%    \item OpenCypher~\cite{Kuijpers:2019:ESC:3335783.3335791}
%    \item J.Hellings. CFPQ~\cite{hellingsRelational,hellings2015querying,Hellings2015PathRF}
%    \item Zhang. CFPQ on rdf graphs~\cite{10.1007/978-3-319-46523-4_38}
%    \item Bradford~\cite{bradford2007quickest,ward2008distributed,bradford2016fast,Bradford:2008:LCG:1373936.1373946}
%\end{itemize}


%\section{Вопросы и задачи}
%\begin{enumerate}
%    \item Пусть есть граф. Задайте грамматику для поиска всех путей, таких, что....
%    \item Существует ли в графе !!! путь из А в Б, такой что!!!
%    \item Для графа !!! постройте все пути, удовлетворяющие !!!!
%    
%    \item Задача 1
%    \item Задача 2
%\end{enumerate}

%\input{RegularLanguages}
\chapter{Контекстно-свободные грамматики и языки}\label{CFG}

Из всего многообразия нас будут интересовать прежде всего контекстно-свободные грамматики.

\begin{definition}
\textit{Контекстно-свободная грамматика} --- это четвёрка вида $\langle \Sigma, N, P, S \rangle$, где
\begin{itemize}
  \item $\Sigma$ --- это терминальный алфавит;
  \item $N$ --- это нетерминальный алфавит;
  \item $P$ --- это множество правил или продукций, таких что каждая продукция имеет вид $N_i \to \alpha$, где $N_i \in N$ и $\alpha \in \{\Sigma \cup N\}^* \cup {\varepsilon}$;
  \item $S$ --- стартовый нетерминал.
  Отметим, что $\Sigma \cap N = \varnothing$.
\end{itemize}
\end{definition}

\begin{example}
Грамматика, задающая язык целых чисел в двоичной записи без лидирующих нулей: $G = \langle \{0, 1, -\}, \{S, N, A\}, P, S \rangle$, где $P$ определено следующим образом:

\[
\begin{array}{rcl}
S& \rightarrow & 0 \mid N \mid - N  \\
N& \rightarrow & 1 A \\
A& \rightarrow & 0 A \mid 1 A  \mid \varepsilon\\
\end{array}
\]
\end{example}

При спецификации грамматики часто опускают множество терминалов и нетерминалов, оставляя только множество правил. При этом нетерминалы часто обозначаются прописными латинскими буквами, терминалы --- строчными, а стартовый нетерминал обозначается буквой~$S$. Мы будем следовать этим обозначениям, если не указано иное.


\begin{definition}\label{def derivability in CFG}
  \textit{Отношение непосредственной выводимости}. Мы говорим, что последовательность терминалов и нетерминалов $\gamma \alpha \delta$ \textit{непосредственно выводится из} $\gamma \beta \delta$ \textit{при помощи правила} $\alpha \rightarrow \beta$ ($\gamma \alpha \delta \Rightarrow \gamma \beta \delta$), если
  \begin{itemize}
    \item $\alpha \rightarrow \beta \in P$
    \item $\gamma, \delta \in \{\Sigma \cup N\}^* \cup {\varepsilon}$
  \end{itemize}
\end{definition}

\begin{definition}
  \textit{Рефлексивно-транзитивное замыкание отношения} --- это наименьшее рефлексивное и транзитивное отношение, содержащее исходное.
\end{definition}

\begin{definition}
\textit{Отношение выводимости} является рефлексивно-транзитивным замыканием отношения непосредственной выводимости
\begin{itemize}
  \item $\alpha \derives \beta$ означает $\exists \gamma_0, \dots \gamma_k: \ \alpha \derives[] \gamma_0 \derives[] \gamma_1 \derives[] \dots \derives[] \gamma_{k-1} \derives[] \gamma_{k} \derives[] \beta$
  \item Транзитивность: $\forall \alpha, \beta, \gamma \in \{\Sigma \cup N\}^* \cup {\varepsilon}: \ \alpha \derives \beta, \beta \derives \gamma \Rightarrow \alpha \derives \gamma$
  \item Рефлексивность: $\forall \alpha \in \{\Sigma \cup N\}^* \cup {\varepsilon}: \ \alpha \derives \alpha$
  \item $\alpha \derives \beta$ --- $\alpha$ выводится из $\beta$
  \item $\alpha \derives[k] \beta$ --- $\alpha$ выводится из $\beta$ за $k$ шагов
  \item $\alpha \derives[+] \beta$ --- при выводе использовалось хотя бы одно правило грамматики
\end{itemize}
\end{definition}


\begin{example}
Пример вывода цепочки $-1101$ в грамматике:

  \[
  \begin{array}{rcl}
  S& \rightarrow & 0 \mid N \mid - N  \\
  N& \rightarrow & 1 A \\
  A& \rightarrow & 0 A \mid 1 A  \mid \varepsilon\\
  \end{array}
  \]

  \[ S \Rightarrow - N \Rightarrow - 1 A \Rightarrow - 1 1 A \derives - 1 1 0 1 A \Rightarrow - 1 1 0 1 \]
\end{example}


\begin{definition}[Вывод слова в грамматике]
Слово $\omega \in \Sigma^*$ \textit{выводимо в грамматике} $\langle \Sigma, N, P, S \rangle$, если существует некоторый вывод этого слова из начального нетерминала $S \derives \omega$.

\end{definition}

\begin{definition}
\textit{Левосторонний вывод}. На каждом шаге вывода заменяется самый левый нетерминал.
\end{definition}

\begin{example}
Пример левостороннего вывода цепочки в грамматике

  \[
    \begin{array}{rcl}
    S& \rightarrow & A A \mid s  \\
    A& \rightarrow & A A \mid B b \mid a \\
    B& \rightarrow & c \mid d
    \end{array}
  \]

  \[ \boldsymbol{S} \derives[] \boldsymbol{A} A \derives[] \boldsymbol{B} b A \derives[] c b \boldsymbol{A} \derives[] c b \boldsymbol{A} A \derives[] c b a \boldsymbol{A} \derives[] c b a a \]
\end{example}

Аналогично можно определить правосторонний вывод.

\begin{definition}
\textit{Язык, задаваемый грамматикой} --- множество строк, выводимых в грамматике $L(G) = \{ \omega \in \Sigma^* \mid S \derives \omega \}$.
\end{definition}

\begin{definition}
  Грамматики $G_1$ и $G_2$ называются \textit{эквивалентными}, если они задают один и тот же язык: $L(G_1) = L(G_2)$
\end{definition}


\begin{example}  Пример эквивалентных грамматик для языка целых чисел в двоичной системе счисления.

  \begin{tabular}{p{0.4\textwidth} | p{0.5\textwidth}}

    \[
      \begin{array}{rcl}
      \Sigma &=& \{ 0, 1, - \} \\
      N &=& \{ S, N, A \} \\~\\
      S& \rightarrow & 0 \mid N \mid - N  \\
      N& \rightarrow & 1 A \\
      A& \rightarrow & 0 A \mid 1 A  \mid \varepsilon\\
      \end{array}
    \]

    &

    \[
      \begin{array}{rcl}
      \Sigma &=& \{ 0, 1, - \} \\
      N &=& \{ S, A \} \\~\\
      S& \rightarrow & 0 \mid 1 A  \mid - 1 A  \\
      A& \rightarrow &  0 A \mid 1 A  \mid \varepsilon\\
      \end{array}
    \]
    \end{tabular}

\end{example}


\begin{definition}
  \textit{Неоднозначная грамматика} --- грамматика, в которой существует 2 и более левосторонних (правосторонних) выводов для одного слова.
\end{definition}

\begin{example}
  Неоднозначная грамматика для правильных скобочных последовательностей

\[
    S \to (S) \mid S S \mid \varepsilon
\]
\end{example}

\begin{definition}
  \textit{Однозначная грамматика} --- грамматика, в которой существует не более одного левостороннего (правостороннего) вывода для каждого слова.
\end{definition}

\begin{example}
  Однозначная грамматика для правильных скобочных последовательностей

\[
    S \to (S)S \mid \varepsilon
\]
\end{example}

\begin{definition}
  \textit{Существенно неоднозначные языки} --- языки, для которых невозможно построить однозначную грамматику.
\end{definition}

\begin{example}
  Пример существенно неоднозначного языка

\[\{a^n b^n c^m \mid n, m \in \mathds{Z}\} \cup \{a^n b^m c^m \mid n,m \in \mathds{Z}\}\]
\end{example}

\section{Дерево вывода}\label{sect:DerivTree}
В некоторых случаях не достаточно знать порядок применения правил.
Необходимо структурное представление вывода цепочки в грамматике.
Таким представлением является \textit{дерево вывода}.
\begin{definition}
Деревом вывода цепочки $\omega$ в грамматике $G=\langle \Sigma, N, S, P \rangle$ называется дерево, удовлетворяющее следующим свойствам.

\begin{enumerate}
  \item Помеченное: метка каждого внутреннего узла --- нетерминал, метка каждого листа --- терминал или $\varepsilon$.
  \item Корневое: корень помечен стартовым нетерминалом.
  \item Упорядоченное.
  \item В дереве может существовать узел с меткой $N_i$ и сыновьями $M_j \dots M_k$ только тогда, когда в грамматике есть правило вида $N_i \to M_j \dots M_k$.
  \item Крона образует исходную цепочку $\omega$.
\end{enumerate}
\end{definition}

\begin{example}
  Построим дерево вывода цепочки $ababab$ в грамматике

  \[ G = \langle \{a,b\}, \{S\}, S, \{S \to a \ S \ b \ S, S \to \varepsilon\} \rangle \]

\begin{center}

    \begin{tikzpicture}[sibling distance=4em,
    every node/.style = {shape=rectangle, rounded corners,
      draw, align=center,
      top color=white, bottom color=blue!20}]]
    \node {S}
      child { node {a} }
      child { node {S}
        child { node {$\varepsilon$}}
      }
      child { node {b} }
      child { node {S}
        child {node {a}}
        child { node {S}
          child { node {$\varepsilon$}}
        }
        child { node {b} }
        child { node {S}
          child {node {a}}
          child {node {S}
            child {node {$\varepsilon$}}
          }
          child {node {b}}
          child {node {S}
            child {node {$\varepsilon$}}
          }
        }
      };
  \end{tikzpicture}
\end{center}

\end{example}

\begin{theorem}
  Пусть $G = \langle \Sigma, N, P, S \rangle$ --- КС-грамматика.
  Вывод $S \derives \alpha$, где $\alpha \in (N \cup \Sigma)^*, \alpha \neq \varepsilon$ существует $\Leftrightarrow$ существует дерево вывода в грамматике $G$ с кроной $\alpha$.
\end{theorem}

\section{Пустота КС-языка}

\begin{theorem}
  Существует алгоритм, определяющий, является ли язык, порождаемый КС грамматикой, пустым.
\end{theorem}

\begin{proof}
  Следующая лемма утверждает, что если в КС языке есть выводимое слово, то существует другое выводимое слово с деревом вывода не глубже количества нетерминалов грамматики.
  Для доказательства теоремы достаточно привести алгоритм, последовательно строящий все деревья глубины не больше количества нетерминалов грамматики, и проверяющий, являются ли такие деревья деревьями вывода.
  Если в результате работы алгоритма не удалось построить ни одного дерева, то грамматика порождает пустой язык.
\end{proof}

\begin{lemma}
  Если в данной грамматике выводится некоторая цепочка, то существует цепочка, дерево вывода которой не содержит ветвей длиннее m, где m --- количество нетерминалов грамматики.
\end{lemma}

\begin{proof}
  Рассмотрим дерево вывода цепочки $\omega$. Если в нем есть 2 узла, соответствующих одному нетерминалу A, обозначим их $n_1$ и $n_2$.

  Предположим, $n_1$ расположен ближе к корню дерева, чем $n_2$.

  $S \derives \alpha A_{n_1} \beta \derives \alpha \omega_1 \beta; S \derives \alpha \gamma A_{n_2} \delta \beta \derives \alpha \gamma \omega_2 \delta \beta$, при этом $\omega_2$ является подцепочкой $\omega_1$.

  Заменим в изначальном дереве узел $n_1$ на $n_2$. Полученное дерево является деревом вывода $\alpha \omega_2 \delta$.

  Повторяем процесс замены одинаковых нетерминалов до тех пор, пока в дереве не останутся только уникальные нетерминалы.

  В полученном дереве не может быть ветвей длины большей, чем m.

  По построению оно является деревом вывода.
\end{proof}


\section{Нормальная форма Хомского}
\label{section:CNF}

\begin{definition}
Контекстно-свободная грамматика $\langle \Sigma, N, P, S\rangle$ находится в \textit{Нормальной Форме Хомского}, если она содержит только правила следующего вида:

\begin{itemize}
  \item $A \to B C \text{, где } A, B, C \in N \text{, S не содержится в правой части правила }$
  \item $A \to a \text{, где } A \in N, a \in \Sigma$
  \item $S \to \varepsilon$
\end{itemize}
\end{definition}

\begin{theorem}
Любую КС грамматику можно преобразовать в НФХ.
\end{theorem}

\begin{proof}
  Алгоритм преобразования в НФХ состоит из следующих шагов:

  \begin{itemize}
    \item Замена неодиночных терминалов
    \item Удаление длинных правил
    \item Удаление $\varepsilon$-правил
    \item Удаление цепных правил
    \item Удаление бесполезных нетерминалов
  \end{itemize}

  То, что каждый из этих шагов преобразует грамматику к эквивалентной, при этом является алгоритмом, доказано в следующих леммах.
\end{proof}

\begin{lemma}
  Для любой КС-грамматики можно построить эквивалентную, которая не содержит правила с неодиночными терминалами.
\end{lemma}

\begin{proof}
  Каждое правило $A \to B_0 B_1 \dots B_k, k \geq 1$ заменить на множество правил:
  \begin{itemize}
    \item $A \to C_0 C_1 \dots C_k$
    \item $\{ C_i \to B_i \mid B_i \in \Sigma, C_i \text{ --- новый нетерминал} \}$
  \end{itemize}
\end{proof}

\begin{lemma}
  Для любой КС-грамматики можно построить эквивалентную, которая не содержит правил длины больше 2.
\end{lemma}

\begin{proof}
  Каждое правило $A \to B_0 B_1 \dots B_k, k \geq 2$ заменить на множество правил:
  \begin{itemize}
    \item $A \to B_0 C_0$
    \item $C_0 \to B_1 C_1$
    \item $\dots$
    \item $C_{k-3} \to B_{k-2} C_{k-2}$
    \item $C_{k-2} \to B_{k-1} B_k$
  \end{itemize}
\end{proof}


\begin{lemma}
  Для любой КС-грамматики можно построить эквивалентную, не содержащую $\varepsilon$-правил.
\end{lemma}

\begin{proof}
  Определим $\varepsilon$-правила:
  \begin{itemize}
    \item $A \to \varepsilon$
    \item $A \to B_0 \dots B_k, \forall i: \ B_i$ --- $\varepsilon$-правило.
  \end{itemize}

  Каждое правило $A \to B_0 B_1 \dots B_k$ заменяем на множество правил, где каждое $\varepsilon$-правило удалено во всех возможных комбинациях.
\end{proof}

\begin{lemma}
  Можно удалить все цепные правила
\end{lemma}

\begin{proof}
  \textit{Цепное правило} --- правило вида $A \to B\text{, где } A, B \in N\\$.
  \textit{Цепная пара} --- упорядоченная пара $(A,B)$, в которой $A\derives B$, используя только цепные правила.
  
  Алгоритм:
  \begin{enumerate}
  \item Найти все цепные пары в грамматике $G$.
  Найти все цепные пары можно по индукции:
  Базис: $(A,A)$ --- цепная пара для любого нетерминала, так как $A\derives A$ за ноль шагов.
  Индукция: Если пара $(A,B_0)$ --- цепная, и есть правило $B_0 \to B_1$, то $(A,B_1)$ --- цепная пара.
  \item Для каждой цепной пары $(A,B)$ добавить в грамматику $G'$ все правила вида $A \to a$, где $B \to a$ --- нецепное правило из $G$.
  \item Удалить все цепные правила
\end{enumerate}
Пусть $G$ --- контекстно-свободная грамматика. $G'$ --- грамматика, полученная в результате применения алгоритма к $G$. Тогда $L(G)=L(G')$.
\end{proof}

\begin{definition}
Нетерминал $A$ называется \textit{порождающим}, если из него может быть выведена конечная терминальная цепочка. Иначе он называется \textit{непорождающим}.
\end{definition}

\begin{lemma}
  Можно удалить все бесполезные (непорождающие) нетерминалы
\end{lemma}

\begin{proof}
  После удаления из грамматики правил, содержащих непорождающие нетерминалы, язык не изменится, так как непорождающие нетерминалы по определению не могли участвовать в выводе какого-либо слова.
  
  Алгоритм нахождения порождающих нетерминалов:
  \begin{enumerate}
  \item Множество порождающих нетерминалов пустое.
  \item Найти правила, не содержащие нетерминалов в правых частях и добавить нетерминалы, встречающихся в левых частях таких правил, в множество.
  \item Если найдено такое правило, что все нетерминалы, стоящие в его правой части, уже входят в множество, то добавить в множество нетерминалы, стоящие в его левой части.
  \item Повторить предыдущий шаг, если множество порождающих нетерминалов изменилось.
\end{enumerate}
В результате получаем множество всех порождающих нетерминалов грамматики, а все нетерминалы, не попавшие в него, являются непорождающими. Их можно удалить.
\end{proof}

\begin{example}
  Приведем в Нормальную Форму Хомского однозначную грамматику правильных скобочных последовательностей: $S \to a S b S \mid \varepsilon$

  Первым шагом добавим новый нетерминал и сделаем его стартовым: 

  \begin{align*}
    S_0 &\to S  \\ 
    S   &\to a S b S \mid \varepsilon
  \end{align*}

  Заменим все терминалы на новые нетерминалы: 

  \begin{align*}
    S_0 &\to S \\ 
    S   &\to L S R S \mid \varepsilon \\ 
    L   &\to a \\ 
    R   &\to b
  \end{align*}

  Избавимся от длинных правил: 

  \begin{align*}
    S_0 &\to S \\ 
    S   &\to L S' \mid \varepsilon \\ 
    S'  &\to S S'' \\ 
    S'' &\to R S \\
    L   &\to a \\ 
    R   &\to b
  \end{align*}

  Избавимся от $\varepsilon$-продукций: 

  \begin{align*}
    S_0 &\to S \mid \varepsilon \\ 
    S   &\to L S' \\ 
    S'  &\to S'' \mid S S'' \\ 
    S'' &\to R   \mid R S \\
    L   &\to a \\ 
    R   &\to b
  \end{align*}

  Избавимся от цепных правил: 

  \begin{align*}
    S_0 &\to L S' \mid \varepsilon \\ 
    S   &\to L S' \\ 
    S'  &\to b \mid R S \mid S S'' \\ 
    S'' &\to b \mid R S \\
    L   &\to a \\ 
    R   &\to b
  \end{align*}
\end{example}

\begin{definition}\label{defn:wCNF}
Контекстно-свободная грамматика $\langle \Sigma, N, P, S\rangle$ находится в \textit{ослабленной Нормальной Форме Хомского}, если она содержит только правила следующего вида:

\begin{itemize}
  \item $A \to B C \text{, где } A, B, C \in N$
  \item $A \to a \text{, где } A \in N, a \in \Sigma$
  \item $A \to \varepsilon \text{, где } A \in N$
\end{itemize}

То есть ослабленная НФХ отличается от НФХ тем, что:
\begin{enumerate}
  \item $\varepsilon$ может выводиться из любого нетерминала
  \item $S$ может появляться в правых частях правил
\end{enumerate}
\end{definition}

\section{Лемма о накачке}

\begin{lemma}
Пусть $L$ --- контекстно-свободный язык над алфавитом $\Sigma$, тогда существует такое $n$, что для любого слова $\omega \in L$, $|\omega| \geq n$ найдутся слова $u,v,x,y,z\in \Sigma^*$, для которых верно: $uvxyz = \omega, vy\neq \varepsilon,|vxy|\leq n$ и для любого $k \geq 0$  $uv^kxy^kz \in L$.
\end{lemma}

Идея доказательства леммы о накачке.

\begin{enumerate}
    \item Для любого КС языка можно найти грамматику в нормальной форме Хомского.
    \item Очевидно, что если брать достаточно длинные цепочки, то в дереве вывода этих цепочек, на пути от корня к какому-то листу обязательно будет нетерминал, встречающийся минимум два раза. Если $m$ --- количество нетерминалов в НФХ, то длины $2^{m+1}$ должно хватить. Это и будет $n$ из леммы.
    \item Возьмём путь, на котором есть хотя бы дважды повторяется некоторый нетерминал. Скажем, это нетерминал  $N_1$. Пойдём от листа по этому пути. Найдём первое появление $N_1$. Цепочка, задаваемая поддеревом для этого узла --- это $x$ из леммы.
    \item Пойдём дальше и найдём второе появление $N_1$. Цепочка, задаваемая поддеревом для этого узла --- это $vxy$ из леммы.
    \item Теперь мы можем копировать кусок дерева между этими повторениями $N_1$ и таким образом накачивать исходную цепочку.
\end{enumerate}

Надо только проверить выполение ограничений на длины.

Нахождение разбиения и пример накачки продемонстрированы на рисунках~\ref{fig:pumping1} и~\ref{fig:pumping2}.

\begin{figure}
\centering
\includegraphics[width=0.5\textwidth]{pics/pumping_tree_1.pdf}
\caption{Разбиение цепочки для леммы о накачке}
\label{fig:pumping1}
\end{figure}

\begin{figure}
\centering
\includegraphics[width=0.5\textwidth]{pics/pumping_tree_2.pdf}
\caption{Пример накачки цепочки с рисунка~\ref{fig:pumping1}}
\label{fig:pumping2}
\end{figure}


Для примера предлагается проверить неконтекстно-свободность языка $L=\{a^nb^nc^n \mid n>0\}$.


\section{Замкнутость КС языков относительно операций}


\begin{theorem}
Контекстно-свободные языки замкнуты относительно следующих операций:
\begin{enumerate}
  \item Объединение: если $L_1$ и $L_2$ --- контекстно-свободные языки, то и $L_3 = L_1 \cup L_2$ --- контекстно-свободный.
  \item Конкатенация: если $L_1$ и $L_2$ --- контекстно-свободные языки, то и $L_3 = L_1 \cdot L_2$ --- контекстно-свободный.
  \item Замыкание Клини: если $L_1$ --- контекстно-свободный, то и $L_2 = \bigcup\limits_{i=0}^{\infty} L_1^i $ --- контекстно-свободный.
  \item Разворот: если $L_1$ --- контекстно-свободный, то и $L_2 = {L_1}^r$ --- контекстно-свободный.
  \item Пересечение с регулярными языками: если $L_1$ --- контекстно-свободный, а $L_2$ --- регулярный, то  $L_3 = L_1 \cap L_2$ --- контекстно-свободный.
  \item Разность с регулярными языками: если $L_1$ --- контекстно-свободный, а $L_2$ --- регулярный, то  $L_3 = L_1 \setminus L_2$ --- контекстно-свободный.
\end{enumerate}
\end{theorem}
Для доказательства пунктов 1--4 можно построить КС граммтику нового языка имея грамматики для исходных. 
Будем предполагать, что множества нетерминальных символов различных граммтик для исходных языков не пересекаются.
\begin{enumerate}
\item $G_1=\langle\Sigma_1,N_1,P_1,S_1\rangle$ --- граммтика для $L_1$, $G_1=\langle\Sigma_2,N_2,P_2,S_2\rangle$ --- граммтика для $L_2$, тогда $G_3=\langle\Sigma_1 \cup \Sigma_2, N_1 \cup N_2 \cup \{S_3\}, P_1 \cup P_2 \cup \{S_3 \to S_1 \mid S_2\} ,S_3\rangle$ --- граммтика для $L_3$. 

\item $G_1=\langle\Sigma_1,N_1,P_1,S_1\rangle$ --- граммтика для $L_1$, $G_1=\langle\Sigma_2,N_2,P_2,S_2\rangle$ --- граммтика для $L_2$, тогда $G_3=\langle\Sigma_1 \cup \Sigma_2, N_1 \cup N_2 \cup \{S_3\}, P_1 \cup P_2 \cup \{S_3 \to S_1 S_2\} ,S_3\rangle$ --- граммтика для $L_3$. 

\item $G_1=\langle\Sigma_1,N_1,P_1,S_1\rangle$ --- граммтика для $L_1$, тогда $G_2=\langle\Sigma_1, N_1 \cup \{S_2\}, P_1 \cup \{S_2 \to S_1 S_2\ \mid \varepsilon\}, S_2\rangle$ --- граммтика для $L_2$. 

\item $G_1=\langle\Sigma_1,N_1,P_1,S_1\rangle$ --- граммтика для $L_1$, тогда $G_2=\langle\Sigma_1, N_1, \{N^i \to \omega^R \mid N^i \to \omega \in P_1 \}, S_1\rangle$ --- граммтика для $L_2$. 
\end{enumerate}

Чтобы доказать замкнутость относительно пересечения с регулярными языками, построим по КС грамматике рекурсивный автомат $R_1$, по регулярному выражению --- детерминированный конечный автомат $R_2$, и построим их прямое произведение $R_3$.
Переходы по терминальным символам в новом автомате возможны тогда и только тогда, когда они возможны одновременно и в исходном рекурсивном автомате и в исходном конечном. 
За рекурсивные вызовы отвечает исходныа рекурсивный автомат. 
Значит цепочка принимается $R_3$ тогда и только тогда, когда она принимается одновременно $R_1$ и $R_2$: так как состояния $R_3$ --- это пары из состояния $R_1$ и $R_2$, то по трассе вычислений $R_3$ мы всегда можем построить трассу для $R_1$ и $R_2$ и наоборот.

Чтобы доказать замкнутость относительно разности с регулятным языком, достаточно вспомнить, что регулярные языки замкнуты относительно дополнения, и выразить разность через пересечение с дополнением: 
$$
L_1 \setminus L_2 = L_1 \cap \overline{L_2}
$$

\qed

\begin{theorem}
Контекстно-свободные языки не замкнуты относительно следующих операций:
\begin{enumerate}
  \item Пересечение: если $L_1$ и $L_2$ --- контекстно-свободные языки, то и $L_3 = L_1 \cap L_2$ --- не контекстно-свободный.
  \item Разность: если $L_1$ и $L_2$ --- контекстно-свободные языки, то и $L_3 = L_1 \setminus L_2$ --- не контекстно-свободный.
\end{enumerate}
\end{theorem}

Чтобы доказать незамкнутость относительно пресечения, рассмотрим языки $L_1 = \{a^n b^n c^k \mid n \geq 0, k \geq 0\}$ и $L_2 = \{a^k b^n c^n \mid n \geq 0, k \geq 0\}$.
Очевидно, что $L_1$ и $L_2$ --- контекстно-свободные языки.
Рассмотрим $L_3 = L_1 \cap L_2 = \{a^n b^n c^n \mid n \geq 0\}$. 
$L_3$ не является контекстно-свободным по лемме о накачке для контекстно-свободных языков.

Чтобы доказать незамкнутость относительно разности проделаем следующее.
\begin{enumerate}
\item Рассмотрим языки $L_4 = \{a^m b^n c^k \mid m \neq n, k \geq 0\}$ и $L_5 = \{a^m b^n c^k \mid n \neq k, m \geq 0\}$. 
Эти языки являются контекстно-свободными.
Это легко заметить, если знать, что язык $L'_4 = \{a^m b^n c^k \mid 0 \leq m < n, k \geq 0\}$ задаётся следующей граммтикой:
\begin{align*}
S \to & S c & T \to & a T b \\
S \to & T &   T \to & T b \\
      &   &   T \to & b. 
\end{align*} 

\item Рассмотрим язык $L_6 = \overline{L'_6} = \overline{\{a^n b^m c^k \mid n \geq 0, m \geq 0, k \geq 0\}}$. Данный язык является регулярным.

\item Рассмотрим язык $L_7 = L_4 \cup L_5 \cup L_6$ --- контектсно свободный, так как является объединением контекстно-свободных.

\item Рассмотрим $\overline{L_7} = \{a^n b^n c^n \mid n \geq 0\} = L_3$: $L_4$ и $L_5$ задают языки с правильным порядком символов, но неравным их количеством, $L_6$ задаёт язык с неправильным порядком символов. 
Из пердыдущего пункта мы знаем, что $L_3$  не является контекстно-свободным.

\end{enumerate}

\qed

%\section{Вопросы и задачи}
%\begin{enumerate}
%  \item Постройте дерево вывода цепочки $w=aababb$ в грамматике $G=\langle\{a,b\},\{S\},\{S\rightarrow \varepsilon \ | \ a \ S \ b \ S \}, S \rangle$.
%  \item Постройте все левосторонние выводы цепочки $w=ababab$ в грамматике $G=\langle\{a,b\},\{S\},\{S\rightarrow \varepsilon \ | \ a \ S \ b \ | S \ S\}, S \rangle$.
%  \item Постройте все правосторонние выводы цепочки $w=ababab$ в грамматике $G=\langle\{a,b\},\{S\},\{S\rightarrow \varepsilon \ | \ a \ S \ b \ | S \ S\}, S \rangle$.
%  \item \label{t1}Постройте все деревья вывода цепочки $w=ababab$ в грамматике $G=\langle\{a,b\},\{S\},\{S\rightarrow \varepsilon \ | \ a \ S \ b \ | S \ S\}, S \rangle$, соответствующие левосторонним выводам.
%  \item \label{t2}Постройте все деревья вывода цепочки $w=ababab$ в грамматике $G=\langle\{a,b\},\{S\},\{S\rightarrow \varepsilon \ | \ a \ S \ b \ | S \ S\}, S \rangle$, соответствующие правосторонним выводам.
%\end{enumerate}

\chapter{CYK для вычисления КС запросов}\label{chpt:CFPQ_CYK}

В данной главе мы рассмотрим алгоритм CYK, позволяющий установить принадлежность слова грамматике и предоставить его вывод, если таковой имеется.

Наш главный интерес заключается в возможности применения данного алгоритма для решения описанной в предыдущей главе задачи --- поиска путей с ограничениями в терминах формальных языков. Как уже было указано выше, будем рассматривать случай контекстно-свободных языков.

\section{Алгоритм CYK}\label{sect:lin_CYK}

Алгоритм CYK (Cocke-Younger-Kasami) --- один из классических алгоритмов синтаксического анализа. Его асимптотическая сложность в худшем случае --- $O(n^3 * |N|)$, где $n$ --- длина входной строки, а $N$ --- количество нетерминалов во входной граммтике~\cite{Hopcroft+Ullman/79/Introduction}. 

Для его применения необходимо, чтобы подаваемая на вход грамматика находилась в Нормальной Форме Хомского (НФХ)~\ref{section:CNF}. Других ограничений нет и, следовательно,данный алгоритм применим для работы с произвольными контекстно-своболными языками.

В основе алгоритма лежит принцип динамического программирования. Используются два соображения:

\begin{enumerate}
\item Для правила вида $A \to a$ очевидно, что из $A$ выводится $\omega$ (с применением этого правила) тогда и только тогда, когда $a = \omega$:

\[
  A \derives \omega \iff \omega = a
\]

\item Для правила вида $A \to B C$ понятно, что из $A$ выводится $\omega$ (с применением этого правила) тогда и только тогда, когда существуют две цепочки $\omega_1$ и $\omega_2$ такие, что $\omega_1$ выводима из $B$, $\omega_2$ выводима из $C$ и при этом $\omega = \omega_1 \omega_2$:

\[
A \derives[] B C \derives \omega \iff \exists \omega_1, \omega_2 : \omega = \omega_1 \omega_2, B \derives \omega_1, C \derives \omega_2
\]

Или в терминах позиций в строке:

\[
A \derives[] B C \derives \omega \iff \exists k \in [1 \dots |\omega|] : B \derives \omega[1 \dots k], C \derives \omega[k+1 \dots |\omega|]
\]
\end{enumerate}

В процессе работы алгоритма заполняется булева трехмерная матрица $M$ размера $n \times n \times  |N|$ таким образом, что $$M[i, j, A] = true \iff A \derives \omega[i \dots j]$$.

Первым шагом инициализируем матрицу, заполнив значения $M[i, i, A]$:

\begin{itemize}
  \item $M[i, i, A] = true \text{, если в грамматике есть правило } A \to \omega[i]$.
  \item $M[i, i, A] = false$, иначе.
\end{itemize}

Далее используем динамику: на шаге $m > 1$ предполагаем, что ячейки матрицы $M[i', j', A]$ заполнены для всех нетерминалов $A$ и пар $i', j': j' - i' < m$.
Тогда можно заполнить ячейки матрицы $M[i, j, A] \text{, где } j - i = m$ следующим образом:

\[ M[i, j, A] = \bigvee_{A \to B C}^{}{\bigvee_{k=i}^{j-1}{M[i, k, B] \wedge M[k, j, C]}} \]

По итогу работы алгоритма значение в ячейке $M[0, |\omega|, S]$, где $S$ --- стартовый нетерминал грамматики, отвечает на вопрос о выводимости цепочки $\omega$ в грамматике.

\begin{example}\label{exampl:CYK}
  Рассмотрим пример работы алгоритма CYK на грамматике правильных скобочных последовательностей в Нормальной Форме Хомского.


\begin{align*}
S &\to A S_2 \mid \varepsilon & S_2  &\to b \mid B S_1 \mid S_1 S_3   & A   &\to a \\
S_1   &\to A S_2              & S_3  &\to b \mid B S_1              & B   &\to b\\      
\end{align*}

Проверим выводимость цепочки $\omega = a a b b a b$.

Так как трехмерные матрицы рисовать на двумерной бумаге не очень удобно, мы будем иллюстрировать работу алгоритма двумерными матрицами размера $n \times n$, где в ячейках указано множество нетерминалов, выводящих соответствующую подстроку.

Шаг 1. Инициализируем матрицу элементами на главной диагонали:

\[
\begin{pmatrix}
\{A\}       & \varnothing & \varnothing    & \varnothing      & \varnothing & \varnothing    \\
\varnothing & \{A\}       & \varnothing    & \varnothing      & \varnothing & \varnothing    \\
\varnothing & \varnothing & \{B, S_2, S_3\} & \varnothing     & \varnothing & \varnothing    \\
\varnothing & \varnothing & \varnothing    & \{B, S_2, S_3\}   & \varnothing & \varnothing   \\
\varnothing & \varnothing & \varnothing    & \varnothing      & \{A\}       & \varnothing    \\
\varnothing & \varnothing & \varnothing    & \varnothing      & \varnothing & \{B, S_2, S_3\} \\
\end{pmatrix}
\]

Шаг 2. Заполняем диагональ, находящуюся над главной:

\[
\begin{pmatrix}
\{A\}       & \varnothing & \varnothing                             & \varnothing      & \varnothing & \varnothing    \\
\varnothing & \{A\}       & \cellcolor{lightgray}\{S_1\}            & \varnothing      & \varnothing & \varnothing    \\
\varnothing & \varnothing & \{B, S_2, S_3\} & \varnothing     & \varnothing & \varnothing    \\
\varnothing & \varnothing & \varnothing    & \{B, S_2, S_3\}   & \varnothing & \varnothing   \\
\varnothing & \varnothing & \varnothing    & \varnothing      & \{A\}       & \cellcolor{lightgray}\{S_1\}            \\
\varnothing & \varnothing & \varnothing    & \varnothing      & \varnothing & \{B, S_2, S_3\} \\
\end{pmatrix}
\]

В двух ячейках появилисб нетерминалы $S_1$ благодяря присутствиб в грамматике правила $S_1 \to A S_2$.

Шаг 3. Заполняем следующую диагональ:

\[
\begin{pmatrix}
\{A\}       & \varnothing & \varnothing    & \varnothing      & \varnothing & \varnothing    \\
\varnothing & \{A\}       & \{S_1\}        & \cellcolor{lightgray}\{S_2\}          & \varnothing & \varnothing    \\
\varnothing & \varnothing & \{B, S_2, S_3\} & \varnothing     & \varnothing & \varnothing    \\
\varnothing & \varnothing & \varnothing    & \{B, S_2, S_3\}   & \varnothing & \cellcolor{lightgray}\{S_2, S_3\}  \\
\varnothing & \varnothing & \varnothing    & \varnothing      & \{A\}       & \{S_1\}            \\
\varnothing & \varnothing & \varnothing    & \varnothing      & \varnothing & \{B, S_2, S_3\} \\
\end{pmatrix}
\]

Шаг 4. И следующую за ней:

\[
\begin{pmatrix}
\{A\}       & \varnothing & \varnothing    & \cellcolor{lightgray}\{S_1, S\}       & \varnothing & \varnothing    \\
\varnothing & \{A\}       & \{S_1\}            & \{S_2\}          & \varnothing & \varnothing    \\
\varnothing & \varnothing & \{B, S_2, S_3\} & \varnothing     & \varnothing & \varnothing    \\
\varnothing & \varnothing & \varnothing    & \{B, S_2, S_3\}   & \varnothing & \{S_2, S_3\}  \\
\varnothing & \varnothing & \varnothing    & \varnothing      & \{A\}       & \{S_1\}            \\
\varnothing & \varnothing & \varnothing    & \varnothing      & \varnothing & \{B, S_2, S_3\} \\
\end{pmatrix}
\]

Шаг 5 Заполняем предпоследнюю диагональ:

\[
\begin{pmatrix}
\{A\}       & \varnothing & \varnothing    & \{S_1, S\}       & \varnothing & \varnothing    \\
\varnothing & \{A\}       & \{S_1\}            & \{S_2\}          & \varnothing & \cellcolor{lightgray}\{S_2\}        \\
\varnothing & \varnothing & \{B, S_2, S_3\} & \varnothing     & \varnothing & \varnothing    \\
\varnothing & \varnothing & \varnothing    & \{B, S_2, S_3\}   & \varnothing & \{S_2, S_3\}  \\
\varnothing & \varnothing & \varnothing    & \varnothing      & \{A\}       & \{S_1\}            \\
\varnothing & \varnothing & \varnothing    & \varnothing      & \varnothing & \{B, S_2, S_3\} \\
\end{pmatrix}
\]

\bigbreak
Шаг 6. Завершаем выполнение алгоритма:

\[
\begin{pmatrix}
\{A\}       & \varnothing & \varnothing    & \{S_1, S\}       & \varnothing & \cellcolor{lightgray}\{S_1, S\}     \\
\varnothing & \{A\}       & \{S_1\}            & \{S_2\}          & \varnothing & \{S_2\}        \\
\varnothing & \varnothing & \{B, S_2, S_3\} & \varnothing     & \varnothing & \varnothing    \\
\varnothing & \varnothing & \varnothing    & \{B, S_2, S_3\}   & \varnothing & \{S_2, S_3\}  \\
\varnothing & \varnothing & \varnothing    & \varnothing      & \{A\}       & \{S_1\}            \\
\varnothing & \varnothing & \varnothing    & \varnothing      & \varnothing & \{B, S_2, S_3\} \\
\end{pmatrix}
\]


Стартовый нетерминал находится в верхней правой ячейке, а значит цепочка $a a b b a b$ выводима в нашей грамматике.
\end{example}

\begin{example}
Теперь выполним алгоритм на цепочке $\omega=abaa$.

Шаг 1. Инициализируем таблицу:

\[
\begin{pmatrix}
\{A\}       & \varnothing    & \varnothing & \varnothing    \\
\varnothing & \{B, S_2, S_3\} & \varnothing & \varnothing       \\
\varnothing & \varnothing    & \{A\}       & \varnothing    \\
\varnothing & \varnothing    & \varnothing & \{A\}          \\
\end{pmatrix}
\]

Шаг 2. Заполняем следующую диагональ:

\[
\begin{pmatrix}
\{A\}       & \cellcolor{lightgray}\{S_1, S\}     & \varnothing & \varnothing    \\
\varnothing & \{B, S_2, S_3\} & \varnothing & \varnothing       \\
\varnothing & \varnothing    & \{A\}       & \varnothing    \\
\varnothing & \varnothing    & \varnothing & \{A\}          \\
\end{pmatrix}
\]

Больше ни одну ячейку в таблице заполнить нельзя и при этом стартовый нетерминал отсутствует в правой верхней ячейке, а значит эта строка не выводится в грамматике правильных скобочных последовательностей.

\end{example}

\section{Алгоритм для графов на основе CYK}
\label{graph:CYK}
Первым шагом на пути к решению задачи достижимости с использованием CYK является модификация представления входа. Прежде мы сопоставляли каждому символу слова его позицию во входной цепочке, поэтому при инициализации заполняли главную диагональ матрицы. Теперь вместо этого обозначим числами позиции между символами. В результате слово можно представить в виде линейного графа следующим образом(в качестве примера рассмотрим слово $a a b b a b$ из предыдущей главы~\ref{sect:lin_CYK}):

\begin{center}
    \begin{tikzpicture}[shorten >=1pt,on grid,auto]
    \node[state] (q_0) at (0,0)  {$0$};
    \node[state] (q_1) at (2,0)  {$1$};
    \node[state] (q_2) at (4,0)  {$2$};
    \node[state] (q_3) at (6,0)  {$3$};
    \node[state] (q_4) at (8,0)  {$4$};
    \node[state] (q_5) at (10,0) {$5$};
    \node[state] (q_6) at (12,0) {$6$};
    \path[->]
    (q_0) edge  node {$a$} (q_1)
    (q_1) edge  node {$a$} (q_2)
    (q_2) edge  node {$b$} (q_3)
    (q_3) edge  node {$b$} (q_4)
    (q_4) edge  node {$a$} (q_5)
    (q_5) edge  node {$b$} (q_6);
    \end{tikzpicture}
\end{center}

Что нужно изменить в описании алгоритма, чтобы он продолжал работать при подобной нумерации? Каждая буква теперь идентифицируется не одним числом, а парой --- номера слева и справа от нее. При этом чисел стало на одно больше, чем при прежнем способе нумерации.

Возьмем матрицу  $(n + 1) \times (n + 1) \times  |N|$ и при инициализации будем заполнять не главную диагональ, а диагональ прямо над ней. Таким образом, мы начинаем наш алгоритм с определения значений $M[i, j, A] \text{, где } j = i + 1$. При этом наши дальнейшие действия в рамках алгоритма не изменятся.

Для примера~\ref{exampl:CYK} на шаге инициализации матрица выглядит следующим образом:

\[
\begin{pmatrix}
\varnothing & \{A\}       & \varnothing & \varnothing    & \varnothing    & \varnothing & \varnothing    \\
\varnothing & \varnothing & \{A\}     & \varnothing    & \varnothing      & \varnothing & \varnothing    \\
\varnothing & \varnothing & \varnothing & \{B, S_2, S_3\} & \varnothing       & \varnothing & \varnothing    \\
\varnothing & \varnothing & \varnothing & \varnothing    & \{B, S_2, S_3\} & \varnothing & \varnothing   \\
\varnothing & \varnothing & \varnothing & \varnothing    & \varnothing    & \{A\}       & \varnothing    \\
\varnothing & \varnothing & \varnothing & \varnothing    & \varnothing    & \varnothing & \{B, S_2, S_3\} \\
\varnothing & \varnothing & \varnothing & \varnothing    & \varnothing    & \varnothing & \varnothing    \\

\end{pmatrix}
\]

А в результате работы алгоритма имеем:

\[
\begin{pmatrix}
\varnothing & \{A\}       & \varnothing & \varnothing    & \{S_1, S\}     & \varnothing & \{S_1, S\}     \\
\varnothing & \varnothing & \{A\}       & \{S_1\}        & \{S_2\}            & \varnothing & \{S_2\}        \\
\varnothing & \varnothing & \varnothing & \{B, S_2, S_3\} & \varnothing       & \varnothing & \varnothing    \\
\varnothing & \varnothing & \varnothing & \varnothing    & \{B, S_2, S_3\} & \varnothing & \{S_2, S_3\}  \\
\varnothing & \varnothing & \varnothing & \varnothing    & \varnothing    & \{A\}       & \{S_1\}            \\
\varnothing & \varnothing & \varnothing & \varnothing    & \varnothing    & \varnothing & \{B, S_2, S_3\} \\
\varnothing & \varnothing & \varnothing & \varnothing    & \varnothing    & \varnothing & \varnothing    \\
\end{pmatrix}
\]

Мы представили входную строку в виде линейного графа, а на шаге инициализации получили его матрицу смежности. Добавление нового нетерминала в язейку матрицы можно рассматривать как нахождение нового пути между соответствующими вершинами, выводимого из добавленного нетерминала. Таким образом, шаги алгоритма напоминают построение транзитивного замыкания графа. Различие заключается в том, что мы добавляем новые ребра только для тех пар нетерминалов, для которых существует соответстующее правило в грамматике.

Алгоритм можно обобщить и на произвольные графы с метками, рассматриваемые в этом курсе. При этом можно ослабить ограничение на форму входной грамматики: она должна находиться в ослабленной Нормальной Форме Хомского~(\ref{defn:wCNF}).

Шаг инициализации в алгоритме теперь состоит из двух пунктов.
\begin{itemize}
\item Как и раньше, с помощью продукций вида \[A \to a \text{, где } A \in N, a \in \Sigma\]
заменяем терминалы на ребрах входного графа на множества нетерминалов, из которых они выводятся.
\item  Добавляем в каждую вершину петлю, помеченную множеством нетерминалов для которых в данной граммтике есть правила вида $$A \to \varepsilon\text{, где } A \in N.$$ 
\end{itemize}

 Затем используем матрицу смежности получившегося графа (обозначим ее $M$) в качестве начального значения. Дальнейший ход алгоритма можно описать псевдокодом, представленным в листинге~\ref{alg:graphParseCYK}.

\begin{algorithm}[H]
    \begin{algorithmic}[1]
        \caption{Алгоритм КС достижимости на основе CYK}
        \label{alg:graphParseCYK}
        \Function{contextFreePathQuerying}{G, $\mathcal{G}$}

        \State{$n \gets$ the number of nodes in $\mathcal{G}$}
        \State{$M \gets$ the modified adjacency matrix of $\mathcal{G}$}
        \State{$P \gets$ the set of production rules in $G$}
        \While{$M$ is changing}
        \For {$k \in 0..n$}
            \For {$i \in 0..n$}
                \For {$j \in 0..n$}
                    \ForAll {$N_1 \in M[i, k]$, $N_2 \in M[k, j]$}
                        \If {$N_3 \to N_1 N_2 \in P$ }
                            \State{$M[i, j] \mathrel{+}= \{N_3\}$}
                        \EndIf
                    \EndFor
                \EndFor
            \EndFor
        \EndFor
        \EndWhile
        \State \Return $M$
        \EndFunction
    \end{algorithmic}
\end{algorithm}

После завершения алгоритма, если в некоторой ячейке результируюшей матрицы с номером $(i, j)$ находятся стартовый нетерминал, то это означает, что существует путь из вершины $i$ в вершину $j$, удовлетворяющий данной грамматике. Таким образом, полученная матрица является ответом для задачи достижимости для заданных графа и граммтики.

\begin{example}
\label{CYK_algorithm_ex}
Рассмотрим работу алгоритма на графе

\begin{center}
    \begin{tikzpicture}[node distance=3cm,shorten >=1pt,on grid,auto]
    \node[state] (q_0)   {$0$};
    \node[state] (q_1) [above right=of q_0] {$1$};
    \node[state] (q_2) [right=of q_0] {$2$};
    \node[state] (q_3) [right=of q_2] {$3$};
    \path[->]
    (q_0) edge  node {$a$} (q_1)
    (q_1) edge  node {$a$} (q_2)
    (q_2) edge  node {$a$} (q_0)
    (q_2) edge[bend left, above]  node {$b$} (q_3)
    (q_3) edge[bend left, below]  node {$b$} (q_2);
    \end{tikzpicture}
\end{center}

и грамматике:

\begin{align*}
S   & \to A B    & A   & \to a     \\
S   & \to A S_1  & B   & \to b\\
S_1 & \to S B   &&\\
\end{align*}

Данный пример является классическим и еще не раз будет использоваться в рамках данного курса. \\

\textbf{Инициализация.}
Заменяем терминалы на ребрах графа на нетерминалы, из которых они выводятся, и строим матрицу смежности получившегося графа:

\begin{center}
    \begin{tikzpicture}[node distance=3cm,shorten >=1pt,on grid,auto]
    \node[state] (q_0)   {$0$};
    \node[state] (q_1) [above right=of q_0] {$1$};
    \node[state] (q_2) [right=of q_0] {$2$};
    \node[state] (q_3) [right=of q_2] {$3$};
    \path[->]
    (q_0) edge  node {$\{A\}$} (q_1)
    (q_1) edge  node {$\{A\}$} (q_2)
    (q_2) edge  node {$\{A\}$} (q_0)
    (q_2) edge[bend left, above]  node {$\{B\}$} (q_3)
    (q_3) edge[bend left, below]  node {$\{B\}$} (q_2);
    \end{tikzpicture}
\end{center}

\[
\begin{pmatrix}
\varnothing & \{A\}       & \varnothing & \varnothing \\
\varnothing & \varnothing & \{A\}       & \varnothing \\
\{A\}       & \varnothing & \varnothing & \{B\}       \\
\varnothing & \varnothing & \{B\}       & \varnothing \\
\end{pmatrix}
\]

\textbf{Итерация 1.}
Итерируемся по $k$, $i$ и $j$, пытаясь найти пары нетерминалов, для которых существуют правила вывода, их выводящие. Нам интересны следующие случаи:

\begin{itemize}
    \item $k = 2, i = 1, j = 3: A \in M[1, 2], B \in M[2, 3]$, так как в грамматике присутствует правило $S \to A B$, добавляем нетерминал $S$ в ячейку $M[1, 3]$.
    \item $k = 3, i = 1, j = 2: S \in M[1, 3], B \in M[3, 2]$, поскольку в грамматике есть правило $S_1 \to S B$, добавляем нетерминал $S_1$ в ячейку $M[1, 2]$.
\end{itemize}

В остальных случаях либо какая-то из клеток пуста, либо не существует продукции в грамматике, выводящей данные два нетерминала.

Матрица после данной итерации:

\[
\begin{pmatrix}
\varnothing & \{A\}       & \varnothing & \varnothing \\
\varnothing & \varnothing & \cellcolor{lightgray}\{A, \boldsymbol{S_1}\}  & \cellcolor{lightgray}\{S\}       \\
\{A\}       & \varnothing & \varnothing & \{B\}       \\
\varnothing & \varnothing & \{B\}       & \varnothing \\
\end{pmatrix}
\]

\textbf{Итерация 2.}
Снова итерируемся по $k$, $i$, $j$. Рассмотрим случаи:

\begin{itemize}
    \setlength\itemsep{1em}
    \item $k = 1, i = 0, j = 2: A \in M[0, 1], S_1 \in M[1, 2]$, так как в грамматике присутствует правило $S \to A S_1$, добавляем нетерминал $S$ в ячейку $M[0, 2]$.
    \item $k = 2, i = 0, j = 3: S \in M[0, 2], B \in M[2, 3]$, поскольку в грамматике есть правило $S_1 \to S B$, добавляем нетерминал $S_1$ в ячейку $M[0, 3]$.
\end{itemize}

Матрица на данном шаге:

\[
\begin{pmatrix}
\varnothing & \{A\}       & \cellcolor{lightgray}\{S\}       & \cellcolor{lightgray}\{S_1\}     \\
\varnothing & \varnothing & \{A, S_1\}  & \{S\}       \\
\{A\}       & \varnothing & \varnothing & \{B\}       \\
\varnothing & \varnothing & \{B\}       & \varnothing \\
\end{pmatrix}
\]

\textbf{Итерация 3.}
Рассматриваемые на данном шаге случаи:

\begin{itemize}
    \setlength\itemsep{1em}
    \item $k = 0, i = 2, j = 3: A \in M[2, 0], S_1 \in M[0, 3]$, так как в грамматике присутствует правило $S \to A S_1$, добавляем нетерминал $S$ в ячейку $M[2, 3]$.
    \item $k = 3, i = 2, j = 2: S \in M[2, 3], B \in M[3, 2]$, поскольку в грамматике есть правило $S_1 \to S B$, добавляем нетерминал $S_1$ в ячейку $M[2, 2]$.
\end{itemize}

Матрица после этой итерации:

\[
\begin{pmatrix}
\varnothing & \{A\}       & \{S\}      & \{S_1\}     \\
\varnothing & \varnothing & \{A, S_1\} & \{S\}       \\
\{A\}       & \varnothing & \cellcolor{lightgray}\{S_1\}    & \cellcolor{lightgray}\{B, \boldsymbol{S}\}    \\
\varnothing & \varnothing & \{B\}      & \varnothing \\
\end{pmatrix}
\]

\textbf{Итерация 4.}
Рассмариваемые случаи:

\begin{itemize}
    \setlength\itemsep{1em}
    \item $k = 2, i = 1, j = 2: A \in M[1, 2], S_1 \in M[2, 2]$, так как в грамматике присутствует правило $S \to A S_1$, добавляем нетерминал $S$ в ячейку $M[1, 2]$.
    \item $k = 2, i = 1, j = 3: S \in M[1, 2], B \in M[2, 3]$, поскольку в грамматике есть правило $S_1 \to S B$, добавляем нетерминал $S_1$ в ячейку $M[1, 3]$.
\end{itemize}

Матрица:

\[
\begin{pmatrix}
\varnothing & \{A\}       & \{S\}         & \{S_1\}     \\
\varnothing & \varnothing & \cellcolor{lightgray}\{A, \boldsymbol{S}, S_1\} & \cellcolor{lightgray}\{S, \boldsymbol{S_1}\}  \\
\{A\}       & \varnothing & \{S_1\}       & \{B, S\}    \\
\varnothing & \varnothing & \{B\}         & \varnothing \\
\end{pmatrix}
\]

\textbf{Итерация 5.}
Рассмотрим на это шаге:

\begin{itemize}
    \setlength\itemsep{1em}
    \item $k = 1, i = 0, j = 3: A \in M[0, 1], S_1 \in M[1, 3]$, поскольку в грамматике есть правило $S \to A S_1$, добавляем нетерминал $S$ в ячейку $M[0, 3]$.
    \item $k = 3, i = 0, j = 2: S \in M[0, 3], B \in M[3, 2]$, поскольку в грамматике есть правило $S_1 \to S B$, добавляем нетерминал $S_1$ в ячейку $M[0, 2]$.
\end{itemize}

Матрица на этой итерации:
\[
\begin{pmatrix}
\varnothing & \{A\}       & \cellcolor{lightgray}\{S, \boldsymbol{S_1}\}    & \cellcolor{lightgray}\{\boldsymbol{S}, S_1\}  \\
\varnothing & \varnothing & \{A, S, S_1\} & \{S, S_1\}  \\
\{A\}       & \varnothing & \{S_1\}       & \{B, S\}    \\
\varnothing & \varnothing & \{B\}         & \varnothing \\
\end{pmatrix}
\]

\textbf{Итерация 6.}
Интересующие нас на этом шаге случаи:

\begin{itemize}
    \setlength\itemsep{1em}
    \item $k = 0, i = 2, j = 2: A \in M[2, 0], S_1 \in M[0, 2]$, поскольку в грамматике есть правило $S \to A S_1$, добавляем нетерминал $S$ в ячейку $M[2, 2]$.
    \item $k = 2, i = 2, j = 3: S \in M[2, 2], B \in M[2, 3]$, поскольку в грамматике есть правило $S_1 \to S B$, добавляем нетерминал $S_1$ в ячейку $M[2, 3]$.
\end{itemize}

Матрица после данного шага:

\[
\begin{pmatrix}
\varnothing & \{A\}       & \{S, S_1\}    & \{S, S_1\}    \\
\varnothing & \varnothing & \{A, S, S_1\} & \{S, S_1\}    \\
\{A\}       & \varnothing & \cellcolor{lightgray}\{\boldsymbol{S}, S_1\}    & \cellcolor{lightgray}\{B, S, \boldsymbol{S_1}\} \\
\varnothing & \varnothing & \{B\}         & \varnothing   \\
\end{pmatrix}
\]

На следующей итерации матрица не изменяется, поэтому заканчиваем работу алгоритма. В результате, если ячейка $M[i, j]$ содержит стартовый нетерминал $S$, то существует путь из $i$ в $j$, удовлетворяющий ограничениям, заданным грамматикой.
\end{example}

Можно заметить, что мы делаем много лишних итераций.
Можно переписать алгоритм так, чтобы он не просматривал заведомо пустые ячейки.
Данную модификацию предложил Й.Хеллингс в работе~\cite{hellingsRelational}, также она реализована в работе~\cite{10.1007/978-3-319-46523-4_38}.

Псевдокод алгоритма Хеллингса представлен в листинге~\ref{alg:graphParseHellings}.

\begin{algorithm}[H]
    \begin{algorithmic}[1]
        \caption{Алгоритм Хеллингса}
        \label{alg:graphParseHellings}
        \Function{contextFreePathQuerying}{$G= \langle \Sigma, N, P, S \rangle$, $\mathcal{G} = \langle V,E,L \rangle$}

        \State{$r \gets \{(N_i,v,v) \mid v \in V \wedge N_i \to \varepsilon \in P \} \cup \{(N_i,v,u) \mid (v,t,u) \in E \wedge N_i \to t \in P \}$}
        \State{$m \gets r$}
        \While{$m \neq \varnothing$}
        \State{$(N_i,v,u) \gets$ m.pick()}
        \For {$(N_j,v',v) \in r$}
            \For {$N_k \to N_j N_i \in P$ таких что $((N_k, v',u) \notin r)$}
                \State{$m \gets  m \cup \{(N_k, v',u)\}$}
                \State{$r \gets  r \cup \{(N_k, v',u)\}$}                
            \EndFor
        \EndFor
        \For {$(N_j,u,v') \in r$}
            \For {$N_k \to N_i N_j \in P$ таких что $((N_k, v, v') \notin r)$}
                \State{$m \gets  m \cup \{(N_k, v, v')\}$}
                \State{$r \gets  r \cup \{(N_k, v, v')\}$}                
            \EndFor
        \EndFor

        \EndWhile
        \State \Return $r$
        \EndFunction
    \end{algorithmic}
\end{algorithm}


\begin{example}
  Запустим алгоритм Хеллингса на нашем примере.
  
  \textbf{Инициализация}
  $$
  m = r = \{(A,0,1),(A,1,2),(A,2,0),(B,2,3),(B,3,2)\}
  $$
  
  \textbf{Итерации внешнего цикла.} Будем считеть, что $r$ и $m$ --- упорядоченные списки и $pick$ возврпщает его голову, оставляя хвост.
  Новые элементы добавляются в конец.
  \begin{enumerate}
  \item Обрабатываем $(A,0,1)$. 
  Ни один из вложенных циклов не найдёт новых путей, так как для рассматриваемого ребра есть только две возможности достроить путь: $2 \xrightarrow{A} 0 \xrightarrow{A} 1$ и $0 \xrightarrow{A} 1 \xrightarrow{A} 2$
  и ни одна из соответствующих строк не выводтся в заданной граммтике.
  \item Перед началом итерации 
     $$
     m = \{(A,1,2),(A,2,0),(B,2,3),(B,3,2)\},
     $$ $r$ не изменилось.
     Обрабатываем $(A,1,2)$.
     В данной ситуации второй цикл найдёт тройку $(B,2,3)$ и соответсвующее правило $S \to A \ B$. 
     Это значит, что и в $m$ и в $r$ добавится тройка $(S, 1, 3)$.
  \item
   Перед началом итерации 
     $$
     m = \{(A,2,0),(B,2,3),(B,3,2),(S,1,3)\},
     $$ 
     $$
     r= \{(A,0,1),(A,1,2),(A,2,0),(B,2,3),(B,3,2),(S,1,3)\}.
     $$
     Обрабатываем $(A,2,0)$. 
     Внутринние циклы ничего не найдут, новых путей н появится.
   \item
   Перед началом итерации 
     $$
     m = \{(B,2,3),(B,3,2),(S,1,3)\},
     $$ 
     $$
     r= \{(A,0,1),(A,1,2),(A,2,0),(B,2,3),(B,3,2),(S,1,3)\}.
     $$
     Обрабатываем $(B,2,3)$. 
     Первый цикл мог бы найти $(A,1,2)$, однако при проверке во вложенном цикле выяснится, что $(S, 1, 3)$ уже найдена. 
     В итоге, на данной итерации новых путей н появится.
   \item
   Перед началом итерации 
     $$
     m = \{(B,3,2),(S,1,3)\},
     $$ 
     $$
     r= \{(A,0,1),(A,1,2),(A,2,0),(B,2,3),(B,3,2),(S,1,3)\}.
     $$
     Обрабатываем $(B,3,2)$. 
     Первый цикл найдёт $(S,1,3)$ и соответствующее правило $S_1 \to S \ B$. 
     Это значит, что и в $m$ и в $r$ добавится тройка $(S_1, 1, 2)$. 
   \item
   Перед началом итерации 
     $$
     m = \{(S,1,3),(S_1, 1, 2)\},
     $$ 
     $$
     r= \{(A,0,1),(A,1,2),(A,2,0),(B,2,3),(B,3,2),(S,1,3),(S_1, 1, 2)\}.
     $$
     Обрабатываем $(S,1,3)$. 
     Второй цикл мог бы найти $(B,3,2)$, однако при проверке во вложенном цикле выяснится, что $(S_1, 1, 2)$ уже найдена. 
     В итоге, на данной итерации новых путей н появится.
   \item
   Перед началом итерации 
     $$
     m = \{(S_1, 1, 2)\},
     $$ 
     $$
     r= \{(A,0,1),(A,1,2),(A,2,0),(B,2,3),(B,3,2),(S,1,3),(S_1, 1, 2)\}.
     $$
     Обрабатываем $(S_1,1,2)$. 
     Первый цикл найдёт $(A,0,1)$ и соответствующее правило $S \to A \ S_1$. 
     Это значит, что и в $m$ и в $r$ добавится тройка $(S, 0, 2)$. 

   \item
   Перед началом итерации 
     $$
     m = \{(S, 0, 2)\},
     $$ 
     $$
     r= \{(A,0,1),(A,1,2),(A,2,0),(B,2,3),(B,3,2),(S,1,3),(S_1, 1, 2),(S, 0, 2)\}.
     $$
     Обрабатываем $(S, 0, 2)$. 
     Найдено: $(B,2,3)$ и соответствующее правило $S_1 \to S \ B$. 
     B $m$ и в $r$ добавится тройка $(S_1, 0, 3)$. 

   \item
   Перед началом итерации 
     $$
     m = \{(S_1, 0, 3)\},
     $$ 
     \begin{align*}
     r= \{&(A,0,1),(A,1,2),(A,2,0),(B,2,3),(B,3,2),(S,1,3),(S_1, 1, 2),(S, 0, 2),\\
          &(S_1, 0, 3)\}.
     \end{align*}
     Обрабатываем $(S_1, 0, 3)$. 
     Найдено: $(A,2,0)$ и соответствующее правило $S \to A \ S_1$. 
     B $m$ и в $r$ добавится тройка $(S, 2, 3)$. 

   \item
   Перед началом итерации 
     $$
     m = \{(S, 2, 3)\},
     $$ 
     \begin{align*}
     r= \{&(A,0,1),(A,1,2),(A,2,0),(B,2,3),(B,3,2),(S,1,3),(S_1, 1, 2),(S, 0, 2),\\
          &(S_1, 0, 3),(S, 2, 3)\}.
     \end{align*}

     Обрабатываем $(S, 2, 3)$. 
     Найдено: $(B,3,2)$ и соответствующее правило $S_1 \to S \ B$. 
     B $m$ и в $r$ добавится тройка $(S_1, 2, 2)$. 

   \item
   Перед началом итерации 
     $$
     m = \{(S_1, 2, 2)\},
     $$ 
     \begin{align*}
     r= \{&(A,0,1),(A,1,2),(A,2,0),(B,2,3),(B,3,2),(S,1,3),(S_1, 1, 2),(S, 0, 2),\\
          &(S_1, 0, 3),(S, 2, 3),(S_1, 2, 2)\}.
     \end{align*}
     Обрабатываем $(S_1, 2, 2)$. 
     Найдено: $(A,1,2)$ и соответствующее правило $S \to A \ S_1$. 
     B $m$ и в $r$ добавится тройка $(S, 1, 2)$. 

   \item
   Перед началом итерации 
     $$
     m = \{(S, 1, 2)\},
     $$ 
     \begin{align*}
     r= \{&(A,0,1),(A,1,2),(A,2,0),(B,2,3),(B,3,2),(S,1,3),(S_1, 1, 2),(S, 0, 2),\\
          &(S_1, 0, 3),(S, 2, 3),(S_1, 2, 2),(S, 1, 2)\}.
     \end{align*}
     Обрабатываем $(S, 1, 2)$. 
     Найдено: $(B,2,3)$ и соответствующее правило $S_1 \to S \ B$. 
     B $m$ и в $r$ добавится тройка $(S_1, 1, 3)$. 

   \item
   Перед началом итерации 
     $$
     m = \{(S_1, 1, 3)\},
     $$ 
     \begin{align*}
     r= \{&(A,0,1),(A,1,2),(A,2,0),(B,2,3),(B,3,2),(S,1,3),(S_1, 1, 2),(S, 0, 2),\\
          &(S_1, 0, 3),(S, 2, 3),(S_1, 2, 2),(S, 1, 2),(S_1, 1, 3)\}.
     \end{align*}
     Обрабатываем $(S_1, 1, 3)$. 
     Найдено: $(A,0,1)$ и соответствующее правило $S \to A \ S_1$. 
     B $m$ и в $r$ добавится тройка $(S, 0, 3)$. 

   \item
   Перед началом итерации 
     $$
     m = \{(S, 0, 3)\},
     $$ 
     \begin{align*}
     r= \{&(A,0,1),(A,1,2),(A,2,0),(B,2,3),(B,3,2),(S,1,3),(S_1, 1, 2),(S, 0, 2),\\
          &(S_1, 0, 3),(S, 2, 3),(S_1, 2, 2),(S, 1, 2),(S_1, 1, 3),(S, 0, 3)\}.
     \end{align*}
     Обрабатываем $(S, 0, 3)$. 
     Найдено: $(B,3,2)$ и соответствующее правило $S_1 \to S \ B$. 
     B $m$ и в $r$ добавится тройка $(S_1, 0, 2)$. 

   \item
   Перед началом итерации 
     $$
     m = \{(S_1, 0, 2)\},
     $$ 
     \begin{align*}
     r= \{&(A,0,1),(A,1,2),(A,2,0),(B,2,3),(B,3,2),(S,1,3),(S_1, 1, 2),(S, 0, 2),\\
          &(S_1, 0, 3),(S, 2, 3),(S_1, 2, 2),(S, 1, 2),(S_1, 1, 3),(S, 0, 3),(S_1, 0, 2)\}.
     \end{align*}
     Обрабатываем $(S_1, 0, 2)$. 
     Найдено: $(A,2,0)$ и соответствующее правило $S \to A \ S_1$. 
     B $m$ и в $r$ добавится тройка $(S, 2, 2)$. 

   \item
   Перед началом итерации 
     $$
     m = \{(S, 2, 2)\},
     $$ 
     \begin{align*}
     r= \{&(A,0,1),(A,1,2),(A,2,0),(B,2,3),(B,3,2),(S,1,3),(S_1, 1, 2),(S, 0, 2),\\
          &(S_1, 0, 3),(S, 2, 3),(S_1, 2, 2),(S, 1, 2),(S_1, 1, 3),(S, 0, 3),(S_1, 0, 2),\\
          &(S, 2, 2)\}.
     \end{align*}
     Обрабатываем $(S, 2, 2)$. 
     Найдено: $(B,2,3)$ и соответствующее правило $S_1 \to S \ B$. 
     B $m$ и в $r$ добавится тройка $(S_1, 2, 3)$. 

   \item
   Перед началом итерации 
     $$
     m = \{(S_1, 2, 3)\},
     $$ 
     \begin{align*}
     r= \{&(A,0,1),(A,1,2),(A,2,0),(B,2,3),(B,3,2),(S,1,3),(S_1, 1, 2),(S, 0, 2),\\
          &(S_1, 0, 3),(S, 2, 3),(S_1, 2, 2),(S, 1, 2),(S_1, 1, 3),(S, 0, 3),(S_1, 0, 2),\\
          &(S, 2, 2),(S_1, 2, 3)\}.
     \end{align*}
     Обрабатываем $(S_1, 2, 3)$. 
     Могло бы быть найдено: $(A,1,2)$ и соответствующее правило $S \to A \ S_1$, однако тройка $(S, 1, 3)$ уже есть в $r$. 
     А значит никаких новых троек найдено не будет и $m$ становится пустым.
     Это была последняя итерация внешнего цикла, в $r$ на текущий момент уже содержится всё ршение. 

  \end{enumerate}

\end{example}

Как можно заметить, количество итераций внешнего цикла также получилось достаточно большим. 
Проверьте, зависит ли оно от порядка обработки элементов из $m$.
При этом внутренние циклы в нашем случае достаточно короткие, так как просматриваются только ``существенные'' элементы и избегается дублирование.

%\section{Вопросы и задачи}
%\begin{enumerate}
%    \item Проверить работу алгоритма CYK для цепочек на грамматике
%    \begin{flushleft}
%    $E \to E + E$ \\
%    $E \to E * E$ \\
%    $E \to (E)$   \\
%    $E \to n$     \\
%    \end{flushleft}
%    и словах (алфавит $\Sigma = \{n, +, *, (, )\}$)
%    \begin{flushleft}
%    $ (n + n) * n$    \\
%    $ n + n * n$      \\
%    $n + n + n + n$   \\
%    $n + (n * n) + n$ \\
%    \end{flushleft}
%
%    \item Изучить вычислительную сложность алгоритма CYK для матриц в зависимости от размера входного графа (размер грамматики считать фиксированным).
%
%    \item Проверить работу алгоритма CYK для графов на графе
%
%    \begin{center}
%        \begin{tikzpicture}[node distance=3cm,shorten >=1pt,on grid,auto]
%        \node[state] (q_0)  {$0$};
%        \node[state] (q_1) [right=of q_0]  {$1$};
%        \node[state] (q_2) [right=of q_1]  {$2$};
%        \node[state] (q_3) [right=of q_2]  {$3$};
%        \node[state] (q_4) [right=of q_3]  {$4$};
%        \path[->]
%        (q_0) edge  node {$a$} (q_1)
%        (q_1) edge  node {$b$} (q_2)
%        (q_2) edge  node {$a$} (q_3)
%        (q_3) edge  node {$b$} (q_4)
%        (q_1) edge[bend left, above]  node {$b$} (q_3)
%        (q_4) edge[bend left, below]  node {$a$} (q_1);
%        \end{tikzpicture}
%    \end{center}
%
%    И грамматике
%
%    \begin{flushleft}
%        $S \to S S$ \\
%        $S \to A B$ \\
%        $A \to a$   \\
%        $B \to b$     \\
%    \end{flushleft}
%
%    \item Оцените временную сложность алгоритма Хеллингса и сравните её с оценкой для наивного обобщения CYK.
%
%\end{enumerate}
\chapter{КС и конъюнктивная достижимость через произведение матриц}\label{chpt:MatrixBasedAlgos}

В данном разделе мы рассмотрим алгоритм решения задачи контекстно-свободной и конъюнктивной достижимости, основанный на произведении матриц. Будет показано, что при использовании конъюнктивных граммтик, представленный алгоритм находит переапроксимацию истинного решения задачи.

\section{КС достижимость через произведение матриц}
\label{Matrix-CFPQ}
В главе~\ref{graph:CYK}~был изложен алгоритм для решения задачи КС достижимости на основе CYK. Заметим, что обход матрицы напоминает умножение матриц в ячейках которых множества нетерминалов:

\begin{align*}
M_3 = &M_1 \times M_2 \\
M_3[i,j] = &\sum_{k=1}^{n} M[i,k] * M[k,j]
\end{align*}
, где сумма --- это объединение множеств:

$$
\sum_{k=1}^{n} = \bigcup_{k=1}^{n}
$$
, а поэлементное умножение определено следующим образом:
$$
S_1 * S_2 = \{N_1^0 ... N_1^m\} * \{N_2^0 ... N_2^l\} = \{N_3 \mid (N_3 \rightarrow N_1^i N_2^j) \in P\}.
$$

Таким образом, алгоритм решения задачи КС достижимости может быть выражена в терминах перемножения матриц над полукольцом с соответствующими операциями.

Для частного случая этой задачи, синтаксического анализа линейного входа, существует алгоритм Валианта~\cite{Valiant:1975:GCR:1739932.1740048}, использующий эту идею.
Однако он не обобщается на графы из-за того, что существенно использует возможность упорядочить обход матрицы (см. разницу в CYK для линейного случая и для графа). 
Поэтому, хотя для линейного случая алгоритм Валианта является алгоритмом синтаксического анализа для произвольных КС граммтик за субкубическое время, его обобщение до задачи КС достижимости в произвольных графах с сохранением асимптотики является нетривиальной задачей~\cite{Yannakakis}. 
В настоящее время алгоритм с субкубической сложностью получен только для частного случая --- языка Дика с одним типом скобок --- Филипом Брэдфорlом~\cite{8249039}.

В случае с линейным входом, отдельного внимания заслуживает работа Лиллиан Ли (Lillian Lee)~\cite{Lee:2002:FCG:505241.505242}, где она показывает, что задача перемножения матриц сводима к синтаксическому анализу линейного входа. Аналогичных результатов для графов на текущий момент не известно.

Поэтому рассмотрим более простую идею, изложенную в статье Рустама Азимова~\cite{Azimov:2018:CPQ:3210259.3210264}: будем строить транзитивное замыкание графа через наивное (не через возведение в квадрат) умножение матриц.

Пусть $\mathcal{G} = (V, E)$ --- входной граф и $G = (N,\Sigma,P)$ --- входная грамматика. Тогда алгоритм может быть сформулирован как представлено в листинге~\ref{alg:graphParse}.

\begin{algorithm}[H]
\begin{algorithmic}[1]
\caption{Context-free recognizer for graphs}
\label{alg:graphParse}
\Function{contextFreePathQuerying}{$\mathcal{G}$, G}
    
    \State{$n \gets$ количество узлов в $\mathcal{G}$}
    \State{$E \gets$ направленные ребра в $\mathcal{G}$}
    \State{$P \gets$ набор продукций из $G$}
    \State{$T \gets$ матрица $n \times n$, в которой каждый элемент $\varnothing$}
    \ForAll{$(i,x,j) \in E$}
    \Comment{Инициализация матрицы}
        \State{$T_{i,j} \gets T_{i,j} \cup \{A~|~(A \rightarrow x) \in P \}$}
    \EndFor
    \ForAll{$i \in 0\ldots n-1$}
    \Comment{Добавление петель для нетерминалов, порождающих пустую строку}
        \State{$T_{i,i} \gets T_{i,i} \cup \{ A \in N \mid A \to \varepsilon \}$}
    \EndFor    
    \While{матрица $T$ меняется}
       
        \State{$T \gets T \cup (T \times T)$}
        \Comment{Вычисление транзитивного замыкания} 
    \EndWhile
\State \Return $T$
\EndFunction
\end{algorithmic}
\end{algorithm}


\begin{example}[Пример работы]

Пусть есть граф $\mathcal{G}$:
\begin{center}
    \begin{tikzpicture}[node distance=2.5cm,shorten >=1pt,on grid,auto]
    \node[state] (q_0)   {$1$};
    \node[state] (q_1) [above right=of q_0] {$2$};
    \node[state] (q_2) [right=of q_0] {$0$};
    \node[state] (q_3) [right=of q_2] {$3$};
    \path[->]
    (q_0) edge  node {a} (q_1)
    (q_1) edge  node {a} (q_2)
    (q_2) edge  node {a} (q_0)
    (q_2) edge[bend left, above]  node {b} (q_3)
    (q_3) edge[bend left, below]  node {b} (q_2);
    \end{tikzpicture}
    
\end{center}

и грамматика $G$:
\begin{align*}
S   &\to A B    &A  \to a \\ 
S  &\to A S_1   &B  \to b\\ 
S_1 &\to S B 
\end{align*}


Пусть $T_i$ --- матрица, полученная из $T$ после применения цикла, описанного в строках \textbf{8-9} алгоритма~\ref{alg:graphParse}, $i$ раз.
Тогда $T_0$, полученная напрямую из графа, выглядит следующим образом:

\[
T_0 = \begin{pmatrix}
    \varnothing & \{A\}       & \varnothing & \{B\}       \\
    \varnothing & \varnothing & \{A\}       & \varnothing \\
    \{A\}       & \varnothing & \varnothing & \varnothing \\
    \{B\}       & \varnothing & \varnothing & \varnothing \\
\end{pmatrix}
\]

Далее показано получение матрицы $T_1$.

\[
T_0 \times T_0 = \begin{pmatrix}
    \varnothing & \varnothing & \varnothing & \varnothing \\
    \varnothing & \varnothing & \varnothing & \varnothing \\
    \varnothing & \varnothing & \varnothing & \{S\}       \\
    \varnothing & \varnothing & \varnothing & \varnothing \\
\end{pmatrix}
\]

\[
T_1 = T_0 \cup (T_0 \times T_0) = \begin{pmatrix}
    \varnothing & \{A\}       & \varnothing & \{B\}       \\
    \varnothing & \varnothing & \{A\}       & \varnothing \\
    \{A\}       & \varnothing & \varnothing & \cellcolor{lightgray} \{\pmb{S}\}       \\
    \{B\}       & \varnothing & \varnothing & \varnothing \\
\end{pmatrix}
\]

После первой итерации цикла нетерминал в ячейку $T[2,3]$ добавился нетерминал $S$. 
Это означает, что существует такой путь $\pi$ из вершины 2 в вершину 3 в графе $\mathcal{G}$, что $S \xrightarrow{*} \omega(\pi)$. В данном примере путь состоит из двух ребер $2 \xrightarrow{a} 0$ и $ 0 \xrightarrow{b} 3$, так что $S \xrightarrow{*} ab$.

Вычисление транзитивного замыкания заканчивается через $k$ итераций, когда достигается неподвижная точка процесса: $T_{k-1} = T_k$. Для данного примера $k = 13$, так как $T_{13} = T_{12}$. Весь процесс рабты алгоритма (все матрицы $T_i$) показан ниже (на каждой итерации новые элементы выделены жирным).

{\footnotesize
\begin{alignat*}{7}
& &&T_2 &&= \begin{pmatrix}
\varnothing & \{A\}       & \varnothing & \{B\}       \\
\varnothing & \varnothing & \{A\}       & \varnothing \\
\cellcolor{lightgray} \{A, \pmb{S_1}\}  & \varnothing & \varnothing & \{S\}       \\
\{B\}       & \varnothing & \varnothing & \varnothing \\
\end{pmatrix} \ \ \ \ &&T_3 &&= \begin{pmatrix}
\varnothing & \{A\}       & \varnothing & \{B\}       \\
\cellcolor{lightgray} \{\pmb{S}\}       & \varnothing & \{A\}       & \varnothing \\
\{A, S_1\}  & \varnothing & \varnothing & \{S\}       \\
\{B\}       & \varnothing & \varnothing & \varnothing \\
\end{pmatrix} \\ & &&T_4 &&= \begin{pmatrix}
\varnothing & \{A\}       & \varnothing & \{B\}       \\
\{S\}       & \varnothing & \{A\}       & \cellcolor{lightgray} \{\pmb{S_1}\}     \\
\{A, S_1\}  & \varnothing & \varnothing & \{S\}       \\
\{B\}       & \varnothing & \varnothing & \varnothing \\
\end{pmatrix}  \ \ \ \ &&T_5 &&= \begin{pmatrix}
\varnothing & \{A\}       & \varnothing & \cellcolor{lightgray} \{B, \pmb{S}\}    \\
\{S\}       & \varnothing & \{A\}       & \{S_1\}     \\
\{A, S_1\}  & \varnothing & \varnothing & \{S\}       \\
\{B\}       & \varnothing & \varnothing & \varnothing \\
\end{pmatrix} \\ & &&T_6 &&= \begin{pmatrix}
\cellcolor{lightgray} \{\pmb{S_1}\}     & \{A\}       & \varnothing & \{B, S\}    \\
\{S\}       & \varnothing & \{A\}       & \{S_1\}     \\
\{A, S_1\}  & \varnothing & \varnothing & \{S\}       \\
\{B\}       & \varnothing & \varnothing & \varnothing \\
\end{pmatrix} \ \ \ \ &&T_7 &&= \begin{pmatrix}
\{S_1\}     & \{A\}       & \varnothing & \{B, S\}    \\
\{S\}       & \varnothing & \{A\}       & \{S_1\}     \\
\cellcolor{lightgray} \{A, S_1, \pmb{S}\}  & \varnothing & \varnothing & \{S\}    \\
\{B\}       & \varnothing & \varnothing & \varnothing \\
\end{pmatrix}  \\
& &&T_8 &&= \begin{pmatrix}
\{S_1\}     & \{A\}       & \varnothing & \{B, S\}    \\
\{S\}       & \varnothing & \{A\}       & \{S_1\}     \\
\{A, S_1, S\}  & \varnothing & \varnothing & \cellcolor{lightgray} \{S, \pmb{S_1}\} \\
\{B\}       & \varnothing & \varnothing & \varnothing \\
\end{pmatrix} \ \ \ \ &&T_9 &&= \begin{pmatrix}
\{S_1\}     & \{A\}       & \varnothing & \{B, S\}    \\
\{S\}       & \varnothing & \{A\}       & \cellcolor{lightgray} \{S_1, \pmb{S}\}     \\
\{A, S_1, S\}  & \varnothing & \varnothing & \{S, S_1\} \\
\{B\}       & \varnothing & \varnothing & \varnothing \\
\end{pmatrix} \\ & &&T_{10} &&= \begin{pmatrix}
\{S_1\}     & \{A\}       & \varnothing & \{B, S\}    \\
\cellcolor{lightgray} \{S, \pmb{S_1}\}       & \varnothing & \{A\}       & \{S_1, S\}     \\
\{A, S_1, S\}  & \varnothing & \varnothing & \{S, S_1\} \\
\{B\}       & \varnothing & \varnothing & \varnothing \\
\end{pmatrix}  \ \ \ \  &&T_{11} &&= \begin{pmatrix}
\cellcolor{lightgray} \{S_1, \pmb{S}\}     & \{A\}       & \varnothing & \{B, S\}    \\
\{S, S_1\}       & \varnothing & \{A\}       & \{S_1, S\}     \\
\{A, S_1, S\}  & \varnothing & \varnothing & \{S, S_1\} \\
\{B\}       & \varnothing & \varnothing & \varnothing \\
\end{pmatrix} \\ & &&T_{12} &&= \begin{pmatrix}
\{S_1, S\}     & \{A\}       & \varnothing & \cellcolor{lightgray} \{B, S, \pmb{S_1}\}    \\
\{S, S_1\}       & \varnothing & \{A\}       & \{S_1, S\}     \\
\{A, S_1, S\}  & \varnothing & \varnothing & \{S, S_1\} \\
\{B\}       & \varnothing & \varnothing & \varnothing \\
\end{pmatrix} \ \ \ \ &&T_{13} &&= \begin{pmatrix}
\{S_1, S\}     & \{A\}       & \varnothing & \{B, S, S_1\}    \\
\{S, S_1\}       & \varnothing & \{A\}       & \{S_1, S\}     \\
\{A, S_1, S\}  & \varnothing & \varnothing & \{S, S_1\} \\
\{B\}       & \varnothing & \varnothing & \varnothing \\
\end{pmatrix}
\end{alignat*}
}

Таким образом, результат алгоритма~\ref{alg:graphParse} для нашего примера --- это матрица $T_{13} = T_{12}$. Заметим, что для данного алгоритма приведённый пример также является худшим случаем: на каждой итерации в матрицу добавляется ровно один нетерминал, при том, что необходимо заполнить порядка $O(n^2)$ ячеек.

\end{example}


\subsection{Расширение алгоритма для конъюнктивных грамматик}

Матричный алгоритм для конъюнктивных грамматик отличается от алгоритма~\ref{alg:graphParse} для контекстно-свободных грамматик только операцией умножения матриц, в остальном алгоритм остается без изменений. Определим операцию умножения матриц.
\begin{definition}
    Пусть $M_1$ и $M_2$ матрицы размера $n$. Определим операцию $\circ$  сдедующим образом:
     $$M_1 \circ M_2 = M_3,$$ $$M_3 [i,j] = \{A \mid \exists (A \rightarrow B_1 C_1 \& \ldots \& B_m C_m) \in P, (B_k , C_k) \in d[i,j] \forall k = 1,\ldots,m\}$$, где $$d[i,j] = \bigcup_{k = 1}^{n} M_1 [i,k] \times M_2 [k,j].$$
\end{definition}

Важно заметить, что алгоритм для конъюнктивных грамматик, в отличие от алгоритма для контекстно-свободных грамматик, дает лишь верхнюю оценку ответа. То есть все нетерминалы, которые должны быть в ячейках матрицы результата, содержатся там, но вместе с ними содержатся и лишние нетерминалы. Рассмотрим пример, иллюстрирующий появление лишних нетерминалов.

\begin{example}
    Грамматика $G$:
    \begin{align*}
    S &\to AB \& DC & C &\to c \\ 
    A &\to a        & D &\to DC \mid b\\
    B &\to BC \mid b
    \end{align*}
    Очевидно, что грамматика $G$ задает язык из одного слова $L(G) = \{abc\} = \{abc^*\} \cap \{a^* bc\}$.
    
    Пусть есть граф $\mathcal{G}$:
    \begin{center}
        \begin{tikzpicture}[node distance=2.5cm, shorten >=1pt,on grid,auto]
        \node[state] (q_0)   {$0$};
        \node[state] (q_1) [right=of q_0] {$1$};
        \node[state] (q_2) [right=of q_1] {$2$};
        \node[state] (q_6) [below=of q_2] {$6$};
        \node[state] (q_3) [right=of q_2] {$3$};
        \node[state] (q_5) [right=of q_6] {$5$};
        \node[state] (q_4) [right=of q_3] {$4$};
        \path[->]
        (q_0) edge  node {a} (q_1)
        (q_1) edge  node {b} (q_2)
        (q_1) edge  node {a} (q_6)
        (q_2) edge  node {c} (q_3)
        (q_3) edge  node {c} (q_4)
        (q_6) edge  node {b} (q_5)
        (q_5) edge  node {c} (q_4);
        \end{tikzpicture}
    \end{center}
    Применяя алгоритм, получим, что существует путь из вершины 0 в вершину 4, выводимый из нетерминала $S$. Однако очевидно, что в графе такого пути нет. 
    Такое поведение алгоритма наблюдается из-за того, что существует путь ``abcc'', соответствующий $L(AB) = \{abc^*\}$ и путь ``aabc'', соответствующий $L(DC) = \{a^{*}bc\}$, но они различны. Однако алгоритм не может это проверить, так как оперирует понятием достижимости между вершинами, а не наличием различных путей. Более того, в общем случае для конъюнктивных граммтик такую проверку реалиховать нельзя. Поэтому для классической семантики достидимости с ограничениями в терминах конъюнктивных граммтик результат работы алгоритма является оценкой сверху.
    
    Существует альтернативная семантика, когда мы трактуем конъюнкцию в праой части правил как крнъюнкцию в Datalog (подробнее о Datalog в параграфе~\ref{Subsection Datalog}). Т.е если есть правило $S \to AB \& DC$, то должен быть путь принадлежащий языку $L(AB)$ и путь принадлежащий языку $L(DC)$. В такой семантике алгоритм дает точный ответ.
\end{example}

Подробнее алгоритм описан в статье Рустама Азимова и Семёна Григорьева~\cite{565CECD7E8F5C6063935B41DB41797AA37D53B04}. Стоит также отметить, что обобщения данного алогритма для булевых грамматик не существует, хотя и сущетсвует частное решение для случая, когда граф не содержит циклов (является DAG-ом), предложенное Екатериной Шеметовой~\cite{Shemetova2019}.

\section{Особенности реализации}

Алгоритмы, описанные выще, удобны с точки зрения реализации тем, что могут быть эффективно реализованы с использованием высокопроизводительных библиотек линейной алгебры, которые эксплуатируют возможности параллельных вычислений на современных CPU и  GPGPU~\cite{Mishin:2019:ECP:3327964.3328503}. 
Это позволяет с минимальными затратими получить эффективную параллельную реализацию алгоритма для решения задачи КС достижимости в графах. 
Благодаря этому, хотя асимптотически приведенные алгоритмы имеют большую сложность чем, скажем, алгоритм Хеллингса, в результате эффективного распараллеливания на практике они работают быстрее однопоточных алгоритов с лучшей сложностью.

Далее рассмотрим некоторые детали, упрощающие реализацию с использованием современных библиотек и аппаратного обеспечения.

Так как множество нетерминалов и правил конечно, то мы можем свести представленный выше алгоритм к булевым матрицам: для каждого нетерминала заведём матрицу, такую что в ячейке стоит 1 тогда и только тогда, когда в исходной матрице в соответствующей ячейке сожержится этот нетерминал.
Тогда перемножение пары таких матриц, соответсвующих нетерминалам $A$ и $B$, соответствует построению путей, выводимых из нетерминалов, для которых есть правила с правой частью вида $A B$. 

\begin{example}
Представим в виде набора булевых матриц следующую матрицу:
\[
T_0 = \begin{pmatrix}
\varnothing & \{A\}       & \varnothing & \{B\}       \\
\varnothing & \varnothing & \{A\}       & \varnothing \\
\{A\}       & \varnothing & \varnothing & \varnothing \\
\{B\}       & \varnothing & \varnothing & \varnothing \\
\end{pmatrix}
\]

Тогда:
\begin{alignat*}{7}
& &&T_{0\_A} &&= \begin{pmatrix}
0 & 1       & 0 & 0       \\
0 & 0 & 1       & 0 \\
1  & 0 & 0 & 0       \\
0       & 0 & 0 & 0 \\
\end{pmatrix} \ \ \ \ &&T_{0\_B} &&= \begin{pmatrix}
0 & 0       & 0 & 1       \\
0       & 0 & 0       & 0 \\
0  & 0 & 0 & 0       \\
1       & 0 & 0 & 0 \\
\end{pmatrix}
\end{alignat*}
Тогда при наличии правила $S \to A B$ в граммтике получим:
\[
T_{1\_S} =T_{0\_A} \times T_{0\_B} = \begin{pmatrix}
0 & 0       & 0 & 0       \\
0       & 0 & 0       & 0 \\
0  & 0 & 0 & 1       \\
0       & 0 & 0 & 0 \\
\end{pmatrix}
\]
\end{example}

Алгоритм же может быть переформулирован так, как показано в листинге~\ref{lst:algo1}. Такой взгляд на алгоритм позволяет использдвать для его реализации существующие высокорпоизводительные библиотеки для работы с булевыми матрицами (например M4RI\footnote{M4RI --- одна из высокопроизводительных библиотек для работы с логическими матрицами на CPU. Реализует Метод Четырёх Русских. Исходный код библиотеки: \url{https://bitbucket.org/malb/m4ri/src/master/}. Дата посещения: 30.03.2020.}~\cite{DBLP:journals/corr/abs-0811-1714}) или библиотеки для линейной алгебры (например CUSP~\cite{Cusp}).

\begin{algorithm}
  \floatname{algorithm}{Listing}
\begin{algorithmic}[1]
\caption{Context-free path quering algorithm. Boolean matrix version}
\label{lst:algo1}
\Function{evalCFPQ}{$D=(V,E), G=(N,\Sigma,P)$}
    \State{$n \gets$ |V|}
    \State{$T \gets \{T^{A_i} \mid A_i \in N, T^{A_i}$ is a matrix $n \times n$, $T^{A_i}_{k,l} \gets$ \texttt{false}\} }
    \ForAll{$(i,x,j) \in E$, $A_k \mid A_k \to x \in P$}
        %\Comment{Matrices initialization}
        %\For{$A_k \mid A_k \to x \in P$}
          {$T^{A_k}_{i,j} \gets \texttt{true}$}
        %\EndFor
    \EndFor
    \For{$A_k \mid A_k \to \varepsilon \in P$}
       {$T^{A_k}_{i,i} \gets \texttt{true}$}
    \EndFor

    \While{any matrix in $T$ is changing}
        %\Comment{Transitive closure calculation}
        \For{$A_i \to A_j A_k \in P$}
          { $T^{A_i} \gets T^{A_i} + (T^{A_j} \times T^{A_k})$ } 
        \EndFor
    \EndWhile
\State \Return $T$
\EndFunction
\end{algorithmic}
\end{algorithm}

С другой стороны, для запросов, выразимых в терминах граммтик с небольшим количеством нетерминалов, практически может быть выгодно представлять множества нетерминалов в ячейке матрицы в виде битового вектора следующим образом.
Нумеруем все нетерминалы с нуля, в векторе стоит 1 на позиции $i$, если в множестве есть нетерминал с номером $i$.
Таким образом, в каждой ячейке хранится битовый вектор длины $|N|$.
Тогда операция умножения определяется следующим образом:
$$v_1 \times v_2 = \{v \mid \exists (v,v_3) \in P, \textit{append}(v_1, v_2) \& v_3 = v_3\},$$ где $\&$ --- побитовое \texttt{``и''}.

Правила надо кодировать соответственно: продукция это пара, где первый элемент --- битовый вектор длины $|N|$ с единственной единицей в позиции, соответствующей нетерминалу в правой части, а второй элемент --- вектор длины $2|N|$, с двумя единицами кодирующими первый и второй нетерминалы, соответственно.

\begin{example}
Пусть $N = \{S, A, B\}$ и в грамматике есть продукция $S \to A B$. Тогда занумеруем нетерминалы $ (S, 0), (A, 1), (B, 2)$. Продукция примет вид $[1, 0, 0] \to [0, 1, 0, 0, 0, 1]$. Матрица $T_0$ примет вид (здесь ``$.$'' означает, что в ячейке стоит $[0,0,0]$):
\[
T_0 = \begin{pmatrix}
. & [0,1,0]       & . & [0,0,1]       \\
. & . & [0,1,0]       & . \\
[0,1,0]       & . & . & . \\
[0,0,1]      & . & . & . \\
\end{pmatrix}
\]

После выполнения умножения получим:
\[
T_1 = T_0 + T_0 \times T_0 =
\begin{pmatrix}
. & [0,1,0]       & . & [0,0,1]       \\
. & . & [0,1,0]       & . \\
[0,1,0]       & . & . & \cellcolor{lightgray}[1,0,0] \\
[0,0,1]      & . & . & . \\
\end{pmatrix}
\]
\end{example}


На практике в роли векторов могут выступать беззнаковые целые числа. 
Например, 32 бита под ячейки в матрице и 64 бита под правила (или 8 и 16, если позволяет количество нетерминалов в граммтике, или 16 и 32).
Тогда умножение выражается через битовые операции и сравнение, что довольно эффективно с точки зрения вычислений.

Для небольших запросов такой подход к реализации может оказаться быстрее --- в данном случае скорость зависит от деталей. Минус подхода в том, что нет возможности использовать готовые библиотеку ленейной алгебры без предварительной модификации. Только если они не являются параметризуемыми и не позволяют задать собственный тип и собственную операцию умножения и сложения (иными словами, собственное полукольцо). Такую возможность предусматривает, например, стандарт GraphBLAS\footnote{GraphBLAS --- открытый стандарт, описывающий набор примитивов и операций, необходимый для реализации графовых алгоритмов в терминах лнейной алгебры. Web-страница проекта: \url{https://github.com/gunrock/graphblast}. Дата доступа: 30.03.2020.} и, соответственно, его реализации, такие как SuiteSparce\footnote{SuteSparse --- это специализированное программное обеспечения для работы с разреженными матрицами, которое включает в себя реализацию GraphBLAS API. Web-страница проекта: \url{http://faculty.cse.tamu.edu/davis/suitesparse.html}. Дата доступа: 30.03.2020.}~\cite{Davis2018Algorithm9S}.

Также стоит замеить, что при работе с реальными графами матрицы как правило оказываются разреженными, а значит необходимо использовать соответствующие представления матриц (CRS, покоординатное, Quad Tree~\cite{quadtree}) и библиотеки, работающие с таким представлениями. 

Однако даже при использовании разреженных матриц, могут возникнуть проблемы с размером реальных данных и объёмом памяти. Напрмиер, для вычислений на GPGPU лучше всего, когда все нужные для вычисления матрицы помещаются на одну карту. Тогда можно свести обмен данными сежду хостом и графическим сопроцессором к мимнимуму. Если не помещаются все, то нужно, чтобы помещалясь хотя бы тройка непостредственно обрабатываемых матриц (два операнда и результат). В самом тяжёлом случае в памяти не удаётся раместить даже операнды целиком и тогда приходится прибегать к поблочному умнодению матриц.

Отдельной инженерной проблемой является масштабирование алгоритмов на несколько вычислительных узлов, как на несколько CPU, так и на несколько GPGPU.

Важным свойством рассмотренного алгоритма является то, что описанные проблемы с объёмом памяти и масштабированием могут эффективно решаться в рамках библиотек линейной алгебры общего назначения, что избавляет от необходимости создавать специализированные решения для конкетных задач. 


%\section{Вопросы и задачи}
%\begin{enumerate}
%    \item Находить кратчайшие пути в графах, используя идеи из секции~\ref{Matrix-CFPQ}.
%    \item Превратить граф, использующийся для CFPQ, в дерево.
%    \item Реализовать предложенные идеи на различных архитектурах.
%    \item Замерить производительность и расходы памяти по сравнению с существующими реализациями.
%\end{enumerate}

\chapter{КС достижимость через тензорное произведение}

Предыдущий подход позволяет выразить задачу поиска путей с ограничениями в терминах формальных языков через набор матричных операций.
Это позволяет использовать высокопроизводительные библиотеки, массовопараллельные архитектуры и другие готовые решения для линейной алгебры.
Однако, такой подход требует, чтобы грамматика находилась в ослабленной нормальной форме Хомского, что приводит к её разрастанию.
Можно ли как-то избежать этого?

В данном разделе мы предложим альтернативное сведение задачи поиска путей к матричным операциям.
В результате мы сможем избежать преобразования грамматики в ОНФХ, однако, матрицы, с которыми нам предётся работать, будут существенно б\'{о}льшего размера.

В основе подхода лежит использование рекурсивных сетей или рекурсивных автоматов в качестве представления контекстно-свободных грамматик и использование тензорного (прямого) произведения для нахождения пересечения автоматов.

\section{Рекурсивные автоматы и сети}

Рекурсивный автомат или сеть --- это представление контекстно-свободных грамматик, обобщающее конечные автоматы.
В нашей работе мы будем придерживаться термина \textbf{рекурсивный автомат}. 
Классическое определение рекурсивного автомата выглядит следующим образом.

\begin{definition}
Рекурсивный автомат --- это кортеж вида $\langle N, \Sigma, S, D \rangle$, где
\begin{itemize}
\item $N$ --- нетерминальный алфавит;
\item $\Sigma$ --- терминальный алфавит;
\item $S$ --- стартовый нетерминал;
\item $D$ --- конечный автомат над $N \cup \Sigma$ в котором стартовые и финальные состояния помечены подмножествами $N$.
\end{itemize}
\end{definition}



Построим рекурсивный автомат для грамматики $G$:
\begin{align*}
S   &\to    a S b \\ 
S   &\to    a b \\
\end{align*}


\begin{align}
\label{input1}
    \begin{tikzpicture}[node distance=2.5cm,shorten >=1pt,on grid,auto] 
       \node[state, initial] (q_0)   {$0 \{S\}$}; 
       \node[state] (q_1) [right=of q_0] {$1$}; 
       \node[state] (q_2) [right=of q_1] {$2$}; 
       \node[state, accepting] (q_3) [right=of q_2] {$3\{S\}$};
        \path[->] 
        (q_0) edge  node {a} (q_1)          
        (q_1) edge  node {S} (q_2)
        (q_2) edge  node {b} (q_3)
        (q_1) edge[bend left, above]  node {b} (q_3);
    \end{tikzpicture}
\end{align}

Используем стандартные обозначения для стартовых и финальных состояний. 
Дополнительно в стартовых и финальных состояниях укажем нетерминалы, для которых эти состояния стартовые/финальные.

В некоторых случаях рекурсивный автомат можно рассматривать как конечный автомат над смешанным алфавитом.
Именно такой взгляд мы будем использовать при изложении алгоритма.


\section{Тензорное произведение}
\label{section2}

Тензорное произведение матриц или произведение Кронекера --- это бинарная операция, обозначаемая $\otimes$ и определяемая следующим образом.

\begin{definition}
Пусть даны две матрицы: $A$ размера $m\times n$ и $B$ размера $p\times q$.
Произведение Кронекера или тензорное произведение матриц $A$ и $B$ --- это блочная матрица $C$ размера $mp \times nq$, вычисляемая следующим образом:
$$
C = A \otimes B = 
\begin{pmatrix}
A_{0,0}B   & \cdots & A_{0,n-1}B    \\
\vdots     & \ddots & \vdots       \\
A_{m-1,0}B & \cdots &  A_{m-1,n-1}B
\end{pmatrix}
$$
\end{definition}

\newcommand{\examplemtrx}
{
\begin{pmatrix}
5 & 6 & 7 & 8 \\
9 & 10 & 11 & 12 \\
13 & 14 & 15 & 16 
\end{pmatrix}
}

\begin{example}
\begin{align}
\begin{pmatrix}
1 & 2 \\
3 & 4
\end{pmatrix}
\otimes
\examplemtrx &=
\begin{pmatrix}
1\examplemtrx & 2\examplemtrx \\
3\examplemtrx & 4\examplemtrx
\end{pmatrix}
=\notag \\
&=
\left(\begin{array}{c c c c | c c c c}
5  & 6  & 7  & 8  & 10 & 12 & 14 & 16 \\
9  & 10 & 11 & 12 & 18 & 20 & 22 & 24 \\
13 & 14 & 15 & 16 & 26 & 28 & 30 & 32 \\
\hline
15 & 18 & 21 & 24 & 20 & 24 & 28 & 32 \\
27 & 30 & 33 & 36 & 36 & 40 & 44 & 48 \\
39 & 42 & 45 & 48 & 52 & 56 & 60 & 64 
\end{array}\right)
\end{align}
\end{example}

Заметим, что для определения тензорного произведения матриц достаточно определить операцию умножения на элементах исходных матриц.
Также отметим, что произведение Кронекера не является коммутативным.
При этом всегда существуют две матрицы перестоновок $P$ и $Q$ такие, что $A \otimes B = P(B \otimes A)Q$.
Это свойство потребуется нам в дальнейшем.

Теперь перейдём к графам.
Сперва дадим классическое определение тензорного произведения двух неориентированных графов.

\begin{definition}
Пусть даны два графа: $\mathcal{G}_1 = \langle V_1, E_1\rangle$ и $\mathcal{G}_2 = \langle V_2, E_2\rangle$. 
Тензорным произведением этих графов будем называть граф $\mathcal{G}_3 = \langle V_3, E_3\rangle$, где $V_3 = V_1 \times V_2$, $E_3 = \{ ((v_1,v_2),(u_1,u_2)) \mid (v_1,u_1) \in E_1 \text{ и } (v_2,u_2) \in E_2 \}$.
\end{definition}

Иными словами, тензорным произведением двух графов является граф, такой что:
\begin{enumerate}
 \item множество вершин --- это прямое произведение множеств вершин исходных графов;
 \item ребро между вершинами $v=(v_1,v_2)$ и $u=(u_1,u_2)$ существует тогда и только тогда, когда существуют рёбра между парами вершин $v_1$, $u_1$ и $v_2$, $u_2$ в соответсвующих графах. 
\end{enumerate}

Для того, чтобы построить тензорное произведение ориентированных графов, необходимо в предыдущем определении, в условии существования ребра в результирующем графе, дополнительно потребовать, чтобы направления рёбер совпадали.
Данное требование получается естесвенным образом, если считать, что пары вершин, задающие ребро, упорядочены, поэтому формальное определение отличаться не будет.

Осталось добавить метки к рёбрам.
Это приведёт к логичному усилению требованя к существованию ребра: метки рёбер в исходных графах должны совпадать.
Таким образом, мы получаем следующее определение тензорного произведения ориентированных графов с метками на рёбрах.

\begin{definition}
Пусть даны два ориентированных графа с метками на рёбрах: $\mathcal{G}_1 = \langle V_1, E_1, L_1 \rangle$ и $\mathcal{G}_2 = \langle V_2, E_2, L_2 \rangle$.
Тензорным произведением этих графов будем называть граф $\mathcal{G}_3 = \langle V_3, E_3, L_3\rangle$, где $V_3 = V_1 \times V_2$, $E_3 = \{ ((v_1,v_2),l,(u_1,u_2)) \mid (v_1,l,u_1) \in E_1 \text{ и } (v_2,l,u_2) \in E_2 \}$, $L_3=L_1 \cap L_2$.
\end{definition}

Нетрудно заметить, что матрица смежности графа $\mathcal{G}_3$ равна тензорному произведению матриц смежностей исходных графов $\mathcal{G}_1$ и $\mathcal{G}_2$.

\begin{example}
Рассмотрим пример.
В качестве одного из графов возьмём рекурсивный автомат, построенный ранее (изображение~\ref{input1}).
Его матрица смежности выглядит следующим образом.
$$ M_1 =
\begin{pmatrix} 
. & [a] & . & . \\
. & . & [S] & [b] \\
. & . & . & [b] \\
. & . & . & . 
\end{pmatrix}
$$

\begin{align}
\label{input2}
    \begin{tikzpicture}[node distance=2.5cm,shorten >=1pt,on grid,auto] 
       \node[state] (q_0)   {$0$}; 
       \node[state] (q_1) [above right=of q_0] {$1$}; 
       \node[state] (q_2) [right=of q_0] {$2$}; 
       \node[state] (q_3) [right=of q_2] {$3$};
        \path[->] 
        (q_0) edge  node {a} (q_1)          
        (q_1) edge  node {a} (q_2)
        (q_2) edge  node {a} (q_0)
        (q_2) edge[bend left, above]  node {b} (q_3)
        (q_3) edge[bend left, below]  node {b} (q_2);
    \end{tikzpicture}
\end{align}

Второй граф представлен на изображении~\ref{input2}. 
Его матрица смежности имеет следующий вид.
$$ M_2 =
\begin{pmatrix} 
. & [a] & . & . \\
. & . & [a] & . \\
[a] & . & . & [b] \\
. & . & [b] & . 
\end{pmatrix}
$$

Теперь вычислим $M_1 \otimes M_2$.
\begin{scaledalign}{\footnotesize}{2pt}{0.9}{\notag}
M_3 &= M_1 \otimes M_2 = 
\begin{pmatrix} 
. & [a] & . & . \\
. & . & [S] & [b] \\
. & . & . & [b] \\
. & . & . & . 
\end{pmatrix}
\otimes 
\begin{pmatrix} 
. & [a] & . & . \\
. & . & [a] & . \\
[a] & . & . & [b] \\
. & . & [b] & . 
\end{pmatrix}
=\notag\\
&=
\label{eq:graph_tm}
\left(\begin{array}{c c c c | c c c c | c c c c | c c c c } 
. & . & . & .  &  .   & [a] & .   & .  &  . & . & . & .  &  . & . & . & .   \\
. & . & . & .  &  .   & .   & [a] & .  &  . & . & . & .  &  . & . & . & .   \\
. & . & . & .  &  [a] & .   & .   & .  &  . & . & . & .  &  . & . & . & .   \\
. & . & . & .  &  .   & .   & .   & .  &  . & . & . & .  &  . & . & . & .   \\
\hline
. & . & . & .  &  . & . & . & .    &  . & . & . & .  &  . & . & . & .   \\
. & . & . & .  &  . & . & . & .    &  . & . & . & .  &  . & . & . & .   \\
. & . & . & .  &  . & . & . & .    &  . & . & . & .  &  . & . & . & [b] \\
. & . & . & .  &  . & . & . & .    &  . & . & . & .  &  . & . & [b] & . \\
\hline
. & . & . & .  &  . & . & . & .    &  . & . & . & .  &  . & . & . & .   \\
. & . & . & .  &  . & . & . & .    &  . & . & . & .  &  . & . & . & .   \\
. & . & . & .  &  . & . & . & .    &  . & . & . & .  &  . & . & . & [b] \\
. & . & . & .  &  . & . & . & .    &  . & . & . & .  &  . & . & [b] & . \\
\hline
. & . & . & .  &  . & . & . & .    &  . & . & . & .  &  . & . & . & .   \\
. & . & . & .  &  . & . & . & .    &  . & . & . & .  &  . & . & . & .   \\
. & . & . & .  &  . & . & . & .    &  . & . & . & .  &  . & . & . & .   \\
. & . & . & .  &  . & . & . & .    &  . & . & . & .  &  . & . & . & . 
\end{array}\right)
\end{scaledalign}

\end{example}

\section{Алгоритм}

Идея алгоритма основана на обобщении пересечения двух конечных автоматов до пересечения рекурсивного автомата, построенного по грамматике, со входным графом.

Пересечение двух конечных автоматов --- тензорное произведение соответствующих графов.
Пересечение языков коммутативно, тензорное произведение нет, но, как было сказано в разделе~\ref{section2}, существует решение этой проблемы.

Будем рассматривать два конечных автомата: одни получан из входного графа, второй из грамматики. 
Можно найти их пересечение, вычислив тензорное произведение матриц смежности соответствующих графов.
Однако, одной такой итерации не достаточно для решения исходной задачи. За первую итерацию мы найдём только те пути, которые выводятся в граммтике за одни шаг. После этого необходимо добавить соответствующие рёбра во входной граф и повторить операцию: так мы найдём пути, выводимые за два шага. Данные действия надо повторять до тех пор, пока не перестанут находиться новые пары достижимых вершин.
Псевдокод, описывающий данные действия, представлен в листинге~\ref{lst:algo1}.

\begin{algorithm}
  \floatname{algorithm}{Listing}
\begin{algorithmic}[1]
\caption{Поиск путей через тензорное произведение}
\label{lst:algo1}
\Function{contextFreePathQueryingTP}{G, $\mathcal{G}$}
    \State{$R \gets$ рекурсивный автомат для $G$}
    \State{$N \gets$ нетерминальный алфавит для $R$}
    \State{$S \gets$ стартовые состояния для $R$}
    \State{$F \gets$ конечные состояния для $R$}
    \State{$M_1 \gets$ матрица смежности $R$}
    \State{$M_2 \gets$ матрица смежности $\mathcal{G}$}
    \For{$N_i \in N$}
       \If{$N_i \xrightarrow{*} \varepsilon$}
          \State{for all $j \in \mathcal{G}.V: M_2[j,j] \gets M_2[j,j] \cup \{N_i\}$}
          \Comment{Добавим петли для нетерминалов, выводящих $\varepsilon$}
       \EndIf
    \EndFor
    \While{матрица $M_2$ изменяется}
        \State{$M_3 \gets M_1 \otimes M_2$}
        \Comment{Пересечение графов}
        \State{$tC_3 \gets \textit{transitiveClosure}(M_3)$}
        \State{$n \gets$ количество строк и столбцов матрицы $M_3$}
        \Comment{размер матрицы $M_3$ = $n \times n$}
        \For{$i \in 0..n$}
           \For{$j \in 0..n$}
                \If{$tC_3[i,j]$}
                    \State{$s \gets$ стартовая вершина ребра $tC_3[i,j]$}
                    \State{$f \gets$ конечная вершина ребра $tC_3[i,j]$}
                    \If{$s \in S$ and $f \in F$ }
                        \State{$x, y \gets$ $getCoordinates(i,j)$}
                        \State{$M_2[x,y] \gets M_2[x,y] \cup \{getNonterminals(s,f)\}$}
                    \EndIf
                \EndIf
           \EndFor
        \EndFor
    \EndWhile
\State \Return $M_2$
\EndFunction
\end{algorithmic}
\end{algorithm}


Алгоритм исполняется до тех пор, пока матрица смежности $M_2$ изменяется. 
На каждой итерации цикла алгоритм последовательно проделывает следующие команды: пересечение двух автоматов через тензорное произведение, транзитивное замыкание результата тензорного произведения и итерация по всем ячейкам получившейся после транзитивного замыкания матрицы, что необходимо для поиска новых пар достижимых вершин.
Во время итерации по ячейкам матрицы транзитивного замыкания алгоритм сначала проверяет наличие ребра в данной ячейке, а затем --- принадлежность стартовой и конечной вершин ребра к стартовому и конечному состоянию входного рекурсивного автомата. 
При удовлетворении этих условий алгоритм добавляет нетерминал (или нетерминалы), соответствующие стартовой и конечной вершинам ребра, в ячейку матрицы $M_2$, полученной с благодаря функции $getCoordinates(i,j)$.

Представленный алгоритм не требует преобразования граммтики в ОНФХ, более того, рекурсивный автомат может быть минимизирован. Однако, результатом тензорного рпоизведения является матрица существенно б\'{о}льшего размера, чем в алгоритме, основанном на матричном рпоизведении. кроме этого, необходимо искать транзитивное замыкание этой матрицы.

Ещё одним важным свойством представленного алгоритма является его оптимальность при обработке регулярных запросов. Так как по контекстно свободной граммтике мы не можем поределить, задаёт ли она регулярный язык, то при добавлении в язык запросов возмодности задавать контекстно-свобдные ограничения, возникает проблема: мы не можем в общем случае отличить регулярный запрос от контекстно-свободного. Следовательно, мы вынуждены применять наиболее общий механизм выполнения заросов, что может приводить к существенным накладным расходам при выполнении регулярного запроса. Данный же алгоритм не выполнит лишних действий, так как сразу выполнит классическое пересечение двух автоматов и получит результат. 

\section{Примеры}

Рассмотрим подробно ряд примеров работы описанного алгоритма. 
Будем для каждой итерации внешнего цикла выписывать результаты основных операций: тензорного произведения, транзитивного замыкания, обновления матрицы смежности входного графа.
Новые, по сравнению с предыдущим состоянием, элементы матриц будем выделять.

\begin{example}
\label{algorithm_example}
Теоретически худший случай.
Такой же как и для матричного.

\textbf{Итерация 1 (конец).} Начало в разделе~\ref{section2}, где мы вычислили тензорное произведение матриц смежности.
Теперь нам осталось только вычислить транзитивное замыкание полученной матрицы:

\begin{scaledalign}{\footnotesize}{2pt}{0.9}{\notag}
tc(M_3) =
\left(\begin{array}{c c c c | c c c c | c c c c | c c c c } 
. & . & . & .  &  . & [a] & . & .  &  . & . & . & .  &  . & . & . & .\\
. & . & . & .  &  . & . & [a] & .  &  . & . & . & .  &  . & . & . & \bfgray{[ab]}   \\
. & . & . & .  &  [a] & . & . & .  &  . & . & . & .  &  . & . & . & .   \\
. & . & . & .  &  . & . & . & .    &  . & . & . & .  &  . & . & . & .   \\
\hline
. & . & . & .  &  . & . & . & .    &  . & . & . & .  &  . & . & . & .   \\
. & . & . & .  &  . & . & . & .    &  . & . & . & .  &  . & . & . & .   \\
. & . & . & .  &  . & . & . & .    &  . & . & . & .  &  . & . & . & [b] \\
. & . & . & .  &  . & . & . & .    &  . & . & . & .  &  . & . & [b] & . \\
\hline
. & . & . & .  &  . & . & . & .    &  . & . & . & .  &  . & . & . & .   \\
. & . & . & .  &  . & . & . & .    &  . & . & . & .  &  . & . & . & .   \\
. & . & . & .  &  . & . & . & .    &  . & . & . & .  &  . & . & . & [b] \\
. & . & . & .  &  . & . & . & .    &  . & . & . & .  &  . & . & [b] & . \\
\hline
. & . & . & .  &  . & . & . & .    &  . & . & . & .  &  . & . & . & .   \\
. & . & . & .  &  . & . & . & .    &  . & . & . & .  &  . & . & . & .   \\
. & . & . & .  &  . & . & . & .    &  . & . & . & .  &  . & . & . & .   \\
. & . & . & .  &  . & . & . & .    &  . & . & . & .  &  . & . & . & . 
\end{array}\right).
\end{scaledalign}

Мы видим, что в результате транзитивного замыкания появилось новое ребро с меткой $ab$ из вершины $(0,1)$ в вершину $(3,3)$.
Так как вершина 0 является стартовой в рекурсивном автомате, а 3 является финальной, то слово вдоль пути из вершины 1 в вершину 3 во входном графе выводимо из нетерминала $S$.
Это означает, что в графе должно быть добавлено ребро из $0$ в $3$ с меткой $S$, после чего граф будет выглядеть следующим образом:

\begin{center}
\begin{tikzpicture}[node distance=2.5cm,shorten >=1pt,on grid,auto] 
   \node[state] (q_0)   {$0$}; 
   \node[state] (q_1) [above right=of q_0] {$1$}; 
   \node[state] (q_2) [right=of q_0] {$2$}; 
   \node[state] (q_3) [right=of q_2] {$3$};
    \path[->] 
    (q_0) edge  node {a} (q_1)          
    (q_1) edge  node {a} (q_2)
    (q_2) edge  node {a} (q_0)
    (q_1) edge[bend left, above]  node {\textbf{S}} (q_3)
    (q_2) edge[bend left, above]  node {b} (q_3)
    (q_3) edge[bend left, below]  node {b} (q_2);
\end{tikzpicture}
\end{center}

Матрица смежности обновлённого графа:
$$ M_2 =
\begin{pmatrix} 
. & [a] & . & . \\
. & . & [a] & \textbf{[S]} \\
[a] & . & . & [b] \\
. & . & [b] & . 
\end{pmatrix}
$$

Итерация закончена. 
Возвращаемся к началу цикла и вновь вычисляем тензорное произведение.

\textbf{Итерация 2.}
Вычисляем тензорное произведение матриц смежности.

\begin{scaledalign}{\footnotesize}{2pt}{0.9}{\notag}
M_3 &= M_1 \otimes M_2 = 
\begin{pmatrix} 
. & [a] & . & . \\
. & . & [S] & [b] \\
. & . & . & [b] \\
. & . & . & . 
\end{pmatrix}
\otimes 
\begin{pmatrix} 
. & [a] & . & . \\
. & . & [a] & [S] \\
[a] & . & . & [b] \\
. & . & [b] & . 
\end{pmatrix}
=\notag\\
&=
\left(\begin{array}{c c c c | c c c c | c c c c | c c c c } 
. & . & . & .  &  . & [a] & . & .  &  . & . & . & .    &  . & . & . & .   \\
. & . & . & .  &  . & . & [a] & .  &  . & . & . & .    &  . & . & . & .   \\
. & . & . & .  &  [a] & . & . & .  &  . & . & . & .    &  . & . & . & .   \\
. & . & . & .  &  . & . & . & .    &  . & . & . & .    &  . & . & . & .   \\
\hline
. & . & . & .  &  . & . & . & .    &  . & . & . & .    &  . & . & . & .   \\
. & . & . & .  &  . & . & . & .    &  . & . & . & \bfgray{[S]}  &  . & . & . & .   \\
. & . & . & .  &  . & . & . & .    &  . & . & . & .    &  . & . & . & [b] \\
. & . & . & .  &  . & . & . & .    &  . & . & . & .    &  . & . & [b] & . \\
\hline
. & . & . & .  &  . & . & . & .    &  . & . & . & .    &  . & . & . & .   \\
. & . & . & .  &  . & . & . & .    &  . & . & . & .    &  . & . & . & .   \\
. & . & . & .  &  . & . & . & .    &  . & . & . & .    &  . & . & . & [b] \\
. & . & . & .  &  . & . & . & .    &  . & . & . & .    &  . & . & [b] & . \\
\hline
. & . & . & .  &  . & . & . & .    &  . & . & . & .    &  . & . & . & .   \\
. & . & . & .  &  . & . & . & .    &  . & . & . & .    &  . & . & . & .   \\
. & . & . & .  &  . & . & . & .    &  . & . & . & .    &  . & . & . & .   \\
. & . & . & .  &  . & . & . & .    &  . & . & . & .    &  . & . & . & . 
\end{array}\right)
\end{scaledalign}

Вычисляем транзитивное замыкание полученной матрицы:

\begin{scaledalign}{\footnotesize}{2pt}{0.9}{\notag}
tc(M_3) =
\left(\begin{array}{c c c c | c c c c | c c c c | c c c c } 
. & . & . & .  &  . & [a] & . & .  &  . & . & . & \bfgray{[aS]}  &  . & . & \bfgray{[aSb]} & .   \\
. & . & . & .  &  . & . & [a] & .  &  . & . & . & .              &  . & . & .              & [ab]   \\
. & . & . & .  &  [a] & . & . & .  &  . & . & . & .              &  . & . & .              & .   \\
. & . & . & .  &  . & . & . & .    &  . & . & . & .              &  . & . & .              & .   \\
\hline
. & . & . & .  &  . & . & . & .    &  . & . & . & .              &  . & . & . & .    \\
. & . & . & .  &  . & . & . & .    &  . & . & . & [S]            &  . & . & \bfgray{[Sb]}    & .    \\
. & . & . & .  &  . & . & . & .    &  . & . & . & .              &  . & . & .    & [b]  \\
. & . & . & .  &  . & . & . & .    &  . & . & . & .              &  . & . & [b]  & .    \\
\hline                                                              
. & . & . & .  &  . & . & . & .    &  . & . & . & .              &  . & . & . & .   \\
. & . & . & .  &  . & . & . & .    &  . & . & . & .              &  . & . & . & .   \\
. & . & . & .  &  . & . & . & .    &  . & . & . & .              &  . & . & . & [b] \\
. & . & . & .  &  . & . & . & .    &  . & . & . & .              &  . & . & [b] & . \\
\hline                                                              
. & . & . & .  &  . & . & . & .    &  . & . & . & .              &  . & . & . & .   \\
. & . & . & .  &  . & . & . & .    &  . & . & . & .              &  . & . & . & .   \\
. & . & . & .  &  . & . & . & .    &  . & . & . & .              &  . & . & . & .   \\
. & . & . & .  &  . & . & . & .    &  . & . & . & .              &  . & . & . & . 
\end{array}\right)
\end{scaledalign}

В транзитивном замыкании появилось три новых ребра, однако только одно из них соединяет вершины, соответствующие стартовому и конечному состоянию входного рекурсивного автомата.
Таким образом только это ребро должно быть добавлено во входной граф.
В итоге, обновлённый граф:
\begin{center}
\begin{tikzpicture}[node distance=2.5cm,shorten >=1pt,on grid,auto] 
   \node[state] (q_0)   {$0$}; 
   \node[state] (q_1) [above right=of q_0] {$1$}; 
   \node[state] (q_2) [right=of q_0] {$2$}; 
   \node[state] (q_3) [right=of q_2] {$3$};
    \path[->] 
    (q_0) edge  node {a} (q_1)          
    (q_1) edge  node {a} (q_2)
    (q_2) edge  node {a} (q_0)
    (q_1) edge[bend left, above]  node {S} (q_3)
    (q_0) edge[bend right, below]  node {\textbf{S}} (q_2)
    (q_2) edge[bend left, above]  node {b} (q_3)
    (q_3) edge[bend left, below]  node {b} (q_2);
\end{tikzpicture}
\end{center}

И его матрица смежности:

$$ M_2 =
\begin{pmatrix} 
. & [a] & [S] & . \\
. & . & [a] & [S] \\
[a] & . & . & [b] \\
. & . & [b] & . 
\end{pmatrix}
$$

\textbf{Итерация 3.}
Снова начинаем с тензорного произведения.

\begin{scaledalign}{\footnotesize}{2pt}{0.9}{\notag}
M_3 &= M_1 \otimes M_2 = 
\begin{pmatrix} 
. & [a] & . & . \\
. & . & [S] & [b] \\
. & . & . & [b] \\
. & . & . & . 
\end{pmatrix}
\otimes 
\begin{pmatrix} 
. & [a] & [S] & . \\
. & . & [a] & [S] \\
[a] & . & . & [b] \\
. & . & [b] & . 
\end{pmatrix}
=\notag\\
&=
\left(\begin{array}{c c c c | c c c c | c c c c | c c c c } 
. & . & . & .  &  . & [a] & . & .  &  . & . & . & .    &  . & . & . & .   \\
. & . & . & .  &  . & . & [a] & .  &  . & . & . & .    &  . & . & . & .   \\
. & . & . & .  &  [a] & . & . & .  &  . & . & . & .    &  . & . & . & .   \\
. & . & . & .  &  . & . & . & .    &  . & . & . & .    &  . & . & . & .   \\
\hline
. & . & . & .  &  . & . & . & .    &  . & . & \bfgray{[S]} & .    &  . & . & . & .   \\
. & . & . & .  &  . & . & . & .    &  . & . & .   & [S]  &  . & . & . & .   \\
. & . & . & .  &  . & . & . & .    &  . & . & .   & .    &  . & . & . & [b] \\
. & . & . & .  &  . & . & . & .    &  . & . & .   & .    &  . & . & [b] & . \\
\hline
. & . & . & .  &  . & . & . & .    &  . & . & . & .    &  . & . & . & .   \\
. & . & . & .  &  . & . & . & .    &  . & . & . & .    &  . & . & . & .   \\
. & . & . & .  &  . & . & . & .    &  . & . & . & .    &  . & . & . & [b] \\
. & . & . & .  &  . & . & . & .    &  . & . & . & .    &  . & . & [b] & . \\
\hline
. & . & . & .  &  . & . & . & .    &  . & . & . & .    &  . & . & . & .   \\
. & . & . & .  &  . & . & . & .    &  . & . & . & .    &  . & . & . & .   \\
. & . & . & .  &  . & . & . & .    &  . & . & . & .    &  . & . & . & .   \\
. & . & . & .  &  . & . & . & .    &  . & . & . & .    &  . & . & . & . 
\end{array}\right)
\end{scaledalign}

Затем вычисляем транзитивное замыкание:

\begin{scaledalign}{\footnotesize}{2pt}{0.9}{\notag}
tc(M_3) =
\left(\begin{array}{c c c c | c c c c | c c c c | c c c c } 
. & . & . & .  &  . & [a] & . & .  &  . & . & . & [aS]           &  . & . & [aSb] & .     \\
. & . & . & .  &  . & . & [a] & .  &  . & . & . & .              &  . & . & .     & [ab]  \\
. & . & . & .  &  [a] & . & . & .  &  . & . & \bfgray{[aS]} & .  &  . & . & .     & \bfgray{[aSb]} \\
. & . & . & .  &  . & . & . & .    &  . & . & . & .              &  . & . & .     & .     \\
\hline
. & . & . & .  &  . & . & . & .    &  . & . & [S] & .            &  . & . & .    & \bfgray{[Sb]}    \\
. & . & . & .  &  . & . & . & .    &  . & . & . & [S]            &  . & . & [Sb] & .    \\
. & . & . & .  &  . & . & . & .    &  . & . & . & .              &  . & . & .    & [b]  \\
. & . & . & .  &  . & . & . & .    &  . & . & . & .              &  . & . & [b]  & .    \\
\hline                                                              
. & . & . & .  &  . & . & . & .    &  . & . & . & .              &  . & . & . & .   \\
. & . & . & .  &  . & . & . & .    &  . & . & . & .              &  . & . & . & .   \\
. & . & . & .  &  . & . & . & .    &  . & . & . & .              &  . & . & . & [b] \\
. & . & . & .  &  . & . & . & .    &  . & . & . & .              &  . & . & [b] & . \\
\hline                                                              
. & . & . & .  &  . & . & . & .    &  . & . & . & .              &  . & . & . & .   \\
. & . & . & .  &  . & . & . & .    &  . & . & . & .              &  . & . & . & .   \\
. & . & . & .  &  . & . & . & .    &  . & . & . & .              &  . & . & . & .   \\
. & . & . & .  &  . & . & . & .    &  . & . & . & .              &  . & . & . & . 
\end{array}\right)
\end{scaledalign}

И наконец обновляем граф:
\begin{center}
\begin{tikzpicture}[node distance=2.5cm,shorten >=1pt,on grid,auto] 
   \node[state] (q_0)   {$0$}; 
   \node[state] (q_1) [above right=of q_0] {$1$}; 
   \node[state] (q_2) [right=of q_0] {$2$}; 
   \node[state] (q_3) [right=of q_2] {$3$};
    \path[->] 
    (q_0) edge  node {a} (q_1)          
    (q_1) edge  node {a} (q_2)
    (q_2) edge  node {a} (q_0)
    (q_1) edge[bend left, above]  node {S} (q_3)
    (q_0) edge[bend right, below]  node {S} (q_2)
    (q_2) edge[bend left, above]  node {b,\textbf{S}} (q_3)
    (q_3) edge[bend left, below]  node {b} (q_2);
\end{tikzpicture}
\end{center}

Матрица смежности обновлённого графа:

$$ M_2 =
\begin{pmatrix} 
. & [a] & [S] & . \\
. & . & [a] & [S] \\
[a] & . & . & [b, \textbf{S}] \\
. & . & [b] & . 
\end{pmatrix}
$$

Уже можно заметить закономерность: на каждой итерации мы добавляем ровно одно новое ребро во входной граф, ровно как и в алгоритме, основанном на матричном произведении, и как в алгоритме Хеллингса.
То есть находим ровно одну новую пару вешин, между которыми существует интересующий нас путь.
Попробуйте спрогонозировать, сколько итераций нам ещё осталось сделать.

\textbf{Итерауия 4}.
Продолжаем. Вычисляем тензорное произведение.

\begin{scaledalign}{\footnotesize}{2pt}{0.9}{\notag}
M_3 &= M_1 \otimes M_2 = 
\begin{pmatrix} 
. & [a] & . & . \\
. & . & [S] & [b] \\
. & . & . & [b] \\
. & . & . & . 
\end{pmatrix}
\otimes 
\begin{pmatrix} 
. & [a] & [S] & . \\
. & . & [a] & [S] \\
[a] & . & . & [b,S] \\
. & . & [b] & . 
\end{pmatrix}
=\notag\\
&=
\left(\begin{array}{c c c c | c c c c | c c c c | c c c c } 
. & . & . & .  &  . & [a] & . & .  &  . & . & . & .    &  . & . & . & .   \\
. & . & . & .  &  . & . & [a] & .  &  . & . & . & .    &  . & . & . & .   \\
. & . & . & .  &  [a] & . & . & .  &  . & . & . & .    &  . & . & . & .   \\
. & . & . & .  &  . & . & . & .    &  . & . & . & .    &  . & . & . & .   \\
\hline
. & . & . & .  &  . & . & . & .    &  . & . & [S] & .             &  . & . & . & .   \\
. & . & . & .  &  . & . & . & .    &  . & . & .   & [S]           &  . & . & . & .   \\
. & . & . & .  &  . & . & . & .    &  . & . & .   & \bfgray{[S]}  &  . & . & . & [b] \\
. & . & . & .  &  . & . & . & .    &  . & . & .   & .             &  . & . & [b] & . \\
\hline
. & . & . & .  &  . & . & . & .    &  . & . & . & .    &  . & . & . & .   \\
. & . & . & .  &  . & . & . & .    &  . & . & . & .    &  . & . & . & .   \\
. & . & . & .  &  . & . & . & .    &  . & . & . & .    &  . & . & . & [b] \\
. & . & . & .  &  . & . & . & .    &  . & . & . & .    &  . & . & [b] & . \\
\hline
. & . & . & .  &  . & . & . & .    &  . & . & . & .    &  . & . & . & .   \\
. & . & . & .  &  . & . & . & .    &  . & . & . & .    &  . & . & . & .   \\
. & . & . & .  &  . & . & . & .    &  . & . & . & .    &  . & . & . & .   \\
. & . & . & .  &  . & . & . & .    &  . & . & . & .    &  . & . & . & . 
\end{array}\right)
\end{scaledalign}

Затем транзитивное замыкание:

\begin{scaledalign}{\footnotesize}{2pt}{0.9}{\notag}
tc(M_3) =
\left(\begin{array}{c c c c | c c c c | c c c c | c c c c } 
. & . & . & .  &  . & [a] & . & .  &  . & . & . & [aS]           &  . & . & [aSb]          & .     \\
. & . & . & .  &  . & . & [a] & .  &  . & . & . & \bfgray{[aS]}  &  . & . & \bfgray{[aSb]} & [ab]  \\
. & . & . & .  &  [a] & . & . & .  &  . & . & [aS] & .           &  . & . & .              & [aSb] \\
. & . & . & .  &  . & . & . & .    &  . & . & . & .              &  . & . & .              & .     \\
\hline
. & . & . & .  &  . & . & . & .    &  . & . & [S] & .            &  . & . & .             & [Sb]    \\
. & . & . & .  &  . & . & . & .    &  . & . & . & [S]            &  . & . & [Sb]          & .    \\
. & . & . & .  &  . & . & . & .    &  . & . & . & [S]            &  . & . & \bfgray{[Sb]} & [b]  \\
. & . & . & .  &  . & . & . & .    &  . & . & . & .              &  . & . & [b]           & .    \\
\hline                                                              
. & . & . & .  &  . & . & . & .    &  . & . & . & .              &  . & . & . & .   \\
. & . & . & .  &  . & . & . & .    &  . & . & . & .              &  . & . & . & .   \\
. & . & . & .  &  . & . & . & .    &  . & . & . & .              &  . & . & . & [b] \\
. & . & . & .  &  . & . & . & .    &  . & . & . & .              &  . & . & [b] & . \\
\hline                                                              
. & . & . & .  &  . & . & . & .    &  . & . & . & .              &  . & . & . & .   \\
. & . & . & .  &  . & . & . & .    &  . & . & . & .              &  . & . & . & .   \\
. & . & . & .  &  . & . & . & .    &  . & . & . & .              &  . & . & . & .   \\
. & . & . & .  &  . & . & . & .    &  . & . & . & .              &  . & . & . & . 
\end{array}\right)
\end{scaledalign}

И снова обновляем граф, так как в транзитивном замыкании появился один (и снова ровно один) новый элемент, соответствующий принимающему пути в автомате.
\begin{center}
\begin{tikzpicture}[node distance=2.5cm,shorten >=1pt,on grid,auto] 
   \node[state] (q_0)   {$0$}; 
   \node[state] (q_1) [above right=of q_0] {$1$}; 
   \node[state] (q_2) [right=of q_0] {$2$}; 
   \node[state] (q_3) [right=of q_2] {$3$};
    \path[->] 
    (q_0) edge  node {a} (q_1)          
    (q_1) edge  node {a,\textbf{S}} (q_2)
    (q_2) edge  node {a} (q_0)
    (q_1) edge[bend left, above]  node {S} (q_3)
    (q_0) edge[bend right, below]  node {S} (q_2)
    (q_2) edge[bend left, above]  node {b,S} (q_3)
    (q_3) edge[bend left, below]  node {b} (q_2);
\end{tikzpicture}
\end{center}
  

Матрица смежности обновлённого графа:

$$ M_2 =
\begin{pmatrix} 
. & [a] & [S] & . \\
. & . & [a, \textbf{S}] & [S] \\
[a] & . & . & [b,S] \\
. & . & [b] & . 
\end{pmatrix}
$$

\textbf{Итерация 5.}
Приступаем к выполнению следующей итерации основного цикла.
Вычисляем тензорное произведение.


\begin{scaledalign}{\footnotesize}{2pt}{0.9}{\notag}
M_3 &= M_1 \otimes M_2 = 
\begin{pmatrix} 
. & [a] & . & . \\
. & . & [S] & [b] \\
. & . & . & [b] \\
. & . & . & . 
\end{pmatrix}
\otimes 
\begin{pmatrix} 
. & [a] & [S] & . \\
. & . & [a,S] & [S] \\
[a] & . & . & [b,S] \\
. & . & [b] & . 
\end{pmatrix}
=\notag\\
&=
\left(\begin{array}{c c c c | c c c c | c c c c | c c c c } 
. & . & . & .  &  . & [a] & . & .  &  . & . & . & .    &  . & . & . & .   \\
. & . & . & .  &  . & . & [a] & .  &  . & . & . & .    &  . & . & . & .   \\
. & . & . & .  &  [a] & . & . & .  &  . & . & . & .    &  . & . & . & .   \\
. & . & . & .  &  . & . & . & .    &  . & . & . & .    &  . & . & . & .   \\
\hline
. & . & . & .  &  . & . & . & .    &  . & . & [S]          & .    &  . & . & . & .   \\
. & . & . & .  &  . & . & . & .    &  . & . & \bfgray{[S]} & [S]  &  . & . & . & .   \\
. & . & . & .  &  . & . & . & .    &  . & . & .            & [S]  &  . & . & . & [b] \\
. & . & . & .  &  . & . & . & .    &  . & . & .            & .    &  . & . & [b] & . \\
\hline
. & . & . & .  &  . & . & . & .    &  . & . & . & .    &  . & . & . & .   \\
. & . & . & .  &  . & . & . & .    &  . & . & . & .    &  . & . & . & .   \\
. & . & . & .  &  . & . & . & .    &  . & . & . & .    &  . & . & . & [b] \\
. & . & . & .  &  . & . & . & .    &  . & . & . & .    &  . & . & [b] & . \\
\hline
. & . & . & .  &  . & . & . & .    &  . & . & . & .    &  . & . & . & .   \\
. & . & . & .  &  . & . & . & .    &  . & . & . & .    &  . & . & . & .   \\
. & . & . & .  &  . & . & . & .    &  . & . & . & .    &  . & . & . & .   \\
. & . & . & .  &  . & . & . & .    &  . & . & . & .    &  . & . & . & . 
\end{array}\right)
\end{scaledalign}

Затем вычисляем транзитивное замыкание:

\begin{scaledalign}{\footnotesize}{2pt}{0.9}{\notag}
tc(M_3) =
\left(\begin{array}{c c c c | c c c c | c c c c | c c c c } 
. & . & . & .  &  . & [a] & . & .  &  . & . & \bfgray{[aS]} & [aS]  &  . & . & [aSb] & \bfgray{[aSb]}  \\
. & . & . & .  &  . & . & [a] & .  &  . & . & .             & [aS]  &  . & . & [aSb] & [ab]          \\
. & . & . & .  &  [a] & . & . & .  &  . & . & [aS]          & .     &  . & . & .     & [aSb]         \\
. & . & . & .  &  . & . & . & .    &  . & . & .             & .     &  . & . & .     & .             \\
\hline
. & . & . & .  &  . & . & . & .    &  . & . & [S] & .             &  . & . & .    & [Sb]    \\
. & . & . & .  &  . & . & . & .    &  . & . & [S] & [S]           &  . & . & [Sb] & \bfgray{[Sb]}    \\
. & . & . & .  &  . & . & . & .    &  . & . & .   & [S]           &  . & . & [Sb] & [b]  \\
. & . & . & .  &  . & . & . & .    &  . & . & .   & .             &  . & . & [b]  & .    \\
\hline                                                              
. & . & . & .  &  . & . & . & .    &  . & . & . & .               &  . & . & .    & .   \\
. & . & . & .  &  . & . & . & .    &  . & . & . & .               &  . & . & .    & .   \\
. & . & . & .  &  . & . & . & .    &  . & . & . & .               &  . & . & .    & [b] \\
. & . & . & .  &  . & . & . & .    &  . & . & . & .               &  . & . & [b]  & . \\
\hline                                                              
. & . & . & .  &  . & . & . & .    &  . & . & . & .               &  . & . & . & .   \\
. & . & . & .  &  . & . & . & .    &  . & . & . & .               &  . & . & . & .   \\
. & . & . & .  &  . & . & . & .    &  . & . & . & .               &  . & . & . & .   \\
. & . & . & .  &  . & . & . & .    &  . & . & . & .               &  . & . & . & . 
\end{array}\right)
\end{scaledalign}

Обновляем граф:
\begin{center}
\begin{tikzpicture}[node distance=2.5cm,shorten >=1pt,on grid,auto] 
   \node[state] (q_0)   {$0$}; 
   \node[state] (q_1) [above right=of q_0] {$1$}; 
   \node[state] (q_2) [right=of q_0] {$2$}; 
   \node[state] (q_3) [right=of q_2] {$3$};
    \path[->] 
    (q_0) edge  node {a} (q_1)          
    (q_1) edge  node {a,S} (q_2)
    (q_2) edge[bend right, above]  node {a} (q_0)
    (q_1) edge[bend left, above]  node {S} (q_3)
    (q_0) edge[bend right, above]  node {S} (q_2)
    (q_2) edge[bend left, above]  node {b,S} (q_3)
    (q_0) edge[bend right, below]  node {\textbf{S}} (q_3)
    (q_3) edge[bend left, above]  node {b} (q_2);
\end{tikzpicture}
\end{center}
  

Матрица смежности обновлённого графа:

$$ M_2 =
\begin{pmatrix} 
. & [a] & [S] & \textbf{[S]} \\
. & . & [a, S] & [S] \\
[a] & . & . & [b,S] \\
. & . & [b] & . 
\end{pmatrix}
$$

\textbf{Итерация 6.}
И наконец последняя содержательная итерация основного цикла.

\begin{scaledalign}{\footnotesize}{2pt}{0.9}{\notag}
M_3 &= M_1 \otimes M_2 = 
\begin{pmatrix} 
. & [a] & . & . \\
. & . & [S] & [b] \\
. & . & . & [b] \\
. & . & . & . 
\end{pmatrix}
\otimes 
\begin{pmatrix} 
. & [a] & [S] & [S] \\
. & . & [a,S] & [S] \\
[a] & . & . & [b,S] \\
. & . & [b] & . 
\end{pmatrix}
=\\
&=
\left(\begin{array}{c c c c | c c c c | c c c c | c c c c } 
. & . & . & .  &  . & [a] & . & .  &  . & . & . & .    &  . & . & . & .   \\
. & . & . & .  &  . & . & [a] & .  &  . & . & . & .    &  . & . & . & .   \\
. & . & . & .  &  [a] & . & . & .  &  . & . & . & .    &  . & . & . & .   \\
. & . & . & .  &  . & . & . & .    &  . & . & . & .    &  . & . & . & .   \\
\hline
. & . & . & .  &  . & . & . & .    &  . & . & [S] & \bfgray{[S]}    &  . & . & . & .   \\
. & . & . & .  &  . & . & . & .    &  . & . & [S] & [S]             &  . & . & . & .   \\
. & . & . & .  &  . & . & . & .    &  . & . & .   & [S]             &  . & . & . & [b] \\
. & . & . & .  &  . & . & . & .    &  . & . & .   & .               &  . & . & [b] & . \\
\hline
. & . & . & .  &  . & . & . & .    &  . & . & . & .    &  . & . & . & .   \\
. & . & . & .  &  . & . & . & .    &  . & . & . & .    &  . & . & . & .   \\
. & . & . & .  &  . & . & . & .    &  . & . & . & .    &  . & . & . & [b] \\
. & . & . & .  &  . & . & . & .    &  . & . & . & .    &  . & . & [b] & . \\
\hline
. & . & . & .  &  . & . & . & .    &  . & . & . & .    &  . & . & . & .   \\
. & . & . & .  &  . & . & . & .    &  . & . & . & .    &  . & . & . & .   \\
. & . & . & .  &  . & . & . & .    &  . & . & . & .    &  . & . & . & .   \\
. & . & . & .  &  . & . & . & .    &  . & . & . & .    &  . & . & . & . 
\end{array}\right)
\end{scaledalign}

Транзитивное замыкание:

\begin{scaledalign}{\footnotesize}{2pt}{0.9}{\notag}
tc(M_3) =
\left(\begin{array}{c c c c | c c c c | c c c c | c c c c } 
. & . & . & .  &  . & [a] & . & .  &  . & . & [aS] & [aS]           &  . & . & [aSb]          & [aSb]  \\
. & . & . & .  &  . & . & [a] & .  &  . & . & .    & [aS]           &  . & . & [aSb]          & [ab]          \\
. & . & . & .  &  [a] & . & . & .  &  . & . & [aS] & \bfgray{[aS]}  &  . & . & \bfgray{[aSb]} & [aSb]         \\
. & . & . & .  &  . & . & . & .    &  . & . & .    & .              &  . & . & .              & .             \\
\hline
. & . & . & .  &  . & . & . & .    &  . & . & [S] & \bfgray{[S]}    &  . & . & \bfgray{[Sb]}  & [Sb]    \\
. & . & . & .  &  . & . & . & .    &  . & . & [S] & [S]             &  . & . & [Sb] & [Sb]    \\
. & . & . & .  &  . & . & . & .    &  . & . & .   & [S]             &  . & . & [Sb] & [b]  \\
. & . & . & .  &  . & . & . & .    &  . & . & .   & .               &  . & . & [b]  & .    \\
\hline                                                              
. & . & . & .  &  . & . & . & .    &  . & . & . & .               &  . & . & .    & .   \\
. & . & . & .  &  . & . & . & .    &  . & . & . & .               &  . & . & .    & .   \\
. & . & . & .  &  . & . & . & .    &  . & . & . & .               &  . & . & .    & [b] \\
. & . & . & .  &  . & . & . & .    &  . & . & . & .               &  . & . & [b]  & . \\
\hline                                                              
. & . & . & .  &  . & . & . & .    &  . & . & . & .               &  . & . & . & .   \\
. & . & . & .  &  . & . & . & .    &  . & . & . & .               &  . & . & . & .   \\
. & . & . & .  &  . & . & . & .    &  . & . & . & .               &  . & . & . & .   \\
. & . & . & .  &  . & . & . & .    &  . & . & . & .               &  . & . & . & . 
\end{array}\right)
\end{scaledalign}

Обновлённый граф:
\begin{center}
\begin{tikzpicture}[node distance=3.5cm,shorten >=1pt,on grid,auto] 
   \node[state] (q_0)   {$0$}; 
   \node[state] (q_1) [above right=of q_0] {$1$}; 
   \node[state] (q_2) [right=of q_0] {$2$}; 
   \node[state] (q_3) [right=of q_2] {$3$};
    \path[->] 
    (q_0) edge  node {a} (q_1)          
    (q_1) edge  node {a,S} (q_2)
    (q_2) edge[bend right, above]  node {a} (q_0)
    (q_2) edge[loop right]  node {\textbf{S}} (q_2)
    (q_1) edge[bend left, above]  node {S} (q_3)
    (q_0) edge[bend right, above]  node {S} (q_2)
    (q_2) edge[bend left, above]  node {b,S} (q_3)
    (q_0) edge[bend right, below]  node {S} (q_3)
    (q_3) edge[bend left, above]  node {b} (q_2);
\end{tikzpicture}
\end{center}
  

И матрица смежности:

$$ M_2 =
\begin{pmatrix} 
. & [a] & [S] & [S] \\
. & . & [a, S] & [S] \\
[a] & . & \textbf{[S]} & [b,S] \\
. & . & [b] & . 
\end{pmatrix}
$$


Следующая итерация не приведёт к изменению графа.
Читатель может убедиться в этом самостоятельно.
Соответственно, алгоритм можно завершать.
Нам потребовалось семь итераций (шесть содержательных и одна проверочная), на каждой из которых вычисляются тензорное произведение двух матриц смежности и транзитивное замыкание результата.

Матрица смежности получилась такая же, как и раньше, ответ правильный.
Мы видим, что количество итераций внешнего цикла такое же как и у алгоритма CYK (пример~\ref{CYK_algorithm_ex}).

\end{example}


\begin{example}

В данном примере мы увидим, как структура грамматики и, соответственно, рекурсивного автомата, влияет на процесс вычислений.

Интуитивно понятно, что чем меньше состояний в рекурсивной сети, тем лучше.
То есть желательно получить как можно более компактное описание контекстно-свободного языка.

Для примера возьмём в качестве КС языка язык Дика на одном типе скобок и опишем его двумя различными грамматиками.
Первая граммтика классическая:
$$
G_1 = \langle \{a,\ b\}, \{ S \}, \{S \to a \ S \ b \ S \mid \varepsilon  \} \rangle
$$

Во второй грамматике мы будем использовать конструкции регулярных выражений в правой части правил.
То есть вторая грамматика находитмся в EBNF (Расширенная форма Бэкуса-Наура~\cite{Hemerik2009, Wirth1977}).
$$
G_2 = \langle \{a, \ b\}, \{S\}, \{S \to (a \ S \ b)^{*}\} \rangle
$$

Построим рекурсивные автоматы $N_1$ и $N_2$ и их матрицы смежности для этих грамматик.

Рекурсивный автомат $N_1$ для грамматики $G_1$:
\begin{center}
\begin{tikzpicture}[node distance=2cm,shorten >=1pt,on grid,auto] 
   \node[state, initial, accepting] (q_0)   {$0$}; 
   \node[state] (q_1) [right=of q_0] {$1$}; 
   \node[state] (q_2) [right=of q_1] {$2$}; 
   \node[state] (q_3) [right=of q_2] {$3$}; 
   \node[state, accepting] (q_4) [right=of q_3] {$4$}; 
    \path[->] 
    (q_0) edge  node {a} (q_1)          
    (q_1) edge  node {S} (q_2)
    (q_2) edge  node {b} (q_3)
    (q_3) edge  node {S} (q_4);
\end{tikzpicture}
\end{center}

Матрица смежности $N_1$:

$$
M_1^1 =
\begin{pmatrix}
. & [a] & .   & .   & .  \\
. & .   & [S] & .   & .  \\
. & .   & .   & [b] & .  \\
. & .   & .   & .   & [S] \\
. & .   & .   & .   & .
\end{pmatrix}
$$


Рекурсивный автомат $N_2$ для грамматики $G_2$:
\begin{center}
\begin{tikzpicture}[node distance=3cm,shorten >=1pt,on grid,auto] 
   \node[state, initial, accepting] (q_0)   {$0$}; 
   \node[state] (q_1) [above right=of q_0] {$1$}; 
   \node[state] (q_2) [right=of q_0] {$2$}; 
    \path[->] 
    (q_0) edge  node {a} (q_1)          
    (q_1) edge  node {S} (q_2)
    (q_2) edge  node {b} (q_0);
\end{tikzpicture}
\end{center}

Матрица смежности $N_2$:

$$
M_1^2 =
\begin{pmatrix}
.   & [a] & .    \\
.   & .   & [S]  \\
[b] & .   & . 
\end{pmatrix}
$$


Первое очевидное наблюдение --- количество состояний в $N_2$ меньше, чем в $N_1$.
Это значит, что матрица смежности, а значит, и результат тензорного произведения будут меньше, следовательно, вычисления будут производиться быстрее.

Выполним пошагово алгоритм для двух заданных рекурсивных сетей на линейном входе:
\begin{center}
\begin{tikzpicture}[node distance=2cm,shorten >=1pt,on grid,auto] 
   \node[state] (q_0)   {$0$}; 
   \node[state] (q_1) [right=of q_0] {$1$}; 
   \node[state] (q_2) [right=of q_1] {$2$}; 
   \node[state] (q_3) [right=of q_2] {$3$}; 
   \node[state] (q_4) [right=of q_3] {$4$}; 
   \node[state] (q_5) [right=of q_4] {$5$}; 
   \node[state] (q_6) [right=of q_5] {$6$}; 
    \path[->] 
    (q_0) edge  node {a} (q_1)          
    (q_1) edge  node {b} (q_2)
    (q_2) edge  node {a} (q_3)
    (q_3) edge  node {b} (q_4)          
    (q_4) edge  node {a} (q_5)
    (q_5) edge  node {b} (q_6);
\end{tikzpicture}
\end{center}


Сразу дополним матрицу смежности нетерминалами, выводящими пустую строку, и получим следующую матрицу:

\begin{scaledalign}{\footnotesize}{0.5pt}{0.9}{\notag}
M_2 =
\begin{pmatrix}
[S] & [a] & .   & .   & .   & .   & .   \\
.   & [S] & [b] & .   & .   & .   & .   \\
.   & .   & [S] & [a] & .   & .   & .   \\
.   & .   & .   & [S] & [b] & .   & .   \\
.   & .   & .   & .   & [S] & [a] & .   \\
.   & .   & .   & .   & .   & [S] & [b] \\
.   & .   & .   & .   & .   & .   & [S] 
\end{pmatrix}
\end{scaledalign}

Сперва запустим алгоритм на данном входе и $N_2$. 
Первый шаг первой итерации --- вычисление тензорного произведения $M_1^2 \otimes M_2$.

\begin{scaledalign}{\footnotesize}{2pt}{0.9}{\notag}
M_3 &= M_1^2 \otimes M_2 = 
\begin{pmatrix}
.   & [a] & .    \\
.   & .   & [S]  \\
[b] & .   & . 
\end{pmatrix}
\otimes 
\begin{pmatrix}
[S] & [a] & .   & .   & .   & .   & .   \\
.   & [S] & [b] & .   & .   & .   & .   \\
.   & .   & [S] & [a] & .   & .   & .   \\
.   & .   & .   & [S] & [b] & .   & .   \\
.   & .   & .   & .   & [S] & [a] & .   \\
.   & .   & .   & .   & .   & [S] & [b] \\
.   & .   & .   & .   & .   & .   & [S] 
\end{pmatrix}
=\notag\\
&=
\left(\begin{array}{c c c c c c c | c c c c c c c | c c c c c c c } 
. & . & . & . & . & . & .  &  . & [a] & . & .   & . & .   & .  &  . & . & . & . & . & . & . \\
. & . & . & . & . & . & .  &  . & .   & . & .   & . & .   & .  &  . & . & . & . & . & . & . \\
. & . & . & . & . & . & .  &  . & .   & . & [a] & . & .   & .  &  . & . & . & . & . & . & . \\
. & . & . & . & . & . & .  &  . & .   & . & .   & . & .   & .  &  . & . & . & . & . & . & . \\
. & . & . & . & . & . & .  &  . & .   & . & .   & . & [a] & .  &  . & . & . & . & . & . & . \\
. & . & . & . & . & . & .  &  . & .   & . & .   & . & .   & .  &  . & . & . & . & . & . & . \\
. & . & . & . & . & . & .  &  . & .   & . & .   & . & .   & .  &  . & . & . & . & . & . & . \\
\hline
. & . & . & . & . & . & .  &  . & . & . & . & . & . & .  &  [S] & . & . & . & . & . & . \\
. & . & . & . & . & . & .  &  . & . & . & . & . & . & .  &  . & [S] & . & . & . & . & . \\
. & . & . & . & . & . & .  &  . & . & . & . & . & . & .  &  . & . & [S] & . & . & . & . \\
. & . & . & . & . & . & .  &  . & . & . & . & . & . & .  &  . & . & . & [S] & . & . & . \\
. & . & . & . & . & . & .  &  . & . & . & . & . & . & .  &  . & . & . & . & [S] & . & . \\
. & . & . & . & . & . & .  &  . & . & . & . & . & . & .  &  . & . & . & . & . & [S] & . \\
. & . & . & . & . & . & .  &  . & . & . & . & . & . & .  &  . & . & . & . & . & . & [S] \\
\hline
. & . & .   & . & .   & . & .    &  . & . & . & . & . & . & .  &  . & . & . & . & . & . & . \\
. & . & [b] & . & .   & . & .    &  . & . & . & . & . & . & .  &  . & . & . & . & . & . & . \\
. & . & .   & . & .   & . & .    &  . & . & . & . & . & . & .  &  . & . & . & . & . & . & . \\
. & . & .   & . & [b] & . & .    &  . & . & . & . & . & . & .  &  . & . & . & . & . & . & . \\
. & . & .   & . & .   & . & .    &  . & . & . & . & . & . & .  &  . & . & . & . & . & . & . \\
. & . & .   & . & .   & . & [b]  &  . & . & . & . & . & . & .  &  . & . & . & . & . & . & . \\
. & . & .   & . & .   & . & .    &  . & . & . & . & . & . & .  &  . & . & . & . & . & . & . 
\end{array}\right)
\end{scaledalign}


\newcommand{\tinybf}[1]{\cellcolor{lightgray}\textbf{\tiny{[#1]}}}
\newcommand{\tntm}[1]{\text{\tiny{#1}}}

Опустим промежуточные шаги вычисления транзитивного замыкания $M_3$ и сразу представим конечный результат:
\begin{scaledalign}{\footnotesize}{0.5pt}{0.5}{\notag}
&tc(M_3)=
\left(\begin{array}{c c c c c c c | c c c c c c c | c c c c c c c } 
. & . & \tinybf{aSb} & . & \tinybf{aSbaSb} & . & \tinybf{aSbaSbaSb}           &         . & [a] & . & \tinybf{aSba} & . & \tinybf{aSbaSba} & .         &           .   & \tinybf{aS} & .   & \tinybf{aSbaS} & .   & \tinybf{aSbaSbaS} & . \\
. & . & .            & . & .               & . & .                            &         . & .   & . & .             & . & .                & .         &           .   & .           & .   & .              & .   & .                 & . \\
. & . & .            & . & \tinybf{aSb}    & . & \tinybf{aSbaSb}              &         . & .   & . & [a]           & . & \tinybf{aSba}    & .         &           .   & .           & .   & \tinybf{aS}    & .   & \tinybf{aSbaS}    & . \\
. & . & .            & . & .               & . & .                            &         . & .   & . & .             & . & .                & .         &           .   & .           & .   & .              & .   & .                 & . \\
. & . & .            & . & .               & . & \tinybf{aSb}                 &         . & .   & . & .             & . & [a]              & .         &           .   & .           & .   & .              & .   & \tinybf{aS}       & . \\
. & . & .            & . & .               & . & .                            &         . & .   & . & .             & . & .                & .         &           .   & .           & .   & .              & .   & .                 & . \\
. & . & .            & . & .               & . & .                            &         . & .   & . & .             & . & .                & .         &           .   & .           & .   & .              & .   & .                 & . \\
\hline                                                                                              
. & . & .            & . & .               & . & .                            &         . & .   & . & .             & . & .                & .         &           [S] & .           & .   & .              & .   & .                 & . \\
. & . & .            & . & .               & . & .                            &         . & .   & . & .             & . & .                & .         &           .   & [S]         & .   & .              & .   & .                 & . \\
. & . & .            & . & .               & . & .                            &         . & .   & . & .             & . & .                & .         &           .   & .           & [S] & .              & .   & .                 & . \\
. & . & .            & . & .               & . & .                            &         . & .   & . & .             & . & .                & .         &           .   & .           & .   & [S]            & .   & .                 & . \\
. & . & .            & . & .               & . & .                            &         . & .   & . & .             & . & .                & .         &           .   & .           & .   & .              & [S] & .                 & . \\
. & . & .            & . & .               & . & .                            &         . & .   & . & .             & . & .                & .         &           .   & .           & .   & .              & .   & [S]               & . \\
. & . & .            & . & .               & . & .                            &         . & .   & . & .             & . & .                & .         &           .   & .           & .   & .              & .   & .                 & [S] \\
\hline                                                                                              
. & . & .            & . & .               & . & .                            &         . & .   & . & .             & . & .                & .         &           .   & .           & .   & .              & .   & .                 & . \\
. & . & [b]          & . & .               & . & .                            &         . & .   & . & .             & . & .                & .         &           .   & .           & .   & .              & .   & .                 & . \\
. & . & .            & . & .               & . & .                            &         . & .   & . & .             & . & .                & .         &           .   & .           & .   & .              & .   & .                 & . \\
. & . & .            & . & [b]             & . & .                            &         . & .   & . & .             & . & .                & .         &           .   & .           & .   & .              & .   & .                 & . \\
. & . & .            & . & .               & . & .                            &         . & .   & . & .             & . & .                & .         &           .   & .           & .   & .              & .   & .                 & . \\
. & . & .            & . & .               & . & [b]                          &         . & .   & . & .             & . & .                & .         &           .   & .           & .   & .              & .   & .                 & . \\
. & . & .            & . & .               & . & .                            &         . & .   & . & .             & . & .                & .         &           .   & .           & .   & .              & .   & .                 & . 
\end{array}\right)
\end{scaledalign}

В результате вычисления транзитивного замыкания появилось большое количество новых рёбер, однако нас будут интересновать только те, информация о которых храниться в левом верхнем блоке.
Остальные рёбра не соответствуют принимающим путям в рекурсивном автомате (убедитесь в этом самостоятельно).

После добавления соответствующих рёбер, мы получим следующий граф:
\begin{center}
\begin{tikzpicture}[node distance=2cm,shorten >=1pt,on grid,auto] 
   \node[state] (q_0)   {$0$}; 
   \node[state] (q_1) [right=of q_0] {$1$}; 
   \node[state] (q_2) [right=of q_1] {$2$}; 
   \node[state] (q_3) [right=of q_2] {$3$}; 
   \node[state] (q_4) [right=of q_3] {$4$}; 
   \node[state] (q_5) [right=of q_4] {$5$}; 
   \node[state] (q_6) [right=of q_5] {$6$}; 
    \path[->] 
    (q_0) edge  node {a} (q_1)
    (q_0) edge[bend left, above]  node {\textbf{S}} (q_2)
    (q_0) edge[bend right, above]  node {\textbf{S}} (q_4)
    (q_0) edge[bend right, above]  node {\textbf{S}} (q_6)
    (q_1) edge  node {b} (q_2)
    (q_2) edge  node {a} (q_3)
    (q_2) edge[bend left, above]  node {\textbf{S}} (q_4)
    (q_2) edge[bend right, above]  node {\textbf{S}} (q_6)
    (q_3) edge  node {b} (q_4)          
    (q_4) edge  node {a} (q_5)
    (q_4) edge[bend left, above]  node {\textbf{S}} (q_6)
    (q_5) edge  node {b} (q_6);
\end{tikzpicture}
\end{center}


Его матрица смежности:

\begin{scaledalign}{\footnotesize}{0.5pt}{0.9}{\notag}
M_2 =
\begin{pmatrix}
[S] & [a] & \bfgray{[S]} & .   & \bfgray{[S]} & .   & \bfgray{[S]} \\
.   & [S] & [b]          & .   & .            & .   & .            \\
.   & .   & [S]          & [a] & \bfgray{[S]} & .   & \bfgray{[S]} \\
.   & .   & .            & [S] & [b]          & .   & .            \\
.   & .   & .            & .   & [S]          & [a] & \bfgray{[S]} \\
.   & .   & .            & .   & .            & [S] & [b]          \\
.   & .   & .            & .   & .            & .   & [S] 
\end{pmatrix}
\end{scaledalign}

Таким образом видно, что для выбранных входных данных алгоритму достаточно двух итераций основного цикла: первая содержательная, вторая, как обычно, проверочная.
Читателю предлагается самостоятельно выяснить, сколько умножений матриц потребуется, чтобы вычислить транзитивное замыкание на первой итерации.

Теперь запустим алгоритм на второй грамматике и том же входе.

\begin{scaledalign}{\footnotesize}{0.5pt}{0.8}{\notag}
&M_3 = M_1^1 \otimes M_2 = 
\begin{pmatrix}
. & [a] & .   & .   & .  \\
. & .   & [S] & .   & .  \\
. & .   & .   & [b] & .  \\
. & .   & .   & .   & [S] \\
. & .   & .   & .   & .
\end{pmatrix}
\otimes 
\begin{pmatrix}
[S] & [a] & .   & .   & .   & .   & .   \\
.   & [S] & [b] & .   & .   & .   & .   \\
.   & .   & [S] & [a] & .   & .   & .   \\
.   & .   & .   & [S] & [b] & .   & .   \\
.   & .   & .   & .   & [S] & [a] & .   \\
.   & .   & .   & .   & .   & [S] & [b] \\
.   & .   & .   & .   & .   & .   & [S] 
\end{pmatrix}
=\notag\\
&=
\left(\begin{array}{c c c c c c c | c c c c c c c | c c c c c c c | c c c c c c c | c c c c c c c} 
. & . & . & . & . & . & .  &  . & [a] & . & .   & . & .   & .  &  . & . & . & . & . & . & .  &  . & . & . & . & . & . & .  &  . & . & . & . & . & . & .   \\
. & . & . & . & . & . & .  &  . & .   & . & .   & . & .   & .  &  . & . & . & . & . & . & .  &  . & . & . & . & . & . & .  &  . & . & . & . & . & . & .   \\
. & . & . & . & . & . & .  &  . & .   & . & [a] & . & .   & .  &  . & . & . & . & . & . & .  &  . & . & . & . & . & . & .  &  . & . & . & . & . & . & .   \\
. & . & . & . & . & . & .  &  . & .   & . & .   & . & .   & .  &  . & . & . & . & . & . & .  &  . & . & . & . & . & . & .  &  . & . & . & . & . & . & .   \\
. & . & . & . & . & . & .  &  . & .   & . & .   & . & [a] & .  &  . & . & . & . & . & . & .  &  . & . & . & . & . & . & .  &  . & . & . & . & . & . & .   \\
. & . & . & . & . & . & .  &  . & .   & . & .   & . & .   & .  &  . & . & . & . & . & . & .  &  . & . & . & . & . & . & .  &  . & . & . & . & . & . & .   \\
. & . & . & . & . & . & .  &  . & .   & . & .   & . & .   & .  &  . & . & . & . & . & . & .  &  . & . & . & . & . & . & .  &  . & . & . & . & . & . & .   \\
\hline
. & . & . & . & . & . & .  &  . & . & . & . & . & . & .  &  [S] & . & . & . & . & . & .  &  . & . & . & . & . & . & .  &  . & . & . & . & . & . & .   \\
. & . & . & . & . & . & .  &  . & . & . & . & . & . & .  &  . & [S] & . & . & . & . & .  &  . & . & . & . & . & . & .  &  . & . & . & . & . & . & .   \\
. & . & . & . & . & . & .  &  . & . & . & . & . & . & .  &  . & . & [S] & . & . & . & .  &  . & . & . & . & . & . & .  &  . & . & . & . & . & . & .   \\
. & . & . & . & . & . & .  &  . & . & . & . & . & . & .  &  . & . & . & [S] & . & . & .  &  . & . & . & . & . & . & .  &  . & . & . & . & . & . & .   \\
. & . & . & . & . & . & .  &  . & . & . & . & . & . & .  &  . & . & . & . & [S] & . & .  &  . & . & . & . & . & . & .  &  . & . & . & . & . & . & .   \\
. & . & . & . & . & . & .  &  . & . & . & . & . & . & .  &  . & . & . & . & . & [S] & .  &  . & . & . & . & . & . & .  &  . & . & . & . & . & . & .   \\
. & . & . & . & . & . & .  &  . & . & . & . & . & . & .  &  . & . & . & . & . & . & [S]  &  . & . & . & . & . & . & .  &  . & . & . & . & . & . & .   \\
\hline
. & . & . & . & . & . & .  &  . & . & . & . & . & . & .  &  . & . & . & . & . & . & .  &  . & . & .   & . & .   & . & .    &  . & . & . & . & . & . & .   \\
. & . & . & . & . & . & .  &  . & . & . & . & . & . & .  &  . & . & . & . & . & . & .  &  . & . & [b] & . & .   & . & .    &  . & . & . & . & . & . & .   \\
. & . & . & . & . & . & .  &  . & . & . & . & . & . & .  &  . & . & . & . & . & . & .  &  . & . & .   & . & .   & . & .    &  . & . & . & . & . & . & .   \\
. & . & . & . & . & . & .  &  . & . & . & . & . & . & .  &  . & . & . & . & . & . & .  &  . & . & .   & . & [b] & . & .    &  . & . & . & . & . & . & .   \\
. & . & . & . & . & . & .  &  . & . & . & . & . & . & .  &  . & . & . & . & . & . & .  &  . & . & .   & . & .   & . & .    &  . & . & . & . & . & . & .   \\
. & . & . & . & . & . & .  &  . & . & . & . & . & . & .  &  . & . & . & . & . & . & .  &  . & . & .   & . & .   & . & [b]  &  . & . & . & . & . & . & .   \\
. & . & . & . & . & . & .  &  . & . & . & . & . & . & .  &  . & . & . & . & . & . & .  &  . & . & .   & . & .   & . & .    &  . & . & . & . & . & . & .   \\
\hline
. & . & . & . & . & . & .  &  . & . & . & . & . & . & .  &  . & . & . & . & . & . & .  &  . & . & . & . & . & . & .  &  [S] & . & . & . & . & . & .   \\
. & . & . & . & . & . & .  &  . & . & . & . & . & . & .  &  . & . & . & . & . & . & .  &  . & . & . & . & . & . & .  &  . & [S] & . & . & . & . & .   \\
. & . & . & . & . & . & .  &  . & . & . & . & . & . & .  &  . & . & . & . & . & . & .  &  . & . & . & . & . & . & .  &  . & . & [S] & . & . & . & .   \\
. & . & . & . & . & . & .  &  . & . & . & . & . & . & .  &  . & . & . & . & . & . & .  &  . & . & . & . & . & . & .  &  . & . & . & [S] & . & . & .   \\
. & . & . & . & . & . & .  &  . & . & . & . & . & . & .  &  . & . & . & . & . & . & .  &  . & . & . & . & . & . & .  &  . & . & . & . & [S] & . & .   \\
. & . & . & . & . & . & .  &  . & . & . & . & . & . & .  &  . & . & . & . & . & . & .  &  . & . & . & . & . & . & .  &  . & . & . & . & . & [S] & .   \\
. & . & . & . & . & . & .  &  . & . & . & . & . & . & .  &  . & . & . & . & . & . & .  &  . & . & . & . & . & . & .  &  . & . & . & . & . & . & [S]   \\
\hline
. & . & . & . & . & . & .  &  . & . & . & . & . & . & .  &  . & . & . & . & . & . & .  &  . & . & . & . & . & . & .  &  . & . & . & . & . & . & .   \\
. & . & . & . & . & . & .  &  . & . & . & . & . & . & .  &  . & . & . & . & . & . & .  &  . & . & . & . & . & . & .  &  . & . & . & . & . & . & .   \\
. & . & . & . & . & . & .  &  . & . & . & . & . & . & .  &  . & . & . & . & . & . & .  &  . & . & . & . & . & . & .  &  . & . & . & . & . & . & .   \\
. & . & . & . & . & . & .  &  . & . & . & . & . & . & .  &  . & . & . & . & . & . & .  &  . & . & . & . & . & . & .  &  . & . & . & . & . & . & .   \\
. & . & . & . & . & . & .  &  . & . & . & . & . & . & .  &  . & . & . & . & . & . & .  &  . & . & . & . & . & . & .  &  . & . & . & . & . & . & .   \\
. & . & . & . & . & . & .  &  . & . & . & . & . & . & .  &  . & . & . & . & . & . & .  &  . & . & . & . & . & . & .  &  . & . & . & . & . & . & .   \\
. & . & . & . & . & . & .  &  . & . & . & . & . & . & .  &  . & . & . & . & . & . & .  &  . & . & . & . & . & . & .  &  . & . & . & . & . & . & .   
\end{array}\right)
\end{scaledalign}

Уже сейчас можно заметить, что размер матриц, с которыми нам придётся работать, существенно увеличился, по сравнению с предыдущим вариантом.
Это, конечно, закономерно, ведь в рекурсивном автомате для предыдущего варианта меньше состояний, а значит и матрица смежности имеет меньший размер.

Транзитивное замыкание:
\begin{scaledalign}{\footnotesize}{0.3pt}{0.5}{\notag}
&tc(M_3)=
\left(\begin{array}{c c c c c c c | c c c c c c c | c c c c c c c | c c c c c c c | c c c c c c c} 
. & . & . & . & . & . & .   &   . & [a] & . & .   & . & .   & .   &   . & \tinybf{aS} & . & .           & . & .           & .  &  . & . & \tinybf{aSb} & . & .            & . & .             &  . & . & \tinybf{aSbS} & . & .             & . & .   \\
. & . & . & . & . & . & .   &   . & .   & . & .   & . & .   & .   &   . & .           & . & .           & . & .           & .  &  . & . & .            & . & .            & . & .             &  . & . & .             & . & .             & . & .   \\
. & . & . & . & . & . & .   &   . & .   & . & [a] & . & .   & .   &   . & .           & . & \tinybf{aS} & . & .           & .  &  . & . & .            & . & \tinybf{aSb} & . & .             &  . & . & .             & . & \tinybf{aSbS} & . & .   \\
. & . & . & . & . & . & .   &   . & .   & . & .   & . & .   & .   &   . & .           & . & .           & . & .           & .  &  . & . & .            & . & .            & . & .             &  . & . & .             & . & .             & . & .   \\
. & . & . & . & . & . & .   &   . & .   & . & .   & . & [a] & .   &   . & .           & . & .           & . & \tinybf{aS} & .  &  . & . & .            & . & .            & . & \tinybf{aSb}  &  . & . & .             & . & .             & . & \tinybf{aSbS}   \\
. & . & . & . & . & . & .   &   . & .   & . & .   & . & .   & .   &   . & .           & . & .           & . & .           & .  &  . & . & .            & . & .            & . & .             &  . & . & .             & . & .             & . & .   \\
. & . & . & . & . & . & .   &   . & .   & . & .   & . & .   & .   &   . & .           & . & .           & . & .           & .  &  . & . & .            & . & .            & . & .             &  . & . & .             & . & .             & . & .   \\
\hline                                                                                
. & . & . & . & . & . & .   &   . & . & . & . & . & . & .   &   [S] & .   & .   & .   & .   & .   & .    &  . & . & . & . & . & . & .  &  . & . & . & . & . & . & .   \\
. & . & . & . & . & . & .   &   . & . & . & . & . & . & .   &   .   & [S] & .   & .   & .   & .   & .    &  . & . & . & . & . & . & .  &  . & . & . & . & . & . & .   \\
. & . & . & . & . & . & .   &   . & . & . & . & . & . & .   &   .   & .   & [S] & .   & .   & .   & .    &  . & . & . & . & . & . & .  &  . & . & . & . & . & . & .   \\
. & . & . & . & . & . & .   &   . & . & . & . & . & . & .   &   .   & .   & .   & [S] & .   & .   & .    &  . & . & . & . & . & . & .  &  . & . & . & . & . & . & .   \\
. & . & . & . & . & . & .   &   . & . & . & . & . & . & .   &   .   & .   & .   & .   & [S] & .   & .    &  . & . & . & . & . & . & .  &  . & . & . & . & . & . & .   \\
. & . & . & . & . & . & .   &   . & . & . & . & . & . & .   &   .   & .   & .   & .   & .   & [S] & .    &  . & . & . & . & . & . & .  &  . & . & . & . & . & . & .   \\
. & . & . & . & . & . & .   &   . & . & . & . & . & . & .   &   .   & .   & .   & .   & .   & .   & [S]  &  . & . & . & . & . & . & .  &  . & . & . & . & . & . & .   \\
\hline                                                                                
. & . & . & . & . & . & .  &  . & . & . & . & . & . & .  &  . & . & . & . & . & . & .  &  . & . & .   & . & .   & . & .    &  . & . & . & . & . & . & .   \\
. & . & . & . & . & . & .  &  . & . & . & . & . & . & .  &  . & . & . & . & . & . & .  &  . & . & [b] & . & .   & . & .    &  . & . & . & . & . & . & .   \\
. & . & . & . & . & . & .  &  . & . & . & . & . & . & .  &  . & . & . & . & . & . & .  &  . & . & .   & . & .   & . & .    &  . & . & . & . & . & . & .   \\
. & . & . & . & . & . & .  &  . & . & . & . & . & . & .  &  . & . & . & . & . & . & .  &  . & . & .   & . & [b] & . & .    &  . & . & . & . & . & . & .   \\
. & . & . & . & . & . & .  &  . & . & . & . & . & . & .  &  . & . & . & . & . & . & .  &  . & . & .   & . & .   & . & .    &  . & . & . & . & . & . & .   \\
. & . & . & . & . & . & .  &  . & . & . & . & . & . & .  &  . & . & . & . & . & . & .  &  . & . & .   & . & .   & . & [b]  &  . & . & . & . & . & . & .   \\
. & . & . & . & . & . & .  &  . & . & . & . & . & . & .  &  . & . & . & . & . & . & .  &  . & . & .   & . & .   & . & .    &  . & . & . & . & . & . & .   \\
\hline
. & . & . & . & . & . & .  &  . & . & . & . & . & . & .  &  . & . & . & . & . & . & .  &  . & . & . & . & . & . & .  &  [S] & . & . & . & . & . & .   \\
. & . & . & . & . & . & .  &  . & . & . & . & . & . & .  &  . & . & . & . & . & . & .  &  . & . & . & . & . & . & .  &  . & [S] & . & . & . & . & .   \\
. & . & . & . & . & . & .  &  . & . & . & . & . & . & .  &  . & . & . & . & . & . & .  &  . & . & . & . & . & . & .  &  . & . & [S] & . & . & . & .   \\
. & . & . & . & . & . & .  &  . & . & . & . & . & . & .  &  . & . & . & . & . & . & .  &  . & . & . & . & . & . & .  &  . & . & . & [S] & . & . & .   \\
. & . & . & . & . & . & .  &  . & . & . & . & . & . & .  &  . & . & . & . & . & . & .  &  . & . & . & . & . & . & .  &  . & . & . & . & [S] & . & .   \\
. & . & . & . & . & . & .  &  . & . & . & . & . & . & .  &  . & . & . & . & . & . & .  &  . & . & . & . & . & . & .  &  . & . & . & . & . & [S] & .   \\
. & . & . & . & . & . & .  &  . & . & . & . & . & . & .  &  . & . & . & . & . & . & .  &  . & . & . & . & . & . & .  &  . & . & . & . & . & . & [S]   \\
\hline
. & . & . & . & . & . & .  &  . & . & . & . & . & . & .  &  . & . & . & . & . & . & .  &  . & . & . & . & . & . & .  &  . & . & . & . & . & . & .   \\
. & . & . & . & . & . & .  &  . & . & . & . & . & . & .  &  . & . & . & . & . & . & .  &  . & . & . & . & . & . & .  &  . & . & . & . & . & . & .   \\
. & . & . & . & . & . & .  &  . & . & . & . & . & . & .  &  . & . & . & . & . & . & .  &  . & . & . & . & . & . & .  &  . & . & . & . & . & . & .   \\
. & . & . & . & . & . & .  &  . & . & . & . & . & . & .  &  . & . & . & . & . & . & .  &  . & . & . & . & . & . & .  &  . & . & . & . & . & . & .   \\
. & . & . & . & . & . & .  &  . & . & . & . & . & . & .  &  . & . & . & . & . & . & .  &  . & . & . & . & . & . & .  &  . & . & . & . & . & . & .   \\
. & . & . & . & . & . & .  &  . & . & . & . & . & . & .  &  . & . & . & . & . & . & .  &  . & . & . & . & . & . & .  &  . & . & . & . & . & . & .   \\
. & . & . & . & . & . & .  &  . & . & . & . & . & . & .  &  . & . & . & . & . & . & .  &  . & . & . & . & . & . & .  &  . & . & . & . & . & . & .   
\end{array}\right)
\end{scaledalign}

Обновлённый граф:
\begin{center}
\begin{tikzpicture}[node distance=2cm,shorten >=1pt,on grid,auto] 
   \node[state] (q_0)   {$0$}; 
   \node[state] (q_1) [right=of q_0] {$1$}; 
   \node[state] (q_2) [right=of q_1] {$2$}; 
   \node[state] (q_3) [right=of q_2] {$3$}; 
   \node[state] (q_4) [right=of q_3] {$4$}; 
   \node[state] (q_5) [right=of q_4] {$5$}; 
   \node[state] (q_6) [right=of q_5] {$6$}; 
    \path[->] 
    (q_0) edge  node {a} (q_1)
    (q_0) edge[bend left, above]  node {\textbf{S}} (q_2)
    (q_1) edge  node {b} (q_2)
    (q_2) edge  node {a} (q_3)
    (q_2) edge[bend left, above]  node {\textbf{S}} (q_4)
    (q_3) edge  node {b} (q_4)          
    (q_4) edge  node {a} (q_5)
    (q_4) edge[bend left, above]  node {\textbf{S}} (q_6)
    (q_5) edge  node {b} (q_6);
\end{tikzpicture}
\end{center}


Его матрица смежности:

\begin{scaledalign}{\footnotesize}{0.5pt}{0.9}{\notag}
M_2 =
\begin{pmatrix}
[S] & [a] & \bfgray{[S]} & .   & .            & .   & .            \\
.   & [S] & [b]          & .   & .            & .   & .            \\
.   & .   & [S]          & [a] & \bfgray{[S]} & .   & .            \\
.   & .   & .            & [S] & [b]          & .   & .            \\
.   & .   & .            & .   & [S]          & [a] & \bfgray{[S]} \\
.   & .   & .            & .   & .            & [S] & [b]          \\
.   & .   & .            & .   & .            & .   & [S] 
\end{pmatrix}
\end{scaledalign}

Потребуется ещё одна итерация.

\begin{scaledalign}{\footnotesize}{0.5pt}{0.8}{\notag}
&M_3 = M_1^1 \otimes M_2 = 
\begin{pmatrix}
. & [a] & .   & .   & .  \\
. & .   & [S] & .   & .  \\
. & .   & .   & [b] & .  \\
. & .   & .   & .   & [S] \\
. & .   & .   & .   & .
\end{pmatrix}
\otimes 
\begin{pmatrix}
[S] & [a] & [S] & .   & .   & .   & .   \\
.   & [S] & [b] & .   & .   & .   & .   \\
.   & .   & [S] & [a] & [S] & .   & .   \\
.   & .   & .   & [S] & [b] & .   & .   \\
.   & .   & .   & .   & [S] & [a] & [S] \\
.   & .   & .   & .   & .   & [S] & [b] \\
.   & .   & .   & .   & .   & .   & [S] 
\end{pmatrix}
=\notag\\
&=
\left(\begin{array}{c c c c c c c | c c c c c c c | c c c c c c c | c c c c c c c | c c c c c c c} 
. & . & . & . & . & . & .  &  . & [a] & . & .   & . & .   & .  &  . & . & . & . & . & . & .  &  . & . & . & . & . & . & .  &  . & . & . & . & . & . & .   \\
. & . & . & . & . & . & .  &  . & .   & . & .   & . & .   & .  &  . & . & . & . & . & . & .  &  . & . & . & . & . & . & .  &  . & . & . & . & . & . & .   \\
. & . & . & . & . & . & .  &  . & .   & . & [a] & . & .   & .  &  . & . & . & . & . & . & .  &  . & . & . & . & . & . & .  &  . & . & . & . & . & . & .   \\
. & . & . & . & . & . & .  &  . & .   & . & .   & . & .   & .  &  . & . & . & . & . & . & .  &  . & . & . & . & . & . & .  &  . & . & . & . & . & . & .   \\
. & . & . & . & . & . & .  &  . & .   & . & .   & . & [a] & .  &  . & . & . & . & . & . & .  &  . & . & . & . & . & . & .  &  . & . & . & . & . & . & .   \\
. & . & . & . & . & . & .  &  . & .   & . & .   & . & .   & .  &  . & . & . & . & . & . & .  &  . & . & . & . & . & . & .  &  . & . & . & . & . & . & .   \\
. & . & . & . & . & . & .  &  . & .   & . & .   & . & .   & .  &  . & . & . & . & . & . & .  &  . & . & . & . & . & . & .  &  . & . & . & . & . & . & .   \\
\hline
. & . & . & . & . & . & .  &  . & . & . & . & . & . & .  &  [S] & .   & \bfgray{[S]} & .   & .            & .   & .             &  . & . & . & . & . & . & .  &  . & . & . & . & . & . & .   \\
. & . & . & . & . & . & .  &  . & . & . & . & . & . & .  &  .   & [S] & .            & .   & .            & .   & .             &  . & . & . & . & . & . & .  &  . & . & . & . & . & . & .   \\
. & . & . & . & . & . & .  &  . & . & . & . & . & . & .  &  .   & .   & [S]          & .   & \bfgray{[S]} & .   & .             &  . & . & . & . & . & . & .  &  . & . & . & . & . & . & .   \\
. & . & . & . & . & . & .  &  . & . & . & . & . & . & .  &  .   & .   & .            & [S] & .            & .   & .             &  . & . & . & . & . & . & .  &  . & . & . & . & . & . & .   \\
. & . & . & . & . & . & .  &  . & . & . & . & . & . & .  &  .   & .   & .            & .   & [S]          & .   & \bfgray{[S]}  &  . & . & . & . & . & . & .  &  . & . & . & . & . & . & .   \\
. & . & . & . & . & . & .  &  . & . & . & . & . & . & .  &  .   & .   & .            & .   & .            & [S] & .             &  . & . & . & . & . & . & .  &  . & . & . & . & . & . & .   \\
. & . & . & . & . & . & .  &  . & . & . & . & . & . & .  &  .   & .   & .            & .   & .            & .   & [S]           &  . & . & . & . & . & . & .  &  . & . & . & . & . & . & .   \\
\hline
. & . & . & . & . & . & .  &  . & . & . & . & . & . & .  &  . & . & . & . & . & . & .  &  . & . & .   & . & .   & . & .    &  . & . & . & . & . & . & .   \\
. & . & . & . & . & . & .  &  . & . & . & . & . & . & .  &  . & . & . & . & . & . & .  &  . & . & [b] & . & .   & . & .    &  . & . & . & . & . & . & .   \\
. & . & . & . & . & . & .  &  . & . & . & . & . & . & .  &  . & . & . & . & . & . & .  &  . & . & .   & . & .   & . & .    &  . & . & . & . & . & . & .   \\
. & . & . & . & . & . & .  &  . & . & . & . & . & . & .  &  . & . & . & . & . & . & .  &  . & . & .   & . & [b] & . & .    &  . & . & . & . & . & . & .   \\
. & . & . & . & . & . & .  &  . & . & . & . & . & . & .  &  . & . & . & . & . & . & .  &  . & . & .   & . & .   & . & .    &  . & . & . & . & . & . & .   \\
. & . & . & . & . & . & .  &  . & . & . & . & . & . & .  &  . & . & . & . & . & . & .  &  . & . & .   & . & .   & . & [b]  &  . & . & . & . & . & . & .   \\
. & . & . & . & . & . & .  &  . & . & . & . & . & . & .  &  . & . & . & . & . & . & .  &  . & . & .   & . & .   & . & .    &  . & . & . & . & . & . & .   \\
\hline
. & . & . & . & . & . & .  &  . & . & . & . & . & . & .  &  . & . & . & . & . & . & .  &  . & . & . & . & . & . & .  &  [S] & .   & \bfgray{[S]} & .   & .            & .   & .   \\
. & . & . & . & . & . & .  &  . & . & . & . & . & . & .  &  . & . & . & . & . & . & .  &  . & . & . & . & . & . & .  &  .   & [S] & .            & .   & .            & .   & .   \\
. & . & . & . & . & . & .  &  . & . & . & . & . & . & .  &  . & . & . & . & . & . & .  &  . & . & . & . & . & . & .  &  .   & .   & [S]          & .   & \bfgray{[S]} & .   & .   \\
. & . & . & . & . & . & .  &  . & . & . & . & . & . & .  &  . & . & . & . & . & . & .  &  . & . & . & . & . & . & .  &  .   & .   & .            & [S] & .            & .   & .   \\
. & . & . & . & . & . & .  &  . & . & . & . & . & . & .  &  . & . & . & . & . & . & .  &  . & . & . & . & . & . & .  &  .   & .   & .            & .   & [S]          & .   & \bfgray{[S]}   \\
. & . & . & . & . & . & .  &  . & . & . & . & . & . & .  &  . & . & . & . & . & . & .  &  . & . & . & . & . & . & .  &  .   & .   & .            & .   & .            & [S] & .   \\
. & . & . & . & . & . & .  &  . & . & . & . & . & . & .  &  . & . & . & . & . & . & .  &  . & . & . & . & . & . & .  &  .   & .   & .            & .   & .            & .   & [S]   \\
\hline
. & . & . & . & . & . & .  &  . & . & . & . & . & . & .  &  . & . & . & . & . & . & .  &  . & . & . & . & . & . & .  &  . & . & . & . & . & . & .   \\
. & . & . & . & . & . & .  &  . & . & . & . & . & . & .  &  . & . & . & . & . & . & .  &  . & . & . & . & . & . & .  &  . & . & . & . & . & . & .   \\
. & . & . & . & . & . & .  &  . & . & . & . & . & . & .  &  . & . & . & . & . & . & .  &  . & . & . & . & . & . & .  &  . & . & . & . & . & . & .   \\
. & . & . & . & . & . & .  &  . & . & . & . & . & . & .  &  . & . & . & . & . & . & .  &  . & . & . & . & . & . & .  &  . & . & . & . & . & . & .   \\
. & . & . & . & . & . & .  &  . & . & . & . & . & . & .  &  . & . & . & . & . & . & .  &  . & . & . & . & . & . & .  &  . & . & . & . & . & . & .   \\
. & . & . & . & . & . & .  &  . & . & . & . & . & . & .  &  . & . & . & . & . & . & .  &  . & . & . & . & . & . & .  &  . & . & . & . & . & . & .   \\
. & . & . & . & . & . & .  &  . & . & . & . & . & . & .  &  . & . & . & . & . & . & .  &  . & . & . & . & . & . & .  &  . & . & . & . & . & . & .   
\end{array}\right)
\end{scaledalign}

Транзитивное замыкание:
\begin{scaledalign}{\footnotesize}{0.3pt}{0.5}{\notag}
&tc(M_3)=
\left(\begin{array}{c c c c c c c | c c c c c c c | c c c c c c c | c c c c c c c | c c c c c c c} 
. & . & . & . & . & . & .   &   . & [a] & . & .   & . & .   & .   &   . & \tntm{[aS]} & . & \tinybf{aS} & . & .           & .  &  . & . & \tntm{[aSb]} & . & \tinybf{aSb} & . & .             &  . & . & \tntm{[aSbS]} & . & \tinybf{aSbS} & . & .               \\
. & . & . & . & . & . & .   &   . & .   & . & .   & . & .   & .   &   . & .           & . & .           & . & .           & .  &  . & . & .            & . & .            & . & .             &  . & . & .             & . & .             & . & .               \\
. & . & . & . & . & . & .   &   . & .   & . & [a] & . & .   & .   &   . & .           & . & \tntm{[aS]} & . & \tinybf{aS} & .  &  . & . & .            & . & \tntm{[aSb]} & . & \tinybf{aSb}  &  . & . & .             & . & \tntm{[aSbS]} & . & \tinybf{aSbS}   \\
. & . & . & . & . & . & .   &   . & .   & . & .   & . & .   & .   &   . & .           & . & .           & . & .           & .  &  . & . & .            & . & .            & . & .             &  . & . & .             & . & .             & . & .               \\
. & . & . & . & . & . & .   &   . & .   & . & .   & . & [a] & .   &   . & .           & . & .           & . & \tntm{[aS]} & .  &  . & . & .            & . & .            & . & \tntm{[aSb]}  &  . & . & .             & . & .             & . & \tntm{[aSbS]}   \\
. & . & . & . & . & . & .   &   . & .   & . & .   & . & .   & .   &   . & .           & . & .           & . & .           & .  &  . & . & .            & . & .            & . & .             &  . & . & .             & . & .             & . & .               \\
. & . & . & . & . & . & .   &   . & .   & . & .   & . & .   & .   &   . & .           & . & .           & . & .           & .  &  . & . & .            & . & .            & . & .             &  . & . & .             & . & .             & . & .               \\
\hline                                                                                
. & . & . & . & . & . & .   &   . & . & . & . & . & . & .   &   [S] & .   & [S] & .   & .   & .   & .    &  . & . & . & . & . & . & .  &  . & . & . & . & . & . & .   \\
. & . & . & . & . & . & .   &   . & . & . & . & . & . & .   &   .   & [S] & .   & .   & .   & .   & .    &  . & . & . & . & . & . & .  &  . & . & . & . & . & . & .   \\
. & . & . & . & . & . & .   &   . & . & . & . & . & . & .   &   .   & .   & [S] & .   & [S] & .   & .    &  . & . & . & . & . & . & .  &  . & . & . & . & . & . & .   \\
. & . & . & . & . & . & .   &   . & . & . & . & . & . & .   &   .   & .   & .   & [S] & .   & .   & .    &  . & . & . & . & . & . & .  &  . & . & . & . & . & . & .   \\
. & . & . & . & . & . & .   &   . & . & . & . & . & . & .   &   .   & .   & .   & .   & [S] & .   & [S]  &  . & . & . & . & . & . & .  &  . & . & . & . & . & . & .   \\
. & . & . & . & . & . & .   &   . & . & . & . & . & . & .   &   .   & .   & .   & .   & .   & [S] & .    &  . & . & . & . & . & . & .  &  . & . & . & . & . & . & .   \\
. & . & . & . & . & . & .   &   . & . & . & . & . & . & .   &   .   & .   & .   & .   & .   & .   & [S]  &  . & . & . & . & . & . & .  &  . & . & . & . & . & . & .   \\
\hline                                                                                
. & . & . & . & . & . & .  &  . & . & . & . & . & . & .  &  . & . & . & . & . & . & .  &  . & . & .   & . & .   & . & .    &  . & . & . & . & . & . & .   \\
. & . & . & . & . & . & .  &  . & . & . & . & . & . & .  &  . & . & . & . & . & . & .  &  . & . & [b] & . & .   & . & .    &  . & . & . & . & . & . & .   \\
. & . & . & . & . & . & .  &  . & . & . & . & . & . & .  &  . & . & . & . & . & . & .  &  . & . & .   & . & .   & . & .    &  . & . & . & . & . & . & .   \\
. & . & . & . & . & . & .  &  . & . & . & . & . & . & .  &  . & . & . & . & . & . & .  &  . & . & .   & . & [b] & . & .    &  . & . & . & . & . & . & .   \\
. & . & . & . & . & . & .  &  . & . & . & . & . & . & .  &  . & . & . & . & . & . & .  &  . & . & .   & . & .   & . & .    &  . & . & . & . & . & . & .   \\
. & . & . & . & . & . & .  &  . & . & . & . & . & . & .  &  . & . & . & . & . & . & .  &  . & . & .   & . & .   & . & [b]  &  . & . & . & . & . & . & .   \\
. & . & . & . & . & . & .  &  . & . & . & . & . & . & .  &  . & . & . & . & . & . & .  &  . & . & .   & . & .   & . & .    &  . & . & . & . & . & . & .   \\
\hline
. & . & . & . & . & . & .  &  . & . & . & . & . & . & .  &  . & . & . & . & . & . & .  &  . & . & . & . & . & . & .  &  [S] & .   & [S] & .   & .   & .   & .   \\
. & . & . & . & . & . & .  &  . & . & . & . & . & . & .  &  . & . & . & . & . & . & .  &  . & . & . & . & . & . & .  &  .   & [S] & .   & .   & .   & .   & .   \\
. & . & . & . & . & . & .  &  . & . & . & . & . & . & .  &  . & . & . & . & . & . & .  &  . & . & . & . & . & . & .  &  .   & .   & [S] & .   & [S] & .   & .   \\
. & . & . & . & . & . & .  &  . & . & . & . & . & . & .  &  . & . & . & . & . & . & .  &  . & . & . & . & . & . & .  &  .   & .   & .   & [S] & .   & .   & .   \\
. & . & . & . & . & . & .  &  . & . & . & . & . & . & .  &  . & . & . & . & . & . & .  &  . & . & . & . & . & . & .  &  .   & .   & .   & .   & [S] & .   & [S]   \\
. & . & . & . & . & . & .  &  . & . & . & . & . & . & .  &  . & . & . & . & . & . & .  &  . & . & . & . & . & . & .  &  .   & .   & .   & .   & .   & [S] & .   \\
. & . & . & . & . & . & .  &  . & . & . & . & . & . & .  &  . & . & . & . & . & . & .  &  . & . & . & . & . & . & .  &  .   & .   & .   & .   & .   & .   & [S]   \\
\hline
. & . & . & . & . & . & .  &  . & . & . & . & . & . & .  &  . & . & . & . & . & . & .  &  . & . & . & . & . & . & .  &  . & . & . & . & . & . & .   \\
. & . & . & . & . & . & .  &  . & . & . & . & . & . & .  &  . & . & . & . & . & . & .  &  . & . & . & . & . & . & .  &  . & . & . & . & . & . & .   \\
. & . & . & . & . & . & .  &  . & . & . & . & . & . & .  &  . & . & . & . & . & . & .  &  . & . & . & . & . & . & .  &  . & . & . & . & . & . & .   \\
. & . & . & . & . & . & .  &  . & . & . & . & . & . & .  &  . & . & . & . & . & . & .  &  . & . & . & . & . & . & .  &  . & . & . & . & . & . & .   \\
. & . & . & . & . & . & .  &  . & . & . & . & . & . & .  &  . & . & . & . & . & . & .  &  . & . & . & . & . & . & .  &  . & . & . & . & . & . & .   \\
. & . & . & . & . & . & .  &  . & . & . & . & . & . & .  &  . & . & . & . & . & . & .  &  . & . & . & . & . & . & .  &  . & . & . & . & . & . & .   \\
. & . & . & . & . & . & .  &  . & . & . & . & . & . & .  &  . & . & . & . & . & . & .  &  . & . & . & . & . & . & .  &  . & . & . & . & . & . & .   
\end{array}\right)
\end{scaledalign}

Обновлённый граф:
\begin{center}
\begin{tikzpicture}[node distance=2cm,shorten >=1pt,on grid,auto] 
   \node[state] (q_0)   {$0$}; 
   \node[state] (q_1) [right=of q_0] {$1$}; 
   \node[state] (q_2) [right=of q_1] {$2$}; 
   \node[state] (q_3) [right=of q_2] {$3$}; 
   \node[state] (q_4) [right=of q_3] {$4$}; 
   \node[state] (q_5) [right=of q_4] {$5$}; 
   \node[state] (q_6) [right=of q_5] {$6$}; 
    \path[->] 
    (q_0) edge  node {a} (q_1)
    (q_0) edge[bend left, above]  node {S} (q_2)
    (q_0) edge[bend right, above]  node {\textbf{S}} (q_4)
    (q_1) edge  node {b} (q_2)
    (q_2) edge  node {a} (q_3)
    (q_2) edge[bend left, above]  node {S} (q_4)
    (q_2) edge[bend right, above]  node {\textbf{S}} (q_6)
    (q_3) edge  node {b} (q_4)          
    (q_4) edge  node {a} (q_5)
    (q_4) edge[bend left, above]  node {S} (q_6)
    (q_5) edge  node {b} (q_6);
\end{tikzpicture}
\end{center}

На этом шаге мы смогли ``склеить'' из подстрок, выводимых из $S$, более длинные пути.
Однако, согласно правилам грамматики, мы смогли ``склеить'' только две подстроки в единое целое.

Матрица смежности обновлённого графа:

\begin{scaledalign}{\footnotesize}{0.5pt}{0.9}{\notag}
M_2 =
\begin{pmatrix}
[S] & [a] & [S]          & .   & \bfgray{[S]} & .   &              \\
.   & [S] & [b]          & .   & .            & .   & .            \\
.   & .   & [S]          & [a] & [S]          & .   & \bfgray{[S]} \\
.   & .   & .            & [S] & [b]          & .   & .            \\
.   & .   & .            & .   & [S]          & [a] & [S]          \\
.   & .   & .            & .   & .            & [S] & [b]          \\
.   & .   & .            & .   & .            & .   & [S] 
\end{pmatrix}
\end{scaledalign}

И, наконец, последняя содержательная итерация.

\begin{scaledalign}{\footnotesize}{0.5pt}{0.8}{\notag}
&M_3 = M_1^1 \otimes M_2 = 
\begin{pmatrix}
. & [a] & .   & .   & .  \\
. & .   & [S] & .   & .  \\
. & .   & .   & [b] & .  \\
. & .   & .   & .   & [S] \\
. & .   & .   & .   & .
\end{pmatrix}
\otimes 
\begin{pmatrix}
[S] & [a] & [S] & .   & [S] & .   & .   \\
.   & [S] & [b] & .   & .   & .   & .   \\
.   & .   & [S] & [a] & [S] & .   & [S] \\
.   & .   & .   & [S] & [b] & .   & .   \\
.   & .   & .   & .   & [S] & [a] & [S] \\
.   & .   & .   & .   & .   & [S] & [b] \\
.   & .   & .   & .   & .   & .   & [S] 
\end{pmatrix}
=\notag\\
&=
\left(\begin{array}{c c c c c c c | c c c c c c c | c c c c c c c | c c c c c c c | c c c c c c c} 
. & . & . & . & . & . & .  &  . & [a] & . & .   & . & .   & .  &  . & . & . & . & . & . & .  &  . & . & . & . & . & . & .  &  . & . & . & . & . & . & .   \\
. & . & . & . & . & . & .  &  . & .   & . & .   & . & .   & .  &  . & . & . & . & . & . & .  &  . & . & . & . & . & . & .  &  . & . & . & . & . & . & .   \\
. & . & . & . & . & . & .  &  . & .   & . & [a] & . & .   & .  &  . & . & . & . & . & . & .  &  . & . & . & . & . & . & .  &  . & . & . & . & . & . & .   \\
. & . & . & . & . & . & .  &  . & .   & . & .   & . & .   & .  &  . & . & . & . & . & . & .  &  . & . & . & . & . & . & .  &  . & . & . & . & . & . & .   \\
. & . & . & . & . & . & .  &  . & .   & . & .   & . & [a] & .  &  . & . & . & . & . & . & .  &  . & . & . & . & . & . & .  &  . & . & . & . & . & . & .   \\
. & . & . & . & . & . & .  &  . & .   & . & .   & . & .   & .  &  . & . & . & . & . & . & .  &  . & . & . & . & . & . & .  &  . & . & . & . & . & . & .   \\
. & . & . & . & . & . & .  &  . & .   & . & .   & . & .   & .  &  . & . & . & . & . & . & .  &  . & . & . & . & . & . & .  &  . & . & . & . & . & . & .   \\
\hline
. & . & . & . & . & . & .  &  . & . & . & . & . & . & .  &  [S] & .   & [S] & .   & \bfgray{[S]} & .   & .             &  . & . & . & . & . & . & .  &  . & . & . & . & . & . & .   \\
. & . & . & . & . & . & .  &  . & . & . & . & . & . & .  &  .   & [S] & .   & .   & .            & .   & .             &  . & . & . & . & . & . & .  &  . & . & . & . & . & . & .   \\
. & . & . & . & . & . & .  &  . & . & . & . & . & . & .  &  .   & .   & [S] & .   & [S]          & .   & \bfgray{[S]}  &  . & . & . & . & . & . & .  &  . & . & . & . & . & . & .   \\
. & . & . & . & . & . & .  &  . & . & . & . & . & . & .  &  .   & .   & .   & [S] & .            & .   & .             &  . & . & . & . & . & . & .  &  . & . & . & . & . & . & .   \\
. & . & . & . & . & . & .  &  . & . & . & . & . & . & .  &  .   & .   & .   & .   & [S]          & .   & [S]           &  . & . & . & . & . & . & .  &  . & . & . & . & . & . & .   \\
. & . & . & . & . & . & .  &  . & . & . & . & . & . & .  &  .   & .   & .   & .   & .            & [S] & .             &  . & . & . & . & . & . & .  &  . & . & . & . & . & . & .   \\
. & . & . & . & . & . & .  &  . & . & . & . & . & . & .  &  .   & .   & .   & .   & .            & .   & [S]           &  . & . & . & . & . & . & .  &  . & . & . & . & . & . & .   \\
\hline
. & . & . & . & . & . & .  &  . & . & . & . & . & . & .  &  . & . & . & . & . & . & .  &  . & . & .   & . & .   & . & .    &  . & . & . & . & . & . & .   \\
. & . & . & . & . & . & .  &  . & . & . & . & . & . & .  &  . & . & . & . & . & . & .  &  . & . & [b] & . & .   & . & .    &  . & . & . & . & . & . & .   \\
. & . & . & . & . & . & .  &  . & . & . & . & . & . & .  &  . & . & . & . & . & . & .  &  . & . & .   & . & .   & . & .    &  . & . & . & . & . & . & .   \\
. & . & . & . & . & . & .  &  . & . & . & . & . & . & .  &  . & . & . & . & . & . & .  &  . & . & .   & . & [b] & . & .    &  . & . & . & . & . & . & .   \\
. & . & . & . & . & . & .  &  . & . & . & . & . & . & .  &  . & . & . & . & . & . & .  &  . & . & .   & . & .   & . & .    &  . & . & . & . & . & . & .   \\
. & . & . & . & . & . & .  &  . & . & . & . & . & . & .  &  . & . & . & . & . & . & .  &  . & . & .   & . & .   & . & [b]  &  . & . & . & . & . & . & .   \\
. & . & . & . & . & . & .  &  . & . & . & . & . & . & .  &  . & . & . & . & . & . & .  &  . & . & .   & . & .   & . & .    &  . & . & . & . & . & . & .   \\
\hline
. & . & . & . & . & . & .  &  . & . & . & . & . & . & .  &  . & . & . & . & . & . & .  &  . & . & . & . & . & . & .  &  [S] & .   & [S] & .   & \bfgray{[S]} & .   & .             \\
. & . & . & . & . & . & .  &  . & . & . & . & . & . & .  &  . & . & . & . & . & . & .  &  . & . & . & . & . & . & .  &  .   & [S] & .   & .   & .            & .   & .             \\
. & . & . & . & . & . & .  &  . & . & . & . & . & . & .  &  . & . & . & . & . & . & .  &  . & . & . & . & . & . & .  &  .   & .   & [S] & .   & [S]          & .   & \bfgray{[S]}  \\
. & . & . & . & . & . & .  &  . & . & . & . & . & . & .  &  . & . & . & . & . & . & .  &  . & . & . & . & . & . & .  &  .   & .   & .   & [S] & .            & .   & .             \\
. & . & . & . & . & . & .  &  . & . & . & . & . & . & .  &  . & . & . & . & . & . & .  &  . & . & . & . & . & . & .  &  .   & .   & .   & .   & [S]          & .   & [S]           \\
. & . & . & . & . & . & .  &  . & . & . & . & . & . & .  &  . & . & . & . & . & . & .  &  . & . & . & . & . & . & .  &  .   & .   & .   & .   & .            & [S] & .             \\
. & . & . & . & . & . & .  &  . & . & . & . & . & . & .  &  . & . & . & . & . & . & .  &  . & . & . & . & . & . & .  &  .   & .   & .   & .   & .            & .   & [S]           \\
\hline
. & . & . & . & . & . & .  &  . & . & . & . & . & . & .  &  . & . & . & . & . & . & .  &  . & . & . & . & . & . & .  &  . & . & . & . & . & . & .   \\
. & . & . & . & . & . & .  &  . & . & . & . & . & . & .  &  . & . & . & . & . & . & .  &  . & . & . & . & . & . & .  &  . & . & . & . & . & . & .   \\
. & . & . & . & . & . & .  &  . & . & . & . & . & . & .  &  . & . & . & . & . & . & .  &  . & . & . & . & . & . & .  &  . & . & . & . & . & . & .   \\
. & . & . & . & . & . & .  &  . & . & . & . & . & . & .  &  . & . & . & . & . & . & .  &  . & . & . & . & . & . & .  &  . & . & . & . & . & . & .   \\
. & . & . & . & . & . & .  &  . & . & . & . & . & . & .  &  . & . & . & . & . & . & .  &  . & . & . & . & . & . & .  &  . & . & . & . & . & . & .   \\
. & . & . & . & . & . & .  &  . & . & . & . & . & . & .  &  . & . & . & . & . & . & .  &  . & . & . & . & . & . & .  &  . & . & . & . & . & . & .   \\
. & . & . & . & . & . & .  &  . & . & . & . & . & . & .  &  . & . & . & . & . & . & .  &  . & . & . & . & . & . & .  &  . & . & . & . & . & . & .   
\end{array}\right)
\end{scaledalign}

Транзитивное замыкание:
\begin{scaledalign}{\footnotesize}{0.3pt}{0.5}{\notag}
&tc(M_3)=
\left(\begin{array}{c c c c c c c | c c c c c c c | c c c c c c c | c c c c c c c | c c c c c c c} 
. & . & . & . & . & . & .   &   . & [a] & . & .   & . & .   & .   &   . & \tntm{[aS]} & . & \tntm{[aS]} & . & \tinybf{aS} & .  &  . & . & \tntm{[aSb]} & . & \tntm{[aSb]} & . & \tinybf{aSb}  &  . & . & \tntm{[aSbS]} & . & \tntm{[aSbS]} & . & \tinybf{aSbS}   \\
. & . & . & . & . & . & .   &   . & .   & . & .   & . & .   & .   &   . & .           & . & .           & . & .           & .  &  . & . & .            & . & .            & . & .             &  . & . & .             & . & .             & . & .               \\
. & . & . & . & . & . & .   &   . & .   & . & [a] & . & .   & .   &   . & .           & . & \tntm{[aS]} & . & \tntm{[aS]} & .  &  . & . & .            & . & \tntm{[aSb]} & . & \tntm{[aSb]}  &  . & . & .             & . & \tntm{[aSbS]} & . & \tntm{[aSbS]}   \\
. & . & . & . & . & . & .   &   . & .   & . & .   & . & .   & .   &   . & .           & . & .           & . & .           & .  &  . & . & .            & . & .            & . & .             &  . & . & .             & . & .             & . & .               \\
. & . & . & . & . & . & .   &   . & .   & . & .   & . & [a] & .   &   . & .           & . & .           & . & \tntm{[aS]} & .  &  . & . & .            & . & .            & . & \tntm{[aSb]}  &  . & . & .             & . & .             & . & \tntm{[aSbS]}   \\
. & . & . & . & . & . & .   &   . & .   & . & .   & . & .   & .   &   . & .           & . & .           & . & .           & .  &  . & . & .            & . & .            & . & .             &  . & . & .             & . & .             & . & .               \\
. & . & . & . & . & . & .   &   . & .   & . & .   & . & .   & .   &   . & .           & . & .           & . & .           & .  &  . & . & .            & . & .            & . & .             &  . & . & .             & . & .             & . & .               \\
\hline                                                                                
. & . & . & . & . & . & .   &   . & . & . & . & . & . & .   &   [S] & .   & [S] & .   & [S] & .   & .    &  . & . & . & . & . & . & .  &  . & . & . & . & . & . & .   \\
. & . & . & . & . & . & .   &   . & . & . & . & . & . & .   &   .   & [S] & .   & .   & .   & .   & .    &  . & . & . & . & . & . & .  &  . & . & . & . & . & . & .   \\
. & . & . & . & . & . & .   &   . & . & . & . & . & . & .   &   .   & .   & [S] & .   & [S] & .   & [S]  &  . & . & . & . & . & . & .  &  . & . & . & . & . & . & .   \\
. & . & . & . & . & . & .   &   . & . & . & . & . & . & .   &   .   & .   & .   & [S] & .   & .   & .    &  . & . & . & . & . & . & .  &  . & . & . & . & . & . & .   \\
. & . & . & . & . & . & .   &   . & . & . & . & . & . & .   &   .   & .   & .   & .   & [S] & .   & [S]  &  . & . & . & . & . & . & .  &  . & . & . & . & . & . & .   \\
. & . & . & . & . & . & .   &   . & . & . & . & . & . & .   &   .   & .   & .   & .   & .   & [S] & .    &  . & . & . & . & . & . & .  &  . & . & . & . & . & . & .   \\
. & . & . & . & . & . & .   &   . & . & . & . & . & . & .   &   .   & .   & .   & .   & .   & .   & [S]  &  . & . & . & . & . & . & .  &  . & . & . & . & . & . & .   \\
\hline                                                                                
. & . & . & . & . & . & .  &  . & . & . & . & . & . & .  &  . & . & . & . & . & . & .  &  . & . & .   & . & .   & . & .    &  . & . & . & . & . & . & .   \\
. & . & . & . & . & . & .  &  . & . & . & . & . & . & .  &  . & . & . & . & . & . & .  &  . & . & [b] & . & .   & . & .    &  . & . & . & . & . & . & .   \\
. & . & . & . & . & . & .  &  . & . & . & . & . & . & .  &  . & . & . & . & . & . & .  &  . & . & .   & . & .   & . & .    &  . & . & . & . & . & . & .   \\
. & . & . & . & . & . & .  &  . & . & . & . & . & . & .  &  . & . & . & . & . & . & .  &  . & . & .   & . & [b] & . & .    &  . & . & . & . & . & . & .   \\
. & . & . & . & . & . & .  &  . & . & . & . & . & . & .  &  . & . & . & . & . & . & .  &  . & . & .   & . & .   & . & .    &  . & . & . & . & . & . & .   \\
. & . & . & . & . & . & .  &  . & . & . & . & . & . & .  &  . & . & . & . & . & . & .  &  . & . & .   & . & .   & . & [b]  &  . & . & . & . & . & . & .   \\
. & . & . & . & . & . & .  &  . & . & . & . & . & . & .  &  . & . & . & . & . & . & .  &  . & . & .   & . & .   & . & .    &  . & . & . & . & . & . & .   \\
\hline
. & . & . & . & . & . & .  &  . & . & . & . & . & . & .  &  . & . & . & . & . & . & .  &  . & . & . & . & . & . & .  &  [S] & .   & [S] & .   & [S] & .   & .   \\
. & . & . & . & . & . & .  &  . & . & . & . & . & . & .  &  . & . & . & . & . & . & .  &  . & . & . & . & . & . & .  &  .   & [S] & .   & .   & .   & .   & .   \\
. & . & . & . & . & . & .  &  . & . & . & . & . & . & .  &  . & . & . & . & . & . & .  &  . & . & . & . & . & . & .  &  .   & .   & [S] & .   & [S] & .   & [S] \\
. & . & . & . & . & . & .  &  . & . & . & . & . & . & .  &  . & . & . & . & . & . & .  &  . & . & . & . & . & . & .  &  .   & .   & .   & [S] & .   & .   & .   \\
. & . & . & . & . & . & .  &  . & . & . & . & . & . & .  &  . & . & . & . & . & . & .  &  . & . & . & . & . & . & .  &  .   & .   & .   & .   & [S] & .   & [S] \\
. & . & . & . & . & . & .  &  . & . & . & . & . & . & .  &  . & . & . & . & . & . & .  &  . & . & . & . & . & . & .  &  .   & .   & .   & .   & .   & [S] & .   \\
. & . & . & . & . & . & .  &  . & . & . & . & . & . & .  &  . & . & . & . & . & . & .  &  . & . & . & . & . & . & .  &  .   & .   & .   & .   & .   & .   & [S] \\
\hline
. & . & . & . & . & . & .  &  . & . & . & . & . & . & .  &  . & . & . & . & . & . & .  &  . & . & . & . & . & . & .  &  . & . & . & . & . & . & .   \\
. & . & . & . & . & . & .  &  . & . & . & . & . & . & .  &  . & . & . & . & . & . & .  &  . & . & . & . & . & . & .  &  . & . & . & . & . & . & .   \\
. & . & . & . & . & . & .  &  . & . & . & . & . & . & .  &  . & . & . & . & . & . & .  &  . & . & . & . & . & . & .  &  . & . & . & . & . & . & .   \\
. & . & . & . & . & . & .  &  . & . & . & . & . & . & .  &  . & . & . & . & . & . & .  &  . & . & . & . & . & . & .  &  . & . & . & . & . & . & .   \\
. & . & . & . & . & . & .  &  . & . & . & . & . & . & .  &  . & . & . & . & . & . & .  &  . & . & . & . & . & . & .  &  . & . & . & . & . & . & .   \\
. & . & . & . & . & . & .  &  . & . & . & . & . & . & .  &  . & . & . & . & . & . & .  &  . & . & . & . & . & . & .  &  . & . & . & . & . & . & .   \\
. & . & . & . & . & . & .  &  . & . & . & . & . & . & .  &  . & . & . & . & . & . & .  &  . & . & . & . & . & . & .  &  . & . & . & . & . & . & .   
\end{array}\right)
\end{scaledalign}


В конечном итоге мы получаем такой же результат, как и при первом запуске.
Однако нам потребовалось выполнить существенно больше итераций внешнего цикла, а именно четыре (три содержательных и одна проверочная).
При этом, в ходе работы нам пришлось оперировать существенно б\'{о}льшими матрицами: $35 \times 35$ против $21 \times 21$.

Таким образом, видно, что минимизация представления запроса, в частности, минимизация рекурсивного автомата как конечного автомата над смешанным алфавитом может улучшить производительность выполнения запросов.
\end{example}

\section{Особенности реализации}

Как и алгоритмы, представленные в разделе~\ref{chpt:MatrixBasedAlgos}, представленный здесь алгоритм оперирует разреженными матрицами, поэтому, к нему применимы все те же соображения, что и к алгоритмам, основанным на произведении матриц. Более того, так как результат тензорного произведения является блочной матрицей, то могут оказаться полезными различные форматы для хранения блочно-разреженных матриц. Вместе с этим, в некоторых случаях матрицу смежности рекурсивного автомата удобнее представлять в классическом, плотном, виде, так как для некоторых запросов её размер мал и накладные расходы на представление в разреженном формате и работе с ним будут больше, чем выигрыш от его использования.


Также заметим, что блочная структура матриц даёт хорошую основу для распределённого умножения матриц при построении транзитивного замыкания.

Вместо того, чтобы перезаписывать каждый раз матрицу смежности входного графа $M_2$ можно вычислять только разницу с предыдущим шагом.
Для этого, правда, потребуется хранить в памяти ещё одну матрицу.
Поэтому нужно проверять, что вычислительно дешевле: поддерживать разницу и потом каждый раз поэлементно складывать две матрицы или каждый раз вычислять полностью произведение.

Заметим, что для решения задачи достижимости нам не нужно накапливать пути вдоль рёбер, как мы это делали в примерах, соответственно, во-первых, можно переопределить тензорное произведение так, чтобы его результатом являлась булева матрица, во-вторых, как следствие первого изменеия, транзитивное замыкание для булевой матрицы можно искать с применением соответствующих оптимизаций.

%\section{Вопросы и задачи}
%
%\begin{enumerate}
%    \item Оценить пространсвенную сложность алгоритма.
%    \item Оценить временную сложность алгоритма.
%    \item Найти библиотеку для тензорного произведения. Реализовать алгоритм. Можно предпологать, что запросы содержат ограниченное число терминалов и нетерминалов. Провести замеры. Сравнить с матричным.
%    \item Реализовать распределённое решение.
%См. блочную структуру
%\end{enumerate}
\chapter{Сжатое представление леса разбора}

Матричный алгоритм даёт нам ответ на вопрос о достижимости, но не предоставляет самих путей.
Что делать, если мы хотим построить все пути, удовлетворяющие ограичениям?

Проблема в том, что искомое множество путей может быть бесконечным.
Можем ли мы предложить конечную структуру, однозначно описывающую такое множество?
Вспомним, что пересечение контекстно-свободного языка с регулярным --- это контекстно-свободный язык.
Мы знаем, что конекстно-свободный язык можно описать коньекстно-своюодной граммтикой, которая конечна.
Это и есть решение нашего вопроса. 
Осталось толко научиться строить такую грамматику.

Прежде, чем двинуться дальше, рекомендуется вспомнить всё, что касается деревьев вывода~\ref{sect:DerivTree}.

\section{Лес разбора как представление контекстно-свободной грамматики}

Для начала нам потребуется внести некоторые изменения в конструкцию дерева вывода.

Во-первых, заметим, что в дереве вывода каждый узел соответсвует выводу какой-то подстроки с известными позициями начала и конца.
Давайте будем сохранять эту информацию в узлах дерева. 
Таким образом, метка любого узла это тройка вида $(i,q,j)$, где $i$ --- координата начала подстроки, соответствующей этому узлу, $j$ --- координата конца, $q \in \Sigma \cup N$ --- метка как в исходном определении.

Во-вторых, заметим, что внутренний узел со своими сыновьями связаны с продукцией в граммтике: узел появляется благодаря применению конкретной продукции в процессе вывода.
Давайте занумеруем все продукции в граммтике и добавим в дерево вывода ещё один тип узлов (дополнительные узлы), в которых будем хранить номер применённой продукции.
Получим следующую конструкцию: непосредственный предок дополнительного узла --- это левая часть продукции, а непосредственные сыновья дополнительного узла --- это правая часть продукции.  

\begin{example}
  Построим модифицированное дерево вывода цепочки $_0a_1b_2a_3b_4a_5b_6$ в грамматике

  \begin{align*}
  G_0 = \langle \{a,b\}, \{S\},  S, \{ & \\
       & \ \ (0) S \to a \ S \ b \ S, \\
       & \ \ (1) S \to \varepsilon \\
  & \}  \rangle  
  \end{align*}
  


\begin{center}
\resizebox{0.9\textwidth}{!}{
\begin{tikzpicture}[shorten >=1pt,node distance=1.2cm] 
   \node[symbol_node] (s_0_6)   {$(0,S,6)$}; 
   \node[prod_node] (p_0_1) [below=of s_0_6] {$0$}; 
   \node[prod_node,draw=none] (dummy1) [below =of p_0_1] {}; 
   \node[symbol_node] (s_1_1) [below left=of p_0_1] {$(1,S,1)$}; 
   \node[symbol_node] (s_2_6) [below right=of p_0_1]  {$(2,S,6)$}; 

   \node[prod_node] (p_0_2) [below right=of s_2_6] {$0$}; 

   \node[symbol_node] (s_3_3) [below left=of p_0_2] {$(3,S,3)$}; 
   \node[symbol_node] (s_4_6) [below right=of p_0_2] {$(4,S,6)$};

   \node[prod_node] (p_0_3) [below right =of s_4_6] {0}; 

   \node[symbol_node] (s_5_5) [below  =of p_0_3] {$(5,S,5)$}; 
   \node[symbol_node] (s_6_6) [below right=of p_0_3] {$(6,S,6)$}; 

%   \node[state,draw=none] (dummy1) [below =of s_0_6] {}; 
%   \node[state,draw=none] (dummy2) [below =of dummy1] {}; 
   
%   \node[state,draw=none] (dummy3) [below left=of p_0_3] {}; 
%   \node[state,draw=none] (dummy4) [below right=of p_0_4] {}; 


%   \node[symbol_node] (s_0_2) [left=of dummy3] {$(0,S,2)$};    
%   
%   \node[symbol_node] (s_2_4) [between=s_0_2 and s_4_6] {$(2,S,4)$};  
   

   \node[prod_node] (p_1_1) [below left =of s_1_1] {$1$}; 
   \node[prod_node] (p_1_2) [below left =of s_3_3] {$1$}; 
   \node[prod_node] (p_1_3) [below  =of s_5_5] {$1$}; 
   \node[prod_node] (p_1_4) [below right=of s_6_6] {$1$}; 
   
   

%   \node[prod_node] (p_2_1) [below =of s_1_1] {$2$}; 
%   \node[prod_node] (p_2_2) [below =of s_3_3] {$2$}; 
%   \node[prod_node] (p_2_3) [below =of s_5_5] {$2$}; 

   \node[symbol_node] (eps_6_6) [below right=of p_1_4] {$(6,\varepsilon,6)$}; 
   \node[symbol_node] (b_5_6)   [left=of eps_6_6] {$(5,b,6)$}; 
   \node[symbol_node] (eps_5_5) [left =of b_5_6] {$(5,\varepsilon,5)$}; 
   \node[symbol_node] (a_4_5)   [left=of eps_5_5] {$(4,a,5)$}; 
   \node[symbol_node] (b_3_4)   [left=of a_4_5] {$(3,b,4)$}; 
   \node[symbol_node] (eps_3_3) [left =of b_3_4] {$(3,\varepsilon,3)$}; 
   \node[symbol_node] (a_2_3)   [left=of eps_3_3] {$(2,a,3)$}; 
   \node[symbol_node] (b_1_2)   [left=of a_2_3] {$(1,b,2)$}; 
   \node[symbol_node] (eps_1_1) [left =of b_1_2] {$(1,\varepsilon,1)$}; 
   \node[symbol_node] (a_0_1) [left=of eps_1_1] {$(0,a,1)$};    


    \path[->] 
    (s_0_6) edge (p_0_1)          

    (p_0_1) edge [bend right] (a_0_1)
    (p_0_1) edge (s_1_1)
    (p_0_1) edge  (b_1_2)
    (p_0_1) edge (s_2_6)

    (s_2_6) edge (p_0_2)          

    (p_0_2) edge [bend right] (a_2_3)
    (p_0_2) edge (s_3_3)
    (p_0_2) edge (b_3_4)
    (p_0_2) edge (s_4_6)

    (s_4_6) edge (p_0_3)          

    (p_0_3) edge (a_4_5)
    (p_0_3) edge (s_5_5)
    (p_0_3) edge (b_5_6)
    (p_0_3) edge (s_6_6)

    (s_1_1) edge (p_1_1)
    (p_1_1) edge (eps_1_1)

    (s_3_3) edge (p_1_2)
    (p_1_2) edge (eps_3_3)

    (s_5_5) edge (p_1_3)
    (p_1_3) edge (eps_5_5)

    (s_6_6) edge (p_1_4)
    (p_1_4) edge (eps_6_6)


    ;
\end{tikzpicture}
}
\end{center}


\end{example}


Сохраняемая нами дополнительная информация позволит переиспользовать узлы в том случае, если деревьев вывода оказалось несколько (в случае неоднозначной грамматики).
При этом мы можем не бояться, что переиспользование узлов может привести к появлению ранее несуществовавших деревьев вывода, так как дополнительная информация позволяет делать только ``безопасные'' склейки и затем восстанавливать только корректные деревья. Таким образом, мы можем представить лес вывода в виде единой структуры данных без дублирования информации.


\begin{example}
  Сжатие леса вывода.
  Построим несколько деревьев вывода цепочки $_0a_1b_2a_3b_4a_5b_6$ в грамматике

  \begin{align*}
   G_1 = \langle \{a,b\}, \{S\},  S, \{ & \\
       & \ \ (0) S \to S S, \\
       & \ \ (1) S \to a \ S \ b, \\
       & \ \ (2) S \to \varepsilon \\
  & \}  \rangle  
  \end{align*}

Пердположим, что мы строим левосторонний вывод.
Тогда после первого применеия продукции 0 у нас есть два варианта переписывания первого нетерминала: либо с применением продукции 0, либо с применением продукции 1:
\begin{align*}
&\textbf{S} \xrightarrow{0} \textbf{S}S \xrightarrow{0} \textbf{S}SS \xrightarrow{1} a\textbf{S}bSS \xrightarrow{2} ab\textbf{S}S \xrightarrow{1} aba\textbf{S}bS \xrightarrow{2} abab\textbf{S} \xrightarrow{1} ababa\textbf{S}b \xrightarrow{2} ababab
\\
&\textbf{S} \xrightarrow{0} \textbf{S}S \xrightarrow{1} a\textbf{S}bS \xrightarrow{2} ab\textbf{S} \xrightarrow{0} ab\textbf{S}S \xrightarrow{1} aba\textbf{S}bS \xrightarrow{2} abab\textbf{S} \xrightarrow{1} ababa\textbf{S}b \xrightarrow{2} ababab
\end{align*}

Сначал рассмотрим первый вариант (применили переписываение по подукции 0).
Все остальные шаги вывода деретерминированы и в результате мы получим следующее дерево разбора:

\begin{center}
\resizebox{0.9\textwidth}{!}{
\begin{tikzpicture}[shorten >=1pt,on grid,auto,node distance=1.8cm] 
   \node[symbol_node] (s_0_6)   {$(0,S,6)$}; 
   \node[prod_node] (p_0_1) [below left=of s_0_6] {$0$}; 
   \node[prod_node,draw=none] (p_0_2) [below right=of s_0_6] {}; 
   \node[symbol_node] (s_0_4) [below left=of p_0_1]  {$(0,S,4)$}; 
   \node[symbol_node,draw=none] (s_2_6) [below right=of p_0_2]  {}; 
   \node[prod_node] (p_0_3) [below =of s_0_4] {$0$}; 
   \node[prod_node,draw=none] (p_0_4) [below =of s_2_6] {}; 

   \node[state,draw=none] (dummy1) [below =of s_0_6] {}; 
   \node[state,draw=none] (dummy2) [below =of dummy1] {}; 
   
   \node[state,draw=none] (dummy3) [below left=of p_0_3] {}; 
   \node[state,draw=none] (dummy4) [below right=of p_0_4] {}; 


   \node[symbol_node] (s_0_2) [left=of dummy3] {$(0,S,2)$};    
   \node[symbol_node] (s_4_6) [right=of dummy4] {$(4,S,6)$};
   \node[symbol_node] (s_2_4) [between=s_0_2 and s_4_6] {$(2,S,4)$};  
   
   \node[prod_node] (p_1_1) [below =of s_0_2] {$1$}; 
   \node[prod_node] (p_1_2) [below =of s_2_4] {$1$}; 
   \node[prod_node] (p_1_3) [below =of s_4_6] {$1$}; 

   \node[symbol_node] (s_1_1) [below =of p_1_1] {$(1,S,1)$}; 
   \node[symbol_node] (s_3_3) [below =of p_1_2] {$(3,S,3)$}; 
   \node[symbol_node] (s_5_5) [below =of p_1_3] {$(5,S,5)$}; 

   \node[prod_node] (p_2_1) [below =of s_1_1] {$2$}; 
   \node[prod_node] (p_2_2) [below =of s_3_3] {$2$}; 
   \node[prod_node] (p_2_3) [below =of s_5_5] {$2$}; 

   \node[symbol_node] (eps_1_1) [below =of p_2_1] {$(1,\varepsilon,1)$}; 
   \node[symbol_node] (eps_3_3) [below =of p_2_2] {$(3,\varepsilon,3)$}; 
   \node[symbol_node] (eps_5_5) [below =of p_2_3] {$(5,\varepsilon,5)$}; 

   \node[symbol_node] (a_0_1) [left=of eps_1_1] {$(0,a,1)$}; 
   \node[symbol_node] (a_2_3) [left=of eps_3_3] {$(2,a,3)$}; 
   \node[symbol_node] (a_4_5) [left=of eps_5_5] {$(4,a,5)$}; 

   \node[symbol_node] (b_1_2) [right=of eps_1_1] {$(1,b,2)$}; 
   \node[symbol_node] (b_3_4) [right=of eps_3_3] {$(3,b,4)$}; 
   \node[symbol_node] (b_5_6) [right=of eps_5_5] {$(5,b,6)$}; 


    \path[->] 
    (s_0_6) edge (p_0_1)          
%    (s_0_6) edge (p_0_2)
    (p_0_1) edge (s_0_4)
    (p_0_1) edge (s_4_6)
%    (p_0_2) edge (s_0_2)
%    (p_0_2) edge (s_2_6)
    (s_0_4) edge (p_0_3)          
%    (s_2_6) edge (p_0_4)
    (p_0_3) edge (s_0_2)
    (p_0_3) edge (s_2_4)
%    (p_0_4) edge (s_2_4)
%    (p_0_4) edge (s_4_6)

    (s_0_2) edge (p_1_1)
    (s_2_4) edge (p_1_2)
    (s_4_6) edge (p_1_3)

    (p_1_1) edge [bend right] (a_0_1)
    (p_1_1) edge (s_1_1)
    (p_1_1) edge [bend left] (b_1_2)

    (p_1_2) edge [bend right] (a_2_3)
    (p_1_2) edge (s_3_3)
    (p_1_2) edge [bend left] (b_3_4)

    (p_1_3) edge [bend right] (a_4_5)
    (p_1_3) edge (s_5_5)
    (p_1_3) edge [bend left] (b_5_6)

    (s_1_1) edge (p_2_1)
    (p_2_1) edge (eps_1_1)

    (s_3_3) edge (p_2_2)
    (p_2_2) edge (eps_3_3)

    (s_5_5) edge (p_2_3)
    (p_2_3) edge (eps_5_5)

    ;
\end{tikzpicture}
}
\end{center}

Теперь рассмотрим второй вариант --- применить продукцию 1.
Остальные шаги вывода всё также детерминированы.
В результате мы получим следующее дерево вывода:

\begin{center}
\resizebox{0.9\textwidth}{!}{
\begin{tikzpicture}[shorten >=1pt,on grid,auto,node distance=1.8cm] 
   \node[symbol_node] (s_0_6)   {$(0,S,6)$}; 
   \node[prod_node,draw=none] (p_0_1) [below left=of s_0_6] {}; 
   \node[prod_node] (p_0_2) [below right=of s_0_6] {$0$}; 
   \node[symbol_node,draw=none] (s_0_4) [below left=of p_0_1]  {}; 
   \node[symbol_node] (s_2_6) [below right=of p_0_2]  {$(2,S,6)$}; 
   \node[prod_node,draw=none] (p_0_3) [below =of s_0_4] {}; 
   \node[prod_node] (p_0_4) [below =of s_2_6] {$0$}; 

   \node[state,draw=none] (dummy1) [below =of s_0_6] {}; 
   \node[state,draw=none] (dummy2) [below =of dummy1] {}; 
   
   \node[state,draw=none] (dummy3) [below left=of p_0_3] {}; 
   \node[state,draw=none] (dummy4) [below right=of p_0_4] {}; 


   \node[symbol_node] (s_0_2) [left=of dummy3] {$(0,S,2)$};    
   \node[symbol_node] (s_4_6) [right=of dummy4] {$(4,S,6)$};
   \node[symbol_node] (s_2_4) [between=s_0_2 and s_4_6] {$(2,S,4)$};  
   
   \node[prod_node] (p_1_1) [below =of s_0_2] {$1$}; 
   \node[prod_node] (p_1_2) [below =of s_2_4] {$1$}; 
   \node[prod_node] (p_1_3) [below =of s_4_6] {$1$}; 

   \node[symbol_node] (s_1_1) [below =of p_1_1] {$(1,S,1)$}; 
   \node[symbol_node] (s_3_3) [below =of p_1_2] {$(3,S,3)$}; 
   \node[symbol_node] (s_5_5) [below =of p_1_3] {$(5,S,5)$}; 

   \node[prod_node] (p_2_1) [below =of s_1_1] {$2$}; 
   \node[prod_node] (p_2_2) [below =of s_3_3] {$2$}; 
   \node[prod_node] (p_2_3) [below =of s_5_5] {$2$}; 

   \node[symbol_node] (eps_1_1) [below =of p_2_1] {$(1,\varepsilon,1)$}; 
   \node[symbol_node] (eps_3_3) [below =of p_2_2] {$(3,\varepsilon,3)$}; 
   \node[symbol_node] (eps_5_5) [below =of p_2_3] {$(5,\varepsilon,5)$}; 

   \node[symbol_node] (a_0_1) [left=of eps_1_1] {$(0,a,1)$}; 
   \node[symbol_node] (a_2_3) [left=of eps_3_3] {$(2,a,3)$}; 
   \node[symbol_node] (a_4_5) [left=of eps_5_5] {$(4,a,5)$}; 

   \node[symbol_node] (b_1_2) [right=of eps_1_1] {$(1,b,2)$}; 
   \node[symbol_node] (b_3_4) [right=of eps_3_3] {$(3,b,4)$}; 
   \node[symbol_node] (b_5_6) [right=of eps_5_5] {$(5,b,6)$}; 


    \path[->] 
    %(s_0_6) edge (p_0_1)          
    (s_0_6) edge (p_0_2)
    %(p_0_1) edge (s_0_4)
    %(p_0_1) edge (s_4_6)
    (p_0_2) edge (s_0_2)
    (p_0_2) edge (s_2_6)
    %(s_0_4) edge (p_0_3)          
    (s_2_6) edge (p_0_4)
    %(p_0_3) edge (s_0_2)
    %(p_0_3) edge (s_2_4)
    (p_0_4) edge (s_2_4)
    (p_0_4) edge (s_4_6)

    (s_0_2) edge (p_1_1)
    (s_2_4) edge (p_1_2)
    (s_4_6) edge (p_1_3)

    (p_1_1) edge [bend right] (a_0_1)
    (p_1_1) edge (s_1_1)
    (p_1_1) edge [bend left] (b_1_2)

    (p_1_2) edge [bend right] (a_2_3)
    (p_1_2) edge (s_3_3)
    (p_1_2) edge [bend left] (b_3_4)

    (p_1_3) edge [bend right] (a_4_5)
    (p_1_3) edge (s_5_5)
    (p_1_3) edge [bend left] (b_5_6)

    (s_1_1) edge (p_2_1)
    (p_2_1) edge (eps_1_1)

    (s_3_3) edge (p_2_2)
    (p_2_2) edge (eps_3_3)

    (s_5_5) edge (p_2_3)
    (p_2_3) edge (eps_5_5)

    ;
\end{tikzpicture}
}
\end{center}

В двух построенных деревьях большое количество одинаковых узлов.
Построим структуру, которая содержит оба дерева и при этом никакие нетерминальные и терминальные узлы не встречаются дважды.
В результате мы молучим следующий граф:

\begin{center}
\resizebox{0.9\textwidth}{!}{
\begin{tikzpicture}[shorten >=1pt,on grid,auto,node distance=1.8cm] 
   \node[symbol_node] (s_0_6)   {$(0,S,6)$}; 
   \node[prod_node] (p_0_1) [below left=of s_0_6] {$0$}; 
   \node[prod_node] (p_0_2) [below right=of s_0_6] {$0$}; 
   \node[symbol_node] (s_0_4) [below left=of p_0_1]  {$(0,S,4)$}; 
   \node[symbol_node] (s_2_6) [below right=of p_0_2]  {$(2,S,6)$}; 
   \node[prod_node] (p_0_3) [below =of s_0_4] {$0$}; 
   \node[prod_node] (p_0_4) [below =of s_2_6] {$0$}; 

   \node[state,draw=none] (dummy1) [below =of s_0_6] {}; 
   \node[state,draw=none] (dummy2) [below =of dummy1] {}; 
   
   \node[state,draw=none] (dummy3) [below left=of p_0_3] {}; 
   \node[state,draw=none] (dummy4) [below right=of p_0_4] {}; 


   \node[symbol_node] (s_0_2) [left=of dummy3] {$(0,S,2)$};    
   \node[symbol_node] (s_4_6) [right=of dummy4] {$(4,S,6)$};
   \node[symbol_node] (s_2_4) [between=s_0_2 and s_4_6] {$(2,S,4)$};  
   
   \node[prod_node] (p_1_1) [below =of s_0_2] {$1$}; 
   \node[prod_node] (p_1_2) [below =of s_2_4] {$1$}; 
   \node[prod_node] (p_1_3) [below =of s_4_6] {$1$}; 

   \node[symbol_node] (s_1_1) [below =of p_1_1] {$(1,S,1)$}; 
   \node[symbol_node] (s_3_3) [below =of p_1_2] {$(3,S,3)$}; 
   \node[symbol_node] (s_5_5) [below =of p_1_3] {$(5,S,5)$}; 

   \node[prod_node] (p_2_1) [below =of s_1_1] {$2$}; 
   \node[prod_node] (p_2_2) [below =of s_3_3] {$2$}; 
   \node[prod_node] (p_2_3) [below =of s_5_5] {$2$}; 

   \node[symbol_node] (eps_1_1) [below =of p_2_1] {$(1,\varepsilon,1)$}; 
   \node[symbol_node] (eps_3_3) [below =of p_2_2] {$(3,\varepsilon,3)$}; 
   \node[symbol_node] (eps_5_5) [below =of p_2_3] {$(5,\varepsilon,5)$}; 

   \node[symbol_node] (a_0_1) [left=of eps_1_1] {$(0,a,1)$}; 
   \node[symbol_node] (a_2_3) [left=of eps_3_3] {$(2,a,3)$}; 
   \node[symbol_node] (a_4_5) [left=of eps_5_5] {$(4,a,5)$}; 

   \node[symbol_node] (b_1_2) [right=of eps_1_1] {$(1,b,2)$}; 
   \node[symbol_node] (b_3_4) [right=of eps_3_3] {$(3,b,4)$}; 
   \node[symbol_node] (b_5_6) [right=of eps_5_5] {$(5,b,6)$}; 


    \path[->] 
    (s_0_6) edge (p_0_1)          
    (s_0_6) edge (p_0_2)
    (p_0_1) edge (s_0_4)
    (p_0_1) edge (s_4_6)
    (p_0_2) edge (s_0_2)
    (p_0_2) edge (s_2_6)
    (s_0_4) edge (p_0_3)          
    (s_2_6) edge (p_0_4)
    (p_0_3) edge (s_0_2)
    (p_0_3) edge (s_2_4)
    (p_0_4) edge (s_2_4)
    (p_0_4) edge (s_4_6)

    (s_0_2) edge (p_1_1)
    (s_2_4) edge (p_1_2)
    (s_4_6) edge (p_1_3)

    (p_1_1) edge [bend right] (a_0_1)
    (p_1_1) edge (s_1_1)
    (p_1_1) edge [bend left] (b_1_2)

    (p_1_2) edge [bend right] (a_2_3)
    (p_1_2) edge (s_3_3)
    (p_1_2) edge [bend left] (b_3_4)

    (p_1_3) edge [bend right] (a_4_5)
    (p_1_3) edge (s_5_5)
    (p_1_3) edge [bend left] (b_5_6)

    (s_1_1) edge (p_2_1)
    (p_2_1) edge (eps_1_1)

    (s_3_3) edge (p_2_2)
    (p_2_2) edge (eps_3_3)

    (s_5_5) edge (p_2_3)
    (p_2_3) edge (eps_5_5)

    ;
\end{tikzpicture}
}
\end{center}


\end{example}


Мы получили очень простой вариант сжатого представления леса разбора (Shared Packed Parse Forest, SPPF). 
Впервые подобная идея была предложена Джоаном Рекерсом в его кандидатской диссертации~\cite{SPPF}.
В дальнейшем она нашла широкое применеие в обобщённом (generalized) синтаксическом анализе и получила серьёзное развитие.
В частности, наш вариант, хоть и позволяет избежать экспоненциального разростания леса разбора, всё же не является оптимальным.
Оптимальное асимптотическое поведение достигается при использовании бинаризованного SPPF~\cite{Billot:1989:SSF:981623.981641} --- в этом случае объём леса составляет $O(n^3)$, где $n$ --- это длина входной строки.

Различные модификации SPPF применяются в таких алгоритмах синтаксического анализа, как RNGLR~\cite{Scott:2006:RNG:1146809.1146810}, бинаризованная верся SPPF в BRNGLR~\cite{Scott:2007:BCT:1289813.1289815} и GLL~\cite{Scott:2010:GP:1860132.1860320,10.1007/978-3-662-46663-6_5}\footnote{Ещё немного полезной информации про SPPF: \url{http://www.bramvandersanden.com/post/2014/06/shared-packed-parse-forest/}.}.

В действительности SPPF может содержать в себе циклы. Для линейного входа их можно получить, когда есть возможность выводить по грамматике бесконечные эпсилон-цепочки. Циклы будут вырожденными, но они будут. 

Мы, кроме традиционного использования, будем применять SPPF для представления результатов КС запросов к графам.

В графе может существовать множество способов получить путь из одной вершины в другую. И точно так же при построении деревьев вывода путей может появиться несколько одинаковых нетерминалов, получаемых в разных деревьях по-разному. При объединении в SPPF может оказаться, что какой-то путь из вершины $a$ в вершину $b$ является подпутем другого пути из вершины $a$ в вершину $b$, просто более длинного. То есть появятся циклические зависимости.

\begin{example}
    Рассмотрим пример SPPF для задачи поиска путей с КС ограничениями.
    Пусть дан граф $\mathcal{G}:$
    
    \begin{center}
        \begin{tikzpicture}[node distance=3cm,shorten >=1pt,on grid,auto]
        \node[state] (q_0)   {$0$};
        \node[state] (q_1) [above right=of q_0] {$1$};
        \node[state] (q_2) [right=of q_0] {$2$};
        \node[state] (q_3) [right=of q_2] {$3$};
        \path[->]
        (q_0) edge  node {a} (q_1)
        (q_1) edge  node {a} (q_2)
        (q_2) edge  node {a} (q_0)
        (q_2) edge[bend left, above]  node {b} (q_3)
        (q_3) edge[bend left, below]  node {b} (q_2);
        \end{tikzpicture}
        
    \end{center}
    
    Дана грамматика
    
    \begin{align*}
        G = \langle \{a,b\}, \{S\},  S, \ & \{ \\
            & \ \ (0)\ S  \to a \ S \ b, \\
            & \ \ (1)\ S  \to a \ b, \\
            \ & \} \rangle 
    \end{align*}
    
    
    Попробуем найти все пути из вершины 2 в вершину 2, выводимые из нетерминала $S$. 
    Проверить наличие такого пути можно используя уже известные нам алгоритмы, однако сами пути пока будем строить ``методом пристального взгляда''.  Найдем один из них. Пусть это будет 
    $$2 \xrightarrow{a} 0 \xrightarrow{a} 1 \xrightarrow{a} 2 \xrightarrow{a} 0 \xrightarrow{a} 1 \xrightarrow{a} 2 \xrightarrow{b} 3 \xrightarrow{b} 2 \xrightarrow{b} 3 \xrightarrow{b} 2 \xrightarrow{b} 3 \xrightarrow{b} 2.
    $$ 
    Построим дерево его вывода.
    
    \begin{center}
    \resizebox{0.3\textwidth}{!}{
    \begin{tikzpicture}[shorten >=1pt,on grid,auto,node distance=1.8cm]
        \node[symbol_node] (s_2_2)   {$(2,S,2)$};
        
        \node[prod_node] (p_0_1) [below =of s_0_6] {$0$}; 
        \node[symbol_node] (s_0_3) [below =of p_0_1]   {$(0,S,3)$}; 
        \node[symbol_node] (a_2_0_1) [left =of s_0_3]   {$(2,a,0)$};
        \node[symbol_node] (b_3_2_1) [right=of s_0_3]   {$(3,b,2)$};
        
        \node[prod_node] (p_0_2) [below =of s_0_3] {$0$}; 
        \node[symbol_node] (s_1_2) [below =of p_0_2]   {$(1,S,2)$}; 
        \node[symbol_node] (a_0_1_1) [left =of s_1_2]   {$(0,a,1)$};
        \node[symbol_node] (b_2_3_1) [right=of s_1_2]   {$(2,b,3)$};
        
        \node[prod_node] (p_0_3) [below =of s_1_2] {$0$}; 
        \node[symbol_node] (s_2_3) [below =of p_0_3]   {$(2,S,3)$}; 
        \node[symbol_node] (a_1_2_1) [left =of s_2_3]   {$(1,a,2)$};
        \node[symbol_node] (b_3_2_2) [right=of s_2_3]   {$(3,b,2)$};
        
        \node[prod_node] (p_0_4) [below =of s_2_3] {$0$}; 
        \node[symbol_node] (s_0_2) [below =of p_0_4]   {$(0,S,2)$}; 
        \node[symbol_node] (a_2_0_2) [left =of s_0_2]   {$(2,a,0)$};
        \node[symbol_node] (b_2_3_2) [right=of s_0_2]   {$(2,b,3)$};
        
        \node[prod_node] (p_0_5) [below =of s_0_2] {$0$}; 
        \node[symbol_node] (s_1_3) [below =of p_0_5]   {$(1,S,3)$}; 
        \node[symbol_node] (a_0_1_2) [left =of s_1_3]   {$(0,a,1)$};
        \node[symbol_node] (b_3_2_3) [right=of s_1_3]   {$(3,b,2)$};
        
        \node[prod_node] (p_1_1) [below =of s_1_3] {$1$};
        \node[symbol_node] (a_1_2_2) [below left =of p_1_1]   {$(1,a,2)$};
        \node[symbol_node] (b_2_3_3) [below right=of p_1_1]   {$(2,b,3)$};
        
        \path[->] 
        (s_2_2) edge (p_0_1)
        
        (p_0_1) edge (s_0_3)
        (p_0_1) edge (a_2_0_1)
        (p_0_1) edge (b_3_2_1)
        
        (s_0_3) edge (p_0_2)
        
        (p_0_2) edge (s_1_2)
        (p_0_2) edge (a_0_1_1)
        (p_0_2) edge (b_2_3_1)
        
        (s_1_2) edge (p_0_3)
        
        (p_0_3) edge (s_2_3)
        (p_0_3) edge (a_1_2_1)
        (p_0_3) edge (b_3_2_2)
        
        (s_2_3) edge (p_0_4)
        
        (p_0_4) edge (s_0_2)
        (p_0_4) edge (a_2_0_2)
        (p_0_4) edge (b_2_3_2)
        
        (s_0_2) edge (p_0_5)
        
        (p_0_5) edge (s_1_3)
        (p_0_5) edge (a_0_1_2)
        (p_0_5) edge (b_3_2_3)
        
        (s_1_3) edge (p_1_1)
        
        (p_1_1) edge (a_1_2_2)
        (p_1_1) edge (b_2_3_3)
        
        ;
    \end{tikzpicture}
    }
    \end{center}
    
    
    Мы построили дерево вывода для одного пути из вершины 2 в неё же. 
    Но можно заметить, что таких путей бесконечно моного: мы можем бесконечное число раз повтроять уже выполненный обход и получать всё более длинные пути. 
    В терминах дерева вывода это будет  означать, что к узлу $_1S_3$ мы добавим сына, соответствующего применению продукции 0, а не 1 для нетерминала $S$. 
    В таком случае мы получим узел $_2S_2$, который уже существует в дереве и таким образом замкнём цикл.
    
    \begin{center}
    \resizebox{0.5\textwidth}{!}{
    \begin{tikzpicture}[shorten >=1pt,on grid,auto,node distance=1.8cm]
        \node[symbol_node] (s_2_2)   {$(2,S,2)$};
        
        \node[prod_node] (p_0_1) [below =of s_0_6] {$0$}; 
        \node[symbol_node] (s_0_3) [below =of p_0_1]   {$(0,S,3)$}; 
        \node[symbol_node] (a_2_0_1) [left =of s_0_3]   {$(2,a,0)$};
        \node[symbol_node] (b_3_2_1) [right=of s_0_3]   {$(3,b,2)$};
        
        \node[prod_node] (p_0_2) [below =of s_0_3] {$0$}; 
        \node[symbol_node] (s_1_2) [below =of p_0_2]   {$(1,S,2)$}; 
        \node[symbol_node] (a_0_1_1) [left =of s_1_2]   {$(0,a,1)$};
        \node[symbol_node] (b_2_3_1) [right=of s_1_2]   {$(2,b,3)$};
        
        \node[prod_node] (p_0_3) [below =of s_1_2] {$0$}; 
        \node[symbol_node] (s_2_3) [below =of p_0_3]   {$(2,S,3)$}; 
        \node[symbol_node] (a_1_2_1) [left =of s_2_3]   {$(1,a,2)$};
        \node[symbol_node] (b_3_2_2) [right=of s_2_3]   {$(3,b,2)$};
        
        \node[prod_node] (p_0_4) [below =of s_2_3] {$0$}; 
        \node[symbol_node] (s_0_2) [below =of p_0_4]   {$(0,S,2)$}; 
        \node[symbol_node] (a_2_0_2) [left =of s_0_2]   {$(2,a,0)$};
        \node[symbol_node] (b_2_3_2) [right=of s_0_2]   {$(2,b,3)$};
        
        \node[prod_node] (p_0_5) [below =of s_0_2] {$0$}; 
        \node[symbol_node] (s_1_3) [below =of p_0_5]   {$(1,S,3)$}; 
        \node[symbol_node] (a_0_1_2) [left =of s_1_3]   {$(0,a,1)$};
        \node[symbol_node] (b_3_2_3) [right=of s_1_3]   {$(3,b,2)$};
        
        \node[state,draw=none] (dummy1) [below left=of s_1_3] {};
        \node[state,draw=none] (dummy2) [below right=of s_1_3] {};
        \node[prod_node] (p_1_1) [left =of dummy1] {$1$};
        \node[symbol_node] (a_1_2_2) [below left =of p_1_1]   {$(1,a,2)$};
        \node[symbol_node] (b_2_3_3) [below right=of p_1_1]   {$(2,b,3)$};
        \node[prod_node] (p_0_6) [right =of dummy2] {$0$};
        \node[symbol_node] (a_1_2_3) [below left =of p_0_6]   {$(1,a,2)$};
        \node[symbol_node] (b_2_3_4) [below right=of p_0_6]   {$(2,b,3)$};
        
        
        \path[->] 
        (s_2_2) edge (p_0_1)
        
        (p_0_1) edge (s_0_3)
        (p_0_1) edge (a_2_0_1)
        (p_0_1) edge (b_3_2_1)
        
        (s_0_3) edge (p_0_2)
        
        (p_0_2) edge (s_1_2)
        (p_0_2) edge (a_0_1_1)
        (p_0_2) edge (b_2_3_1)
        
        (s_1_2) edge (p_0_3)
        
        (p_0_3) edge (s_2_3)
        (p_0_3) edge (a_1_2_1)
        (p_0_3) edge (b_3_2_2)
        
        (s_2_3) edge (p_0_4)
        
        (p_0_4) edge (s_0_2)
        (p_0_4) edge (a_2_0_2)
        (p_0_4) edge (b_2_3_2)
        
        (s_0_2) edge (p_0_5)
        
        (p_0_5) edge (s_1_3)
        (p_0_5) edge (a_0_1_2)
        (p_0_5) edge (b_3_2_3)
        
        (s_1_3) edge (p_1_1)
        
        (p_1_1) edge (a_1_2_2)
        (p_1_1) edge (b_2_3_3)
        
        (s_1_3) edge (p_0_6)
        
        (p_0_6) edge (a_1_2_3)
        (p_0_6) edge (b_2_3_4)
        (p_0_6) edge [bend right] (s_2_2)
        
        ;
    \end{tikzpicture}
    } 
    \end{center}
    
    Таким образом мы построили SPPF. Обойдя эту структуру необходимое количество раз, мы можем получить любой путь, удовлетворяющий условию. Более того, в полученном графе можно получать любые другие пути по соответствующим нетерминалам и парам вершин, содержащимся в узлах леса.
    \end{example}

    \begin{note}
    SPPF построенный для данной контекстно-свободной грамматки $G$ и графа $\mathcal{G}$ 
    \begin{enumerate}
      \item содержит терминальный узел вида $(i,t_k,j)$ тогда и только тогда, когда в графе $\mathcal{G}$ есть ребро $(i,t_k,j)$;
      \item содержит нетерминальный узел вида $(i,S_k,j)$ тогда и только тогда, когда в графе $\mathcal{G}$ есть путь из вершины $i$ в вершину $j$, выводимый из нетерминала $S_k$ в грамматике $G$.
    \end{enumerate}
    \end{note}
    
    Осталось увидеть, что SPPF является представлением контекстно-свободной грамматики, описывающей результат пересечения исходных графа и грамматики. Для этого просто построим грамматику $G_{\textit{SPPF}} = \langle \Sigma_{\textit{SPPF}}, N_{\textit{SPPF}}, S_{\textit{SPPF}}, P_{\textit{SPPF}}\rangle$ по SPPF следующим образом:
    \begin{itemize}
      \item $\Sigma_{\textit{SPPF}}$ --- все листья SPPF;
      \item $N_{\textit{SPPF}}$ --- все нетерминальные узлы SPPF;
      \item $S_{\textit{SPPF}}$ --- нетерминал, соответствующий пути, который нас будет интересовать;
      \item $P_{\textit{SPPF}}$ --- для каждого дополнительного узла (с номером продукции) добавляем продукцию, левая часть которой --- непосредственный предок этого узла, а правая чать --- непосредственные потомки. 
    \end{itemize} 

    \begin{example}
    Построим грамматику для полученного SPPF:
    \begin{align*}
        (0)\ _2S_2  & \to\ _2a_0\ _0S_3\ _3b_2   &(4)\ _0S_2 \to\ &_0a_1\ _1S_3\ _3b_2 \\ 
        (1)\ _0S_3  & \to\ _0a_1\ _1S_2\ _2b_3   &(5)\ _1S_3 \to\ &_1a_2\ _2S_2\ _2b_3 \\ 
        (2)\ _1S_2  & \to\ _1a_2\ _2S_3\ _3b_2   &(6)\ _1S_3 \to\ &_1a_2\ _2b_3 \\ 
        (3)\ _2S_3  & \to\ _2a_0\ _0S_2\ _2b_3   &
    \end{align*}
    Видим, что для одного единственного нетерминала $_1S_3$ существует 2 правила, одно из которых рекурсивное. Попробуем получить левосторонний вывод какой-нибудь цепочки в этой грамматике: 
    \begin{align*}
        & \boldsymbol{_2S_2} \xRightarrow{(0)\ } \\
        & {_2a_0}\ \boldsymbol{_0S_3}\ {_3b_2} \xRightarrow{(1)\ } \\
        & {_2a_0}\ {_0a_1}\ \boldsymbol{_1S_2}\ {_2b_3}\ {_3b_2} \xRightarrow{(2)\ } \\
        & {_2a_0}\ {_0a_1}\ {_1a_2}\ \boldsymbol{_2S_3}\ {_3b_2}\ {_2b_3}\ {_3b_2} \xRightarrow{(3)\ } \\
        & {_2a_0}\ {_0a_1}\ {_1a_2}\ {_2a_0}\ \boldsymbol{_0S_2}\ {_2b_3}\ {_3b_2}\ {_2b_3}\ {_3b_2} \xRightarrow{(4)\ } \\
        & {_2a_0}\ {_0a_1}\ {_1a_2}\ {_2a_0}\ {_0a_1}\ \boldsymbol{_1S_3}\ {_3b_2}\ {_2b_3}\ {_3b_2}\ {_2b_3}\ {_3b_2} \xRightarrow{(5)\ } \\
        & {_2a_0}\ {_0a_1}\ {_1a_2}\ {_2a_0}\ {_0a_1}\ {_1a_2}\ \boldsymbol{_2S_2}\ {_2b_3}\ {_3b_2}\ {_2b_3}\ {_3b_2}\ {_2b_3}\ {_3b_2} \xRightarrow{(0)\ } \\
        & {_2a_0}\ {_0a_1}\ {_1a_2}\ {_2a_0}\ {_0a_1}\ {_1a_2}\ {_2a_0}\ \boldsymbol{_0S_3}\ {_3b_2}\ {_2b_3}\ {_3b_2}\ {_2b_3}\ {_3b_2}\ {_2b_3}\ {_3b_2} \xRightarrow{(1)\ } \\
        & {_2a_0}\ {_0a_1}\ {_1a_2}\ {_2a_0}\ {_0a_1}\ {_1a_2}\ {_2a_0}\ {_0a_1}\ \boldsymbol{_1S_2}\ {_2b_3}\ {_3b_2}\ {_2b_3}\ {_3b_2}\ {_2b_3}\ {_3b_2}\ {_2b_3}\ {_3b_2} \xRightarrow{(2)\ } \\
        & {_2a_0}\ {_0a_1}\ {_1a_2}\ {_2a_0}\ {_0a_1}\ {_1a_2}\ {_2a_0}\ {_0a_1}\ {_1a_2}\ \boldsymbol{_2S_3}\ {_3b_2}\ {_2b_3}\ {_3b_2}\ {_2b_3}\ {_3b_2}\ {_2b_3}\ {_3b_2}\ {_2b_3}\ {_3b_2} \xRightarrow{(3)\ } \\
        & {_2a_0}\ {_0a_1}\ {_1a_2}\ {_2a_0}\ {_0a_1}\ {_1a_2}\ {_2a_0}\ {_0a_1}\ {_1a_2}\ {_2a_0}\ \boldsymbol{_0S_2}\ {_2b_3}\ {_3b_2}\ {_2b_3}\ {_3b_2}\ {_2b_3}\ {_3b_2}\ {_2b_3}\ {_3b_2}\ {_2b_3}\ {_3b_2} \xRightarrow{(4)\ } \\
        & {_2a_0}\ {_0a_1}\ {_1a_2}\ {_2a_0}\ {_0a_1}\ {_1a_2}\ {_2a_0}\ {_0a_1}\ {_1a_2}\ {_2a_0}\ {_0a_1}\ \boldsymbol{_1S_3}\ {_3b_2}\ {_2b_3}\ {_3b_2}\ {_2b_3}\ {_3b_2}\ {_2b_3}\ {_3b_2}\ {_2b_3}\ {_3b_2}\ {_2b_3}\ {_3b_2} \xRightarrow{(6)\ } \\
        & {_2a_0}\ {_0a_1}\ {_1a_2}\ {_2a_0}\ {_0a_1}\ {_1a_2}\ {_2a_0}\ {_0a_1}\ {_1a_2}\ {_2a_0}\ {_0a_1}\ {_1a_2}\ {_2b_3}\ {_3b_2}\ {_2b_3}\ {_3b_2}\ {_2b_3}\ {_3b_2}\ {_2b_3}\ {_3b_2}\ {_2b_3}\ {_3b_2}\ {_2b_3}\ {_3b_2}
    \end{align*}
     
    Мы получили цепочку, которая действительно является путем из вершины 2 в вершину 2 в заданном графе. Таким образом выводятся и любые другие соответствующие пути.
    
\end{example}


%\section{Вопросы и задачи}
%\begin{enumerate}
%  \item Постройте дерево вывода цепочки $w=aababb$ в грамматике $G=\langle\{a,b\},\{S\},\{S\rightarrow \varepsilon \ | \ a \ S \ b \ S \}, S \rangle$.
%  \item Постройте все левосторонние выводы цепочки $w=ababab$ в грамматике $G=\langle\{a,b\},\{S\},\{S\rightarrow \varepsilon \ | \ a \ S \ b \ | S \ S\}, S \rangle$.
%  \item Постройте все правосторонние выводы цепочки $w=ababab$ в грамматике $G=\langle\{a,b\},\{S\},\{S\rightarrow \varepsilon \ | \ a \ S \ b \ | S \ S\}, S \rangle$.
%  \item \label{t1}Постройте все деревья вывода цепочки $w=ababab$ в грамматике $G=\langle\{a,b\},\{S\},\{S\rightarrow \varepsilon \ | \ a \ S \ b \ | S \ S\}, S \rangle$, соответствующие левосторонним выводам.
%  \item \label{t2}Постройте все деревья вывода цепочки $w=ababab$ в грамматике $G=\langle\{a,b\},\{S\},\{S\rightarrow \varepsilon \ | \ a \ S \ b \ | S \ S\}, S \rangle$, соответствующие правосторонним выводам.
%  \item Как связаны между собой леса, полученные в предыдущих двух задачах (\ref{t1} и \ref{t2})? Какие выводы можно сделать из такой связи?
%  \item Постройте сжатое представление леса разбора, полученного в задаче~\ref{t1}.
%  \item Постройте сжатое представление леса разбора, полученного в задаче~\ref{t2}.
%  \item \label{t3}Предъявите контекстно-свободную граммтику существенно неоднозначного языка. 
%        Возьмите цепочку длины болше пяти, при надлежащую этому языку, и постройте все деревья вывода этой цепочки в предъявленной граммтике. 
%  \item Постройте сжатое представление леса, полученного в задаче~\ref{t3}.
%\end{enumerate}

\chapter{Алгоритм на основе нисходящего анализа}

В данном разделе мы рассмотрим семейство алгоритмов нисходящего синтаксического (рекурсивный спуск, LL, GLL~\cite{Scott:2010:GP:1860132.1860320,10.1007/978-3-662-46663-6_5}) рассмотрим их обощение для задачи поиска путей с контекстно-свободными ограничениями. 

% и GLL~\cite{Grigorev:2017:CPQ:3166094.3166104} Другие реализации~\cite{MEDEIROS201975}

\section{Рекурсивный спуск}

Идея рекурсивного спуска основана на использовании программного стека вызовов в качестве стека магазинного автомата. Достигается это следующим образом.
\begin{itemize}
  \item Для каждого нетерминала создаётся функция, принимающая ещё не обработанный остаток строки и возвращающая пару: результат вывода префикса данной строки из соответствующего нетерминала и не обработанный остаток строки. В случае распзнователя результат вывода --- логическое значение (выводится/не выводится).
  \item Каждая такая функция реализовывает обработку цепочки согласно правым частям правил для соответствующих нетерминалов: считывание символа входа при обработке терминального символа, вызов соответствующей функции при обработке нетерминального.
\end{itemize}

У твкого подхода есть два ограничения:
\begin{enumerate}
  \item Не работает с леворекурсивными грамматиками, грамматиками, в которых вывод может принимать следующий вид: 
  $$
  S \to \cdots \to \underline{N_i} \alpha \to \cdots \underline{N_i} \beta \to \cdots \omega
  $$
  \item Шаги должны быть однозначными.
\end{enumerate}

\begin{example}

Постороим функцию рекурсивного спуска для продукции $S \rightarrow aSbS \mid \varepsilon$.

\begin{algorithm}
  \floatname{algorithm}{Listing}
\begin{algorithmic}[1]
\caption{Функция рекурсивного спуска}
\Function{S}{$\omega$}    
    \If {(len$(\omega)=0$)}
    \Comment{{\footnotesize Пустая цепочка выводима из $S$}}
    \State{\Return \textit{(true, $\omega$)}}
    \EndIf 
       
    \If{$(\omega = a :: tl)$}
        \Comment{{\footnotesize Выводимая из $S$ подстрока должна начинаться с $a$}}
        \State{$res,tl' = $ S($tl$)}
        \Comment{{\footnotesize Затем должна идти подстрока, выводимая из $S$}}
        \If{res \&\&  $tl' = b :: tl''$}
           \Comment{{\footnotesize Если вызов закончился успешно, то надо проверить, что следующий символ --- это $b$}}
           \State{\Return $S(tl'')$}
           \Comment{{\footnotesize И снова попробовать вывести перфикс из $S$}}
         \Else
           \State{\Return \textit{(false, $tl'$)}}
        \EndIf
    \Else
        \State{\Return \textit{(false, $\omega$)}}
    \EndIf           
\EndFunction

\end{algorithmic}
\end{algorithm}
\end{example}

Если возвращаеммое значение этой функции  --- пара вида \textit{(true, [])}, то разбор завершился успехом.

Данный подход применяется как для ручного написания синтаксических нализаторов, так и при генерации анализаторов по грамматике. 

\section{LL(k)-алгоритм синтаксического анализа}

LL(k) --- алгоритм синтаксического анализа --- нисходящий анализ без отката, но с предпросмотром. 
Решение о том, какую продукцию применять, принимается на основании k следующих за текущим символом. 
Временная сложность алгоритма $O(n)$, где $n$~--- длина слова. 

Алгоритм использует входной буфер, стек для хранения промежуточных данных и таблицу анализатора, которая управляет процессом разбора. 
В ячейке таблицы указано правило, которое нужно применять, если рассматривается нетерминал $A$, а следующие $m$ символов строки~--- $t_{1} \dots t_{m}$, где $m \leq k$. 
Также в таблице выделена отдельная колонка для $\$$~--- маркера конца строки. 

\begin{center}
  \begin{tabular}{ c || c | c | c | c }
             & $\dots$ & $t_{1} \dots t_{m}$ & $\dots$ & $\$$ \\ \hline  
    $\dots$  & $\dots$ & $\dots$ & $\dots$ & $\dots$ \\ \hline  
    $A$  & $\dots$ & $A \to \alpha$ & $\dots$ & $\dots$ \\ \hline  
    $\dots$  & $\dots$ & $\dots$ & $\dots$ & $\dots$ 
  \end{tabular}  
\end{center}

Для построения таблицы вычисляются множества $\first[k]$ и $\follow[k]$. Идейно их можно понимать, как первые или, соответственно, последующие $k$ символов в результирующем выводе, при использовании нетерминала $A$. Данную мысль хорошо иллюстрирует рисунок:

\begin{center}
    \begin{tikzpicture}
        \draw[black, thick] (0,0) -- (2,4);
        \draw[black, thick] (2,4) -- (4,0);
        \draw[black, thick] (1,0) -- (2,2);
        \draw[black, thick] (2,2) -- (3,0);
        \draw[black, thick] (2,2) -- (3,0);
        \draw[black, thick] (0,0) -- (1,0);
        \draw[red, ultra thick] (1,0) -- (1.75,0);
        \draw[black, thick] (1.75,0) -- (3,0);
        \draw[red, ultra thick] (3,0) -- (3.75,0);
        \draw[black, thick] (3.75,0) -- (4,0);
        \filldraw[black] (2,4) circle (1pt) node[anchor=west] {S};
        \filldraw[black] (2,2) circle (1pt) node[anchor=west] {A};
        \filldraw[black] (0,0) circle (0pt)
        node[anchor=east] {\textbf{$\omega$}};
        \filldraw[red] (1.5,0) circle (0pt)
        node[anchor=north] {\footnotesize $\first[k](A)$};
        \filldraw[red] (3.75,0) circle (0pt)
        node[anchor=north] {\footnotesize $\follow[k](A)$};
    \end{tikzpicture}
\end{center}

Определим их формально:

\begin{definition}
  Пусть $G = \langle N, \Sigma, P, S \rangle$~--- КС-грамматика. Множество $\first[k]$ определено для сентециальной формы $\alpha$ следующим образом:
  \[ \first[k](\alpha) = \{ \omega \in \Sigma^* \mid \alpha \derives{} \omega \text{ и } |\omega| < k \text{ либо } \exists \beta: \alpha \derives{} \omega \beta \text{ и } |\omega| = k \}
  \]
  , где $\alpha, \beta \in (N \cup \Sigma)^*.$ 
\end{definition}

\begin{definition}
  Пусть $G = \langle N, \Sigma, P, S \rangle$~--- КС-грамматика. Множество $\follow[k]$ определено для сентециальной формы $\beta$ следующим образом:
  \[\follow[k](\beta) = \{ \omega \in \Sigma^* \mid \exists \gamma, \alpha: S \derives{} \gamma \beta \alpha \text{ и } \omega \in \first[k](\alpha) \} \]
\end{definition}

В частном случае для $k = 1$:

\[ \first(\alpha) = \{ a \in \Sigma \mid \exists \gamma \in (N \cup \Sigma)^*: \alpha \derives{} a \gamma \} \text{, где } \alpha \in (N \cup \Sigma)^* \]

\[ \follow(\beta) = \{ a \in \Sigma \mid \exists \gamma, \alpha \in (N \cup \Sigma)^* : S \derives{} \gamma \beta a \alpha \} \text{, где } \beta \in (N \cup \Sigma)^*  \]

Множество $\first$ можно вычислить, пользуясь следующими соотношениями:

\begin{itemize}
  \item $\first(a \alpha) = \{a\}, a \in \Sigma, \alpha \in (N \cup \Sigma)^* $
  \item $\first(\varepsilon) = \{\varepsilon\}$
  \item $\first(\alpha \beta) = \first(\alpha) \cup (\first(\beta) \text{, если } \varepsilon \in \first(\alpha))$
  \item $\first(A) = \first(\alpha) \cup \first(\beta) \text{, если в грамматике есть правило } A \to \alpha \mid\beta$
\end{itemize}

Алгоритм для вычисления множества $\follow$:

\begin{itemize}
  \item Положим $\follow(X) = \varnothing, \forall X \in N$
  \item $\follow(S) = \follow(S) \cup \{\$\} \text{, где } S \text{--- стартовый нетерминал}$
  \item Для всех правил вида $A \to \alpha X \beta: \follow(X) = \follow(X) \cup (\first(\beta) \setminus \{\varepsilon\} )$
  \item Для всех правил вида $A \to \alpha X \text{ и } A \to \alpha X \beta \text{, где } \varepsilon \in \first(\beta): \follow(X) = \follow(X) \cup \follow(A)$
  \item Последние два пункта применяются пока есть что добавлять в строящиеся множества.
\end{itemize}


\begin{example}

Рассмотрим грамматику $G$ со следующими продукциями:
\begin{align*}
  S  &\to a S' & A' \to b \mid a \\
  S' &\to A b B S' \mid \varepsilon &  B  \to c \mid \varepsilon\\
  A  &\to a A' \mid \varepsilon 
\end{align*}


Пример множеств $\first$ для нетерминалов грамматики $G$:

\begin{multicols}{2}

\columnbreak

\begin{align*}
  \first(S)  &= \{ a \}  & \first(B)  &= \{ c, \varepsilon \} \\
  \first(A)  &= \{ a, \varepsilon \} & \first(S') &= \{ a, b, \varepsilon \}\\
  \first(A') &= \{ a, b \}   
\end{align*}
\end{multicols}

Пример множеств $\follow$ для нетерминалов грамматики $G$:

\begin{align*}
  \follow(S)  &= \{ \$ \} & \\
  \follow(S') &= \{ \$ \} &(S \to a S')\\
  \follow(A)  &= \{ b \}  &(S' \to A b B S') \\
  \follow(A') &= \{ b \}  &(A \to a A')\\
  \follow(B)  &= \{ a, b, \$ \} &(S' \to A b B S', \varepsilon \in \first(S'))
\end{align*}

\end{example}

Управляющая таблица LL(k) анализатора заполняется следующим образом: продукции $A \to \alpha, \alpha \neq \varepsilon$ помещаются в ячейки $(A, a)$, где $a \in \first(\alpha)$, продукции $A \to \alpha$~--- в ячейки $(A, a)$, где $a \in \follow(A)$, если $\varepsilon \in \first(\alpha)$

\begin{example}

Пример таблицы для грамматики $S \to aSbS \mid \varepsilon$

\begin{center}
\begin{tabular}{ r || c | c || c | c | c }
N & $\first$ & $\follow$ & a & b & $\$ $ \\ \hline  
$S$ & $\{ a, \varepsilon \}$ & $\{ b, \$ \}$ & $S \rightarrow aSbS$ & $S \rightarrow \varepsilon$ & $S \rightarrow \varepsilon$ 
\end{tabular}  
\end{center}

\end{example}

Однако, не для всех грамматик по множествам $\first[k]$ и $\follow[k]$ возможно выбрать применяемую продукцию, а значит, нельзя однозначно построить таблицу, необходимую для работы алгоритма, поэтому данный алгоритм применим только для грамматик особого класса --- LL(k).

\begin{definition}
  LL(k) грамматика --- грамматика, для которой на основании множеств $\first[k]$ и $\follow[k]$ можно однозначно определить, какую продукцию применять.
\end{definition}

Важно заметить, что при больших $k$ управляющая таблица сильно разрастается, поэтому на практике данный алгоритм применим для небольших значений $k$.

Интерпретатор автомата принимает входную строку и построенную управляющую таблицу и работает следующим образом. 
В каждый момент времени конфигурация автомата это позиция во входной строке и стек. 
В начальный момент времени стэк пуст, а позиция во входной строке соответствует её началу.
На певом шаге в стек добавляются последовательно сперва симаол концы строки, затем стартовый нетерминал.
На каждом шаге анализируется существующая конфигурация и совершается одно из действий.
\begin{itemize}
\item Если текущая позиция --- конец строки и вершина стека --- символ конца строки, то успешно завершаем разбор.
\item Если текушая вершина стека --- терминал, то проверяем, что позиция в строке соответствует этому терминалу. Если да, то снимаем элемент со стека, сдвигаем позицию на единицу и продолжаем разбор. Иначе завершаем разбор с ошибкой.
\item Если текущая врешина стека --- нетерминал $N_i$ и текущий входной символ $t_j$, то ищем в управляющей таблице ячейку с координатами $(N_i, t_j)$ и записываем на стек содержимое этой ячейки.
\end{itemize}

\begin{example}Пример работы LL анализатора.
Рассмотрим грамматику $S \to aSbS \mid \varepsilon$ и выводимое слово $\omega = abab$.

Расмотрим пошагово работу алгоритма, будем использовать таблицу, построенную в предыдущем примере:

\begin{enumerate}
  \item Начало работы.
  
  Стек: \,
    \begin{tabular}[c]{ |c| } 
        \\ \hline
        \$ \\ \hline
    \end{tabular}  
    \qquad  \qquad \qquad  \qquad входное слово: \,
    \begin{tabular}[c]{ |c|c|c|c|c| } 
        \hline
        \textcolor{red}{a} & b & a & b & \$ \\ \hline
    \end{tabular}
    
Финальный символ лежит на стеке, а указатель указывает на первый символ слова.

  \item кладем стартовый символ на стек

    Стек: \,
    \begin{tabular}[c]{ |c| } 
        \\ \hline
        $S$ \\ \hline
        \$ \\ \hline
    \end{tabular}  
    \qquad  \qquad \qquad  \qquad входное слово: \,
    \begin{tabular}[c]{ |c|c|c|c|c| } 
        \hline
        \textcolor{red}{a} & b & a & b & \$ \\ \hline
    \end{tabular}
    
  \item Ищем ячейку с координатами (S, a), применяем продукцию из ячейки.

    Стек: \,
    \begin{tabular}[c]{ |c| } 
        \\ \hline
        $a$ \\ \hline
        $S$ \\ \hline
        $b$ \\ \hline
        $S$ \\ \hline
        \$ \\ \hline
    \end{tabular}  
    \qquad  \qquad \qquad  \qquad входное слово: \,
    \begin{tabular}[c]{ |c|c|c|c|c| } 
        \hline
        \textcolor{red}{a} & b & a & b & \$ \\ \hline
    \end{tabular}

\item Снимаем терминал $a$ со стека и двигаем указатель.
    
    Стек: \,
    \begin{tabular}[c]{ |c| } 
        \\ \hline
        $S$ \\ \hline
        $b$ \\ \hline
        $S$ \\ \hline
        \$ \\ \hline
    \end{tabular}  
    \qquad  \qquad \qquad  \qquad входное слово: \,
    \begin{tabular}[c]{ |c|c|c|c|c| } 
        \hline
        a & \textcolor{red}{b} & a & b & \$ \\ \hline
    \end{tabular}

\item Ищем ячейку с координатами (S, b), применяем продукцию из ячейки.

    Стек: \,
    \begin{tabular}[c]{ |c| } 
        \\ \hline
        $b$ \\ \hline
        $S$ \\ \hline
        \$ \\ \hline
    \end{tabular}  
    \qquad  \qquad \qquad  \qquad входное слово: \,
    \begin{tabular}[c]{ |c|c|c|c|c| } 
        \hline
        a & \textcolor{red}{b} & a & b & \$ \\ \hline
    \end{tabular}

\item Снимаем терминал $b$ со стека и двигаем указатель.

    Стек: \,
    \begin{tabular}[c]{ |c| } 
        \\ \hline
        $S$ \\ \hline
        \$ \\ \hline
    \end{tabular}  
    \qquad  \qquad \qquad  \qquad входное слово: \,
    \begin{tabular}[c]{ |c|c|c|c|c| } 
        \hline
        a & b & \textcolor{red}{a} & b & \$ \\ \hline
    \end{tabular}
  
  \item Ищем ячейку с координатами (S, a), применяем продукцию из ячейки.

    Стек: \,
    \begin{tabular}[c]{ |c| } 
        \\ \hline
        $a$ \\ \hline
        $S$ \\ \hline
        $b$ \\ \hline
        $S$ \\ \hline
        \$ \\ \hline
    \end{tabular}  
    \qquad  \qquad \qquad  \qquad входное слово: \,
    \begin{tabular}[c]{ |c|c|c|c|c| } 
        \hline
        a & b & \textcolor{red}{a} & b & \$ \\ \hline
    \end{tabular}

\item Снимаем терминал $a$ со стека и двигаем указатель.
    
    Стек: \,
    \begin{tabular}[c]{ |c| } 
        \\ \hline
        $S$ \\ \hline
        $b$ \\ \hline
        $S$ \\ \hline
        \$ \\ \hline
    \end{tabular}  
    \qquad  \qquad \qquad  \qquad входное слово: \,
    \begin{tabular}[c]{ |c|c|c|c|c| } 
        \hline
        a & b & a & \textcolor{red}{b} & \$ \\ \hline
    \end{tabular}

\item Ищем ячейку с координатами (S, b), применяем продукцию из ячейки.

    Стек: \,
    \begin{tabular}[c]{ |c| } 
        \\ \hline
        $b$ \\ \hline
        $S$ \\ \hline
        \$ \\ \hline
    \end{tabular}  
    \qquad  \qquad \qquad  \qquad входное слово: \,
    \begin{tabular}[c]{ |c|c|c|c|c| } 
        \hline
        a & b & a & \textcolor{red}{b} & \$ \\ \hline
    \end{tabular}

\item Снимаем терминал $b$ со стека и двигаем указатель.

    Стек: \,
    \begin{tabular}[c]{ |c| } 
        \\ \hline
        $S$ \\ \hline
        \$ \\ \hline
    \end{tabular}  
    \qquad  \qquad \qquad  \qquad входное слово: \,
    \begin{tabular}[c]{ |c|c|c|c|c| } 
        \hline
        a & b & a & b & \textcolor{red}{\$} \\ \hline
    \end{tabular}

\item Ищем ячейку с координатами (S, \$), применяем продукцию из ячейки.

    Стек: \,
    \begin{tabular}[c]{ |c| } 
        \\ \hline
        \$ \\ \hline
    \end{tabular}  
    \qquad  \qquad \qquad  \qquad входное слово: \,
    \begin{tabular}[c]{ |c|c|c|c|c| } 
        \hline
        a & b & a & b & \textcolor{red}{\$} \\ \hline
    \end{tabular}    
 
\item Оказались в конце строки и на вершине стека символ конца --- завершаем разбор.

\end{enumerate}

\end{example}

Можно расширить данный алгоритм так, чтобы он строил дерево вывода. Дерево будет строиться сверху вниз, от корня к листьям. Для этого необходимо расширить шаги алгоритма.
\begin{itemize}
  \item В ситуации, когда мы читаем очередной нетерминал (на вершине стека и во входе одинаковые терминалы), мы создаём лист с соответствующим терминалом.
  \item В ситуации, когда мы заменяем нетерминал на стеке на правую часть продукции в соответствии с управляющей таблицей, мы создаём нетерминальный узел соответствующий применяемой продукции.
\end{itemize} 

Данное семейство всё так же не работает с леворекурсивными грамматиками и с неоднозначными грамматиками.

Таким образом, по некоторым граммтикам можно построить LL(k) анализатор (назовём их LL(k) граммтиками), но не по всем.
С левой рекурсией, конечно, можно бороться, так как существуют алгоритмы устранения левой и скрытой левой рекурсии, а вот с неодносзначностями ничего не поделаешь.



\section{Алгоритм Generalized LL}

Можно построить анализатор, работающий с произвольными КС-грамматиками.
Generalized LL (GLL)~\cite{Scott:2010:GP:1860132.1860320,10.1007/978-3-662-46663-6_5}

Принцип работы остается абсолютно таким же как и для табличного LL: 
\begin{itemize}
  \item Сначала по грамматике строится \textit{управляющая} таблица
  \item Затем построенная таблица команд и непосредственно анализируемое слово поступают на вход абстрактному интерпретатору.
  \item Для своей работы интерпретатор поддерживает некоторую вспомогательную структуру данных (стек для LL).
  \item Один шаг разбора состоит в том, чтобы рассмотреть текущую позицию в слове, применить все соответствующие ей правила из таблицы и при возможности сдвинуть позицию разбора вправо.
\end{itemize}

Где в этой схеме возникают ограничения на вид обрабатываемой грамматики для алгоритма LL? На самом первом шаге --- при построении таблицы может возникнуть ситуация, когда одному нетерминалу $N_j$ и последовательности $first_k(N_j)$ соответствует несколько продукций грамматики. В этом случае грамматика признавалась не соответствующей классу LL(k) и отвергалась анализатором.

Теперь же мы разрешим такую ситуацию и в этом случае в ячейку таблицы будем записывать все продукции грамматики, соответствующие этой ячейке. Однако сразу же возникает вопрос --- а что делать интерпретатору, когда при разборе ему необходимо применить правило, состоящее из нескольких продукций? Общий ответ такой --- необходим некоторый вид недетерминизма, при котором интерпретатор мог бы ``параллельно'' обрабатывать несколько возможных вариантов синтаксического разбора.

Эти два свойства (модифицированная управляющая таблица и недетерминизм) суть главные принципиальные отличия GLL(k) от LL(k). Далее мы перейдем к рассмотрению непосредственно технической реализации описанного алгоритма.

Нам необходимо научиться задавать различные ветви (пути) синтаксического разбора и переключаться между ними. Заметим, что состояние любой ветви в любой момент времени суть следующее: необходимо распознать символ $N_j \in N \cup \Sigma$ из продукции $X$, начиная с элемента слова под индексом $i$. Т.е. имеем позицию в слове и позицию символа в продукции. Последнее принято называть \textit{слотом грамматики}. 

\begin{definition}
  Пусть $G = \langle N, \Sigma, P, S \rangle$~--- КС-грамматика. \textit{Слотом грамматики} $G$ (позицией грамматики $G$) назовем пару из продукции $X \in P$ и позиции $0 \leq q \leq length(body(X))$ тела продукции $X$. При этом введем следующее обозначение $X ::= \alpha \cdot \beta, \quad \alpha,\beta \in (N \cup \Sigma)^*$, где $ \cdot $ указывает на позицию в продукции.
\end{definition}

Описанная пара позиций уже однозначно задает состояние синтаксического разбора. Имеем множество состояний и переходов между ними --- возникает естественное желание воспользоваться терминами графов для представления этой структуры. Такую конструкцию называют \textit{граф-структурированный стек} или \textit{GSS} (Graph Structured Stack), который впервые был предложен Масару Томитой ~\cite{tomita1988graph} в контексте восходящего анализа. GSS будет являться рабочей структурой нашего нового интерпретатора вместо стека для LL. Состояние разбора вместе с узлом GSS мы будет называть \textit{дескриптором}.

\begin{definition} 
  Пусть $G = \langle N, \Sigma, P, S \rangle$~--- КС-грамматика, $X$ слот грамматики $G$, $i$ позиция в слове $ w $ над алфавитом $\Sigma$, а $ u $ узел GSS. \textit{Дескриптором} назовём тройку $ (X, u, i) $.
\end{definition}

Есть несколько способов задания GSS для алгоритма GLL. Вариант, предложенный самими авторами алгоритма, оперирует непосредственно парами из позиции слова и слота грамматики в качестве состояний (и узлов графа) --- такой метод является довольно простым и наглядным, но, как описано в работе \cite{10.1007/978-3-662-46663-6_5}, не самым эффективным. Предложим сразу чуть более оптимальное представление: заметим, что шаги разбора, соответствующие одному и тому же нетерминалу и позиции слова, должны выдавать один и тот же результат независимо от конкретной продукции грамматики, в которой стоит этот нетерминал. Поэтому заводить по узлу на каждый слот грамматики довольно избыточно --- вместо этого в качестве состояния будет использовать пары из нетерминала и позиции слова, а позиции грамматики будем записывать на рёбрах.  

Итак, мы научились задавать состояния с помощью дескрипторов, а также определились со вспомогательной структурой GSS. Теперь можно перейти к рассмотрению непосредственно самого алгоритма, суть которого довольно проста и напоминает BFS по неявному графу.

Дескриптор задает состояние, которое необходимо обработать. При этом мы без какой-либо дополнительной информации можем продолжить анализ входа из состояния, задаваемого этим дескриптором. В процессе обработки мы можем получить несколько новых состояний. Поэтому будем поддерживать множество $ R $ дескрипторов на обработку --- на каждом шаге извлекаем один из множества, проводим анализ и кладем в множество новые полученные. 

При каких условиях этот процесс будет конечен? Ну, например, если мы каждое состояние будем обрабатывать не более одного раза. И действительно, поскольку наш интерпретатор является ``чистым'' в том смысле, что для оного и того же состояния каждый раз будут получены одинаковые результаты, проводить анализ дважды не имеет смысла. Поэтому будем также поддерживать множество $ U $ всех полученных в ходе разбора дескрипторов, и добавлять в $ R $ только те, которых еще нет в $ U $.

И наконец, заключительная и самая главная часть --- как происходит обработка дескриптора?
Пусть дескриптор имеет вид $ (X, u, i) $, а входное слово обозначим $ W $. Есть три возможных варианта, в зависимости от вида позиции грамматики $ X $ --- разберем каждый из них по отдельности; 

\begin{itemize}
  \item $ X ::= \alpha \cdot t \beta $, т.е. указатель смотрит на терминал --- в этом случае новых дескрипторов добавлено не будет. Если $ W[i] = t $, то мы сдвигаем указатель слота, переходя к рассмотрению $ X ::= \alpha t \cdot \beta $, и инкрементируем позицию $ i $ в слове. В противном же случае сразу переходим к следующему дескриптору, т.о. терминируя текущую ветвь разбора.
  
  \item $ X ::= \alpha \cdot A \beta $, т.е. указатель смотрит на нетерминал. Нам нужен GSS узел $ v $ вида $ (A, i) $ и ребро $ (u, X ::= \alpha A \cdot \beta, v) $ (ребро из $ u $ в $ v $ с пометкой $ X ::= \alpha A \cdot \beta $). Если такой узел и ребро уже существуют в нашем GSS, берем их, иначе --- создаём. Далее в $ R $ добавляем по дескриптору для узла $ v $ и каждого правила грамматики из ячейки управляющей таблицы для нетерминала $ A $ (конечно, если их еще не было в $ U $). На этом обработка текущего дескриптора завершается.
  
  \item $ X ::= \alpha \cdot $, т.е. указатель находится в конце продукции. Продукция разобрана, а значит, интерпретатору необходимо вернуться из разбора $ X $ к вызывающему правилу и продолжить разбор там (это, в некотором смысле, соответствует возврату из функции разбора нетерминала в методе рекурсивного спуска). По каждому исходящему ребру $ (u, Y, v) $ добавляем (если уже не существует) дескриптор $(Y, v, i)$. 
\end{itemize}

Результатом синтаксического разбора является успех тогда и только тогда, когда был достигнут дескриптор вида $ (S ::= \alpha \cdot, s, n) $, где слот грамматики представляет собой любое правило для аксиомы $ S $, узел GSS $ s $ состоит из аксиомы $ S $ и 0, а позиция входного слова равна его длине $ n $. Если же после разбора всех полученных дескрипторов указанный найден не был, результатом будет являться провал.

Давайте посмотрим, как такой алгоритм справится с неоднозачной грамматикой с леворекурсивным правилом. 

\begin{example}
  \label{gll:example1}
  Пусть грамматика $ G $ имеет вид $ S \to SSS \mid SS \mid a $, а разбираемое слово $ w = aaa $. Тогда GSS, соответствующий разбору $S \Rightarrow SSS \Rightarrow aSS \Rightarrow aaS \Rightarrow aaa$, будет выглядеть следующим образом (для удобства каждое ребро дополнительно пронумеровано):
  
  \begin{center}             
    \begin{tikzpicture}[x=0.75pt,y=0.75pt,yscale=-1,xscale=1,scale=1.4]
        \draw   (246,139) .. controls (246,127.95) and (261.67,119) .. (281,119) .. controls (300.33,119) and (316,127.95) .. (316,139) .. controls (316,150.05) and (300.33,159) .. (281,159) .. controls (261.67,159) and (246,150.05) .. (246,139) -- cycle ;
        \draw    (281,119) .. controls (317.14,79.07) and (236.15,84.55) .. (274.31,117.66) ;
        \draw [shift={(275.5,118.67)}, rotate = 219.67000000000002] [fill={rgb, 255:red, 0; green, 0; blue, 0 }  ][line width=0.75]  [draw opacity=0] (8.93,-4.29) -- (0,0) -- (8.93,4.29) -- cycle    ;
        
        \draw    (281,159) .. controls (319.12,192.33) and (238.15,190.7) .. (274.37,158.65) ;
        \draw [shift={(275.5,157.67)}, rotate = 499.76] [fill={rgb, 255:red, 0; green, 0; blue, 0 }  ][line width=0.75]  [draw opacity=0] (8.93,-4.29) -- (0,0) -- (8.93,4.29) -- cycle    ;
        
        \draw   (379,139) .. controls (379,127.95) and (394.67,119) .. (414,119) .. controls (433.33,119) and (449,127.95) .. (449,139) .. controls (449,150.05) and (433.33,159) .. (414,159) .. controls (394.67,159) and (379,150.05) .. (379,139) -- cycle ;
        \draw    (414,119) .. controls (450.14,79.07) and (369.15,84.55) .. (407.31,117.66) ;
        \draw [shift={(408.5,118.67)}, rotate = 219.67000000000002] [fill={rgb, 255:red, 0; green, 0; blue, 0 }  ][line width=0.75]  [draw opacity=0] (8.93,-4.29) -- (0,0) -- (8.93,4.29) -- cycle    ;
        
        \draw    (414,159) .. controls (452.12,192.33) and (371.15,190.7) .. (407.37,158.65) ;
        \draw [shift={(408.5,157.67)}, rotate = 499.76] [fill={rgb, 255:red, 0; green, 0; blue, 0 }  ][line width=0.75]  [draw opacity=0] (8.93,-4.29) -- (0,0) -- (8.93,4.29) -- cycle    ;
        
        \draw    (381.5,131.33) .. controls (368.76,115.65) and (338.73,112.46) .. (313.07,129.28) ;
        \draw [shift={(311.5,130.33)}, rotate = 325.3] [fill={rgb, 255:red, 0; green, 0; blue, 0 }  ][line width=0.75]  [draw opacity=0] (8.93,-4.29) -- (0,0) -- (8.93,4.29) -- cycle    ;
        
        \draw    (382.5,148.33) .. controls (373.68,163.03) and (335.09,171.01) .. (314.72,147.79) ;
        \draw [shift={(313.5,146.33)}, rotate = 411.34000000000003] [fill={rgb, 255:red, 0; green, 0; blue, 0 }  ][line width=0.75]  [draw opacity=0] (8.93,-4.29) -- (0,0) -- (8.93,4.29) -- cycle    ;
        
        \draw   (112,139.33) .. controls (112,128.29) and (127.67,119.33) .. (147,119.33) .. controls (166.33,119.33) and (182,128.29) .. (182,139.33) .. controls (182,150.38) and (166.33,159.33) .. (147,159.33) .. controls (127.67,159.33) and (112,150.38) .. (112,139.33) -- cycle ;
        \draw    (147,119.33) .. controls (183.14,79.4) and (102.15,84.88) .. (140.31,117.99) ;
        \draw [shift={(141.5,119)}, rotate = 219.67000000000002] [fill={rgb, 255:red, 0; green, 0; blue, 0 }  ][line width=0.75]  [draw opacity=0] (8.93,-4.29) -- (0,0) -- (8.93,4.29) -- cycle    ;
        
        \draw    (147,159.33) .. controls (185.12,192.66) and (104.15,191.04) .. (140.37,158.98) ;
        \draw [shift={(141.5,158)}, rotate = 499.76] [fill={rgb, 255:red, 0; green, 0; blue, 0 }  ][line width=0.75]  [draw opacity=0] (8.93,-4.29) -- (0,0) -- (8.93,4.29) -- cycle    ;
        
        \draw    (182,139.33) -- (244,139.01) ;
        \draw [shift={(246,139)}, rotate = 539.7] [fill={rgb, 255:red, 0; green, 0; blue, 0 }  ][line width=0.75]  [draw opacity=0] (8.93,-4.29) -- (0,0) -- (8.93,4.29) -- cycle    ;
        
        \draw (281,139) node [scale=1.2]  {$S,\ 0$};
        \draw (414,139) node [scale=1.2]  {$S,\ 1$};
        \draw (350,107) node {$S \rightarrow SS\mathbf{\cdot } S$};
        \draw (346,172) node {$S \rightarrow SS\mathbf{\cdot }$};
        \draw (280,81) node {$S \rightarrow S\mathbf{\cdot } SS$};
        \draw (415,81) node {$S \rightarrow S\mathbf{\cdot } SS$};
        \draw (278,191) node{$S \rightarrow S\mathbf{\cdot } S$};
        \draw (412,191) node {$S \rightarrow S\mathbf{\cdot } S$};
        \draw (279,101) node {$1$};
        \draw (280,170) node {$2$};
        \draw (349,126) node {$3$};
        \draw (350,154) node {$4$};
        \draw (412,101) node {$5$};
        \draw (413,171) node {$6$};
        \draw (147,139.33) node [scale=1.2] {$S,\ 2$};
        \draw (214,125.33) node {$S \rightarrow SSS\mathbf{\cdot }$};
        \draw (148,81.33) node {$S \rightarrow S\mathbf{\cdot } SS$};
        \draw (145,101.33) node {$8$};
        \draw (146,171.33) node {$9$};
        \draw (210,147.33) node {$7$};
        \draw (149,191) node {$S \rightarrow S\mathbf{\cdot } S$};
    \end{tikzpicture}
  \end{center}
  
  Далее мы пошагово рассмотрим процесс его построения, а пока отметим несколько особенностей:
  
  \begin{itemize}
    \item Это \textit{неполный} GSS. Для задачи синтаксического анализа такого достаточно, поскольку если в какой-то момент был достигнут финальный дескриптор, то обрабатывать все последующие уже не нужно. Однако, для задачи построения SPPF, как мы отметим далее, это уже не так, поскольку она требует агрегирования всех возможных путей разбора.
  
    \item Обратите особое внимание на наличие петель. Они как раз-таки и обеспечивают эффективную работу с леворекурсивными правилами, поскольку переиспользуются уже существующие узлы. При этом кратных петель, понятно, не создается, т.к. мы запоминаем все достигнутые дескрипторы в множестве $ U $ и дублирующих дескрипторов в рабочее множество $ R $ не добавляем. 
  
    \item В GSS не создаются узлы, соответствующие разбору терминалов (например, $a, 0$). В действительности так можно было бы сделать. Но тогда при обработке слота, указывающего на терминал, сначала бы создался узел GSS, затем интерпретатор сверил бы терминал и символ в слове, после чего, если они совпали, произошел бы возврат из узла, а если нет, узел был бы отброшен и интерпретатор перешел бы к другому дескриптору. Таким образом, при любом случае сначала создается узел, затем выполняется проверка, после чего узел сразу отбрасывается. Для того, чтобы не создавать такие ``одноразовые'' узлы, проверка терминалов выполняется in-place.
  \end{itemize}
  
  Пронумеруем продукции и выпишем управляющую таблицу: 
  
  \begin{table}[!htb]
    \begin{minipage}{.5\linewidth}
      \centering
      \begin{tabular}{lc}
        $S \to S S S$ & (0) \\
        $S \to S S$   & (1) \\ 
        $S \to a$     & (2)
      \end{tabular}
    \end{minipage}%
    \begin{minipage}{.5\linewidth}
      \centering
      \begin{tabular}{ r || c || c | c}
        N   & $\first$  & a     & $\$ $ \\ \hline
        $S$ & $\{ a \}$ & 0,1,2 &
      \end{tabular}
    \end{minipage} 
  \end{table}
  
  Разумеется, что конкретный порядок исполнения алгоритма будет зависеть, например, от используемой в качестве рабочего множества $ R $ структуры данных и от порядка обработки правил из ячейки управляющей таблицы. Рассмотрим лишь один из возможных вариантов:
  
  \begin{enumerate}
    \item Для начала мы создаем узел GSS $ s_0 = (S, 0) $ и дескрипторы для правил из ячейки таблицы $ S, a $: $ (S \to \cdot SSS, s_0, 0), (S \to \cdot SS, s_0, 0), (S \to \cdot a, s_0, 0) $. 
    
    \begin{center}
        \begin{tikzpicture}[x=0.75pt,y=0.75pt,yscale=-1,xscale=1]
        \draw   (210,96) .. controls (210,84.95) and (225.67,76) .. (245,76) .. controls (264.33,76) and (280,84.95) .. (280,96) .. controls (280,107.05) and (264.33,116) .. (245,116) .. controls (225.67,116) and (210,107.05) .. (210,96) -- cycle ;
        
        \draw (245,96) node   {$S,\ 0$};
        \end{tikzpicture}
    \end{center}
     
    \item При обработке $ (S \to \cdot S S S, s_0, 0) $ образовываются петля 1 и дескрипторы \linebreak $ (S \to \cdot SSS, s_0, 0), (S \to \cdot SS, s_0, 0), (S \to \cdot a, s_0, 0) $, которые уже содержатся в множестве $ U $ после шага 1 и поэтому не добавляются повторно. 
    
    \begin{center}
     \begin{tikzpicture}[x=0.75pt,y=0.75pt,yscale=-1,xscale=1]
      \draw   (210,96) .. controls (210,84.95) and (225.67,76) .. (245,76) .. controls (264.33,76) and (280,84.95) .. (280,96) .. controls (280,107.05) and (264.33,116) .. (245,116) .. controls (225.67,116) and (210,107.05) .. (210,96) -- cycle ;
    
      \draw    (245,76) .. controls (281.14,36.07) and (200.15,41.55) .. (238.31,74.66) ;
      \draw [shift={(239.5,75.67)}, rotate = 219.67000000000002] [fill={rgb, 255:red, 0; green, 0; blue, 0 }  ][line width=0.75]  [draw opacity=0] (8.93,-4.29) -- (0,0) -- (8.93,4.29) -- cycle    ;
      
      \draw (245,96) node   {$S,\ 0$};
      \draw (244,38) node   {$S\ \rightarrow S\mathbf{\cdot } SS$};
      \draw (243,58) node   {$1$};
     \end{tikzpicture}
    \end{center}
     
    \item При обработке $ (S \to \cdot S S, s_0, 0) $ образовывается петля 2, а в остальном аналогично \mbox{шагу 2.}
    
    \begin{center}
        \begin{tikzpicture}[x=0.75pt,y=0.75pt,yscale=-1,xscale=1]
        \draw   (210,96) .. controls (210,84.95) and (225.67,76) .. (245,76) .. controls (264.33,76) and (280,84.95) .. (280,96) .. controls (280,107.05) and (264.33,116) .. (245,116) .. controls (225.67,116) and (210,107.05) .. (210,96) -- cycle ;
        
        \draw    (245,76) .. controls (281.14,36.07) and (200.15,41.55) .. (238.31,74.66) ;
        \draw [shift={(239.5,75.67)}, rotate = 219.67000000000002] [fill={rgb, 255:red, 0; green, 0; blue, 0 }  ][line width=0.75]  [draw opacity=0] (8.93,-4.29) -- (0,0) -- (8.93,4.29) -- cycle    ;
        
        \draw    (245,116) .. controls (283.12,149.33) and (202.15,147.7) .. (238.37,115.65) ;
        \draw [shift={(239.5,114.67)}, rotate = 499.76] [fill={rgb, 255:red, 0; green, 0; blue, 0 }  ][line width=0.75]  [draw opacity=0] (8.93,-4.29) -- (0,0) -- (8.93,4.29) -- cycle    ;
        
        \draw (245,96)  node   {$S,\ 0$};
        \draw (244,38)  node   {$S\ \rightarrow S\mathbf{\cdot } SS$};
        \draw (242,148) node   {$S\ \rightarrow S\mathbf{\cdot } S$};
        \draw (243,58)  node   {$1$};
        \draw (244,127) node   {$2$};
        \end{tikzpicture}
    \end{center}
     
    \item При обработке $ (S \to \cdot a, s_0, 0) $ мы распознаем терминал $a$ на позиции 0 и, возвращаясь по петлям 1 и 2, добавляем дескрипторы $ (S \to S \cdot S S, s_0, 1), (S \to S \cdot S, s_0, 1) $.
    
    \item При обработке $ (S \to S \cdot S S, s_0, 1) $ образовываем узел $s_1 = (S, 1)$ с исходящим ребром 3 и добавляем дескрипторы $ (S \to \cdot SSS, s_1, 1), (S \to \cdot SS, s_1, 1), (S \to \cdot a, s_1, 1) $.
    
    \begin{center}
     \begin{tikzpicture}[x=0.75pt,y=0.75pt,yscale=-1,xscale=1]
      \draw   (210,96) .. controls (210,84.95) and (225.67,76) .. (245,76) .. controls (264.33,76) and (280,84.95) .. (280,96) .. controls (280,107.05) and (264.33,116) .. (245,116) .. controls (225.67,116) and (210,107.05) .. (210,96) -- cycle ;
      \draw    (245,76) .. controls (281.14,36.07) and (200.15,41.55) .. (238.31,74.66) ;
      \draw [shift={(239.5,75.67)}, rotate = 219.67000000000002] [fill={rgb, 255:red, 0; green, 0; blue, 0 }  ][line width=0.75]  [draw opacity=0] (8.93,-4.29) -- (0,0) -- (8.93,4.29) -- cycle    ;
      
      \draw    (245,116) .. controls (283.12,149.33) and (202.15,147.7) .. (238.37,115.65) ;
      \draw [shift={(239.5,114.67)}, rotate = 499.76] [fill={rgb, 255:red, 0; green, 0; blue, 0 }  ][line width=0.75]  [draw opacity=0] (8.93,-4.29) -- (0,0) -- (8.93,4.29) -- cycle    ;
      
      \draw   (343,96) .. controls (343,84.95) and (358.67,76) .. (378,76) .. controls (397.33,76) and (413,84.95) .. (413,96) .. controls (413,107.05) and (397.33,116) .. (378,116) .. controls (358.67,116) and (343,107.05) .. (343,96) -- cycle ;
      \draw    (345.5,88.33) .. controls (332.76,72.65) and (302.73,69.46) .. (277.07,86.28) ;
      \draw [shift={(275.5,87.33)}, rotate = 325.3] [fill={rgb, 255:red, 0; green, 0; blue, 0 }  ][line width=0.75]  [draw opacity=0] (8.93,-4.29) -- (0,0) -- (8.93,4.29) -- cycle    ;
      
      \draw (245,96)  node   {$S,\ 0$};
      \draw (378,96)  node   {$S,\ 1$};
      \draw (314,64)  node   {$S\ \rightarrow SS\mathbf{\cdot } S$};
      \draw (244,38)  node   {$S\ \rightarrow S\mathbf{\cdot } SS$};
      \draw (242,148) node   {$S\ \rightarrow S\mathbf{\cdot } S$};
      \draw (243,58)  node   {$1$};
      \draw (244,127) node   {$2$};
      \draw (313,83)  node   {$3$};
     \end{tikzpicture}
    \end{center}
    
    \item При обработке $ (S \to S \cdot S, s_0, 1) $ образовываем ребро 4, новых дескрипторов не добавляется.   
    
    \begin{center}
     \begin{tikzpicture}[x=0.75pt,y=0.75pt,yscale=-1,xscale=1]
      \draw   (210,96) .. controls (210,84.95) and (225.67,76) .. (245,76) .. controls (264.33,76) and (280,84.95) .. (280,96) .. controls (280,107.05) and (264.33,116) .. (245,116) .. controls (225.67,116) and (210,107.05) .. (210,96) -- cycle ;
      \draw    (245,76) .. controls (281.14,36.07) and (200.15,41.55) .. (238.31,74.66) ;
      \draw [shift={(239.5,75.67)}, rotate = 219.67000000000002] [fill={rgb, 255:red, 0; green, 0; blue, 0 }  ][line width=0.75]  [draw opacity=0] (8.93,-4.29) -- (0,0) -- (8.93,4.29) -- cycle    ;
      
      \draw    (245,116) .. controls (283.12,149.33) and (202.15,147.7) .. (238.37,115.65) ;
      \draw [shift={(239.5,114.67)}, rotate = 499.76] [fill={rgb, 255:red, 0; green, 0; blue, 0 }  ][line width=0.75]  [draw opacity=0] (8.93,-4.29) -- (0,0) -- (8.93,4.29) -- cycle    ;
      
      \draw   (343,96) .. controls (343,84.95) and (358.67,76) .. (378,76) .. controls (397.33,76) and (413,84.95) .. (413,96) .. controls (413,107.05) and (397.33,116) .. (378,116) .. controls (358.67,116) and (343,107.05) .. (343,96) -- cycle ;
      \draw    (345.5,88.33) .. controls (332.76,72.65) and (302.73,69.46) .. (277.07,86.28) ;
      \draw [shift={(275.5,87.33)}, rotate = 325.3] [fill={rgb, 255:red, 0; green, 0; blue, 0 }  ][line width=0.75]  [draw opacity=0] (8.93,-4.29) -- (0,0) -- (8.93,4.29) -- cycle    ;
      
      \draw    (346.5,105.33) .. controls (337.68,120.03) and (299.09,128.01) .. (278.72,104.79) ;
      \draw [shift={(277.5,103.33)}, rotate = 411.34000000000003] [fill={rgb, 255:red, 0; green, 0; blue, 0 }  ][line width=0.75]  [draw opacity=0] (8.93,-4.29) -- (0,0) -- (8.93,4.29) -- cycle    ;
      
      \draw (245,96)  node   {$S,\ 0$};
      \draw (378,96)  node   {$S,\ 1$};
      \draw (314,64)  node   {$S\ \rightarrow SS\mathbf{\cdot } S$};
      \draw (310,129) node   {$S\ \rightarrow SS\mathbf{\cdot }$};
      \draw (244,38)  node   {$S\ \rightarrow S\mathbf{\cdot } SS$};
      \draw (242,148) node   {$S\ \rightarrow S\mathbf{\cdot } S$};
      \draw (243,58)  node   {$1$};
      \draw (244,127) node   {$2$};
      \draw (313,83)  node   {$3$};
      \draw (314,111) node   {$4$};
     \end{tikzpicture}
    \end{center}
     
    \item Обработка дескриптора $ (S \to \cdot S S S, s_1, 1) $ аналогична шагу 2 с добавлением петли 5.
    
    \begin{center}       
        \begin{tikzpicture}[x=0.75pt,y=0.75pt,yscale=-1,xscale=1]
            \draw   (210,96) .. controls (210,84.95) and (225.67,76) .. (245,76) .. controls (264.33,76) and (280,84.95) .. (280,96) .. controls (280,107.05) and (264.33,116) .. (245,116) .. controls (225.67,116) and (210,107.05) .. (210,96) -- cycle ;
            \draw    (245,76) .. controls (281.14,36.07) and (200.15,41.55) .. (238.31,74.66) ;
            \draw [shift={(239.5,75.67)}, rotate = 219.67000000000002] [fill={rgb, 255:red, 0; green, 0; blue, 0 }  ][line width=0.75]  [draw opacity=0] (8.93,-4.29) -- (0,0) -- (8.93,4.29) -- cycle    ;
            
            \draw    (245,116) .. controls (283.12,149.33) and (202.15,147.7) .. (238.37,115.65) ;
            \draw [shift={(239.5,114.67)}, rotate = 499.76] [fill={rgb, 255:red, 0; green, 0; blue, 0 }  ][line width=0.75]  [draw opacity=0] (8.93,-4.29) -- (0,0) -- (8.93,4.29) -- cycle    ;
            
            \draw   (343,96) .. controls (343,84.95) and (358.67,76) .. (378,76) .. controls (397.33,76) and (413,84.95) .. (413,96) .. controls (413,107.05) and (397.33,116) .. (378,116) .. controls (358.67,116) and (343,107.05) .. (343,96) -- cycle ;
            \draw    (378,76) .. controls (414.14,36.07) and (333.15,41.55) .. (371.31,74.66) ;
            \draw [shift={(372.5,75.67)}, rotate = 219.67000000000002] [fill={rgb, 255:red, 0; green, 0; blue, 0 }  ][line width=0.75]  [draw opacity=0] (8.93,-4.29) -- (0,0) -- (8.93,4.29) -- cycle    ;
            
            \draw    (345.5,88.33) .. controls (332.76,72.65) and (302.73,69.46) .. (277.07,86.28) ;
            \draw [shift={(275.5,87.33)}, rotate = 325.3] [fill={rgb, 255:red, 0; green, 0; blue, 0 }  ][line width=0.75]  [draw opacity=0] (8.93,-4.29) -- (0,0) -- (8.93,4.29) -- cycle    ;
            
            \draw    (346.5,105.33) .. controls (337.68,120.03) and (299.09,128.01) .. (278.72,104.79) ;
            \draw [shift={(277.5,103.33)}, rotate = 411.34000000000003] [fill={rgb, 255:red, 0; green, 0; blue, 0 }  ][line width=0.75]  [draw opacity=0] (8.93,-4.29) -- (0,0) -- (8.93,4.29) -- cycle    ;
            
            \draw (245,96)  node   {$S,\ 0$};
            \draw (378,96)  node   {$S,\ 1$};
            \draw (314,64)  node {$S\ \rightarrow SS\mathbf{\cdot } S$};
            \draw (310,129) node {$S\ \rightarrow SS\mathbf{\cdot }$};
            \draw (244,38)  node {$S\ \rightarrow S\mathbf{\cdot } SS$};
            \draw (379,38)  node {$S\ \rightarrow S\mathbf{\cdot } SS$};
            \draw (242,148) node {$S\ \rightarrow S\mathbf{\cdot } S$};
            \draw (243,58)  node {$1$};
            \draw (244,127) node {$2$};
            \draw (313,83)  node {$3$};
            \draw (314,111) node {$4$};
            \draw (376,58)  node {$5$};
        \end{tikzpicture}
    \end{center}
    
    \item Обработка дескриптора $ (S \to \cdot S S, s_1, 1) $ аналогична шагу 3 с добавлением петли 6.
    
    \begin{center}
        \begin{tikzpicture}[x=0.75pt,y=0.75pt,yscale=-1,xscale=1]
            \draw   (210,96) .. controls (210,84.95) and (225.67,76) .. (245,76) .. controls (264.33,76) and (280,84.95) .. (280,96) .. controls (280,107.05) and (264.33,116) .. (245,116) .. controls (225.67,116) and (210,107.05) .. (210,96) -- cycle ;
            \draw    (245,76) .. controls (281.14,36.07) and (200.15,41.55) .. (238.31,74.66) ;
            \draw [shift={(239.5,75.67)}, rotate = 219.67000000000002] [fill={rgb, 255:red, 0; green, 0; blue, 0 }  ][line width=0.75]  [draw opacity=0] (8.93,-4.29) -- (0,0) -- (8.93,4.29) -- cycle    ;
            
            \draw    (245,116) .. controls (283.12,149.33) and (202.15,147.7) .. (238.37,115.65) ;
            \draw [shift={(239.5,114.67)}, rotate = 499.76] [fill={rgb, 255:red, 0; green, 0; blue, 0 }  ][line width=0.75]  [draw opacity=0] (8.93,-4.29) -- (0,0) -- (8.93,4.29) -- cycle    ;
            
            \draw   (343,96) .. controls (343,84.95) and (358.67,76) .. (378,76) .. controls (397.33,76) and (413,84.95) .. (413,96) .. controls (413,107.05) and (397.33,116) .. (378,116) .. controls (358.67,116) and (343,107.05) .. (343,96) -- cycle ;
            \draw    (378,76) .. controls (414.14,36.07) and (333.15,41.55) .. (371.31,74.66) ;
            \draw [shift={(372.5,75.67)}, rotate = 219.67000000000002] [fill={rgb, 255:red, 0; green, 0; blue, 0 }  ][line width=0.75]  [draw opacity=0] (8.93,-4.29) -- (0,0) -- (8.93,4.29) -- cycle    ;
            
            \draw    (378,116) .. controls (416.12,149.33) and (335.15,147.7) .. (371.37,115.65) ;
            \draw [shift={(372.5,114.67)}, rotate = 499.76] [fill={rgb, 255:red, 0; green, 0; blue, 0 }  ][line width=0.75]  [draw opacity=0] (8.93,-4.29) -- (0,0) -- (8.93,4.29) -- cycle    ;
            
            \draw    (345.5,88.33) .. controls (332.76,72.65) and (302.73,69.46) .. (277.07,86.28) ;
            \draw [shift={(275.5,87.33)}, rotate = 325.3] [fill={rgb, 255:red, 0; green, 0; blue, 0 }  ][line width=0.75]  [draw opacity=0] (8.93,-4.29) -- (0,0) -- (8.93,4.29) -- cycle    ;
            
            \draw    (346.5,105.33) .. controls (337.68,120.03) and (299.09,128.01) .. (278.72,104.79) ;
            \draw [shift={(277.5,103.33)}, rotate = 411.34000000000003] [fill={rgb, 255:red, 0; green, 0; blue, 0 }  ][line width=0.75]  [draw opacity=0] (8.93,-4.29) -- (0,0) -- (8.93,4.29) -- cycle    ;
            
            
            \draw (245,96)  node   {$S,\ 0$};
            \draw (378,96)  node   {$S,\ 1$};
            \draw (314,64)  node {$S\ \rightarrow SS\mathbf{\cdot } S$};
            \draw (310,129) node {$S\ \rightarrow SS\mathbf{\cdot }$};
            \draw (244,38)  node {$S\ \rightarrow S\mathbf{\cdot } SS$};
            \draw (379,38)  node {$S\ \rightarrow S\mathbf{\cdot } SS$};
            \draw (242,148) node {$S\ \rightarrow S\mathbf{\cdot } S$};
            \draw (376,148) node {$S\ \rightarrow S\mathbf{\cdot } S$};
            \draw (243,58)  node {$1$};
            \draw (244,127) node {$2$};
            \draw (313,83)  node {$3$};
            \draw (314,111) node {$4$};
            \draw (376,58)  node {$5$};
            \draw (377,128) node {$6$};
        \end{tikzpicture}
    \end{center}
    
    \item При обработке $ (S \to \cdot a, s_1, 1) $ мы распознаем терминал $a$ на позиции 1 и, возвращаясь по ребрам 3 и 4, добавляем дескрипторы $ (S \to S S \cdot S, s_0, 2), (S \to S S \cdot, s_0, 2) $, а также, возвращаясь по петлям 5 и 6, добавляем дескрипторы $ (S \to S \cdot S S, s_1, 2), (S \to S \cdot S, s_1, 2) $.
    
    \item При обработке $ (S \to S S \cdot S, s_0, 2) $ образовываем узел $s_2 = (S, 2)$ с исходящим ребром 7 и добавляем дескрипторы $ (S \to \cdot SSS, s_2, 2), (S \to \cdot SS, s_2, 2), (S \to \cdot a, s_2, 2) $.
    
    \begin{center}
        \begin{tikzpicture}[x=0.75pt,y=0.75pt,yscale=-1,xscale=1]
            \draw   (210,96) .. controls (210,84.95) and (225.67,76) .. (245,76) .. controls (264.33,76) and (280,84.95) .. (280,96) .. controls (280,107.05) and (264.33,116) .. (245,116) .. controls (225.67,116) and (210,107.05) .. (210,96) -- cycle ;
            \draw    (245,76) .. controls (281.14,36.07) and (200.15,41.55) .. (238.31,74.66) ;
            \draw [shift={(239.5,75.67)}, rotate = 219.67000000000002] [fill={rgb, 255:red, 0; green, 0; blue, 0 }  ][line width=0.75]  [draw opacity=0] (8.93,-4.29) -- (0,0) -- (8.93,4.29) -- cycle    ;
            
            \draw    (245,116) .. controls (283.12,149.33) and (202.15,147.7) .. (238.37,115.65) ;
            \draw [shift={(239.5,114.67)}, rotate = 499.76] [fill={rgb, 255:red, 0; green, 0; blue, 0 }  ][line width=0.75]  [draw opacity=0] (8.93,-4.29) -- (0,0) -- (8.93,4.29) -- cycle    ;
            
            \draw   (343,96) .. controls (343,84.95) and (358.67,76) .. (378,76) .. controls (397.33,76) and (413,84.95) .. (413,96) .. controls (413,107.05) and (397.33,116) .. (378,116) .. controls (358.67,116) and (343,107.05) .. (343,96) -- cycle ;
            \draw    (378,76) .. controls (414.14,36.07) and (333.15,41.55) .. (371.31,74.66) ;
            \draw [shift={(372.5,75.67)}, rotate = 219.67000000000002] [fill={rgb, 255:red, 0; green, 0; blue, 0 }  ][line width=0.75]  [draw opacity=0] (8.93,-4.29) -- (0,0) -- (8.93,4.29) -- cycle    ;
            
            \draw [shift={(372.5,114.67)}, rotate = 499.76] [fill={rgb, 255:red, 0; green, 0; blue, 0 }  ][line width=0.75]  [draw opacity=0] (8.93,-4.29) -- (0,0) -- (8.93,4.29) -- cycle    ;
            
            \draw    (345.5,88.33) .. controls (332.76,72.65) and (302.73,69.46) .. (277.07,86.28) ;
            \draw [shift={(275.5,87.33)}, rotate = 325.3] [fill={rgb, 255:red, 0; green, 0; blue, 0 }  ][line width=0.75]  [draw opacity=0] (8.93,-4.29) -- (0,0) -- (8.93,4.29) -- cycle    ;
            
            \draw    (346.5,105.33) .. controls (337.68,120.03) and (299.09,128.01) .. (278.72,104.79) ;
            \draw [shift={(277.5,103.33)}, rotate = 411.34000000000003] [fill={rgb, 255:red, 0; green, 0; blue, 0 }  ][line width=0.75]  [draw opacity=0] (8.93,-4.29) -- (0,0) -- (8.93,4.29) -- cycle    ;
            
            \draw   (76,96.33) .. controls (76,85.29) and (91.67,76.33) .. (111,76.33) .. controls (130.33,76.33) and (146,85.29) .. (146,96.33) .. controls (146,107.38) and (130.33,116.33) .. (111,116.33) .. controls (91.67,116.33) and (76,107.38) .. (76,96.33) -- cycle ;
            \draw    (146,96.33) -- (208,96.01) ;
            \draw [shift={(210,96)}, rotate = 539.7] [fill={rgb, 255:red, 0; green, 0; blue, 0 }  ][line width=0.75]  [draw opacity=0] (8.93,-4.29) -- (0,0) -- (8.93,4.29) -- cycle    ;
            
            \draw (245,96)  node   {$S,\ 0$};
            \draw (378,96)  node   {$S,\ 1$};
            \draw (314,64)  node   {$S\ \rightarrow SS\mathbf{\cdot } S$};
            \draw (310,129) node   {$S\ \rightarrow SS\mathbf{\cdot }$};
            \draw (244,38)  node   {$S\ \rightarrow S\mathbf{\cdot } SS$};
            \draw (379,38)  node   {$S\ \rightarrow S\mathbf{\cdot } SS$};
            \draw (242,148) node   {$S\ \rightarrow S\mathbf{\cdot } S$};
            \draw (376,148) node   {$S\ \rightarrow S\mathbf{\cdot } S$};
            \draw (243,58)  node   {$1$};
            \draw (244,127) node   {$2$};
            \draw (313,83)  node   {$3$};
            \draw (314,111) node   {$4$};
            \draw (376,58)  node   {$5$};
            \draw (377,128) node   {$6$};
            \draw (111,96)  node   {$S,\ 2$};
            \draw (177,80)  node   {$S\ \rightarrow SSS\mathbf{\cdot }$};
            \draw (174,104) node   {$7$};
        \end{tikzpicture}
    \end{center}
    
    \item Обработка дескриптора $ (S \to \cdot S S S, s_2, 2) $ аналогична шагу 2 с добавлением петли 8.
    
    \begin{center}
        \begin{tikzpicture}[x=0.75pt,y=0.75pt,yscale=-1,xscale=1]
            \draw   (210,96) .. controls (210,84.95) and (225.67,76) .. (245,76) .. controls (264.33,76) and (280,84.95) .. (280,96) .. controls (280,107.05) and (264.33,116) .. (245,116) .. controls (225.67,116) and (210,107.05) .. (210,96) -- cycle ;
            \draw    (245,76) .. controls (281.14,36.07) and (200.15,41.55) .. (238.31,74.66) ;
            \draw [shift={(239.5,75.67)}, rotate = 219.67000000000002] [fill={rgb, 255:red, 0; green, 0; blue, 0 }  ][line width=0.75]  [draw opacity=0] (8.93,-4.29) -- (0,0) -- (8.93,4.29) -- cycle    ;
            
            \draw    (245,116) .. controls (283.12,149.33) and (202.15,147.7) .. (238.37,115.65) ;
            \draw [shift={(239.5,114.67)}, rotate = 499.76] [fill={rgb, 255:red, 0; green, 0; blue, 0 }  ][line width=0.75]  [draw opacity=0] (8.93,-4.29) -- (0,0) -- (8.93,4.29) -- cycle    ;
            
            \draw   (343,96) .. controls (343,84.95) and (358.67,76) .. (378,76) .. controls (397.33,76) and (413,84.95) .. (413,96) .. controls (413,107.05) and (397.33,116) .. (378,116) .. controls (358.67,116) and (343,107.05) .. (343,96) -- cycle ;
            \draw    (378,76) .. controls (414.14,36.07) and (333.15,41.55) .. (371.31,74.66) ;
            \draw [shift={(372.5,75.67)}, rotate = 219.67000000000002] [fill={rgb, 255:red, 0; green, 0; blue, 0 }  ][line width=0.75]  [draw opacity=0] (8.93,-4.29) -- (0,0) -- (8.93,4.29) -- cycle    ;
            
            \draw    (378,116) .. controls (416.12,149.33) and (335.15,147.7) .. (371.37,115.65) ;
            \draw [shift={(372.5,114.67)}, rotate = 499.76] [fill={rgb, 255:red, 0; green, 0; blue, 0 }  ][line width=0.75]  [draw opacity=0] (8.93,-4.29) -- (0,0) -- (8.93,4.29) -- cycle    ;
            
            \draw    (345.5,88.33) .. controls (332.76,72.65) and (302.73,69.46) .. (277.07,86.28) ;
            \draw [shift={(275.5,87.33)}, rotate = 325.3] [fill={rgb, 255:red, 0; green, 0; blue, 0 }  ][line width=0.75]  [draw opacity=0] (8.93,-4.29) -- (0,0) -- (8.93,4.29) -- cycle    ;
            
            \draw    (346.5,105.33) .. controls (337.68,120.03) and (299.09,128.01) .. (278.72,104.79) ;
            \draw [shift={(277.5,103.33)}, rotate = 411.34000000000003] [fill={rgb, 255:red, 0; green, 0; blue, 0 }  ][line width=0.75]  [draw opacity=0] (8.93,-4.29) -- (0,0) -- (8.93,4.29) -- cycle    ;
            
            \draw   (76,96.33) .. controls (76,85.29) and (91.67,76.33) .. (111,76.33) .. controls (130.33,76.33) and (146,85.29) .. (146,96.33) .. controls (146,107.38) and (130.33,116.33) .. (111,116.33) .. controls (91.67,116.33) and (76,107.38) .. (76,96.33) -- cycle ;
            \draw    (111,76.33) .. controls (147.14,36.4) and (66.15,41.88) .. (104.31,74.99) ;
            \draw [shift={(105.5,76)}, rotate = 219.67000000000002] [fill={rgb, 255:red, 0; green, 0; blue, 0 }  ][line width=0.75]  [draw opacity=0] (8.93,-4.29) -- (0,0) -- (8.93,4.29) -- cycle    ;
            
            \draw    (146,96.33) -- (208,96.01) ;
            \draw [shift={(210,96)}, rotate = 539.7] [fill={rgb, 255:red, 0; green, 0; blue, 0 }  ][line width=0.75]  [draw opacity=0] (8.93,-4.29) -- (0,0) -- (8.93,4.29) -- cycle    ;
            
            \draw (245,96)  node   {$S,\ 0$};
            \draw (378,96)  node   {$S,\ 1$};
            \draw (314,64)  node   {$S\ \rightarrow SS\mathbf{\cdot } S$};
            \draw (310,129) node   {$S\ \rightarrow SS\mathbf{\cdot }$};
            \draw (244,38)  node   {$S\ \rightarrow S\mathbf{\cdot } SS$};
            \draw (379,38)  node   {$S\ \rightarrow S\mathbf{\cdot } SS$};
            \draw (242,148) node   {$S\ \rightarrow S\mathbf{\cdot } S$};
            \draw (376,148) node   {$S\ \rightarrow S\mathbf{\cdot } S$};
            \draw (243,58)  node   {$1$};
            \draw (244,127) node   {$2$};
            \draw (313,83)  node   {$3$};
            \draw (314,111) node   {$4$};
            \draw (376,58)  node   {$5$};
            \draw (377,128) node   {$6$};
            \draw (111,96)  node   {$S,\ 2$};
            \draw (177,80)  node   {$S\ \rightarrow SSS\mathbf{\cdot }$};
            \draw (112,38)  node   {$S\ \rightarrow S\mathbf{\cdot } SS$};
            \draw (109,58)  node   {$8$};
            \draw (174,104) node   {$7$};
        \end{tikzpicture}
    \end{center}
            
    \item Обработка дескриптора $ (S \to \cdot S S, s_2, 2) $ аналогична шагу 3 с добавлением петли 9.
    
    \begin{center}
        \begin{tikzpicture}[x=0.75pt,y=0.75pt,yscale=-1,xscale=1]
            \draw   (210,96) .. controls (210,84.95) and (225.67,76) .. (245,76) .. controls (264.33,76) and (280,84.95) .. (280,96) .. controls (280,107.05) and (264.33,116) .. (245,116) .. controls (225.67,116) and (210,107.05) .. (210,96) -- cycle ;
            \draw    (245,76) .. controls (281.14,36.07) and (200.15,41.55) .. (238.31,74.66) ;
            \draw [shift={(239.5,75.67)}, rotate = 219.67000000000002] [fill={rgb, 255:red, 0; green, 0; blue, 0 }  ][line width=0.75]  [draw opacity=0] (8.93,-4.29) -- (0,0) -- (8.93,4.29) -- cycle    ;
            
            \draw    (245,116) .. controls (283.12,149.33) and (202.15,147.7) .. (238.37,115.65) ;
            \draw [shift={(239.5,114.67)}, rotate = 499.76] [fill={rgb, 255:red, 0; green, 0; blue, 0 }  ][line width=0.75]  [draw opacity=0] (8.93,-4.29) -- (0,0) -- (8.93,4.29) -- cycle    ;
            
            \draw   (343,96) .. controls (343,84.95) and (358.67,76) .. (378,76) .. controls (397.33,76) and (413,84.95) .. (413,96) .. controls (413,107.05) and (397.33,116) .. (378,116) .. controls (358.67,116) and (343,107.05) .. (343,96) -- cycle ;
            \draw    (378,76) .. controls (414.14,36.07) and (333.15,41.55) .. (371.31,74.66) ;
            \draw [shift={(372.5,75.67)}, rotate = 219.67000000000002] [fill={rgb, 255:red, 0; green, 0; blue, 0 }  ][line width=0.75]  [draw opacity=0] (8.93,-4.29) -- (0,0) -- (8.93,4.29) -- cycle    ;
            
            \draw    (378,116) .. controls (416.12,149.33) and (335.15,147.7) .. (371.37,115.65) ;
            \draw [shift={(372.5,114.67)}, rotate = 499.76] [fill={rgb, 255:red, 0; green, 0; blue, 0 }  ][line width=0.75]  [draw opacity=0] (8.93,-4.29) -- (0,0) -- (8.93,4.29) -- cycle    ;
            
            \draw    (345.5,88.33) .. controls (332.76,72.65) and (302.73,69.46) .. (277.07,86.28) ;
            \draw [shift={(275.5,87.33)}, rotate = 325.3] [fill={rgb, 255:red, 0; green, 0; blue, 0 }  ][line width=0.75]  [draw opacity=0] (8.93,-4.29) -- (0,0) -- (8.93,4.29) -- cycle    ;
            
            \draw    (346.5,105.33) .. controls (337.68,120.03) and (299.09,128.01) .. (278.72,104.79) ;
            \draw [shift={(277.5,103.33)}, rotate = 411.34000000000003] [fill={rgb, 255:red, 0; green, 0; blue, 0 }  ][line width=0.75]  [draw opacity=0] (8.93,-4.29) -- (0,0) -- (8.93,4.29) -- cycle    ;
            
            \draw   (76,96.33) .. controls (76,85.29) and (91.67,76.33) .. (111,76.33) .. controls (130.33,76.33) and (146,85.29) .. (146,96.33) .. controls (146,107.38) and (130.33,116.33) .. (111,116.33) .. controls (91.67,116.33) and (76,107.38) .. (76,96.33) -- cycle ;
            \draw    (111,76.33) .. controls (147.14,36.4) and (66.15,41.88) .. (104.31,74.99) ;
            \draw [shift={(105.5,76)}, rotate = 219.67000000000002] [fill={rgb, 255:red, 0; green, 0; blue, 0 }  ][line width=0.75]  [draw opacity=0] (8.93,-4.29) -- (0,0) -- (8.93,4.29) -- cycle    ;
            
            \draw    (111,116.33) .. controls (149.12,149.66) and (68.15,148.04) .. (104.37,115.98) ;
            \draw [shift={(105.5,115)}, rotate = 499.76] [fill={rgb, 255:red, 0; green, 0; blue, 0 }  ][line width=0.75]  [draw opacity=0] (8.93,-4.29) -- (0,0) -- (8.93,4.29) -- cycle    ;
            
            \draw    (146,96.33) -- (208,96.01) ;
            \draw [shift={(210,96)}, rotate = 539.7] [fill={rgb, 255:red, 0; green, 0; blue, 0 }  ][line width=0.75]  [draw opacity=0] (8.93,-4.29) -- (0,0) -- (8.93,4.29) -- cycle    ;
            
            \draw (245,96)  node   {$S,\ 0$};
            \draw (378,96)  node   {$S,\ 1$};
            \draw (314,64)  node   {$S\ \rightarrow SS\mathbf{\cdot } S$};
            \draw (310,129) node   {$S\ \rightarrow SS\mathbf{\cdot }$};
            \draw (244,38)  node   {$S\ \rightarrow S\mathbf{\cdot } SS$};
            \draw (379,38)  node   {$S\ \rightarrow S\mathbf{\cdot } SS$};
            \draw (242,148) node   {$S\ \rightarrow S\mathbf{\cdot } S$};
            \draw (376,148) node   {$S\ \rightarrow S\mathbf{\cdot } S$};
            \draw (243,58)  node   {$1$};
            \draw (244,127) node   {$2$};
            \draw (313,83)  node   {$3$};
            \draw (314,111) node   {$4$};
            \draw (376,58)  node   {$5$};
            \draw (377,128) node   {$6$};
            \draw (111,96)  node   {$S,\ 2$};
            \draw (177,80)  node   {$S\ \rightarrow SSS\mathbf{\cdot }$};
            \draw (112,38)  node   {$S\ \rightarrow S\mathbf{\cdot } SS$};
            \draw (109,58)  node   {$8$};
            \draw (110,128) node   {$9$};
            \draw (174,104) node   {$7$};
            \draw (113,148) node   {$S\ \rightarrow S\mathbf{\cdot } S$};
        \end{tikzpicture}
    \end{center}
    
    \item При обработке $ (S \to \cdot a, s_2, 2) $ мы распознаем терминал $a$ на позиции 2 и, возвращаясь по ребру 7, добавляем дескриптор $ (S \to S S S \cdot, s_0, 3) $, а также, возвращаясь по петлям 8 и 9, добавляем дескрипторы $ (S \to S \cdot S S, s_2, 3), (S \to S \cdot S, s_2, 3) $.
    
    \item Мы достигли финального дескриптора $ (S \to S S S \cdot, s_0, 3) $, синтаксический разбор успешен.
  \end{enumerate}

\end{example}

Внимательный читатель мог заметить, что если бы в этом примере шаг 4 был выполнен перед шагом 2, разбор довольно быстро бы завершился неудачей. Отсюда вытекает следующее наблюдение: если в какой-то момент из существующего узла появилось новое ребро, необходимо пересчитать все входящие в него пути. 

Для построения SPPF требуется внести лишь несколько небольших добавлений:

\begin{enumerate}
  \item В дескриптор необходимо добавить узел SPPF $ w $, который будет представлять уже разобранный префикс. 
  \item Необходимо поддерживать множество $ P $ из элементов вида $ (u, z) $, где $ u $ это узел GSS, а $ z $ соответствующий ему узел SPPF, для того, чтобы переиспользовать результаты разбора, ассоциированные с узлами GSS. 
  \item При обработке терминала $ t $ на позиции $ i $ ищется узел вида $ (t, i, i + 1) $, либо создается, если такого еще нет.
  \item При обработке нетерминала с помощью $ P $ ищется или при необходимости создается промежуточный узел вида $ (X, l, r) $, где $ X $ соответствующий слот грамматики, а $ l $ и $ r $ узлы SPPF, отвечающие за разбор левой и правой частей слота соответственно.
\end{enumerate}

Конкретные шаги построения SPPF будут зависеть от выбранного для него формата. Описание эффективного бинаризованного SPPF и детали его построения при выполнении GLL представлены в работе~\cite{10.1007/978-3-662-46663-6_5}.

 \section{Алгоритм вычисления КС запросов на основе GLL}

GLL довольно естественно обобщается на граф~\cite{Grigorev:2017:CPQ:3166094.3166104}: позициями входа теперь будем считать не индексы линейного слова, а вершины графа. В самом же алгоритме требуется внести лишь два небольших дополнения:

\begin{enumerate}
  \item Теперь при обработке терминала ``следующих'' символов может быть несколько --- рассматриваем каждый из них отдельно, сдвигаясь по соответствующему ребру. В результате, при одном чтении можем получить несколько новых дескрипторов, но они независимы, потому просто ставим их в рабочее множество $ R $.
  \item При обработке нетерминала, аналогично, правила управляющей таблицы применяются для каждого из ``следующих'' символов в графе. Соответственно новых дескрипторов будет сгенерировано больше, но все они по-прежнему независимы и просто добавляются в рабочее множество $ R $.  
\end{enumerate}

Подробное описание алгоритма и псевдокод представлены в работе~\cite{Grigorev:2017:CPQ:3166094.3166104}. Существует ещё одно обобщение нисходящего синтаксического анализа для решения задачи КС достижимости~\cite{MEDEIROS201975}, которое предполагает непосредственное обобщение LL(k) алгоритма, что приводит к аналогичному результату, однако теряется связь с некоторыми построениями. 

Основанный на нисходящем анализе алгоритма поиска путей с контекстно-свободными ограничениями имеет следующие особенности.
\begin{enumerate}
  \item Необходимо явно задавать начальную вершину, поэтому хорошо подходит для задач поиска путей с одним источником (single-source) или с небольшим количеством источников. Для поиска путей между всеми парами вершин необходимо явным образом все указать стартовыми.
  \item Является направленным сверху вниз --- обходит граф последовательно начиная с указанной стартовой вершины и строит вывод, начиная со стартового нетерминала. Как следствие, в отличие от алгоритмов на основе линейной алгебры и Хеллингса, обойдёт только подграф, необходимый для построения ответа. В среднем это меньше, чем весь граф, который обрабанывается другими алгоритмами.
  \item Естественным образом строит множество путей в виде сжатого леса разбора.
  \item Использует существенно более тяжеловесные стуктуры данных и плохо распараллеливается (на практике). Как следствие, при решении задачи досижимости для всех пар путей проигрывает алгоритмам на основе линейной алгебры.
\end{enumerate}

Частным случаем применения задачи КС достижимости является синтаксический анализ с неоднозначной токенизацией, то есть ситуацией, когда несколько пересекающихся подстрок во входной строке символов могут задавать разные лексические единицы и не возможно сделать однозначный выбор на этапе лексического анализа.
Например, для строки \verb|x = a---b| возможны несколько вариантов токенизации.
\begin{enumerate}
  \item \verb|ID(x) OP_EQ ID(a) OP_MINUS OP_DECRIMENT ID(b)|
  %\item \verb|ID(x) OP_EQ ID(A) OP_MINUS OP_UN_MINUS ID(b)|
  \item \verb|ID(x) OP_EQ ID(a) OP_DECRIMENT OP_MINUS ID(b)|  
\end{enumerate} 

В таком случае на вход синтаксическому анализатору можно подать DAG, содержащий все возможные варианты токенизации. Для нашего примера он может выглядить следующим образом:

\begin{center}
    \begin{tikzpicture}[node distance=4cm,shorten >=1pt,on grid,auto]
    \node[state] (q_0) at (0,0)  {$0$};
    \node[state] (q_1) at (2.5,0)  {$1$};
    \node[state] (q_2) at (5.5,0)  {$2$};
    \node[state] (q_3) at (8,0)  {$3$};
    \node[state] (q_4) at (10,-2)  {$4$};
    \node[state] (q_5) at (11.5,0) {$6$};
    \node[state] (q_6) at (10,2) {$5$};
    \node[state] (q_7) at (14,0) {$7$};
    \path[->]
    (q_0) edge  node {ID(x)} (q_1)
    (q_1) edge  node {OP\_EQ} (q_2)
    (q_2) edge  node {ID(a)} (q_3)
    (q_3) edge[left]  node {OP\_MINUS} (q_4)
    (q_4) edge[right]  node {OP\_DECRIMENT} (q_5)
    (q_3) edge  node {OP\_DECRIMENT} (q_6)
    (q_6) edge node {OP\_MINUS} (q_5)
    (q_5) edge  node {ID(b)} (q_7);
    \end{tikzpicture}
\end{center}

Далее будем проверять наличие пути из старовой (нулевой) вершины в конечную (соответствующую концу строки). Если таких путей оказалось несколько, то нужны дополнительные средства для выбора нужного дерева разбора. Даннаяя идея рассматривается в работе~\cite{10.1145/3357766.3359532}.

Напоследок сделаем небольшое замечание об эффективной реализации: в качестве рабочего множества $ R $ можно использовать несколько различных структур данных и, как правило, выбирают очередь. Однако иногда (в особенности для графов) лучше использовать стек дескрипторов, так как в этом случае выше локальность данных --- мы кладём пачку дескрипторов, соответствующих исходящим рёбрам. И если граф представлен списком смежности, то исходящие будут храниться рядом и их лучше обработать сразу.

%\section{Вопросы и задачи}
%\begin{enumerate}
%  \item Проведите алгоритм GLL для грамматики $ S \to a S b S \mid \varepsilon$. Правда ли, что эта грамматика принадлежит классу $ LL(1) $? Пронаблюдайте, как использование GSS вырождается в работу с обычным стеком. 
%  \item Доразберите все не рассмотренные в примере \ref{gll:example1} дескрипторы, постройте полный GSS.
%\end{enumerate}


\chapter{Алгоритм на основе восходящего анализа}\label{chpt:GLR}

В данном разделе будут рассмотрены алгоритмы восходящего синтаксического анализа LR-семейсва, в том числе Generalized LR (GLR). Также будет рассмотрено обощение алгоритма GLR для решения задачи поиска путей с контекстно-свободными ограничениями в графах.

\section{Восходящий синтаксический анализ}

Существует большое семейство LR(k) алгоритмов --- алгоритм восходящего синтаксического анализа. 
Основная идея, лежащая в основе семейства, заключается в следующем: входная последовательность символов считывается слева направо с попутным добавлением в стек и выполнением сворачивания на стеке --- замены последовательности терминалов и нетерминалов, лежащих наверху стека, на нетерминал, если существует соответствующее правило в исходной грамматике.

Как и в случае с LL используется магазинный автомат, управляемый таблицами, построенными по грамматике.
При этом, у LR анализатора есть два типа команд:
\begin{enumerate}
	\item shift --- прочитать следующий символ входной последовательности, положив его в стек, и перейти в следующее состояние;
	\item reduce(k) --- применить k-ое правило грамматики, правая часть которого уже лежит на стеке: снимаем со стека правую часть продукции и кладём левую часть.
\end{enumerate}

А управляющая таблица выглядит следующим образом.

\begin{center}
  \begin{tabular}{c||c|c|c|c|c||c|c|c|c}
     States & $t_0$   &$\dots$ & $t_a$   & $\dots$ & \$      & $N_0$   &$\dots$ & $N_b$   & $\dots$  \\ \hline 
    $\dots$ & $\dots$ &$\dots$ & $\dots$ & $\dots$ & $\dots$ & $\dots$ &$\dots$ & $\dots$ & $\dots$  \\ \hline 
    $10$    & $\dots$ &$\dots$ & $s_i$   & $\dots$ & $r_k$   & $\dots$ &$\dots$ & $j$     & $\dots$ \\ \hline 
    $\dots$ & $\dots$ &$\dots$ & $\dots$ & $\dots$ & $acc$ & $\dots$ &$\dots$ & $\dots$ & $\dots$ 
  \end{tabular}
\end{center}

Здесь
\begin{itemize}
  \item $s_i$~--- shift: перенести соответствующи символ в стек и перейти в состояние $i$.
  \item $r_k$~--- reduce(k): в стеке накопилась правая часть продукции $k$, пора производить свёртку.  
  \item $j$~--- goto: выполняется после reduce. Сама по себе команда reduce не переводит автомат в новое состояние. Команда goto переведёт автомат в состояние $j$.
  \item $acc$~--- accept: разбор завершился успешно. 
\end{itemize}

Если ячейка пустая и в процессе работы мы пропали в неё --- значит произошла ошибка. Для детерминированной работы анализатора требуется, чтобы в каждой ячейке было не более одной команды. Есди это не так, то говорят о возникновении конфликтов.

\begin{itemize}
\item shift-reduce --- ситуация, когда не понятно, читать ли следующий символ или выполнить reduce. Например, если правая часть одного из правил является префиксом правой части другого правила: $N \rightarrow w, M \rightarrow ww'$.
\item reduce-reduce --- ситуация, когда не понятно, к какому правилу нужно применить reduce. Например, если есть два правила с одинаковыми правыми частями: $N \rightarrow w, M \rightarrow w$.
\end{itemize}

Принцип работы LR анализаторов следующий. Пусть у нас есть входная строка, LR-автомат со стеком и управляющая таблица. 
В начальный момент на стеке лежит стартовое состояние LR-автомата, позиция во входной строке соответствует её началу.
На каждом шаге анализируется текущий символ входа и текущее состояние, в котором находится автомат, и совершается одно из действий: 
\begin{itemize}
\item Если в управляющей таблице нет инструкции для текущего состояния автомата и текущего символа на входе, то завершаем разбор с ошибкой.
\item Иначе выполняем одну из инструкций: 
\begin{itemize}
\item в случае acc --- успешно завершаем разбор.
\item в случае shift --- кладем на стек текущий символ входа, сдвигая при этом текущую позицию, и номер нового состояния. Переходим в новое состояние. 
\item в случае reduce(k) --- снимаем со стека 2l элементов: l состояний и l терминалов/нетерминалов (где l --- длина правой части k-ого правила), кладём на стек нетерминал левой части правила. Тперь на вершине стека у нас нетерминал $N_a$, а следующий элемент --- состяние $i$. Если в ячейке $(i,N_a)$ управляющей таблицы лежит состояние $j$, то кладём его на вершину стека. Иначе завершаемся с ошибкой.
\end{itemize}
\end{itemize}


Разные алгоритмы из LR-семейсва строят таблицы разными способами и, соответсвенно, могут избегать тех или иных конфликтов. Рассмотрим некорых представителей.

\subsection{LR(0) алгоритм}

Данный алгоритм самый ``слабый'' из семейства --- разбирает наименьший класс языков.
Для построения используются LR(0) пункты.

\begin{definition}
LR(0) пункт (LR(0) item) --- правило грамматики, в правой части которого имеется точка, отделяющая уже разобранную часть правила (слева от точки) от того, что еще предстоит распознать (справа от точки): $A \to \alpha \cdot \beta$, где $A \to \alpha \beta$~--- правило грамматики.
\qed
\end{definition}

Состояние LR(0) автомата --- множество LR(0) пунктов. Для того чтобы изх построить используется операция \textit{closure} или \textit{замыкание}. 

\begin{definition}
$closure(X)  = closure(X \cup \{M \rightarrow \cdot \gamma \mid N_i \rightarrow \alpha\cdot M\beta \in X \})$
\qed 
\end{definition}

\begin{definition}
Ядро --- исходное множество пунктов, до применения к нему замыкания.
\qed
\end{definition}

Для перемещения точки в пункте используется функция \textit{goto}.

\begin{definition}
$goto(X,p)  = \{N_j \rightarrow \alpha p \cdot \beta \mid N_j \rightarrow \alpha\cdot p\beta \in X \}$
\qed 
\end{definition}

Теперь мы можем построить LR(0) автомат.
Первым шагом необходимо расширить грамматику: добавить к исходной граммтике правило вида $S' \to S \$$, где $S$ --- стартовый нетерминал исходной граммтики, $S'$ --- новый стартовый нетерминал (не использовался ранее в грамматике), $\$$ --- маркер конца строки (не входил в терминальный алфавит исходной граммтики).

Далее строим автомат по следующим принципам.

\begin{itemize}
    \item Состояния~--- множества пунктов.
    \item Переходы между состяниями осуществляются по символам грамматики.
    \item Начальное состояние~--- $closure(\{S'\to 
    \cdot S \$\})$. 
    \item Следующее состояние по текущему состоянию $X$ и смволу $p$ вычисляются как $closure(goto(X, p))$
  \end{itemize}

Управляющая таблица по автомату строится следующим образом.

\begin{itemize}
    \item $acc$ в ячейку, соответствующую финальному состоянию и \$ 
    \item $s_i$ в ячейку $(j,t)$, если в автомате есть переход из состояния $j$ по терминалу $t$ в состояние $i$
    \item $i$ в ячейку $(j, N)$, если в автомате есть переход из состояния $j$ по нетерминалу $N$ в состояние $i$
    \item $r_k$ в ячейку $(j,t)$, если в состоянии $j$ есть пункт $A \to \alpha \cdot$, где $A \to \alpha$~--- $k$-ое правило грамматики, $t$~--- терминал грамматики
  \end{itemize}

\subsection{SLR(1) алгоритм}

SLR(1) анализатор отличается от LR(0) анализатора построением таблицы по автомату (автомат в точности как у LR(0).
А именно, $r_k$ добавляется в ячейку $(j,t)$, если в состоянии $j$ есть пункт $A \to \alpha \cdot$, где $A \to \alpha$~--- $k$-ое правило грамматики, $t \in FOLLOW(A)$

\subsection{CLR(1) алгоритм}

Canonical LR(1), он же LR(1).
Данный алгоритм является дальнейшим расширением SLR(1): к пунктам добавляются множества предпросмотра (lookahead).

\begin{definition}
Множество предпросмотра для правила $P$ --- терминалы, которые должны встретиться в выведенной строке сразу после строки, выводимой из данного правила.  
\qed
\end{definition}

\begin{definition}
CLR пункт: $ [A \to \alpha \cdot \beta, \{ t_0, \dots, t_n\}] $,
где $t_0, \dots, t_n$ --- множество пердпросмотра для правила $A \to \alpha \beta$.
\qed
\end{definition}


\begin{definition}
Пусть дана грамматика $G = \langle \Sigma, N, R, S\rangle$. 
\begin{align*}
 closure(X) = closure(X \cup \{&[B \to \cdot \delta, \{FIRST(\beta t_0), \dots, FIRST(\beta t_n)\}] \\
                               &\mid B \to \beta \in R, [A \to \alpha \cdot B \beta, \{t_0, \dots, t_n\}] \in closure(X)\})
\end{align*}
\qed
\end{definition}

Функция \textit{goto} определяется аналогично LR(0), автомат строится по тем же принципам.

При построении управляющей таблицы усиливается правило добавлеия команды \textit{redice}.
А именно, добавляем $r_k$ в ячейку $(j,t_i)$, если в состоянии $j$ есть пункт $[A \to \alpha \cdot, \{t_0, \dots, t_n\}]$, где $A \to \alpha$~--- $k$-ое правило грамматики.

\subsection{Примеры}

Расмотрим построение автоматов и таблиц для различных модефикаций LR алгоритма.

Возьмем следующую грамматику:
\begin{align*}
0) S & \rightarrow a S b S \\
1) S & \rightarrow \varepsilon
\end{align*}

Расширим вышеупомянутую грамматику, добавив новый стартовый нетерминал S', и далее будем работать с этой расширенной грамматикой:
\begin{align*}
0) & S \rightarrow a S b S \\
1) & S \rightarrow \varepsilon \\
2) & S' \rightarrow S \$
\end{align*}


\begin{example}
Пример ядра и замыкания. 

Возьмем правило 2 нашей грамматики, предположим, что мы только начинаем разбирать данное правило.

Ядром в таком случае является item исходного правила: $S' \rightarrow .S \$$

При замыкании добавятся ещё два item'a с правилами по выводу нетерминала 'S', поэтому получаем три item'a: $S' \rightarrow .S\$$, $S \rightarrow .aSbS$ и $S \rightarrow .\varepsilon$
\end{example}

\begin{example}
Пример построения LR(0)-автомата для нашей грамматики с применением замыкания.
\begin{enumerate}
\item Добавляем стартовое состояние: item правила 0 и его замыкание (вместо item'a $S \rightarrow .\varepsilon$ будем писать $S \rightarrow .$). \\ \\
\tikzset{every picture/.style={line width=0.75pt}}
\begin{tikzpicture}[x=0.75pt,y=0.75pt,yscale=-1,xscale=1]
%Rounded Rect
\draw   (139.33,32.07) .. controls (139.33,25.22) and (144.89,19.67) .. (151.74,19.67) -- (212.44,19.67) .. controls (219.29,19.67) and (224.85,25.22) .. (224.85,32.07) -- (224.85,69.3) .. controls (224.85,76.15) and (219.29,81.71) .. (212.44,81.71) -- (151.74,81.71) .. controls (144.89,81.71) and (139.33,76.15) .. (139.33,69.3) -- cycle ;
\draw (174.09,32.69) node  [align=left] {$S' \rightarrow .S\$$};
\draw (182.09,50.69) node  [align=left] {$S \rightarrow .aSbS$};
\draw (164,69) node  [align=left] {$S \rightarrow .$};
\draw (232.67,49.33) node  [align=left] {\textbf{{\small \textcolor{red}{0}}}};
\end{tikzpicture}

\item По 'S' добавляем переход из стартового состояния в новое состояние 1. \\ \\
\tikzset{every picture/.style={line width=0.75pt}}
\begin{tikzpicture}[x=0.75pt,y=0.75pt,yscale=-1,xscale=1]
%Rounded Rect
\draw   (139.33,32.07) .. controls (139.33,25.22) and (144.89,19.67) .. (151.74,19.67) -- (212.44,19.67) .. controls (219.29,19.67) and (224.85,25.22) .. (224.85,32.07) -- (224.85,69.3) .. controls (224.85,76.15) and (219.29,81.71) .. (212.44,81.71) -- (151.74,81.71) .. controls (144.89,81.71) and (139.33,76.15) .. (139.33,69.3) -- cycle ;
%Rounded Rect
\draw   (145.81,123.98) .. controls (145.81,121.26) and (148.02,119.05) .. (150.74,119.05) -- (211.87,119.05) .. controls (214.6,119.05) and (216.81,121.26) .. (216.81,123.98) -- (216.81,138.78) .. controls (216.81,141.51) and (214.6,143.72) .. (211.87,143.72) -- (150.74,143.72) .. controls (148.02,143.72) and (145.81,141.51) .. (145.81,138.78) -- cycle ;
%Straight Lines 
\draw    (180.81,81.92) -- (180.53,116.51) ;
\draw [shift={(180.51,118.51)}, rotate = 270.47] [color={rgb, 255:red, 0; green, 0; blue, 0 }  ][line width=0.75]    (10.93,-3.29) .. controls (6.95,-1.4) and (3.31,-0.3) .. (0,0) .. controls (3.31,0.3) and (6.95,1.4) .. (10.93,3.29)   ;
\draw (174.09,32.69) node  [align=left] {$S' \rightarrow .S\$$};
\draw (182.09,50.69) node  [align=left] {$S \rightarrow .aSbS$};
\draw (164,69) node  [align=left] {$S \rightarrow .$};
\draw (180.33,131.67) node  [align=left] {$S' \rightarrow S.\$$};
\draw (188,99) node  [align=left] {S};
\draw (232.67,49.33) node  [align=left] {\textbf{{\small \textcolor{red}{0}}}};
\draw (224.67,131.33) node  [align=left] {\textbf{{\small \textcolor{red}{1}}}};
\end{tikzpicture}

\item По '\$' добавляем переход из состояния 1 в новое состояние 2. \\ \\
\tikzset{every picture/.style={line width=0.75pt}}
\begin{tikzpicture}[x=0.75pt,y=0.75pt,yscale=-1,xscale=1]
%Rounded Rect
\draw   (139.33,32.07) .. controls (139.33,25.22) and (144.89,19.67) .. (151.74,19.67) -- (212.44,19.67) .. controls (219.29,19.67) and (224.85,25.22) .. (224.85,32.07) -- (224.85,69.3) .. controls (224.85,76.15) and (219.29,81.71) .. (212.44,81.71) -- (151.74,81.71) .. controls (144.89,81.71) and (139.33,76.15) .. (139.33,69.3) -- cycle ;
%Rounded Rect
\draw   (145.81,123.98) .. controls (145.81,121.26) and (148.02,119.05) .. (150.74,119.05) -- (211.87,119.05) .. controls (214.6,119.05) and (216.81,121.26) .. (216.81,123.98) -- (216.81,138.78) .. controls (216.81,141.51) and (214.6,143.72) .. (211.87,143.72) -- (150.74,143.72) .. controls (148.02,143.72) and (145.81,141.51) .. (145.81,138.78) -- cycle ;
%Rounded Rect
\draw   (145.81,185.98) .. controls (145.81,183.26) and (148.02,181.05) .. (150.74,181.05) -- (211.87,181.05) .. controls (214.6,181.05) and (216.81,183.26) .. (216.81,185.98) -- (216.81,200.78) .. controls (216.81,203.51) and (214.6,205.72) .. (211.87,205.72) -- (150.74,205.72) .. controls (148.02,205.72) and (145.81,203.51) .. (145.81,200.78) -- cycle ;
%Straight Lines 
\draw    (180.81,81.92) -- (180.53,116.51) ;
\draw [shift={(180.51,118.51)}, rotate = 270.47] [color={rgb, 255:red, 0; green, 0; blue, 0 }  ][line width=0.75]    (10.93,-3.29) .. controls (6.95,-1.4) and (3.31,-0.3) .. (0,0) .. controls (3.31,0.3) and (6.95,1.4) .. (10.93,3.29)   ;
%Straight Lines 
\draw    (180.81,143.92) -- (180.53,178.51) ;
\draw [shift={(180.51,180.51)}, rotate = 270.47] [color={rgb, 255:red, 0; green, 0; blue, 0 }  ][line width=0.75]    (10.93,-3.29) .. controls (6.95,-1.4) and (3.31,-0.3) .. (0,0) .. controls (3.31,0.3) and (6.95,1.4) .. (10.93,3.29)   ;
\draw (174.09,32.69) node  [align=left] {$S' \rightarrow .S\$$};
\draw (182.09,50.69) node  [align=left] {$S \rightarrow .aSbS$};
\draw (164,69) node  [align=left] {$S \rightarrow .$};
\draw (180.33,131.67) node  [align=left] {$S' \rightarrow S.\$$};
\draw (180.33,193.67) node  [align=left] {$S' \rightarrow S\$.$};
\draw (188,99) node  [align=left] {S};
\draw (188,161) node  [align=left] {\$};
\draw (232.67,49.33) node  [align=left] {\textbf{{\small \textcolor{red}{0}}}};
\draw (224.67,131.33) node  [align=left] {\textbf{{\small \textcolor{red}{1}}}};
\draw (224.67,193.33) node  [align=left] {\textbf{{\small \textcolor{red}{2}}}};
\end{tikzpicture}

\item По 'a' добавляем переход из стартового состояния в новое состояние 3 и делаем его замыкание. Также добавляем переход по 'a' из этого состояния в себя же. \\ \\
\tikzset{every picture/.style={line width=0.75pt}}
\begin{tikzpicture}[x=0.75pt,y=0.75pt,yscale=-1,xscale=1]
%Rounded Rect
\draw   (139.33,32.07) .. controls (139.33,25.22) and (144.89,19.67) .. (151.74,19.67) -- (212.44,19.67) .. controls (219.29,19.67) and (224.85,25.22) .. (224.85,32.07) -- (224.85,69.3) .. controls (224.85,76.15) and (219.29,81.71) .. (212.44,81.71) -- (151.74,81.71) .. controls (144.89,81.71) and (139.33,76.15) .. (139.33,69.3) -- cycle ;
%Rounded Rect
\draw   (145.81,123.98) .. controls (145.81,121.26) and (148.02,119.05) .. (150.74,119.05) -- (211.87,119.05) .. controls (214.6,119.05) and (216.81,121.26) .. (216.81,123.98) -- (216.81,138.78) .. controls (216.81,141.51) and (214.6,143.72) .. (211.87,143.72) -- (150.74,143.72) .. controls (148.02,143.72) and (145.81,141.51) .. (145.81,138.78) -- cycle ;
%Rounded Rect
\draw   (145.81,185.98) .. controls (145.81,183.26) and (148.02,181.05) .. (150.74,181.05) -- (211.87,181.05) .. controls (214.6,181.05) and (216.81,183.26) .. (216.81,185.98) -- (216.81,200.78) .. controls (216.81,203.51) and (214.6,205.72) .. (211.87,205.72) -- (150.74,205.72) .. controls (148.02,205.72) and (145.81,203.51) .. (145.81,200.78) -- cycle ;
%Rounded Rect 
\draw   (16.67,31.41) .. controls (16.67,24.56) and (22.22,19) .. (29.07,19) -- (89.77,19) .. controls (96.62,19) and (102.18,24.56) .. (102.18,31.41) -- (102.18,68.63) .. controls (102.18,75.48) and (96.62,81.04) .. (89.77,81.04) -- (29.07,81.04) .. controls (22.22,81.04) and (16.67,75.48) .. (16.67,68.63) -- cycle ;
%Straight Lines 
\draw    (180.81,81.92) -- (180.53,116.51) ;
\draw [shift={(180.51,118.51)}, rotate = 270.47] [color={rgb, 255:red, 0; green, 0; blue, 0 }  ][line width=0.75]    (10.93,-3.29) .. controls (6.95,-1.4) and (3.31,-0.3) .. (0,0) .. controls (3.31,0.3) and (6.95,1.4) .. (10.93,3.29)   ;
%Straight Lines 
\draw    (180.81,143.92) -- (180.53,178.51) ;
\draw [shift={(180.51,180.51)}, rotate = 270.47] [color={rgb, 255:red, 0; green, 0; blue, 0 }  ][line width=0.75]    (10.93,-3.29) .. controls (6.95,-1.4) and (3.31,-0.3) .. (0,0) .. controls (3.31,0.3) and (6.95,1.4) .. (10.93,3.29)   ;
%Straight Lines
\draw    (139.5,60.47) -- (122.5,60.47) -- (104.51,60.12) ;
\draw [shift={(102.51,60.09)}, rotate = 361.09000000000003] [color={rgb, 255:red, 0; green, 0; blue, 0 }  ][line width=0.75]    (10.93,-3.29) .. controls (6.95,-1.4) and (3.31,-0.3) .. (0,0) .. controls (3.31,0.3) and (6.95,1.4) .. (10.93,3.29)   ;
%Curve Lines
\draw    (99.51,23.34) .. controls (140.25,22.37) and (122.65,34.57) .. (104.22,40.43) ;
\draw [shift={(102.51,40.96)}, rotate = 343.53999999999996] [color={rgb, 255:red, 0; green, 0; blue, 0 }  ][line width=0.75]    (10.93,-3.29) .. controls (6.95,-1.4) and (3.31,-0.3) .. (0,0) .. controls (3.31,0.3) and (6.95,1.4) .. (10.93,3.29)   ;
\draw (174.09,32.69) node  [align=left] {$S' \rightarrow .S\$$};
\draw (182.09,50.69) node  [align=left] {$S \rightarrow .aSbS$};
\draw (164,69) node  [align=left] {$S \rightarrow .$};
\draw (180.33,131.67) node  [align=left] {$S' \rightarrow S.\$$};
\draw (180.33,193.67) node  [align=left] {$S' \rightarrow S\$.$};
\draw (59.33,32.33) node  [align=left] {$S \rightarrow a.SbS$};
\draw (59.42,50.02) node  [align=left] {$S \rightarrow .aSbS$};
\draw (41,68) node  [align=left] {$S \rightarrow .$};
\draw (188,99) node  [align=left] {S};
\draw (188,161) node  [align=left] {\$};
\draw (123,53) node  [align=left] {a};
\draw (130,26) node  [align=left] {a};
\draw (232.67,49.33) node  [align=left] {\textbf{{\small \textcolor{red}{0}}}};
\draw (224.67,131.33) node  [align=left] {\textbf{{\small \textcolor{red}{1}}}};
\draw (224.67,193.33) node  [align=left] {\textbf{{\small \textcolor{red}{2}}}};
\draw (109.67,50.33) node  [align=left] {\textbf{{\small \textcolor{red}{3}}}};
\end{tikzpicture}

\item По 'S' добавляем переход из состояния 3 в новое состояние 4. \\ \\
\tikzset{every picture/.style={line width=0.75pt}}
\begin{tikzpicture}[x=0.75pt,y=0.75pt,yscale=-1,xscale=1]
%Rounded Rect
\draw   (139.33,32.07) .. controls (139.33,25.22) and (144.89,19.67) .. (151.74,19.67) -- (212.44,19.67) .. controls (219.29,19.67) and (224.85,25.22) .. (224.85,32.07) -- (224.85,69.3) .. controls (224.85,76.15) and (219.29,81.71) .. (212.44,81.71) -- (151.74,81.71) .. controls (144.89,81.71) and (139.33,76.15) .. (139.33,69.3) -- cycle ;
%Rounded Rect
\draw   (145.81,123.98) .. controls (145.81,121.26) and (148.02,119.05) .. (150.74,119.05) -- (211.87,119.05) .. controls (214.6,119.05) and (216.81,121.26) .. (216.81,123.98) -- (216.81,138.78) .. controls (216.81,141.51) and (214.6,143.72) .. (211.87,143.72) -- (150.74,143.72) .. controls (148.02,143.72) and (145.81,141.51) .. (145.81,138.78) -- cycle ;
%Rounded Rect
\draw   (145.81,185.98) .. controls (145.81,183.26) and (148.02,181.05) .. (150.74,181.05) -- (211.87,181.05) .. controls (214.6,181.05) and (216.81,183.26) .. (216.81,185.98) -- (216.81,200.78) .. controls (216.81,203.51) and (214.6,205.72) .. (211.87,205.72) -- (150.74,205.72) .. controls (148.02,205.72) and (145.81,203.51) .. (145.81,200.78) -- cycle ;
%Rounded Rect 
\draw   (16.67,31.41) .. controls (16.67,24.56) and (22.22,19) .. (29.07,19) -- (89.77,19) .. controls (96.62,19) and (102.18,24.56) .. (102.18,31.41) -- (102.18,68.63) .. controls (102.18,75.48) and (96.62,81.04) .. (89.77,81.04) -- (29.07,81.04) .. controls (22.22,81.04) and (16.67,75.48) .. (16.67,68.63) -- cycle ;
%Rounded Rect
\draw   (16.67,102.98) .. controls (16.67,100.26) and (18.88,98.05) .. (21.6,98.05) -- (97.25,98.05) .. controls (99.97,98.05) and (102.18,100.26) .. (102.18,102.98) -- (102.18,117.78) .. controls (102.18,120.51) and (99.97,122.72) .. (97.25,122.72) -- (21.6,122.72) .. controls (18.88,122.72) and (16.67,120.51) .. (16.67,117.78) -- cycle ;
%Straight Lines 
\draw    (180.81,81.92) -- (180.53,116.51) ;
\draw [shift={(180.51,118.51)}, rotate = 270.47] [color={rgb, 255:red, 0; green, 0; blue, 0 }  ][line width=0.75]    (10.93,-3.29) .. controls (6.95,-1.4) and (3.31,-0.3) .. (0,0) .. controls (3.31,0.3) and (6.95,1.4) .. (10.93,3.29)   ;
%Straight Lines 
\draw    (180.81,143.92) -- (180.53,178.51) ;
\draw [shift={(180.51,180.51)}, rotate = 270.47] [color={rgb, 255:red, 0; green, 0; blue, 0 }  ][line width=0.75]    (10.93,-3.29) .. controls (6.95,-1.4) and (3.31,-0.3) .. (0,0) .. controls (3.31,0.3) and (6.95,1.4) .. (10.93,3.29)   ;
%Straight Lines
\draw    (139.5,60.47) -- (122.5,60.47) -- (104.51,60.12) ;
\draw [shift={(102.51,60.09)}, rotate = 361.09000000000003] [color={rgb, 255:red, 0; green, 0; blue, 0 }  ][line width=0.75]    (10.93,-3.29) .. controls (6.95,-1.4) and (3.31,-0.3) .. (0,0) .. controls (3.31,0.3) and (6.95,1.4) .. (10.93,3.29)   ;
%Curve Lines
\draw    (99.51,23.34) .. controls (140.25,22.37) and (122.65,34.57) .. (104.22,40.43) ;
\draw [shift={(102.51,40.96)}, rotate = 343.53999999999996] [color={rgb, 255:red, 0; green, 0; blue, 0 }  ][line width=0.75]    (10.93,-3.29) .. controls (6.95,-1.4) and (3.31,-0.3) .. (0,0) .. controls (3.31,0.3) and (6.95,1.4) .. (10.93,3.29)   ;
%Straight Lines
\draw    (59.51,81.05) -- (59.5,90) -- (59.5,96) ;
\draw [shift={(59.5,98)}, rotate = 270.03] [color={rgb, 255:red, 0; green, 0; blue, 0 }  ][line width=0.75]    (10.93,-3.29) .. controls (6.95,-1.4) and (3.31,-0.3) .. (0,0) .. controls (3.31,0.3) and (6.95,1.4) .. (10.93,3.29)   ;
\draw (174.09,32.69) node  [align=left] {$S' \rightarrow .S\$$};
\draw (182.09,50.69) node  [align=left] {$S \rightarrow .aSbS$};
\draw (164,69) node  [align=left] {$S \rightarrow .$};
\draw (180.33,131.67) node  [align=left] {$S' \rightarrow S.\$$};
\draw (180.33,193.67) node  [align=left] {$S' \rightarrow S\$.$};
\draw (59.33,32.33) node  [align=left] {$S \rightarrow a.SbS$};
\draw (59.42,50.02) node  [align=left] {$S \rightarrow .aSbS$};
\draw (41,68) node  [align=left] {$S \rightarrow .$};
\draw (59.33,110.67) node  [align=left] {$S \rightarrow aS.bS$};
\draw (188,99) node  [align=left] {S};
\draw (188,161) node  [align=left] {\$};
\draw (123,53) node  [align=left] {a};
\draw (130,26) node  [align=left] {a};
\draw (69,90) node  [align=left] {S};
\draw (232.67,49.33) node  [align=left] {\textbf{{\small \textcolor{red}{0}}}};
\draw (224.67,131.33) node  [align=left] {\textbf{{\small \textcolor{red}{1}}}};
\draw (224.67,193.33) node  [align=left] {\textbf{{\small \textcolor{red}{2}}}};
\draw (109.67,50.33) node  [align=left] {\textbf{{\small \textcolor{red}{3}}}};
\draw (109.67,110.33) node  [align=left] {\textbf{{\small \textcolor{red}{4}}}};
\end{tikzpicture}

\item По 'b' добавляем переход из состояния 4 в новое состояние 5 и делаем его замыкание. Также добавляем переход по 'a' из этого состояния в состояние 3. \\ \\
\tikzset{every picture/.style={line width=0.75pt}}
\begin{tikzpicture}[x=0.75pt,y=0.75pt,yscale=-1,xscale=1]
%Rounded Rect
\draw   (139.33,32.07) .. controls (139.33,25.22) and (144.89,19.67) .. (151.74,19.67) -- (212.44,19.67) .. controls (219.29,19.67) and (224.85,25.22) .. (224.85,32.07) -- (224.85,69.3) .. controls (224.85,76.15) and (219.29,81.71) .. (212.44,81.71) -- (151.74,81.71) .. controls (144.89,81.71) and (139.33,76.15) .. (139.33,69.3) -- cycle ;
%Rounded Rect
\draw   (145.81,123.98) .. controls (145.81,121.26) and (148.02,119.05) .. (150.74,119.05) -- (211.87,119.05) .. controls (214.6,119.05) and (216.81,121.26) .. (216.81,123.98) -- (216.81,138.78) .. controls (216.81,141.51) and (214.6,143.72) .. (211.87,143.72) -- (150.74,143.72) .. controls (148.02,143.72) and (145.81,141.51) .. (145.81,138.78) -- cycle ;
%Rounded Rect
\draw   (145.81,185.98) .. controls (145.81,183.26) and (148.02,181.05) .. (150.74,181.05) -- (211.87,181.05) .. controls (214.6,181.05) and (216.81,183.26) .. (216.81,185.98) -- (216.81,200.78) .. controls (216.81,203.51) and (214.6,205.72) .. (211.87,205.72) -- (150.74,205.72) .. controls (148.02,205.72) and (145.81,203.51) .. (145.81,200.78) -- cycle ;
%Rounded Rect 
\draw   (16.67,31.41) .. controls (16.67,24.56) and (22.22,19) .. (29.07,19) -- (89.77,19) .. controls (96.62,19) and (102.18,24.56) .. (102.18,31.41) -- (102.18,68.63) .. controls (102.18,75.48) and (96.62,81.04) .. (89.77,81.04) -- (29.07,81.04) .. controls (22.22,81.04) and (16.67,75.48) .. (16.67,68.63) -- cycle ;
%Rounded Rect
\draw   (16.67,102.98) .. controls (16.67,100.26) and (18.88,98.05) .. (21.6,98.05) -- (97.25,98.05) .. controls (99.97,98.05) and (102.18,100.26) .. (102.18,102.98) -- (102.18,117.78) .. controls (102.18,120.51) and (99.97,122.72) .. (97.25,122.72) -- (21.6,122.72) .. controls (18.88,122.72) and (16.67,120.51) .. (16.67,117.78) -- cycle ;
%Rounded Rect
\draw   (16.67,155.74) .. controls (16.67,148.89) and (22.22,143.33) .. (29.07,143.33) -- (89.77,143.33) .. controls (96.62,143.33) and (102.18,148.89) .. (102.18,155.74) -- (102.18,192.97) .. controls (102.18,199.82) and (96.62,205.37) .. (89.77,205.37) -- (29.07,205.37) .. controls (22.22,205.37) and (16.67,199.82) .. (16.67,192.97) -- cycle ;
%Straight Lines 
\draw    (180.81,81.92) -- (180.53,116.51) ;
\draw [shift={(180.51,118.51)}, rotate = 270.47] [color={rgb, 255:red, 0; green, 0; blue, 0 }  ][line width=0.75]    (10.93,-3.29) .. controls (6.95,-1.4) and (3.31,-0.3) .. (0,0) .. controls (3.31,0.3) and (6.95,1.4) .. (10.93,3.29)   ;
%Straight Lines 
\draw    (180.81,143.92) -- (180.53,178.51) ;
\draw [shift={(180.51,180.51)}, rotate = 270.47] [color={rgb, 255:red, 0; green, 0; blue, 0 }  ][line width=0.75]    (10.93,-3.29) .. controls (6.95,-1.4) and (3.31,-0.3) .. (0,0) .. controls (3.31,0.3) and (6.95,1.4) .. (10.93,3.29)   ;
%Straight Lines
\draw    (139.5,60.47) -- (122.5,60.47) -- (104.51,60.12) ;
\draw [shift={(102.51,60.09)}, rotate = 361.09000000000003] [color={rgb, 255:red, 0; green, 0; blue, 0 }  ][line width=0.75]    (10.93,-3.29) .. controls (6.95,-1.4) and (3.31,-0.3) .. (0,0) .. controls (3.31,0.3) and (6.95,1.4) .. (10.93,3.29)   ;
%Curve Lines
\draw    (99.51,23.34) .. controls (140.25,22.37) and (122.65,34.57) .. (104.22,40.43) ;
\draw [shift={(102.51,40.96)}, rotate = 343.53999999999996] [color={rgb, 255:red, 0; green, 0; blue, 0 }  ][line width=0.75]    (10.93,-3.29) .. controls (6.95,-1.4) and (3.31,-0.3) .. (0,0) .. controls (3.31,0.3) and (6.95,1.4) .. (10.93,3.29)   ;
%Straight Lines
\draw    (59.51,81.05) -- (59.5,90) -- (59.5,96) ;
\draw [shift={(59.5,98)}, rotate = 270.03] [color={rgb, 255:red, 0; green, 0; blue, 0 }  ][line width=0.75]    (10.93,-3.29) .. controls (6.95,-1.4) and (3.31,-0.3) .. (0,0) .. controls (3.31,0.3) and (6.95,1.4) .. (10.93,3.29)   ;
%Straight Lines  
\draw    (60.51,123.05) -- (60.5,132) -- (60.5,141.08) ;
\draw [shift={(60.5,143.08)}, rotate = 270] [color={rgb, 255:red, 0; green, 0; blue, 0 }  ][line width=0.75]    (10.93,-3.29) .. controls (6.95,-1.4) and (3.31,-0.3) .. (0,0) .. controls (3.31,0.3) and (6.95,1.4) .. (10.93,3.29)   ;
%Curve Lines
\draw    (100.5,149.38) .. controls (127.95,117.68) and (124.65,99.76) .. (100.98,78.35) ;
\draw [shift={(99.51,77.03)}, rotate = 401.35] [color={rgb, 255:red, 0; green, 0; blue, 0 }  ][line width=0.75]    (10.93,-3.29) .. controls (6.95,-1.4) and (3.31,-0.3) .. (0,0) .. controls (3.31,0.3) and (6.95,1.4) .. (10.93,3.29)   ;
\draw (174.09,32.69) node  [align=left] {$S' \rightarrow .S\$$};
\draw (182.09,50.69) node  [align=left] {$S \rightarrow .aSbS$};
\draw (164,69) node  [align=left] {$S \rightarrow .$};
\draw (180.33,131.67) node  [align=left] {$S' \rightarrow S.\$$};
\draw (180.33,193.67) node  [align=left] {$S' \rightarrow S\$.$};
\draw (59.33,32.33) node  [align=left] {$S \rightarrow a.SbS$};
\draw (59.42,50.02) node  [align=left] {$S \rightarrow .aSbS$};
\draw (41,68) node  [align=left] {$S \rightarrow .$};
\draw (59.33,110.67) node  [align=left] {$S \rightarrow aS.bS$};
\draw (59.33,156.67) node  [align=left] {$S \rightarrow aSb.S$};
\draw (59.42,174.35) node  [align=left] {$S \rightarrow .aSbS$};
\draw (41,192) node  [align=left] {$S \rightarrow .$};
\draw (188,99) node  [align=left] {S};
\draw (188,161) node  [align=left] {\$};
\draw (123,53) node  [align=left] {a};
\draw (130,26) node  [align=left] {a};
\draw (69,90) node  [align=left] {S};
\draw (70,133) node  [align=left] {b};
\draw (126,109) node  [align=left] {a};
\draw (232.67,49.33) node  [align=left] {\textbf{{\small \textcolor{red}{0}}}};
\draw (224.67,131.33) node  [align=left] {\textbf{{\small \textcolor{red}{1}}}};
\draw (224.67,193.33) node  [align=left] {\textbf{{\small \textcolor{red}{2}}}};
\draw (109.67,50.33) node  [align=left] {\textbf{{\small \textcolor{red}{3}}}};
\draw (109.67,110.33) node  [align=left] {\textbf{{\small \textcolor{red}{4}}}};
\draw (110.67,175.33) node  [align=left] {\textbf{{\small \textcolor{red}{5}}}};
\end{tikzpicture}

\item По 'S' добавляем переход из состояния 5 в новое состояние 6. Завершаем построение LR-автомата. \\ \\
\tikzset{every picture/.style={line width=0.75pt}}
\begin{tikzpicture}[x=0.75pt,y=0.75pt,yscale=-1,xscale=1]
%Rounded Rect
\draw   (139.33,32.07) .. controls (139.33,25.22) and (144.89,19.67) .. (151.74,19.67) -- (212.44,19.67) .. controls (219.29,19.67) and (224.85,25.22) .. (224.85,32.07) -- (224.85,69.3) .. controls (224.85,76.15) and (219.29,81.71) .. (212.44,81.71) -- (151.74,81.71) .. controls (144.89,81.71) and (139.33,76.15) .. (139.33,69.3) -- cycle ;
%Rounded Rect 
\draw   (16.67,31.41) .. controls (16.67,24.56) and (22.22,19) .. (29.07,19) -- (89.77,19) .. controls (96.62,19) and (102.18,24.56) .. (102.18,31.41) -- (102.18,68.63) .. controls (102.18,75.48) and (96.62,81.04) .. (89.77,81.04) -- (29.07,81.04) .. controls (22.22,81.04) and (16.67,75.48) .. (16.67,68.63) -- cycle ;
%Rounded Rect
\draw   (16.67,155.74) .. controls (16.67,148.89) and (22.22,143.33) .. (29.07,143.33) -- (89.77,143.33) .. controls (96.62,143.33) and (102.18,148.89) .. (102.18,155.74) -- (102.18,192.97) .. controls (102.18,199.82) and (96.62,205.37) .. (89.77,205.37) -- (29.07,205.37) .. controls (22.22,205.37) and (16.67,199.82) .. (16.67,192.97) -- cycle ;
%Rounded Rect
\draw   (16.67,102.98) .. controls (16.67,100.26) and (18.88,98.05) .. (21.6,98.05) -- (97.25,98.05) .. controls (99.97,98.05) and (102.18,100.26) .. (102.18,102.98) -- (102.18,117.78) .. controls (102.18,120.51) and (99.97,122.72) .. (97.25,122.72) -- (21.6,122.72) .. controls (18.88,122.72) and (16.67,120.51) .. (16.67,117.78) -- cycle ;
%Rounded Rect
\draw   (16.67,230.65) .. controls (16.67,227.93) and (18.88,225.72) .. (21.6,225.72) -- (97.25,225.72) .. controls (99.97,225.72) and (102.18,227.93) .. (102.18,230.65) -- (102.18,245.45) .. controls (102.18,248.18) and (99.97,250.38) .. (97.25,250.38) -- (21.6,250.38) .. controls (18.88,250.38) and (16.67,248.18) .. (16.67,245.45) -- cycle ;
%Rounded Rect
\draw   (145.81,123.98) .. controls (145.81,121.26) and (148.02,119.05) .. (150.74,119.05) -- (211.87,119.05) .. controls (214.6,119.05) and (216.81,121.26) .. (216.81,123.98) -- (216.81,138.78) .. controls (216.81,141.51) and (214.6,143.72) .. (211.87,143.72) -- (150.74,143.72) .. controls (148.02,143.72) and (145.81,141.51) .. (145.81,138.78) -- cycle ;
%Rounded Rect
\draw   (145.81,185.98) .. controls (145.81,183.26) and (148.02,181.05) .. (150.74,181.05) -- (211.87,181.05) .. controls (214.6,181.05) and (216.81,183.26) .. (216.81,185.98) -- (216.81,200.78) .. controls (216.81,203.51) and (214.6,205.72) .. (211.87,205.72) -- (150.74,205.72) .. controls (148.02,205.72) and (145.81,203.51) .. (145.81,200.78) -- cycle ;
%Straight Lines 
\draw    (180.81,81.92) -- (180.53,116.51) ;
\draw [shift={(180.51,118.51)}, rotate = 270.47] [color={rgb, 255:red, 0; green, 0; blue, 0 }  ][line width=0.75]    (10.93,-3.29) .. controls (6.95,-1.4) and (3.31,-0.3) .. (0,0) .. controls (3.31,0.3) and (6.95,1.4) .. (10.93,3.29)   ;
%Straight Lines 
\draw    (180.81,143.92) -- (180.53,178.51) ;
\draw [shift={(180.51,180.51)}, rotate = 270.47] [color={rgb, 255:red, 0; green, 0; blue, 0 }  ][line width=0.75]    (10.93,-3.29) .. controls (6.95,-1.4) and (3.31,-0.3) .. (0,0) .. controls (3.31,0.3) and (6.95,1.4) .. (10.93,3.29)   ;
%Straight Lines
\draw    (59.51,81.05) -- (59.5,90) -- (59.5,96) ;
\draw [shift={(59.5,98)}, rotate = 270.03] [color={rgb, 255:red, 0; green, 0; blue, 0 }  ][line width=0.75]    (10.93,-3.29) .. controls (6.95,-1.4) and (3.31,-0.3) .. (0,0) .. controls (3.31,0.3) and (6.95,1.4) .. (10.93,3.29)   ;
%Straight Lines  
\draw    (60.51,123.05) -- (60.5,132) -- (60.5,141.08) ;
\draw [shift={(60.5,143.08)}, rotate = 270] [color={rgb, 255:red, 0; green, 0; blue, 0 }  ][line width=0.75]    (10.93,-3.29) .. controls (6.95,-1.4) and (3.31,-0.3) .. (0,0) .. controls (3.31,0.3) and (6.95,1.4) .. (10.93,3.29)   ;
%Straight Lines
\draw    (60.51,205.05) -- (60.5,214) -- (60.5,223.62) ;
\draw [shift={(60.5,225.62)}, rotate = 270.02] [color={rgb, 255:red, 0; green, 0; blue, 0 }  ][line width=0.75]    (10.93,-3.29) .. controls (6.95,-1.4) and (3.31,-0.3) .. (0,0) .. controls (3.31,0.3) and (6.95,1.4) .. (10.93,3.29)   ;
%Straight Lines
\draw    (139.5,60.47) -- (122.5,60.47) -- (104.51,60.12) ;
\draw [shift={(102.51,60.09)}, rotate = 361.09000000000003] [color={rgb, 255:red, 0; green, 0; blue, 0 }  ][line width=0.75]    (10.93,-3.29) .. controls (6.95,-1.4) and (3.31,-0.3) .. (0,0) .. controls (3.31,0.3) and (6.95,1.4) .. (10.93,3.29)   ;
%Curve Lines
\draw    (99.51,23.34) .. controls (140.25,22.37) and (122.65,34.57) .. (104.22,40.43) ;
\draw [shift={(102.51,40.96)}, rotate = 343.53999999999996] [color={rgb, 255:red, 0; green, 0; blue, 0 }  ][line width=0.75]    (10.93,-3.29) .. controls (6.95,-1.4) and (3.31,-0.3) .. (0,0) .. controls (3.31,0.3) and (6.95,1.4) .. (10.93,3.29)   ;
%Curve Lines
\draw    (100.5,149.38) .. controls (127.95,117.68) and (124.65,99.76) .. (100.98,78.35) ;
\draw [shift={(99.51,77.03)}, rotate = 401.35] [color={rgb, 255:red, 0; green, 0; blue, 0 }  ][line width=0.75]    (10.93,-3.29) .. controls (6.95,-1.4) and (3.31,-0.3) .. (0,0) .. controls (3.31,0.3) and (6.95,1.4) .. (10.93,3.29)   ;
\draw (174.09,32.69) node  [align=left] {$S' \rightarrow .S\$$};
\draw (59.33,32.33) node  [align=left] {$S \rightarrow a.SbS$};
\draw (59.33,156.67) node  [align=left] {$S \rightarrow aSb.S$};
\draw (59.33,110.67) node  [align=left] {$S \rightarrow aS.bS$};
\draw (59.33,238.33) node  [align=left] {$S \rightarrow aSbS.$};
\draw (180.33,131.67) node  [align=left] {$S' \rightarrow S.\$$};
\draw (180.33,193.67) node  [align=left] {$S' \rightarrow S\$.$};
\draw (59.42,50.02) node  [align=left] {$S \rightarrow .aSbS$};
\draw (41,68) node  [align=left] {$S \rightarrow .$};
\draw (182.09,50.69) node  [align=left] {$S \rightarrow .aSbS$};
\draw (164,69) node  [align=left] {$S \rightarrow .$};
\draw (59.42,174.35) node  [align=left] {$S \rightarrow .aSbS$};
\draw (41,192) node  [align=left] {$S \rightarrow .$};
\draw (126,109) node  [align=left] {a};
\draw (70,133) node  [align=left] {b};
\draw (123,53) node  [align=left] {a};
\draw (130,26) node  [align=left] {a};
\draw (69,90) node  [align=left] {S};
\draw (70,216) node  [align=left] {S};
\draw (188,99) node  [align=left] {S};
\draw (188,161) node  [align=left] {\$};
\draw (232.67,49.33) node  [align=left] {\textbf{{\small \textcolor{red}{0}}}};
\draw (109.67,50.33) node  [align=left] {\textbf{{\small \textcolor{red}{3}}}};
\draw (110.67,175.33) node  [align=left] {\textbf{{\small \textcolor{red}{5}}}};
\draw (224.67,131.33) node  [align=left] {\textbf{{\small \textcolor{red}{1}}}};
\draw (224.67,193.33) node  [align=left] {\textbf{{\small \textcolor{red}{2}}}};
\draw (109.67,110.33) node  [align=left] {\textbf{{\small \textcolor{red}{4}}}};
\draw (109.67,238.33) node  [align=left] {\textbf{{\small \textcolor{red}{6}}}};
\end{tikzpicture}
\end{enumerate}
\end{example}

Далее будем использовать этот автомат для построения управляющей таблицы.

\begin{example}
Пример управляющей LR(0) таблицы.

\begin{tabular}{|c|c|c|c||c|} 
    \hline   & a            & b   & \$  & S \\ [0.5ex]
    \hline 0 & $s_3$, $r_1$ & $r_1$ & $r_1$ & 1 \\
    \hline 1 &              &       & acc   &   \\
    \hline 2 & $r_2$        & $r_2$ & $r_2$ &   \\
    \hline 3 & $s_3$, $r_1$ & $r_1$ & $r_1$ & 4 \\
    \hline 4 &              & $s_5$ &       &   \\
    \hline 5 & $s_3, r_1$   & $r_1$ & $r_1$ & 6 \\
    \hline 6 & $r_0$        & $r_0$ & $r_0$ &   \\ [1ex] 
    \hline
\end{tabular}

Как видим, в данном случае в таблице присутствуют shift-reduce конфликты. В случае, когда не удаётся построить таблицу без конфликтов, говорят, что грамматика не LR(0).

\qed
\end{example}


\begin{example}
Пример управляющей LR(1) таблицы. Автомат тот же, однако команды \textit{reduce} расставляются с использованием FOLLOW. 

$$
\textit{FOLLOW}_1(S) = \{b, \$\}
$$

\begin{tabular}{|c|c|c|c||c|} 
    \hline   & a        & b     & \$    & S \\ [0.5ex]
    \hline 0 & $s_3$    & $r_1$ & $r_1$ & 1 \\
    \hline 1 &          &       & acc   &   \\
    \hline 2 &          &       &       &   \\
    \hline 3 & $s_3$    & $r_1$ & $r_1$ & 4 \\
    \hline 4 &          & $s_5$ &       &   \\
    \hline 5 & $s_3$    & $r_1$ & $r_1$ & 6 \\
    \hline 6 &          & $r_0$ & $r_0$ &   \\ [1ex] 
    \hline
\end{tabular}

В данном случае в таблице отсутствуют shift-reduce конфликты. То есть наша грамматика SLR(1), но не LR(0).

\qed
\end{example}

\begin{example}
Пример LR-разбора входного слова abab\$ из языка нашей грамматики с использованием построенных ранее LR-автомата и управляющей таблицы.
\begin{enumerate}
\item Начало разбора. На стеке --- стартовое состояние 0. \\ \\
Вход: \,
\begin{tabular}[c]{ |c|c|c|c|c| } 
    \hline \textcolor{red}{a} & b & a & b & \$ \\ \hline
\end{tabular} \\
Стек: \,
\begin{tabular}[c]{ |c|c } 
    \hline 0 & \\ \hline
\end{tabular}  
\\
\item Выполняем shift 3: сдвигаем указатель на входе, кладем на стек 'a', новое состояние 3 и переходим в него. \\ \\
Вход: \,
\begin{tabular}[c]{ |c|c|c|c|c| } 
    \hline a & \textcolor{red}{b} & a & b & \$ \\ \hline
\end{tabular} \\
Стек: \,
\begin{tabular}[c]{ |c|c|c|c } 
    \hline 0 & a & 3 & \\ \hline
\end{tabular}
\\ 
\item Выполняем reduce 1 (кладем на стек 'S'), кладем новое состояние 4 и переходим в него. \\ \\
Вход: \,
\begin{tabular}[c]{ |c|c|c|c|c| } 
    \hline a & \textcolor{red}{b} & a & b & \$ \\ \hline
\end{tabular} \\
Стек: \,
\begin{tabular}[c]{ |c|c|c|c|c|c } 
    \hline 0 & a & 3 & S & 4 & \\ \hline
\end{tabular}
\\ 
\item Выполняем shift 5: сдвигаем указатель на входе, кладем на стек 'b', новое состояние 5 и переходим в него. \\ \\
Вход: \,
\begin{tabular}[c]{ |c|c|c|c|c| } 
    \hline a & b & \textcolor{red}{a} & b & \$ \\ \hline
\end{tabular}\\
Стек: \,
\begin{tabular}[c]{ |c|c|c|c|c|c|c|c } 
    \hline 0 & a & 3 & S & 4 & b & 5 & \\ \hline
\end{tabular}
\\
\item Выполняем shift 3. \\ \\
Вход: \,
\begin{tabular}[c]{ |c|c|c|c|c| } 
    \hline a & b & a & \textcolor{red}{b} & \$ \\ \hline
\end{tabular} \\
Стек: \,
\begin{tabular}[c]{ |c|c|c|c|c|c|c|c|c|c } 
    \hline 0 & a & 3 & S & 4 & b & 5 & a & 3 & \\ \hline
\end{tabular}
\\
\item Выполняем reduce 1, кладем новое состояние 4 и переходим в него. \\ \\
Вход: \,
\begin{tabular}[c]{ |c|c|c|c|c| } 
    \hline a & b & a & \textcolor{red}{b} & \$ \\ \hline
\end{tabular}\\
Стек: \,
\begin{tabular}[c]{ |c|c|c|c|c|c|c|c|c|c|c|c } 
    \hline 0 & a & 3 & S & 4 & b & 5 & a & 3 & S & 4 & \\ \hline
\end{tabular}
\\
\item Выполняем shift 5. \\ \\
Вход: \,
\begin{tabular}[c]{ |c|c|c|c|c| } 
    \hline a & b & a & b & \textcolor{red}{\$} \\ \hline
\end{tabular} \\
Стек: \,
\begin{tabular}[c]{ |c|c|c|c|c|c|c|c|c|c|c|c|c|c } 
    \hline 0 & a & 3 & S & 4 & b & 5 & a & 3 & S & 4 & b & 5 & \\ \hline
\end{tabular}
\\
\item Выполняем reduce 1, кладем новое состояние 6 и переходим в него. \\ \\
Вход: \,
\begin{tabular}[c]{ |c|c|c|c|c| } 
    \hline a & b & a & b & \textcolor{red}{\$} \\ \hline
\end{tabular} \\
Стек: \,
\begin{tabular}[c]{ |c|c|c|c|c|c|c|c|c|c|c|c|c|c|c|c } 
    \hline 0 & a & 3 & S & 4 & b & 5 & a & 3 & S & 4 & b & 5 & S & 6 & \\ \hline
\end{tabular}
\\
\item Выполняем reduce 0 (снимаем со стека 8 элементов и кладем 'S'), оказываемся в состоянии 5 и делаем переход в новое состояние 6 с добавлением его на стек. \\ \\
Вход: \,
\begin{tabular}[c]{ |c|c|c|c|c| } 
    \hline a & b & a & b & \textcolor{red}{\$} \\ \hline
\end{tabular}\\
Стек: \,
\begin{tabular}[c]{ |c|c|c|c|c|c|c|c|c|c } 
    \hline 0 & a & 3 & S & 4 & b & 5 & S & 6 & \\ \hline
\end{tabular}
\\
\item Снова выполняем reduce 0, оказываемся в состоянии 0 и делаем переход в новое состояние 1 с добавлением его на стек. Заканчиваем разбор. \\ \\
Вход: \,
\begin{tabular}[c]{ |c|c|c|c|c| } 
    \hline a & b & a & b & \textcolor{red}{\$} \\ \hline
\end{tabular} \\
Стек: \,
\begin{tabular}[c]{ |c|c|c|c } 
    \hline 0 & S & 1 & \\ \hline
\end{tabular}
\end{enumerate}
\qed
\end{example}

\begin{example}
Пример CLR автомата.

\begin{center}
    \begin{tikzpicture}[> = stealth,node distance=3.25cm, on grid, scale=0.8, every node/.style={scale=0.8}]
      \node[draw=none, fill=none] at (-1.4, 1.2)  {0};
      \node[draw=none, fill=none] at (3.1, 0.55)  {3};
      \node[draw=none, fill=none] at (-1.4, -1.3) {1};
      \node[draw=none, fill=none] at (-1.4, -3.8) {2};
      \node[draw=none, fill=none] at (2.9, -1.95) {4};
      \node[draw=none, fill=none] at (2.9, -4.45) {5};
      \node[draw=none, fill=none] at (6.7, -1.3)  {6};
      \node[draw=none, fill=none] at (6.7, -3.8)  {7};
      \node[draw=none, fill=none] at (11, -1.95)  {8};
      \node[draw=none, fill=none] at (11, -4.45)  {9};
      \node[r_state] (s_0)
      {
        $
        \begin{aligned}
          S' &\to \cdot S, \{\$\}  \\ 
          S  &\to \cdot, \{\$\}\\
          S  &\to \cdot a S b S, \{\$\}
        \end{aligned}
        $
      };
      \node[r_state] (s_1) [below=2cm of s_0] 
      {
        $
        \begin{aligned}
          S  &\to a \cdot S b S, \{\$\} \\ 
          S  &\to \cdot, \{b\} \\
          S  &\to \cdot a S b S, \{b\}
        \end{aligned}
        $
      };
      \node[r_state] (s_2) [below=2cm of s_1] 
      {
        $
        \begin{aligned}
          S  &\to a \cdot S b S, \{b\} \\ 
          S  &\to \cdot, \{b\} \\ 
          S  &\to \cdot a S b S, \{b\}
        \end{aligned}
        $
      };
      \node[r_state] (s_3) [right=of s_0] 
      {
        $ S' \to S \cdot, \{\$\} $
      };
      \node[r_state] (s_4) [right=of s_1]
      {
        $ S \to a S \cdot b S, \{\$\} $
      };
      \node[r_state] (s_5) [right=of s_2] 
      {
        $ S \to a S \cdot b S, \{b\} $
      };
      \node[r_state] (s_6) [right=of s_4] 
      {
        $
        \begin{aligned}
          S  &\to a S b \cdot S, \{\$\} \\ 
          S  &\to \cdot, \{\$\} \\
          S  &\to \cdot a S b S, \{\$\}
        \end{aligned}
        $
      };
      \node[r_state] (s_7) [right=of s_5]
      {
        $
        \begin{aligned}
          S  &\to a S b \cdot S, \{b\} \\ 
          S  &\to \cdot , \{b\} \\
          S  &\to \cdot a S b S, \{b\}
        \end{aligned}
        $
      };
      \node[r_state] (s_8) [right=of s_6] 
      {
        $ S \to a S b S \cdot, \{\$\} $
      };
      \node[r_state] (s_9) [right=of s_7] 
      {
        $ S \to a S b S \cdot, \{b\} $
      };
      
      \path[->] 
        (s_0) edge [left]                 node {$a$} (s_1)
              edge [above]                node {$S$} (s_3)
        (s_1) edge [left]                 node {$a$} (s_2)
              edge [above]                node {$S$} (s_4)
        (s_2) edge [loop below]           node {$a$} ()
              edge [above]                node {$S$} (s_5)
        (s_4) edge [above]                node {$b$} (s_6)
        (s_5) edge [above]                node {$b$} (s_7)
        (s_6) edge [above]                node {$S$} (s_8)
              edge [above, bend right=20] node {$a$} (s_1)
        (s_7) edge [above]                node {$S$} (s_9)
              edge [below, bend left=20]  node {$a$} (s_2)
        ;
    \end{tikzpicture}
  \end{center}
\qed
\end{example}


Существуют и другие модификации, например LALR(1)

На практике конфликты стараются решать ещё и на этапе генерации.
Прикладные инструменты могут сгенерировать парсер по неоднозначной грамматике: из переноса или свёртки выбирать перенос, из нескольких свёрток --- первую в каком-то порядке (обычно в порядке появления соответствующих продукций в грамматике).

\subsection{Сравнение классов LL и LR}

Иерархию языков, распозноваемых различными классами алгоритмов, можно представить следующим образом.
\begin{center}
\includegraphics[width=0.6\textwidth]{pics/LL_LR.pdf}
\end{center}

Из диаграмы видно, что класс яыков, распозноваемых LL(k) алгоритмом уже, чем класс языков, распозноваемый LR(k) алгоритмом, при любом конечном $k$. Приведём несколько примеров.
\begin{enumerate}
\item $L = \{a^mb^nc \mid m \geq n \geq 0\} $ является LR(0), но для него не существует LL(1) граммтики. 
\item $L = \{ a^n b^n + a^n c^n \mid n > 0\}$ является LR, но не LL.
\item Больше примеров можно найти в работе Джона Битти~\cite{BEATTY1980193}.
\end{enumerate}

\section{GLR и его применение для КС запросов}

Алгоритм LR довольно эффективен, однако позволяет работать не со всеми КС-грамматиками, а только с их подмножеством LR(k). Если грамматика находится за рамками допускаемого класса, некоторые ячейки управляющей таблицы могут содержать несколько значений. В этом случае грамматика отвергалась анализатором.

Чтобы допустить множестенные значения в ячейках управляющей таблицы, потребуется некоторый вид недетерминизма, который даст возможность анализатору обрабатывать несколько возможных вариантов синтаксического разбора параллельно. Именно это и предлагает анализатор Generalized LR (GLR)~\cite{tomita-1987-efficient}. Далее мы рассмотрим общий принцип работы, проиллюстрируем его с помощью примера, а также рассмотрим модификации GLR.

\subsection{Классический GLR алгоритм}

Впервые GLR парсер был представлен Масару Томитой в 1987~\cite{tomita-1987-efficient}. В целом, алгоритм работы идентичен LR той разницей, что управляющая таблица модифицирована таким образом, чтобы допускать множественные значения в ячейках. Интерпретатор автомата изменён соответствующим образом.

Для того, чтобы избежать дублирования информации при обработке неоднозначностей, стоит использовать более сложную структуру стека: \textit{граф-структурированный стек} или (\textit{GSS}, Graph Structured Stack). Это направленный граф, в котором вершины соответствуют элементам стека, а ребра их соединяют по правилам управляющей таблицы. У каждой вершины может быть несколько входящих и исходящих дуг: таким образом реализуется то объединение одинаковых состяний и ветвление в случае неоднозначности.

\begin{example}
    \label{glr:example}
    Рассмотрим пример GLR разбора с использованием GSS.
    
    Возьмем грамматику $G$ следующего вида:
    \begin{align*}
        &0.\quad S' \to S\$ \\
        &1.\quad S \to abC \\
        &2.\quad S \to aBC \\
        &3.\quad B \to b \\
        &4.\quad C \to c 
    \end{align*}
    
    Входное слово $ w $:
    \begin{align*}
        w = abc\$
    \end{align*}
    
    Построим для данной грамматики LR автомат:
    
    \begin{tikzpicture}[x=0.75pt,y=0.75pt,yscale=-1,xscale=1]
    %uncomment if require: \path (0,306); %set diagram left start at 0, and has height of 306
    
    
    % Text Node
    \draw    (21.5,33) .. controls (21.5,30.24) and (23.74,28) .. (26.5,28) -- (94.5,28) .. controls (97.26,28) and (99.5,30.24) .. (99.5,33) -- (99.5,89) .. controls (99.5,91.76) and (97.26,94) .. (94.5,94) -- (26.5,94) .. controls (23.74,94) and (21.5,91.76) .. (21.5,89) -- cycle  ;
    \draw (60.5,61) node [color={rgb, 255:red, 62; green, 45; blue, 45 }  ,opacity=1 ]  {$ \begin{array}{l}
        S'\rightarrow .S\$\\
        S\rightarrow .abC\\
        S\rightarrow .aBC
        \end{array}$};
    % Text Node
    \draw    (25,141) .. controls (25,138.24) and (27.24,136) .. (30,136) -- (91,136) .. controls (93.76,136) and (96,138.24) .. (96,141) -- (96,156) .. controls (96,158.76) and (93.76,161) .. (91,161) -- (30,161) .. controls (27.24,161) and (25,158.76) .. (25,156) -- cycle  ;
    \draw (60.5,148.5) node   {$S'\rightarrow S.\$$};
    % Text Node
    \draw (446,39) node [scale=0.9,color={rgb, 255:red, 255; green, 0; blue, 0 }  ,opacity=1 ]  {$5$};
    % Text Node
    \draw (93,18) node [scale=0.9,color={rgb, 255:red, 255; green, 0; blue, 0 }  ,opacity=1 ]  {$0$};
    % Text Node
    \draw (213,18) node [scale=0.9,color={rgb, 255:red, 255; green, 0; blue, 0 }  ,opacity=1 ]  {$2$};
    % Text Node
    \draw (89,126) node [scale=0.9,color={rgb, 255:red, 255; green, 0; blue, 0 }  ,opacity=1 ]  {$1$};
    % Text Node
    \draw (330,18) node [scale=0.9,color={rgb, 255:red, 255; green, 0; blue, 0 }  ,opacity=1 ]  {$3$};
    % Text Node
    \draw (212,117) node [scale=0.9,color={rgb, 255:red, 255; green, 0; blue, 0 }  ,opacity=1 ]  {$4$};
    % Text Node
    \draw    (141.5,33) .. controls (141.5,30.24) and (143.74,28) .. (146.5,28) -- (214.5,28) .. controls (217.26,28) and (219.5,30.24) .. (219.5,33) -- (219.5,89) .. controls (219.5,91.76) and (217.26,94) .. (214.5,94) -- (146.5,94) .. controls (143.74,94) and (141.5,91.76) .. (141.5,89) -- cycle  ;
    \draw (180.5,61) node [color={rgb, 255:red, 62; green, 45; blue, 45 }  ,opacity=1 ]  {$ \begin{array}{l}
        S\rightarrow a.BC\\
        S\rightarrow a.bC\\
        B\rightarrow .b
        \end{array}$};
    % Text Node
    \draw (121,51) node [scale=0.9,color={rgb, 255:red, 0; green, 0; blue, 0 }  ,opacity=1 ]  {$a$};
    % Text Node
    \draw    (261.5,33) .. controls (261.5,30.24) and (263.74,28) .. (266.5,28) -- (332.5,28) .. controls (335.26,28) and (337.5,30.24) .. (337.5,33) -- (337.5,89) .. controls (337.5,91.76) and (335.26,94) .. (332.5,94) -- (266.5,94) .. controls (263.74,94) and (261.5,91.76) .. (261.5,89) -- cycle  ;
    \draw (299.5,61) node [color={rgb, 255:red, 62; green, 45; blue, 45 }  ,opacity=1 ]  {$ \begin{array}{l}
        S\rightarrow ab.C\\
        B\rightarrow b.\\
        C\rightarrow .c
        \end{array}$};
    % Text Node
    \draw    (25,209) .. controls (25,206.24) and (27.24,204) .. (30,204) -- (91,204) .. controls (93.76,204) and (96,206.24) .. (96,209) -- (96,224) .. controls (96,226.76) and (93.76,229) .. (91,229) -- (30,229) .. controls (27.24,229) and (25,226.76) .. (25,224) -- cycle  ;
    \draw (60.5,216.5) node   {$S'\rightarrow S\$.$};
    % Text Node
    \draw (69,111) node [scale=0.9,color={rgb, 255:red, 0; green, 0; blue, 0 }  ,opacity=1 ]  {$S$};
    % Text Node
    \draw (69,178) node [scale=0.9,color={rgb, 255:red, 0; green, 0; blue, 0 }  ,opacity=1 ]  {$\$$};
    % Text Node
    \draw    (378,53.5) .. controls (378,50.74) and (380.24,48.5) .. (383,48.5) -- (449,48.5) .. controls (451.76,48.5) and (454,50.74) .. (454,53.5) -- (454,68.5) .. controls (454,71.26) and (451.76,73.5) .. (449,73.5) -- (383,73.5) .. controls (380.24,73.5) and (378,71.26) .. (378,68.5) -- cycle  ;
    \draw (416,61) node [color={rgb, 255:red, 62; green, 45; blue, 45 }  ,opacity=1 ]  {$S\rightarrow abC.$};
    % Text Node
    \draw (240,51) node [scale=0.9,color={rgb, 255:red, 0; green, 0; blue, 0 }  ,opacity=1 ]  {$b$};
    % Text Node
    \draw (356,51) node [scale=0.9,color={rgb, 255:red, 0; green, 0; blue, 0 }  ,opacity=1 ]  {$C$};
    % Text Node
    \draw    (142,132) .. controls (142,129.24) and (144.24,127) .. (147,127) -- (215,127) .. controls (217.76,127) and (220,129.24) .. (220,132) -- (220,167) .. controls (220,169.76) and (217.76,172) .. (215,172) -- (147,172) .. controls (144.24,172) and (142,169.76) .. (142,167) -- cycle  ;
    \draw (181,149.5) node [color={rgb, 255:red, 62; green, 45; blue, 45 }  ,opacity=1 ]  {$ \begin{array}{l}
        S\rightarrow aB.C\\
        C\rightarrow .c
        \end{array}$};
    % Text Node
    \draw    (142,210) .. controls (142,207.24) and (144.24,205) .. (147,205) -- (215,205) .. controls (217.76,205) and (220,207.24) .. (220,210) -- (220,225) .. controls (220,227.76) and (217.76,230) .. (215,230) -- (147,230) .. controls (144.24,230) and (142,227.76) .. (142,225) -- cycle  ;
    \draw (181,217.5) node [color={rgb, 255:red, 62; green, 45; blue, 45 }  ,opacity=1 ]  {$S\rightarrow aBC.$};
    % Text Node
    \draw    (271,142) .. controls (271,139.24) and (273.24,137) .. (276,137) -- (324,137) .. controls (326.76,137) and (329,139.24) .. (329,142) -- (329,157) .. controls (329,159.76) and (326.76,162) .. (324,162) -- (276,162) .. controls (273.24,162) and (271,159.76) .. (271,157) -- cycle  ;
    \draw (300,149.5) node [color={rgb, 255:red, 62; green, 45; blue, 45 }  ,opacity=1 ]  {$C\rightarrow c.$};
    % Text Node
    \draw (247,139) node [scale=0.9,color={rgb, 255:red, 0; green, 0; blue, 0 }  ,opacity=1 ]  {$c$};
    % Text Node
    \draw (308,111) node [scale=0.9,color={rgb, 255:red, 0; green, 0; blue, 0 }  ,opacity=1 ]  {$c$};
    % Text Node
    \draw (190,184) node [scale=0.9,color={rgb, 255:red, 0; green, 0; blue, 0 }  ,opacity=1 ]  {$C$};
    % Text Node
    \draw (322,127) node [scale=0.9,color={rgb, 255:red, 255; green, 0; blue, 0 }  ,opacity=1 ]  {$6$};
    % Text Node
    \draw (213,195) node [scale=0.9,color={rgb, 255:red, 255; green, 0; blue, 0 }  ,opacity=1 ]  {$7$};
    % Text Node
    \draw (84.5,194) node [scale=0.9,color={rgb, 255:red, 255; green, 0; blue, 0 }  ,opacity=1 ]  {$acc$};
    % Connection
    \draw    (99.5,61) -- (139.5,61) ;
    \draw [shift={(141.5,61)}, rotate = 180] [color={rgb, 255:red, 0; green, 0; blue, 0 }  ][line width=0.75]    (10.93,-3.29) .. controls (6.95,-1.4) and (3.31,-0.3) .. (0,0) .. controls (3.31,0.3) and (6.95,1.4) .. (10.93,3.29)   ;
    
    % Connection
    \draw    (60.5,94) -- (60.5,134) ;
    \draw [shift={(60.5,136)}, rotate = 270] [color={rgb, 255:red, 0; green, 0; blue, 0 }  ][line width=0.75]    (10.93,-3.29) .. controls (6.95,-1.4) and (3.31,-0.3) .. (0,0) .. controls (3.31,0.3) and (6.95,1.4) .. (10.93,3.29)   ;
    
    % Connection
    \draw    (60.5,161) -- (60.5,202) ;
    \draw [shift={(60.5,204)}, rotate = 270] [color={rgb, 255:red, 0; green, 0; blue, 0 }  ][line width=0.75]    (10.93,-3.29) .. controls (6.95,-1.4) and (3.31,-0.3) .. (0,0) .. controls (3.31,0.3) and (6.95,1.4) .. (10.93,3.29)   ;
    
    % Connection
    \draw    (219.5,61) -- (259.5,61) ;
    \draw [shift={(261.5,61)}, rotate = 180] [color={rgb, 255:red, 0; green, 0; blue, 0 }  ][line width=0.75]    (10.93,-3.29) .. controls (6.95,-1.4) and (3.31,-0.3) .. (0,0) .. controls (3.31,0.3) and (6.95,1.4) .. (10.93,3.29)   ;
    
    % Connection
    \draw    (337.5,61) -- (376,61) ;
    \draw [shift={(378,61)}, rotate = 180] [color={rgb, 255:red, 0; green, 0; blue, 0 }  ][line width=0.75]    (10.93,-3.29) .. controls (6.95,-1.4) and (3.31,-0.3) .. (0,0) .. controls (3.31,0.3) and (6.95,1.4) .. (10.93,3.29)   ;
    
    % Connection
    \draw    (180.69,94) -- (180.86,125) ;
    \draw [shift={(180.87,127)}, rotate = 269.68] [color={rgb, 255:red, 0; green, 0; blue, 0 }  ][line width=0.75]    (10.93,-3.29) .. controls (6.95,-1.4) and (3.31,-0.3) .. (0,0) .. controls (3.31,0.3) and (6.95,1.4) .. (10.93,3.29)   ;
    
    % Connection
    \draw    (181,172) -- (181,203) ;
    \draw [shift={(181,205)}, rotate = 270] [color={rgb, 255:red, 0; green, 0; blue, 0 }  ][line width=0.75]    (10.93,-3.29) .. controls (6.95,-1.4) and (3.31,-0.3) .. (0,0) .. controls (3.31,0.3) and (6.95,1.4) .. (10.93,3.29)   ;
    
    % Connection
    \draw    (299.69,94) -- (299.92,135) ;
    \draw [shift={(299.93,137)}, rotate = 269.68] [color={rgb, 255:red, 0; green, 0; blue, 0 }  ][line width=0.75]    (10.93,-3.29) .. controls (6.95,-1.4) and (3.31,-0.3) .. (0,0) .. controls (3.31,0.3) and (6.95,1.4) .. (10.93,3.29)   ;
    
    % Connection
    \draw    (220,149.5) -- (269,149.5) ;
    \draw [shift={(271,149.5)}, rotate = 180] [color={rgb, 255:red, 0; green, 0; blue, 0 }  ][line width=0.75]    (10.93,-3.29) .. controls (6.95,-1.4) and (3.31,-0.3) .. (0,0) .. controls (3.31,0.3) and (6.95,1.4) .. (10.93,3.29)   ;
    
    
    \end{tikzpicture}
    
    И управляющую таблицу:
    
    \begin{tabular}{|c|c|c|c|c||c|c|c|} 
        \hline   & a     & b     & c            & \$    & B & C & S \\ [0.5ex]
        \hline 0 & $s_2$ &       &              &       &   & 1 & \\
        \hline 1 &       &       &              &  acc  &   &   & \\
        \hline 2 &       & $s_3$ &              &       & 4 &   & \\
        \hline 3 &       &       & $s_6$, $r_3$ &       &   & 5 & \\
        \hline 4 &       &       & $s_6$        &       &   & 7 & \\
        \hline 5 &       &       &              & $r_1$ &   &   & \\
        \hline 6 &       &       &              & $r_4$ &   &   & \\
        \hline 7 &       &       &              & $r_2$ &   &   & \\ [1ex] 
        \hline
    \end{tabular}
    
    Разберем слово $w$ с помощью алгоритма GLR. Использована следующая аннотация: вершины-состояния обозначены кругами, вершины-символы --- прямоугольниками.
    \begin{enumerate}
        \item Инициализируем GSS стартовым состоянием $v_0$: \\ \\
        Вход: \,
        \begin{tabular}[c]{ |c|c|c|c| } 
            \hline a & b & c & \$ \\ \hline
        \end{tabular}
        \qquad GSS: \,
        \begin{tikzpicture}[x=0.5pt,y=0.5pt,yscale=-1,xscale=1]
        %uncomment if require: \path (0,306); %set diagram left start at 0, and has height of 306
        
        
        % Text Node
        \draw  [line width=0.75]   (92, 109) circle [x radius= 13.6, y radius= 13.6]   ;
        \draw (92,109) node [color={rgb, 255:red, 62; green, 45; blue, 45 }  ,opacity=1 ]  {$0$};
        % Text Node
        \draw (110,89) node [color={rgb, 255:red, 62; green, 45; blue, 45 }  ,opacity=1 ]  {$v_{0}$};
        
        
        \end{tikzpicture}
        \\
        
        \item Видим входной символ '$a$', ищем соответствующую ему операцию в управляющей таблице --- $shift\ 2$, строим новый узел $v_1$: \\ \\
        Вход: \,
        \begin{tabular}[c]{ |c|c|c|c| } 
            \hline \textcolor{red}{a} & b & c & \$ \\ \hline
        \end{tabular}
        \qquad GSS: \,
        \begin{tikzpicture}[x=0.5pt,y=0.5pt,yscale=-1,xscale=1]
        %uncomment if require: \path (0,306); %set diagram left start at 0, and has height of 306
        
        
        % Text Node
        \draw  [line width=0.75]   (92, 109) circle [x radius= 13.6, y radius= 13.6]   ;
        \draw (92,109) node [color={rgb, 255:red, 62; green, 45; blue, 45 }  ,opacity=1 ]  {$0$};
        % Text Node
        \draw  [line width=0.75]   (138,98) -- (156,98) -- (156,122) -- (138,122) -- cycle  ;
        \draw (147,110) node [scale=1,color={rgb, 255:red, 62; green, 45; blue, 45 }  ,opacity=1 ]  {а};
        % Text Node
        \draw  [line width=0.75]   (203, 110) circle [x radius= 13.6, y radius= 13.6]   ;
        \draw (203,110) node [color={rgb, 255:red, 62; green, 45; blue, 45 }  ,opacity=1 ]  {$2$};
        % Text Node
        \draw (110,89) node [color={rgb, 255:red, 62; green, 45; blue, 45 }  ,opacity=1 ]  {$v_{0}$};
        % Text Node
        \draw (220,89) node [color={rgb, 255:red, 62; green, 45; blue, 45 }  ,opacity=1 ]  {$v_{1}$};
        % Connection
        \draw    (138,109.84) -- (107.6,109.28) ;
        \draw [shift={(105.6,109.25)}, rotate = 361.03999999999996] [color={rgb, 255:red, 0; green, 0; blue, 0 }  ][line width=0.75]    (10.93,-3.29) .. controls (6.95,-1.4) and (3.31,-0.3) .. (0,0) .. controls (3.31,0.3) and (6.95,1.4) .. (10.93,3.29)   ;
        
        % Connection
        \draw    (189.4,110) -- (158,110) ;
        \draw [shift={(156,110)}, rotate = 360] [color={rgb, 255:red, 0; green, 0; blue, 0 }  ][line width=0.75]    (10.93,-3.29) .. controls (6.95,-1.4) and (3.31,-0.3) .. (0,0) .. controls (3.31,0.3) and (6.95,1.4) .. (10.93,3.29)   ;
        
        
        \end{tikzpicture}
        \\
        
        \item Повторяем для символа '$b$', операции $shift\ 3$ и узла $v_2$: \\ \\
        Вход: \,
        \begin{tabular}[c]{ |c|c|c|c| } 
            \hline a & \textcolor{red}{b} & c & \$ \\ \hline
        \end{tabular}
        \qquad GSS: \,
        \begin{tikzpicture}[x=0.5pt,y=0.5pt,yscale=-1,xscale=1]
        %uncomment if require: \path (0,306); %set diagram left start at 0, and has height of 306
        
        
        % Text Node
        \draw  [line width=0.75]   (92, 109) circle [x radius= 13.6, y radius= 13.6]   ;
        \draw (92,109) node [color={rgb, 255:red, 62; green, 45; blue, 45 }  ,opacity=1 ]  {$0$};
        % Text Node
        \draw  [line width=0.75]   (138,98) -- (156,98) -- (156,122) -- (138,122) -- cycle  ;
        \draw (147,110) node [scale=1,color={rgb, 255:red, 62; green, 45; blue, 45 }  ,opacity=1 ]  {a};
        % Text Node
        \draw  [line width=0.75]   (203, 110) circle [x radius= 13.6, y radius= 13.6]   ;
        \draw (203,110) node [color={rgb, 255:red, 62; green, 45; blue, 45 }  ,opacity=1 ]  {$2$};
        % Text Node
        \draw (110,89) node [color={rgb, 255:red, 62; green, 45; blue, 45 }  ,opacity=1 ]  {$v_{0}$};
        % Text Node
        \draw (220,89) node [color={rgb, 255:red, 62; green, 45; blue, 45 }  ,opacity=1 ]  {$v_{1}$};
        % Text Node
        \draw  [line width=0.75]   (258,98) -- (276,98) -- (276,122) -- (258,122) -- cycle  ;
        \draw (267,110) node [scale=1,color={rgb, 255:red, 62; green, 45; blue, 45 }  ,opacity=1 ]  {b};
        % Text Node
        \draw  [line width=0.75]   (323, 110) circle [x radius= 13.6, y radius= 13.6]   ;
        \draw (323,110) node [color={rgb, 255:red, 62; green, 45; blue, 45 }  ,opacity=1 ]  {$3$};
        % Text Node
        \draw (340,90) node [color={rgb, 255:red, 62; green, 45; blue, 45 }  ,opacity=1 ]  {$v_{2}$};
        % Connection
        \draw    (138,109.84) -- (107.6,109.28) ;
        \draw [shift={(105.6,109.25)}, rotate = 361.03999999999996] [color={rgb, 255:red, 0; green, 0; blue, 0 }  ][line width=0.75]    (10.93,-3.29) .. controls (6.95,-1.4) and (3.31,-0.3) .. (0,0) .. controls (3.31,0.3) and (6.95,1.4) .. (10.93,3.29)   ;
        
        % Connection
        \draw    (189.4,110) -- (158,110) ;
        \draw [shift={(156,110)}, rotate = 360] [color={rgb, 255:red, 0; green, 0; blue, 0 }  ][line width=0.75]    (10.93,-3.29) .. controls (6.95,-1.4) and (3.31,-0.3) .. (0,0) .. controls (3.31,0.3) and (6.95,1.4) .. (10.93,3.29)   ;
        
        % Connection
        \draw    (309.4,110) -- (278,110) ;
        \draw [shift={(276,110)}, rotate = 360] [color={rgb, 255:red, 0; green, 0; blue, 0 }  ][line width=0.75]    (10.93,-3.29) .. controls (6.95,-1.4) and (3.31,-0.3) .. (0,0) .. controls (3.31,0.3) and (6.95,1.4) .. (10.93,3.29)   ;
        
        % Connection
        \draw    (258,110) -- (218.6,110) ;
        \draw [shift={(216.6,110)}, rotate = 360] [color={rgb, 255:red, 0; green, 0; blue, 0 }  ][line width=0.75]    (10.93,-3.29) .. controls (6.95,-1.4) and (3.31,-0.3) .. (0,0) .. controls (3.31,0.3) and (6.95,1.4) .. (10.93,3.29)   ;
        
        
        \end{tikzpicture}
        \\
        
        \item При обработке узла $v_3$ у нас возникает конфликт shift-reduce: $s_6,\ r_3$. Мы смотрим на вершины, смежные $v_2$, на управляющую таблицу и на правило вывода под номером 3 для поиска альтернативного построения стека. Находим $goto\ 4$ и строим вершину $v_3$ с соответствующим переходом по нетерминалу $B$ из $v_1$ (т.к. количество символов в правой части правила вывода 3 равняется 1, значит мы в дереве опустимся на глубину 1 по вершинам-состояниям):\\ \\
        Вход: \,
        \begin{tabular}[c]{ |c|c|c|c| } 
            \hline a & b & c & \$ \\ \hline
        \end{tabular}
        \qquad GSS: \,
        \begin{tikzpicture}[x=0.5pt,y=0.5pt,yscale=-1,xscale=1]
        %uncomment if require: \path (0,422); %set diagram left start at 0, and has height of 422
        
        
        % Text Node
        \draw  [line width=0.75]   (92, 110) circle [x radius= 13.6, y radius= 13.6]   ;
        \draw (92,110) node [color={rgb, 255:red, 62; green, 45; blue, 45 }  ,opacity=1 ]  {$0$};
        % Text Node
        \draw  [line width=0.75]   (138,98) -- (156,98) -- (156,122) -- (138,122) -- cycle  ;
        \draw (147,110) node [scale=1,color={rgb, 255:red, 62; green, 45; blue, 45 }  ,opacity=1 ]  {а};
        % Text Node
        \draw  [line width=0.75]   (203, 110) circle [x radius= 13.6, y radius= 13.6]   ;
        \draw (203,110) node [color={rgb, 255:red, 62; green, 45; blue, 45 }  ,opacity=1 ]  {$2$};
        % Text Node
        \draw (110,89) node [color={rgb, 255:red, 62; green, 45; blue, 45 }  ,opacity=1 ]  {$v_{0}$};
        % Text Node
        \draw (220,89) node [color={rgb, 255:red, 62; green, 45; blue, 45 }  ,opacity=1 ]  {$v_{1}$};
        % Text Node
        \draw  [line width=0.75]   (258,98) -- (276,98) -- (276,122) -- (258,122) -- cycle  ;
        \draw (267,110) node [scale=1,color={rgb, 255:red, 62; green, 45; blue, 45 }  ,opacity=1 ]  {b};
        % Text Node
        \draw  [line width=0.75]   (323, 110) circle [x radius= 13.6, y radius= 13.6]   ;
        \draw (323,110) node [color={rgb, 255:red, 62; green, 45; blue, 45 }  ,opacity=1 ]  {$3$};
        % Text Node
        \draw (340,90) node [color={rgb, 255:red, 62; green, 45; blue, 45 }  ,opacity=1 ]  {$v_{2}$};
        % Text Node
        \draw  [line width=0.75]   (258,158) -- (276,158) -- (276,182) -- (258,182) -- cycle  ;
        \draw (267,170) node [scale=1,color={rgb, 255:red, 62; green, 45; blue, 45 }  ,opacity=1 ]  {B};
        % Text Node
        \draw  [line width=0.75]   (323, 170) circle [x radius= 13.6, y radius= 13.6]   ;
        \draw (323,170) node [color={rgb, 255:red, 62; green, 45; blue, 45 }  ,opacity=1 ]  {$4$};
        % Text Node
        \draw (340,149) node [color={rgb, 255:red, 62; green, 45; blue, 45 }  ,opacity=1 ]  {$v_{3}$};
        % Connection
        \draw    (138,110) -- (107.6,110) ;
        \draw [shift={(105.6,110)}, rotate = 360] [color={rgb, 255:red, 0; green, 0; blue, 0 }  ][line width=0.75]    (10.93,-3.29) .. controls (6.95,-1.4) and (3.31,-0.3) .. (0,0) .. controls (3.31,0.3) and (6.95,1.4) .. (10.93,3.29)   ;
        
        % Connection
        \draw    (189.4,110) -- (158,110) ;
        \draw [shift={(156,110)}, rotate = 360] [color={rgb, 255:red, 0; green, 0; blue, 0 }  ][line width=0.75]    (10.93,-3.29) .. controls (6.95,-1.4) and (3.31,-0.3) .. (0,0) .. controls (3.31,0.3) and (6.95,1.4) .. (10.93,3.29)   ;
        
        % Connection
        \draw    (309.4,110) -- (278,110) ;
        \draw [shift={(276,110)}, rotate = 360] [color={rgb, 255:red, 0; green, 0; blue, 0 }  ][line width=0.75]    (10.93,-3.29) .. controls (6.95,-1.4) and (3.31,-0.3) .. (0,0) .. controls (3.31,0.3) and (6.95,1.4) .. (10.93,3.29)   ;
        
        % Connection
        \draw    (258,110) -- (218.6,110) ;
        \draw [shift={(216.6,110)}, rotate = 360] [color={rgb, 255:red, 0; green, 0; blue, 0 }  ][line width=0.75]    (10.93,-3.29) .. controls (6.95,-1.4) and (3.31,-0.3) .. (0,0) .. controls (3.31,0.3) and (6.95,1.4) .. (10.93,3.29)   ;
        
        % Connection
        \draw    (309.4,170) -- (278,170) ;
        \draw [shift={(276,170)}, rotate = 360] [color={rgb, 255:red, 0; green, 0; blue, 0 }  ][line width=0.75]    (10.93,-3.29) .. controls (6.95,-1.4) and (3.31,-0.3) .. (0,0) .. controls (3.31,0.3) and (6.95,1.4) .. (10.93,3.29)   ;
        
        % Connection
        \draw    (258,168.07) .. controls (230.15,163.58) and (213.3,149.18) .. (207.42,124.89) ;
        \draw [shift={(206.99,123.01)}, rotate = 437.91] [color={rgb, 255:red, 0; green, 0; blue, 0 }  ][line width=0.75]    (10.93,-3.29) .. controls (6.95,-1.4) and (3.31,-0.3) .. (0,0) .. controls (3.31,0.3) and (6.95,1.4) .. (10.93,3.29)   ;
        
        
        \end{tikzpicture}
        \\
        
        \item Читаем символ '$c$' и ищем в управляющей таблице переходы из состояний 3 и 4 (так как узлы $v_2$ и $v_3$ находятся на одном уровне, то есть были построены после чтения одного символа из входного слова). Таким переходом оказывается $s_6$ в обоих случаях, поэтому соединяем узел $v_4$ с обоими рассмотренными узлами:\\ \\
        Вход: \,
        \begin{tabular}[c]{ |c|c|c|c| } 
            \hline a & b & \textcolor{red}{c} & \$ \\ \hline
        \end{tabular}
        \qquad GSS: \,
        \begin{tikzpicture}[x=0.5pt,y=0.5pt,yscale=-1,xscale=1]
        %uncomment if require: \path (0,422); %set diagram left start at 0, and has height of 422
        
        
        % Text Node
        \draw  [line width=0.75]   (92, 110) circle [x radius= 13.6, y radius= 13.6]   ;
        \draw (92,110) node [color={rgb, 255:red, 62; green, 45; blue, 45 }  ,opacity=1 ]  {$0$};
        % Text Node
        \draw  [line width=0.75]   (138,98) -- (156,98) -- (156,122) -- (138,122) -- cycle  ;
        \draw (147,110) node [scale=1,color={rgb, 255:red, 62; green, 45; blue, 45 }  ,opacity=1 ]  {а};
        % Text Node
        \draw  [line width=0.75]   (203, 110) circle [x radius= 13.6, y radius= 13.6]   ;
        \draw (203,110) node [color={rgb, 255:red, 62; green, 45; blue, 45 }  ,opacity=1 ]  {$2$};
        % Text Node
        \draw (110,89) node [color={rgb, 255:red, 62; green, 45; blue, 45 }  ,opacity=1 ]  {$v_{0}$};
        % Text Node
        \draw (220,89) node [color={rgb, 255:red, 62; green, 45; blue, 45 }  ,opacity=1 ]  {$v_{1}$};
        % Text Node
        \draw  [line width=0.75]   (258,98) -- (276,98) -- (276,122) -- (258,122) -- cycle  ;
        \draw (267,110) node [scale=1,color={rgb, 255:red, 62; green, 45; blue, 45 }  ,opacity=1 ]  {b};
        % Text Node
        \draw  [line width=0.75]   (323, 110) circle [x radius= 13.6, y radius= 13.6]   ;
        \draw (323,110) node [color={rgb, 255:red, 62; green, 45; blue, 45 }  ,opacity=1 ]  {$3$};
        % Text Node
        \draw (340,90) node [color={rgb, 255:red, 62; green, 45; blue, 45 }  ,opacity=1 ]  {$v_{2}$};
        % Text Node
        \draw  [line width=0.75]   (258,158) -- (276,158) -- (276,182) -- (258,182) -- cycle  ;
        \draw (267,170) node [scale=1,color={rgb, 255:red, 62; green, 45; blue, 45 }  ,opacity=1 ]  {B};
        % Text Node
        \draw  [line width=0.75]   (323, 170) circle [x radius= 13.6, y radius= 13.6]   ;
        \draw (323,170) node [color={rgb, 255:red, 62; green, 45; blue, 45 }  ,opacity=1 ]  {$4$};
        % Text Node
        \draw (340,149) node [color={rgb, 255:red, 62; green, 45; blue, 45 }  ,opacity=1 ]  {$v_{3}$};
        % Text Node
        \draw  [line width=0.75]   (374,98) -- (392,98) -- (392,122) -- (374,122) -- cycle  ;
        \draw (383,110) node [scale=1,color={rgb, 255:red, 62; green, 45; blue, 45 }  ,opacity=1 ]  {c};
        % Text Node
        \draw  [line width=0.75]   (374,158) -- (392,158) -- (392,182) -- (374,182) -- cycle  ;
        \draw (383,170) node [scale=1,color={rgb, 255:red, 62; green, 45; blue, 45 }  ,opacity=1 ]  {c};
        % Text Node
        \draw  [line width=0.75]   (437, 110) circle [x radius= 13.6, y radius= 13.6]   ;
        \draw (437,110) node [color={rgb, 255:red, 62; green, 45; blue, 45 }  ,opacity=1 ]  {$6$};
        % Text Node
        \draw (450,89) node [color={rgb, 255:red, 62; green, 45; blue, 45 }  ,opacity=1 ]  {$v_{4}$};
        % Connection
        \draw    (138,110) -- (107.6,110) ;
        \draw [shift={(105.6,110)}, rotate = 360] [color={rgb, 255:red, 0; green, 0; blue, 0 }  ][line width=0.75]    (10.93,-3.29) .. controls (6.95,-1.4) and (3.31,-0.3) .. (0,0) .. controls (3.31,0.3) and (6.95,1.4) .. (10.93,3.29)   ;
        
        % Connection
        \draw    (189.4,110) -- (158,110) ;
        \draw [shift={(156,110)}, rotate = 360] [color={rgb, 255:red, 0; green, 0; blue, 0 }  ][line width=0.75]    (10.93,-3.29) .. controls (6.95,-1.4) and (3.31,-0.3) .. (0,0) .. controls (3.31,0.3) and (6.95,1.4) .. (10.93,3.29)   ;
        
        % Connection
        \draw    (309.4,110) -- (278,110) ;
        \draw [shift={(276,110)}, rotate = 360] [color={rgb, 255:red, 0; green, 0; blue, 0 }  ][line width=0.75]    (10.93,-3.29) .. controls (6.95,-1.4) and (3.31,-0.3) .. (0,0) .. controls (3.31,0.3) and (6.95,1.4) .. (10.93,3.29)   ;
        
        % Connection
        \draw    (258,110) -- (218.6,110) ;
        \draw [shift={(216.6,110)}, rotate = 360] [color={rgb, 255:red, 0; green, 0; blue, 0 }  ][line width=0.75]    (10.93,-3.29) .. controls (6.95,-1.4) and (3.31,-0.3) .. (0,0) .. controls (3.31,0.3) and (6.95,1.4) .. (10.93,3.29)   ;
        
        % Connection
        \draw    (309.4,170) -- (278,170) ;
        \draw [shift={(276,170)}, rotate = 360] [color={rgb, 255:red, 0; green, 0; blue, 0 }  ][line width=0.75]    (10.93,-3.29) .. controls (6.95,-1.4) and (3.31,-0.3) .. (0,0) .. controls (3.31,0.3) and (6.95,1.4) .. (10.93,3.29)   ;
        
        % Connection
        \draw    (258,168.07) .. controls (230.15,163.58) and (213.3,149.18) .. (207.42,124.89) ;
        \draw [shift={(206.99,123.01)}, rotate = 437.91] [color={rgb, 255:red, 0; green, 0; blue, 0 }  ][line width=0.75]    (10.93,-3.29) .. controls (6.95,-1.4) and (3.31,-0.3) .. (0,0) .. controls (3.31,0.3) and (6.95,1.4) .. (10.93,3.29)   ;
        
        % Connection
        \draw    (374,110) -- (338.6,110) ;
        \draw [shift={(336.6,110)}, rotate = 360] [color={rgb, 255:red, 0; green, 0; blue, 0 }  ][line width=0.75]    (10.93,-3.29) .. controls (6.95,-1.4) and (3.31,-0.3) .. (0,0) .. controls (3.31,0.3) and (6.95,1.4) .. (10.93,3.29)   ;
        
        % Connection
        \draw    (423.4,110) -- (394,110) ;
        \draw [shift={(392,110)}, rotate = 360] [color={rgb, 255:red, 0; green, 0; blue, 0 }  ][line width=0.75]    (10.93,-3.29) .. controls (6.95,-1.4) and (3.31,-0.3) .. (0,0) .. controls (3.31,0.3) and (6.95,1.4) .. (10.93,3.29)   ;
        
        % Connection
        \draw    (435.82,123.55) .. controls (435.9,150.31) and (416.89,165.24) .. (393.78,168.34) ;
        \draw [shift={(392,168.55)}, rotate = 353.9] [color={rgb, 255:red, 0; green, 0; blue, 0 }  ][line width=0.75]    (10.93,-3.29) .. controls (6.95,-1.4) and (3.31,-0.3) .. (0,0) .. controls (3.31,0.3) and (6.95,1.4) .. (10.93,3.29)   ;
        
        % Connection
        \draw    (374,170) -- (338.6,170) ;
        \draw [shift={(336.6,170)}, rotate = 360] [color={rgb, 255:red, 0; green, 0; blue, 0 }  ][line width=0.75]    (10.93,-3.29) .. controls (6.95,-1.4) and (3.31,-0.3) .. (0,0) .. controls (3.31,0.3) and (6.95,1.4) .. (10.93,3.29)   ;
        
        
        \end{tikzpicture}
        \\
        
        \item При обработке узла $v_4$ находим соответствующею 6-ому состоянию редукцию по правилу 4. Его правая часть содержит один символ '$c$', 2 вершины-символа с которым достижимы из $v_4$. Находим вершины-состояния, которые смежны с этими вершинами-символами и обрабатываем переходы по левой части правила 4. Такими переходами по нетерминалу $C$ оказываются $5$ и $7$. Строим соответствующие им вершины $v_5$ и $v_6$:\\ \\
        Вход: \,
        \begin{tabular}[c]{ |c|c|c|c| } 
            \hline a & b & c & \$ \\ \hline
        \end{tabular}
        \qquad GSS: \,
        \begin{tikzpicture}[x=0.5pt,y=0.5pt,yscale=-1,xscale=1]
        %uncomment if require: \path (0,422); %set diagram left start at 0, and has height of 422
        
        
        % Text Node
        \draw  [line width=0.75]   (92, 110) circle [x radius= 13.6, y radius= 13.6]   ;
        \draw (92,110) node [color={rgb, 255:red, 62; green, 45; blue, 45 }  ,opacity=1 ]  {$0$};
        % Text Node
        \draw  [line width=0.75]   (138,98) -- (156,98) -- (156,122) -- (138,122) -- cycle  ;
        \draw (147,110) node [scale=1,color={rgb, 255:red, 62; green, 45; blue, 45 }  ,opacity=1 ]  {а};
        % Text Node
        \draw  [line width=0.75]   (203, 110) circle [x radius= 13.6, y radius= 13.6]   ;
        \draw (203,110) node [color={rgb, 255:red, 62; green, 45; blue, 45 }  ,opacity=1 ]  {$2$};
        % Text Node
        \draw (110,89) node [color={rgb, 255:red, 62; green, 45; blue, 45 }  ,opacity=1 ]  {$v_{0}$};
        % Text Node
        \draw (220,89) node [color={rgb, 255:red, 62; green, 45; blue, 45 }  ,opacity=1 ]  {$v_{1}$};
        % Text Node
        \draw  [line width=0.75]   (258,98) -- (276,98) -- (276,122) -- (258,122) -- cycle  ;
        \draw (267,110) node [scale=1,color={rgb, 255:red, 62; green, 45; blue, 45 }  ,opacity=1 ]  {b};
        % Text Node
        \draw  [line width=0.75]   (323, 110) circle [x radius= 13.6, y radius= 13.6]   ;
        \draw (323,110) node [color={rgb, 255:red, 62; green, 45; blue, 45 }  ,opacity=1 ]  {$3$};
        % Text Node
        \draw (340,90) node [color={rgb, 255:red, 62; green, 45; blue, 45 }  ,opacity=1 ]  {$v_{2}$};
        % Text Node
        \draw  [line width=0.75]   (258,158) -- (276,158) -- (276,182) -- (258,182) -- cycle  ;
        \draw (267,170) node [scale=1,color={rgb, 255:red, 62; green, 45; blue, 45 }  ,opacity=1 ]  {B};
        % Text Node
        \draw  [line width=0.75]   (323, 170) circle [x radius= 13.6, y radius= 13.6]   ;
        \draw (323,170) node [color={rgb, 255:red, 62; green, 45; blue, 45 }  ,opacity=1 ]  {$4$};
        % Text Node
        \draw (340,149) node [color={rgb, 255:red, 62; green, 45; blue, 45 }  ,opacity=1 ]  {$v_{3}$};
        % Text Node
        \draw  [line width=0.75]   (374,98) -- (392,98) -- (392,122) -- (374,122) -- cycle  ;
        \draw (383,110) node [scale=1,color={rgb, 255:red, 62; green, 45; blue, 45 }  ,opacity=1 ]  {c};
        % Text Node
        \draw  [line width=0.75]   (374,158) -- (392,158) -- (392,182) -- (374,182) -- cycle  ;
        \draw (383,170) node [scale=1,color={rgb, 255:red, 62; green, 45; blue, 45 }  ,opacity=1 ]  {c};
        % Text Node
        \draw  [line width=0.75]   (437, 110) circle [x radius= 13.6, y radius= 13.6]   ;
        \draw (437,110) node [color={rgb, 255:red, 62; green, 45; blue, 45 }  ,opacity=1 ]  {$6$};
        % Text Node
        \draw (450,89) node [color={rgb, 255:red, 62; green, 45; blue, 45 }  ,opacity=1 ]  {$v_{4}$};
        % Text Node
        \draw  [line width=0.75]   (372,38) -- (392,38) -- (392,62) -- (372,62) -- cycle  ;
        \draw (382,50) node [scale=1,color={rgb, 255:red, 62; green, 45; blue, 45 }  ,opacity=1 ]  {C};
        % Text Node
        \draw  [line width=0.75]   (372,218) -- (392,218) -- (392,242) -- (372,242) -- cycle  ;
        \draw (382,230) node [scale=1,color={rgb, 255:red, 62; green, 45; blue, 45 }  ,opacity=1 ]  {C};
        % Text Node
        \draw  [line width=0.75]   (437, 50) circle [x radius= 13.6, y radius= 13.6]   ;
        \draw (437,50) node [color={rgb, 255:red, 62; green, 45; blue, 45 }  ,opacity=1 ]  {$5$};
        % Text Node
        \draw  [line width=0.75]   (437, 231) circle [x radius= 13.6, y radius= 13.6]   ;
        \draw (437,231) node [color={rgb, 255:red, 62; green, 45; blue, 45 }  ,opacity=1 ]  {$7$};
        % Text Node
        \draw (452,28) node [color={rgb, 255:red, 62; green, 45; blue, 45 }  ,opacity=1 ]  {$v_{5}$};
        % Text Node
        \draw (452,208) node [color={rgb, 255:red, 62; green, 45; blue, 45 }  ,opacity=1 ]  {$v_{6}$};
        % Connection
        \draw    (138,110) -- (107.6,110) ;
        \draw [shift={(105.6,110)}, rotate = 360] [color={rgb, 255:red, 0; green, 0; blue, 0 }  ][line width=0.75]    (10.93,-3.29) .. controls (6.95,-1.4) and (3.31,-0.3) .. (0,0) .. controls (3.31,0.3) and (6.95,1.4) .. (10.93,3.29)   ;
        
        % Connection
        \draw    (189.4,110) -- (158,110) ;
        \draw [shift={(156,110)}, rotate = 360] [color={rgb, 255:red, 0; green, 0; blue, 0 }  ][line width=0.75]    (10.93,-3.29) .. controls (6.95,-1.4) and (3.31,-0.3) .. (0,0) .. controls (3.31,0.3) and (6.95,1.4) .. (10.93,3.29)   ;
        
        % Connection
        \draw    (309.4,110) -- (278,110) ;
        \draw [shift={(276,110)}, rotate = 360] [color={rgb, 255:red, 0; green, 0; blue, 0 }  ][line width=0.75]    (10.93,-3.29) .. controls (6.95,-1.4) and (3.31,-0.3) .. (0,0) .. controls (3.31,0.3) and (6.95,1.4) .. (10.93,3.29)   ;
        
        % Connection
        \draw    (258,110) -- (218.6,110) ;
        \draw [shift={(216.6,110)}, rotate = 360] [color={rgb, 255:red, 0; green, 0; blue, 0 }  ][line width=0.75]    (10.93,-3.29) .. controls (6.95,-1.4) and (3.31,-0.3) .. (0,0) .. controls (3.31,0.3) and (6.95,1.4) .. (10.93,3.29)   ;
        
        % Connection
        \draw    (309.4,170) -- (278,170) ;
        \draw [shift={(276,170)}, rotate = 360] [color={rgb, 255:red, 0; green, 0; blue, 0 }  ][line width=0.75]    (10.93,-3.29) .. controls (6.95,-1.4) and (3.31,-0.3) .. (0,0) .. controls (3.31,0.3) and (6.95,1.4) .. (10.93,3.29)   ;
        
        % Connection
        \draw    (258,168.07) .. controls (230.15,163.58) and (213.3,149.18) .. (207.42,124.89) ;
        \draw [shift={(206.99,123.01)}, rotate = 437.91] [color={rgb, 255:red, 0; green, 0; blue, 0 }  ][line width=0.75]    (10.93,-3.29) .. controls (6.95,-1.4) and (3.31,-0.3) .. (0,0) .. controls (3.31,0.3) and (6.95,1.4) .. (10.93,3.29)   ;
        
        % Connection
        \draw    (374,110) -- (338.6,110) ;
        \draw [shift={(336.6,110)}, rotate = 360] [color={rgb, 255:red, 0; green, 0; blue, 0 }  ][line width=0.75]    (10.93,-3.29) .. controls (6.95,-1.4) and (3.31,-0.3) .. (0,0) .. controls (3.31,0.3) and (6.95,1.4) .. (10.93,3.29)   ;
        
        % Connection
        \draw    (423.4,110) -- (394,110) ;
        \draw [shift={(392,110)}, rotate = 360] [color={rgb, 255:red, 0; green, 0; blue, 0 }  ][line width=0.75]    (10.93,-3.29) .. controls (6.95,-1.4) and (3.31,-0.3) .. (0,0) .. controls (3.31,0.3) and (6.95,1.4) .. (10.93,3.29)   ;
        
        % Connection
        \draw    (435.82,123.55) .. controls (435.9,150.31) and (416.89,165.24) .. (393.78,168.34) ;
        \draw [shift={(392,168.55)}, rotate = 353.9] [color={rgb, 255:red, 0; green, 0; blue, 0 }  ][line width=0.75]    (10.93,-3.29) .. controls (6.95,-1.4) and (3.31,-0.3) .. (0,0) .. controls (3.31,0.3) and (6.95,1.4) .. (10.93,3.29)   ;
        
        % Connection
        \draw    (374,170) -- (338.6,170) ;
        \draw [shift={(336.6,170)}, rotate = 360] [color={rgb, 255:red, 0; green, 0; blue, 0 }  ][line width=0.75]    (10.93,-3.29) .. controls (6.95,-1.4) and (3.31,-0.3) .. (0,0) .. controls (3.31,0.3) and (6.95,1.4) .. (10.93,3.29)   ;
        
        % Connection
        \draw    (372,50.52) .. controls (341.88,50.18) and (326.06,64.88) .. (324.53,94.64) ;
        \draw [shift={(324.46,96.48)}, rotate = 271.78] [color={rgb, 255:red, 0; green, 0; blue, 0 }  ][line width=0.75]    (10.93,-3.29) .. controls (6.95,-1.4) and (3.31,-0.3) .. (0,0) .. controls (3.31,0.3) and (6.95,1.4) .. (10.93,3.29)   ;
        
        % Connection
        \draw    (372,229.65) .. controls (341.88,230.53) and (326.05,215.79) .. (324.5,185.41) ;
        \draw [shift={(324.42,183.53)}, rotate = 448.21] [color={rgb, 255:red, 0; green, 0; blue, 0 }  ][line width=0.75]    (10.93,-3.29) .. controls (6.95,-1.4) and (3.31,-0.3) .. (0,0) .. controls (3.31,0.3) and (6.95,1.4) .. (10.93,3.29)   ;
        
        % Connection
        \draw    (423.4,50) -- (394,50) ;
        \draw [shift={(392,50)}, rotate = 360] [color={rgb, 255:red, 0; green, 0; blue, 0 }  ][line width=0.75]    (10.93,-3.29) .. controls (6.95,-1.4) and (3.31,-0.3) .. (0,0) .. controls (3.31,0.3) and (6.95,1.4) .. (10.93,3.29)   ;
        
        % Connection
        \draw    (423.4,230.75) -- (394,230.22) ;
        \draw [shift={(392,230.18)}, rotate = 361.03999999999996] [color={rgb, 255:red, 0; green, 0; blue, 0 }  ][line width=0.75]    (10.93,-3.29) .. controls (6.95,-1.4) and (3.31,-0.3) .. (0,0) .. controls (3.31,0.3) and (6.95,1.4) .. (10.93,3.29)   ;
        
        
        \end{tikzpicture}
        \\
        
        \item При обработке узлов $v_5$ и $v_6$ находим редукции с символом '$S$' в левой части и тремя символами в правой. Возвращаемся на 3 вершины-состояния назад и строим вершину $v_7$ с переходом по $S$: \\ \\
        Вход: \,
        \begin{tabular}[c]{ |c|c|c|c| } 
            \hline a & b & c & \$ \\ \hline
        \end{tabular}
        \qquad GSS: \,
        \begin{tikzpicture}[x=0.5pt,y=0.5pt,yscale=-1,xscale=1]
        %uncomment if require: \path (0,422); %set diagram left start at 0, and has height of 422
        
        
        % Text Node
        \draw  [line width=0.75]   (92, 110) circle [x radius= 13.6, y radius= 13.6]   ;
        \draw (92,110) node [color={rgb, 255:red, 62; green, 45; blue, 45 }  ,opacity=1 ]  {$0$};
        % Text Node
        \draw  [line width=0.75]   (138,98) -- (156,98) -- (156,122) -- (138,122) -- cycle  ;
        \draw (147,110) node [scale=1,color={rgb, 255:red, 62; green, 45; blue, 45 }  ,opacity=1 ]  {а};
        % Text Node
        \draw  [line width=0.75]   (203, 110) circle [x radius= 13.6, y radius= 13.6]   ;
        \draw (203,110) node [color={rgb, 255:red, 62; green, 45; blue, 45 }  ,opacity=1 ]  {$2$};
        % Text Node
        \draw (110,89) node [color={rgb, 255:red, 62; green, 45; blue, 45 }  ,opacity=1 ]  {$v_{0}$};
        % Text Node
        \draw (220,89) node [color={rgb, 255:red, 62; green, 45; blue, 45 }  ,opacity=1 ]  {$v_{1}$};
        % Text Node
        \draw  [line width=0.75]   (258,98) -- (276,98) -- (276,122) -- (258,122) -- cycle  ;
        \draw (267,110) node [scale=1,color={rgb, 255:red, 62; green, 45; blue, 45 }  ,opacity=1 ]  {b};
        % Text Node
        \draw  [line width=0.75]   (323, 110) circle [x radius= 13.6, y radius= 13.6]   ;
        \draw (323,110) node [color={rgb, 255:red, 62; green, 45; blue, 45 }  ,opacity=1 ]  {$3$};
        % Text Node
        \draw (340,90) node [color={rgb, 255:red, 62; green, 45; blue, 45 }  ,opacity=1 ]  {$v_{2}$};
        % Text Node
        \draw  [line width=0.75]   (258,158) -- (276,158) -- (276,182) -- (258,182) -- cycle  ;
        \draw (267,170) node [scale=1,color={rgb, 255:red, 62; green, 45; blue, 45 }  ,opacity=1 ]  {B};
        % Text Node
        \draw  [line width=0.75]   (323, 170) circle [x radius= 13.6, y radius= 13.6]   ;
        \draw (323,170) node [color={rgb, 255:red, 62; green, 45; blue, 45 }  ,opacity=1 ]  {$4$};
        % Text Node
        \draw (340,149) node [color={rgb, 255:red, 62; green, 45; blue, 45 }  ,opacity=1 ]  {$v_{3}$};
        % Text Node
        \draw  [line width=0.75]   (374,98) -- (392,98) -- (392,122) -- (374,122) -- cycle  ;
        \draw (383,110) node [scale=1,color={rgb, 255:red, 62; green, 45; blue, 45 }  ,opacity=1 ]  {c};
        % Text Node
        \draw  [line width=0.75]   (374,158) -- (392,158) -- (392,182) -- (374,182) -- cycle  ;
        \draw (383,170) node [scale=1,color={rgb, 255:red, 62; green, 45; blue, 45 }  ,opacity=1 ]  {c};
        % Text Node
        \draw  [line width=0.75]   (437, 110) circle [x radius= 13.6, y radius= 13.6]   ;
        \draw (437,110) node [color={rgb, 255:red, 62; green, 45; blue, 45 }  ,opacity=1 ]  {$6$};
        % Text Node
        \draw (450,89) node [color={rgb, 255:red, 62; green, 45; blue, 45 }  ,opacity=1 ]  {$v_{4}$};
        % Text Node
        \draw  [line width=0.75]   (372,38) -- (392,38) -- (392,62) -- (372,62) -- cycle  ;
        \draw (382,50) node [scale=1,color={rgb, 255:red, 62; green, 45; blue, 45 }  ,opacity=1 ]  {C};
        % Text Node
        \draw  [line width=0.75]   (372,218) -- (392,218) -- (392,242) -- (372,242) -- cycle  ;
        \draw (382,230) node [scale=1,color={rgb, 255:red, 62; green, 45; blue, 45 }  ,opacity=1 ]  {C};
        % Text Node
        \draw  [line width=0.75]   (437, 50) circle [x radius= 13.6, y radius= 13.6]   ;
        \draw (437,50) node [color={rgb, 255:red, 62; green, 45; blue, 45 }  ,opacity=1 ]  {$5$};
        % Text Node
        \draw  [line width=0.75]   (437, 231) circle [x radius= 13.6, y radius= 13.6]   ;
        \draw (437,231) node [color={rgb, 255:red, 62; green, 45; blue, 45 }  ,opacity=1 ]  {$7$};
        % Text Node
        \draw (452,28) node [color={rgb, 255:red, 62; green, 45; blue, 45 }  ,opacity=1 ]  {$v_{5}$};
        % Text Node
        \draw (452,208) node [color={rgb, 255:red, 62; green, 45; blue, 45 }  ,opacity=1 ]  {$v_{6}$};
        % Text Node
        \draw  [line width=0.75]   (373,278) -- (391,278) -- (391,302) -- (373,302) -- cycle  ;
        \draw (382,290) node [scale=1,color={rgb, 255:red, 62; green, 45; blue, 45 }  ,opacity=1 ]  {S};
        % Text Node
        \draw  [line width=0.75]   (437, 290) circle [x radius= 13.6, y radius= 13.6]   ;
        \draw (437,290) node [color={rgb, 255:red, 62; green, 45; blue, 45 }  ,opacity=1 ]  {$1$};
        % Text Node
        \draw (452,268) node [color={rgb, 255:red, 62; green, 45; blue, 45 }  ,opacity=1 ]  {$v_{7}$};
        % Connection
        \draw    (138,110) -- (107.6,110) ;
        \draw [shift={(105.6,110)}, rotate = 360] [color={rgb, 255:red, 0; green, 0; blue, 0 }  ][line width=0.75]    (10.93,-3.29) .. controls (6.95,-1.4) and (3.31,-0.3) .. (0,0) .. controls (3.31,0.3) and (6.95,1.4) .. (10.93,3.29)   ;
        
        % Connection
        \draw    (189.4,110) -- (158,110) ;
        \draw [shift={(156,110)}, rotate = 360] [color={rgb, 255:red, 0; green, 0; blue, 0 }  ][line width=0.75]    (10.93,-3.29) .. controls (6.95,-1.4) and (3.31,-0.3) .. (0,0) .. controls (3.31,0.3) and (6.95,1.4) .. (10.93,3.29)   ;
        
        % Connection
        \draw    (309.4,110) -- (278,110) ;
        \draw [shift={(276,110)}, rotate = 360] [color={rgb, 255:red, 0; green, 0; blue, 0 }  ][line width=0.75]    (10.93,-3.29) .. controls (6.95,-1.4) and (3.31,-0.3) .. (0,0) .. controls (3.31,0.3) and (6.95,1.4) .. (10.93,3.29)   ;
        
        % Connection
        \draw    (258,110) -- (218.6,110) ;
        \draw [shift={(216.6,110)}, rotate = 360] [color={rgb, 255:red, 0; green, 0; blue, 0 }  ][line width=0.75]    (10.93,-3.29) .. controls (6.95,-1.4) and (3.31,-0.3) .. (0,0) .. controls (3.31,0.3) and (6.95,1.4) .. (10.93,3.29)   ;
        
        % Connection
        \draw    (309.4,170) -- (278,170) ;
        \draw [shift={(276,170)}, rotate = 360] [color={rgb, 255:red, 0; green, 0; blue, 0 }  ][line width=0.75]    (10.93,-3.29) .. controls (6.95,-1.4) and (3.31,-0.3) .. (0,0) .. controls (3.31,0.3) and (6.95,1.4) .. (10.93,3.29)   ;
        
        % Connection
        \draw    (258,168.07) .. controls (230.15,163.58) and (213.3,149.18) .. (207.42,124.89) ;
        \draw [shift={(206.99,123.01)}, rotate = 437.91] [color={rgb, 255:red, 0; green, 0; blue, 0 }  ][line width=0.75]    (10.93,-3.29) .. controls (6.95,-1.4) and (3.31,-0.3) .. (0,0) .. controls (3.31,0.3) and (6.95,1.4) .. (10.93,3.29)   ;
        
        % Connection
        \draw    (374,110) -- (338.6,110) ;
        \draw [shift={(336.6,110)}, rotate = 360] [color={rgb, 255:red, 0; green, 0; blue, 0 }  ][line width=0.75]    (10.93,-3.29) .. controls (6.95,-1.4) and (3.31,-0.3) .. (0,0) .. controls (3.31,0.3) and (6.95,1.4) .. (10.93,3.29)   ;
        
        % Connection
        \draw    (423.4,110) -- (394,110) ;
        \draw [shift={(392,110)}, rotate = 360] [color={rgb, 255:red, 0; green, 0; blue, 0 }  ][line width=0.75]    (10.93,-3.29) .. controls (6.95,-1.4) and (3.31,-0.3) .. (0,0) .. controls (3.31,0.3) and (6.95,1.4) .. (10.93,3.29)   ;
        
        % Connection
        \draw    (435.82,123.55) .. controls (435.9,150.31) and (416.89,165.24) .. (393.78,168.34) ;
        \draw [shift={(392,168.55)}, rotate = 353.9] [color={rgb, 255:red, 0; green, 0; blue, 0 }  ][line width=0.75]    (10.93,-3.29) .. controls (6.95,-1.4) and (3.31,-0.3) .. (0,0) .. controls (3.31,0.3) and (6.95,1.4) .. (10.93,3.29)   ;
        
        % Connection
        \draw    (374,170) -- (338.6,170) ;
        \draw [shift={(336.6,170)}, rotate = 360] [color={rgb, 255:red, 0; green, 0; blue, 0 }  ][line width=0.75]    (10.93,-3.29) .. controls (6.95,-1.4) and (3.31,-0.3) .. (0,0) .. controls (3.31,0.3) and (6.95,1.4) .. (10.93,3.29)   ;
        
        % Connection
        \draw    (372,50.52) .. controls (341.88,50.18) and (326.06,64.88) .. (324.53,94.64) ;
        \draw [shift={(324.46,96.48)}, rotate = 271.78] [color={rgb, 255:red, 0; green, 0; blue, 0 }  ][line width=0.75]    (10.93,-3.29) .. controls (6.95,-1.4) and (3.31,-0.3) .. (0,0) .. controls (3.31,0.3) and (6.95,1.4) .. (10.93,3.29)   ;
        
        % Connection
        \draw    (372,229.65) .. controls (341.88,230.53) and (326.05,215.79) .. (324.5,185.41) ;
        \draw [shift={(324.42,183.53)}, rotate = 448.21] [color={rgb, 255:red, 0; green, 0; blue, 0 }  ][line width=0.75]    (10.93,-3.29) .. controls (6.95,-1.4) and (3.31,-0.3) .. (0,0) .. controls (3.31,0.3) and (6.95,1.4) .. (10.93,3.29)   ;
        
        % Connection
        \draw    (423.4,50) -- (394,50) ;
        \draw [shift={(392,50)}, rotate = 360] [color={rgb, 255:red, 0; green, 0; blue, 0 }  ][line width=0.75]    (10.93,-3.29) .. controls (6.95,-1.4) and (3.31,-0.3) .. (0,0) .. controls (3.31,0.3) and (6.95,1.4) .. (10.93,3.29)   ;
        
        % Connection
        \draw    (423.4,230.75) -- (394,230.22) ;
        \draw [shift={(392,230.18)}, rotate = 361.03999999999996] [color={rgb, 255:red, 0; green, 0; blue, 0 }  ][line width=0.75]    (10.93,-3.29) .. controls (6.95,-1.4) and (3.31,-0.3) .. (0,0) .. controls (3.31,0.3) and (6.95,1.4) .. (10.93,3.29)   ;
        
        % Connection
        \draw    (423.4,290) -- (393,290) ;
        \draw [shift={(391,290)}, rotate = 360] [color={rgb, 255:red, 0; green, 0; blue, 0 }  ][line width=0.75]    (10.93,-3.29) .. controls (6.95,-1.4) and (3.31,-0.3) .. (0,0) .. controls (3.31,0.3) and (6.95,1.4) .. (10.93,3.29)   ;
        
        % Connection
        \draw    (373,289.72) .. controls (234.96,286.59) and (143.34,231.38) .. (98.15,124.08) ;
        \draw [shift={(97.47,122.46)}, rotate = 427.47] [color={rgb, 255:red, 0; green, 0; blue, 0 }  ][line width=0.75]    (10.93,-3.29) .. controls (6.95,-1.4) and (3.31,-0.3) .. (0,0) .. controls (3.31,0.3) and (6.95,1.4) .. (10.93,3.29)   ;
        
        
        \end{tikzpicture}
        \\
        \item Наконец, обрабатывая вершину $v_7$, читаем символ '$\$$' и строим узел $v_8$, который соответствует допускающим состоянием: \\ \\
        Вход: \,
        \begin{tabular}[c]{ |c|c|c|c| } 
            \hline a & b & c & \textcolor{red}{\$} \\ \hline
        \end{tabular}
        \qquad GSS: \,
        \begin{tikzpicture}[x=0.5pt,y=0.5pt,yscale=-1,xscale=1]
        %uncomment if require: \path (0,422); %set diagram left start at 0, and has height of 422
        
        
        % Text Node
        \draw  [line width=0.75]   (92, 110) circle [x radius= 13.6, y radius= 13.6]   ;
        \draw (92,110) node [color={rgb, 255:red, 62; green, 45; blue, 45 }  ,opacity=1 ]  {$0$};
        % Text Node
        \draw  [line width=0.75]   (138,98) -- (156,98) -- (156,122) -- (138,122) -- cycle  ;
        \draw (147,110) node [scale=1,color={rgb, 255:red, 62; green, 45; blue, 45 }  ,opacity=1 ]  {а};
        % Text Node
        \draw  [line width=0.75]   (203, 110) circle [x radius= 13.6, y radius= 13.6]   ;
        \draw (203,110) node [color={rgb, 255:red, 62; green, 45; blue, 45 }  ,opacity=1 ]  {$2$};
        % Text Node
        \draw (110,89) node [color={rgb, 255:red, 62; green, 45; blue, 45 }  ,opacity=1 ]  {$v_{0}$};
        % Text Node
        \draw (220,89) node [color={rgb, 255:red, 62; green, 45; blue, 45 }  ,opacity=1 ]  {$v_{1}$};
        % Text Node
        \draw  [line width=0.75]   (258,98) -- (276,98) -- (276,122) -- (258,122) -- cycle  ;
        \draw (267,110) node [scale=1,color={rgb, 255:red, 62; green, 45; blue, 45 }  ,opacity=1 ]  {b};
        % Text Node
        \draw  [line width=0.75]   (323, 110) circle [x radius= 13.6, y radius= 13.6]   ;
        \draw (323,110) node [color={rgb, 255:red, 62; green, 45; blue, 45 }  ,opacity=1 ]  {$3$};
        % Text Node
        \draw (340,90) node [color={rgb, 255:red, 62; green, 45; blue, 45 }  ,opacity=1 ]  {$v_{2}$};
        % Text Node
        \draw  [line width=0.75]   (258,158) -- (276,158) -- (276,182) -- (258,182) -- cycle  ;
        \draw (267,170) node [scale=1,color={rgb, 255:red, 62; green, 45; blue, 45 }  ,opacity=1 ]  {B};
        % Text Node
        \draw  [line width=0.75]   (323, 170) circle [x radius= 13.6, y radius= 13.6]   ;
        \draw (323,170) node [color={rgb, 255:red, 62; green, 45; blue, 45 }  ,opacity=1 ]  {$4$};
        % Text Node
        \draw (340,149) node [color={rgb, 255:red, 62; green, 45; blue, 45 }  ,opacity=1 ]  {$v_{3}$};
        % Text Node
        \draw  [line width=0.75]   (374,98) -- (392,98) -- (392,122) -- (374,122) -- cycle  ;
        \draw (383,110) node [scale=1,color={rgb, 255:red, 62; green, 45; blue, 45 }  ,opacity=1 ]  {c};
        % Text Node
        \draw  [line width=0.75]   (374,158) -- (392,158) -- (392,182) -- (374,182) -- cycle  ;
        \draw (383,170) node [scale=1,color={rgb, 255:red, 62; green, 45; blue, 45 }  ,opacity=1 ]  {c};
        % Text Node
        \draw  [line width=0.75]   (437, 110) circle [x radius= 13.6, y radius= 13.6]   ;
        \draw (437,110) node [color={rgb, 255:red, 62; green, 45; blue, 45 }  ,opacity=1 ]  {$6$};
        % Text Node
        \draw (450,89) node [color={rgb, 255:red, 62; green, 45; blue, 45 }  ,opacity=1 ]  {$v_{4}$};
        % Text Node
        \draw  [line width=0.75]   (372,38) -- (392,38) -- (392,62) -- (372,62) -- cycle  ;
        \draw (382,50) node [scale=1,color={rgb, 255:red, 62; green, 45; blue, 45 }  ,opacity=1 ]  {C};
        % Text Node
        \draw  [line width=0.75]   (372,218) -- (392,218) -- (392,242) -- (372,242) -- cycle  ;
        \draw (382,230) node [scale=1,color={rgb, 255:red, 62; green, 45; blue, 45 }  ,opacity=1 ]  {C};
        % Text Node
        \draw  [line width=0.75]   (437, 50) circle [x radius= 13.6, y radius= 13.6]   ;
        \draw (437,50) node [color={rgb, 255:red, 62; green, 45; blue, 45 }  ,opacity=1 ]  {$5$};
        % Text Node
        \draw  [line width=0.75]   (437, 231) circle [x radius= 13.6, y radius= 13.6]   ;
        \draw (437,231) node [color={rgb, 255:red, 62; green, 45; blue, 45 }  ,opacity=1 ]  {$7$};
        % Text Node
        \draw (452,28) node [color={rgb, 255:red, 62; green, 45; blue, 45 }  ,opacity=1 ]  {$v_{5}$};
        % Text Node
        \draw (452,208) node [color={rgb, 255:red, 62; green, 45; blue, 45 }  ,opacity=1 ]  {$v_{6}$};
        % Text Node
        \draw  [line width=0.75]   (373,278) -- (391,278) -- (391,302) -- (373,302) -- cycle  ;
        \draw (382,290) node [scale=1,color={rgb, 255:red, 62; green, 45; blue, 45 }  ,opacity=1 ]  {S};
        % Text Node
        \draw  [line width=0.75]   (437, 290) circle [x radius= 13.6, y radius= 13.6]   ;
        \draw (437,290) node [color={rgb, 255:red, 62; green, 45; blue, 45 }  ,opacity=1 ]  {$1$};
        % Text Node
        \draw (452,268) node [color={rgb, 255:red, 62; green, 45; blue, 45 }  ,opacity=1 ]  {$v_{7}$};
        % Text Node
        \draw  [line width=0.75]   (488,278) -- (506,278) -- (506,302) -- (488,302) -- cycle  ;
        \draw (497,290) node [scale=1,color={rgb, 255:red, 62; green, 45; blue, 45 }  ,opacity=1 ]  {\$};
        % Text Node
        \draw  [line width=0.75]   (557, 290) circle [x radius= 17.8, y radius= 17.8]   ;
        \draw (557,290) node [color={rgb, 255:red, 62; green, 45; blue, 45 }  ,opacity=1 ]  {acc};
        % Text Node
        \draw (575,262) node [color={rgb, 255:red, 62; green, 45; blue, 45 }  ,opacity=1 ]  {$v_{8}$};
        % Connection
        \draw    (138,110) -- (107.6,110) ;
        \draw [shift={(105.6,110)}, rotate = 360] [color={rgb, 255:red, 0; green, 0; blue, 0 }  ][line width=0.75]    (10.93,-3.29) .. controls (6.95,-1.4) and (3.31,-0.3) .. (0,0) .. controls (3.31,0.3) and (6.95,1.4) .. (10.93,3.29)   ;
        
        % Connection
        \draw    (189.4,110) -- (158,110) ;
        \draw [shift={(156,110)}, rotate = 360] [color={rgb, 255:red, 0; green, 0; blue, 0 }  ][line width=0.75]    (10.93,-3.29) .. controls (6.95,-1.4) and (3.31,-0.3) .. (0,0) .. controls (3.31,0.3) and (6.95,1.4) .. (10.93,3.29)   ;
        
        % Connection
        \draw    (309.4,110) -- (278,110) ;
        \draw [shift={(276,110)}, rotate = 360] [color={rgb, 255:red, 0; green, 0; blue, 0 }  ][line width=0.75]    (10.93,-3.29) .. controls (6.95,-1.4) and (3.31,-0.3) .. (0,0) .. controls (3.31,0.3) and (6.95,1.4) .. (10.93,3.29)   ;
        
        % Connection
        \draw    (258,110) -- (218.6,110) ;
        \draw [shift={(216.6,110)}, rotate = 360] [color={rgb, 255:red, 0; green, 0; blue, 0 }  ][line width=0.75]    (10.93,-3.29) .. controls (6.95,-1.4) and (3.31,-0.3) .. (0,0) .. controls (3.31,0.3) and (6.95,1.4) .. (10.93,3.29)   ;
        
        % Connection
        \draw    (309.4,170) -- (278,170) ;
        \draw [shift={(276,170)}, rotate = 360] [color={rgb, 255:red, 0; green, 0; blue, 0 }  ][line width=0.75]    (10.93,-3.29) .. controls (6.95,-1.4) and (3.31,-0.3) .. (0,0) .. controls (3.31,0.3) and (6.95,1.4) .. (10.93,3.29)   ;
        
        % Connection
        \draw    (258,168.07) .. controls (230.15,163.58) and (213.3,149.18) .. (207.42,124.89) ;
        \draw [shift={(206.99,123.01)}, rotate = 437.91] [color={rgb, 255:red, 0; green, 0; blue, 0 }  ][line width=0.75]    (10.93,-3.29) .. controls (6.95,-1.4) and (3.31,-0.3) .. (0,0) .. controls (3.31,0.3) and (6.95,1.4) .. (10.93,3.29)   ;
        
        % Connection
        \draw    (374,110) -- (338.6,110) ;
        \draw [shift={(336.6,110)}, rotate = 360] [color={rgb, 255:red, 0; green, 0; blue, 0 }  ][line width=0.75]    (10.93,-3.29) .. controls (6.95,-1.4) and (3.31,-0.3) .. (0,0) .. controls (3.31,0.3) and (6.95,1.4) .. (10.93,3.29)   ;
        
        % Connection
        \draw    (423.4,110) -- (394,110) ;
        \draw [shift={(392,110)}, rotate = 360] [color={rgb, 255:red, 0; green, 0; blue, 0 }  ][line width=0.75]    (10.93,-3.29) .. controls (6.95,-1.4) and (3.31,-0.3) .. (0,0) .. controls (3.31,0.3) and (6.95,1.4) .. (10.93,3.29)   ;
        
        % Connection
        \draw    (435.82,123.55) .. controls (435.9,150.31) and (416.89,165.24) .. (393.78,168.34) ;
        \draw [shift={(392,168.55)}, rotate = 353.9] [color={rgb, 255:red, 0; green, 0; blue, 0 }  ][line width=0.75]    (10.93,-3.29) .. controls (6.95,-1.4) and (3.31,-0.3) .. (0,0) .. controls (3.31,0.3) and (6.95,1.4) .. (10.93,3.29)   ;
        
        % Connection
        \draw    (374,170) -- (338.6,170) ;
        \draw [shift={(336.6,170)}, rotate = 360] [color={rgb, 255:red, 0; green, 0; blue, 0 }  ][line width=0.75]    (10.93,-3.29) .. controls (6.95,-1.4) and (3.31,-0.3) .. (0,0) .. controls (3.31,0.3) and (6.95,1.4) .. (10.93,3.29)   ;
        
        % Connection
        \draw    (372,50.52) .. controls (341.88,50.18) and (326.06,64.88) .. (324.53,94.64) ;
        \draw [shift={(324.46,96.48)}, rotate = 271.78] [color={rgb, 255:red, 0; green, 0; blue, 0 }  ][line width=0.75]    (10.93,-3.29) .. controls (6.95,-1.4) and (3.31,-0.3) .. (0,0) .. controls (3.31,0.3) and (6.95,1.4) .. (10.93,3.29)   ;
        
        % Connection
        \draw    (372,229.65) .. controls (341.88,230.53) and (326.05,215.79) .. (324.5,185.41) ;
        \draw [shift={(324.42,183.53)}, rotate = 448.21] [color={rgb, 255:red, 0; green, 0; blue, 0 }  ][line width=0.75]    (10.93,-3.29) .. controls (6.95,-1.4) and (3.31,-0.3) .. (0,0) .. controls (3.31,0.3) and (6.95,1.4) .. (10.93,3.29)   ;
        
        % Connection
        \draw    (423.4,50) -- (394,50) ;
        \draw [shift={(392,50)}, rotate = 360] [color={rgb, 255:red, 0; green, 0; blue, 0 }  ][line width=0.75]    (10.93,-3.29) .. controls (6.95,-1.4) and (3.31,-0.3) .. (0,0) .. controls (3.31,0.3) and (6.95,1.4) .. (10.93,3.29)   ;
        
        % Connection
        \draw    (423.4,230.75) -- (394,230.22) ;
        \draw [shift={(392,230.18)}, rotate = 361.03999999999996] [color={rgb, 255:red, 0; green, 0; blue, 0 }  ][line width=0.75]    (10.93,-3.29) .. controls (6.95,-1.4) and (3.31,-0.3) .. (0,0) .. controls (3.31,0.3) and (6.95,1.4) .. (10.93,3.29)   ;
        
        % Connection
        \draw    (423.4,290) -- (393,290) ;
        \draw [shift={(391,290)}, rotate = 360] [color={rgb, 255:red, 0; green, 0; blue, 0 }  ][line width=0.75]    (10.93,-3.29) .. controls (6.95,-1.4) and (3.31,-0.3) .. (0,0) .. controls (3.31,0.3) and (6.95,1.4) .. (10.93,3.29)   ;
        
        % Connection
        \draw    (373,289.72) .. controls (234.96,286.59) and (143.34,231.38) .. (98.15,124.08) ;
        \draw [shift={(97.47,122.46)}, rotate = 427.47] [color={rgb, 255:red, 0; green, 0; blue, 0 }  ][line width=0.75]    (10.93,-3.29) .. controls (6.95,-1.4) and (3.31,-0.3) .. (0,0) .. controls (3.31,0.3) and (6.95,1.4) .. (10.93,3.29)   ;
        
        % Connection
        \draw    (488,290) -- (452.6,290) ;
        \draw [shift={(450.6,290)}, rotate = 360] [color={rgb, 255:red, 0; green, 0; blue, 0 }  ][line width=0.75]    (10.93,-3.29) .. controls (6.95,-1.4) and (3.31,-0.3) .. (0,0) .. controls (3.31,0.3) and (6.95,1.4) .. (10.93,3.29)   ;
        
        % Connection
        \draw    (539.2,290) -- (508,290) ;
        \draw [shift={(506,290)}, rotate = 360] [color={rgb, 255:red, 0; green, 0; blue, 0 }  ][line width=0.75]    (10.93,-3.29) .. controls (6.95,-1.4) and (3.31,-0.3) .. (0,0) .. controls (3.31,0.3) and (6.95,1.4) .. (10.93,3.29)   ;
        
        
        \end{tikzpicture}
        \\
    \end{enumerate}
    
    
    
\end{example}

\subsection{Модификации GLR}
Алгоритм, представленный Томитой имел большой недостаток: он корректно работал не со всеми КС грамматиками, хоть и расширял класс допустимых LR анализаторами. Объем потребляемой памяти классическим GLR можно оценить как $ O(n^3)$ с учетом оптимизаций, о которых говорилось ранее.

Спустя некоторое время после публикации Томита-парсера, Элизабет Скотт и Эндриан Джонстоун представили $RNGLR$ (Right Nulled GLR)~\cite{Scott:2006:RNG:1146809.1146810} --- модифицированная версия GLR, которая решала проблему скрытых рекурсий. Это позволило расширить класс допускаемых грамматик до КС. Однако объем потребляемой памяти можно оценить сверху уже полиномом $O(n^{k+1})$, где k --- длина самого длинного правила грамматики, что несколько ухудшило оценку классического GLR.

С этой проблемой справился BRNGLR (Binary RNGLR)~\cite{Scott:2007:BCT:1289813.1289815}. За счет бинаризации удалось получить кубическую оценку сложности и при этом также, как и RNGLR, допускать все КС грамматики.

Кроме того, GLR довольно естесственно обобщается до решения задачи поиска путей с КС ограничениями. Это происходит следующим образом: элементами во входной структуре теперь будем считать не позиции символа в слове, а вершины графа (то есть ``позици'' и множество смежных вершин). Это приводит к тому, что при применении операции shift, следующих символов может быть несколько и каждый из них должен быть рассмотрен отдельно, сдвигаясь по соответствующему ребру и проходя входной граф в ширину. Подробное описание алгоритма и псевдокод представлены в работе~\cite{10.1007/978-3-319-41579-6_22}.

%\section{Вопросы и задачи}
%\begin{enumerate}
%    \item Постройте LR автомат и управляющую таблицу для грамматики $G_1$: $S \to a S b$; $S \to \epsilon$.
%    \item Постройте LR автомат и управляющую таблицу для грамматики $G_2$: $S \to S S S$; $S \to S S$; $S \to a$.
%    \item Проведите GLR разбор для грамматики $G_2$ и входного слова $w = aaa\$$.
%    \item Реализуйте LR анализатор на любом языке программирования. Программа должна принимать на вход файл с однозначной грамматикой и входное слово, строить LR автомат и управляющую таблицу (во внутреннем представлении), и сообщать, выводимо ли входное слово в данной грамматике.
%    \item[6*.] Реализуйте GLR анализатор на любом языке программирования. Программа должна принимать на вход файл с однозначной грамматикой и входное слово, работать согласно алгоритму GLR и сохранять GSS, а также сообщать, выводимо ли входное слово в данной грамматике.
%\end{enumerate}

%\input{CombinatorsForCFPQ}
%\input{DerivativesForCFPQ}
%\input{CFPQ_to_Datalog}
%\input{Multiple_Context-Free_Languages}
%\chapter{Конъюнктивные и булевы грамматики}

Впервые конъюнктивные и булевы грамматики были предложены Александром Охотиным~\cite{DBLP:journals/jalc/Okhotin01,Okhotin:2003:BG:1758089.1758123}. Дадим определение конъюнктивной грамматики.

\begin{definition}
    \textit{Конъюнктивной грамматикой} называется $G = (\Sigma,N,P,S)$, где:
    \begin{itemize}
        \item $\Sigma$ и $N$ --- дизъюнктивные конечные непустые множества терминалов и нетерминалов.
        \item $P$ --- конечное множество продукций, каждая вида
        \[
        A\rightarrow \alpha_1\&...\&\alpha_n
        \]
        ,где $A \in N,n \geq 1$ и $\alpha_1,...,\alpha_n \in (\Sigma \cup N)^*$.
        \item $S \in N$  --- стартовый нетерминал.
    \end{itemize}
\end{definition}

Конъюнктивная грамматика генерирует строки, выводя их из начального символа, так же, как это происходит в контекстно-свободных грамматиках в параграфе~\ref{CFG}. Промежуточные строки, используемые в процессе вывода, являются формулами следующего вида:

\begin{definition}\label{Definition of conjunctive formula}
    Пусть $G = (\Sigma,N,P,S)$ --- конъюнктивная грамматика. Множество конъюнктивных формул $ \mathcal{F}$ определяется индуктивно:
    \begin{itemize}
        \item Пустая строка $\varepsilon$ --- конъюнктивная формула. 
        \item Любой символ из $(\Sigma \cup N)$ --- формула.
        \item Если $\mathcal{A}$ и $\mathcal{B}$ непустые формулы, тогда $\mathcal{AB}$ --- формула.
        \item Если $\mathcal{A}_1,\ldots,\mathcal{A}_n$ $(n \geqslant 1)$ --- формула, тогда $(\mathcal{A}_1\&\ldots\&\mathcal{A}_n)$ --- формула.
    \end{itemize}
\end{definition}

\begin{definition}
    Пусть $G = (\Sigma,N,P,S)$ --- конъюнктивная грамматика. Аналогично определению отношения непосредственной выводимости в контекстно-свободной грамматике~\ref{def derivability in CFG} определим $\xRightarrow[G]{}$ как отношение непосредственной выводимости на множестве конъюнктивных формул.
    \begin{itemize}
        \item Любой нетерминал в любой формуле может быть перезаписан телом любого правила для этого терминала заключенным в скобки. То есть для любых $s^{'},s^{''} \in (\Sigma \cup N \cup \{(, \&, )\})^*$ и $A\in N$, таких что $s^{'}As^{''}$ --- формула, и для всех правил вида $A \rightarrow \alpha_1\&\ldots\&\alpha_n \in P$, имеем $s^{'}As^{''}\xRightarrow[G]{}s^{'}(\alpha_1\&\ldots\&\alpha_n)s^{''}$. 
        \item Если формула содержит подформулу в виде конъюнкции одной или нескольких одинаковых терминальных строк, заключенных в скобки, тогда подформула может быть перезаписана терминальной строкой без скобок. То есть для любых $s^{'},s^{''} \in (\Sigma \cup N \cup \{(, \&, )\})^*$, $(n \geqslant 1)$ и $w \in \Sigma^*$, таких что $s^{'}(w\&\ldots\&w)s^{''}$ --- формула, имеем $s^{'}(w\&\ldots\&w)s^{''}\xRightarrow[G]{}s^{'}ws^{''}$.
    \end{itemize}
    Как и в случае контекстно-свободной грамматики обозначим $\xRightarrow[G]{}^*$ рефлексивное транзитивное замыкание отношения $\xRightarrow[G]{}$.
\end{definition}

\begin{definition}
    Пусть $G = (\Sigma,N,P,S)$ --- конъюнктивная грамматика. Язык, порождаемый формулой, это множество всех терминальных строк выводимых из этой формулы: $L_{G}(\mathcal{A}) = \{w\in\Sigma^* \mid \mathcal{A} \xRightarrow[G]{}^*w\}$. Очевидно, что язык порождаемый грамматикой, это язык порождаемый стартовым нетерминалом $S$ : $L(G) = L_{G}(S) = L(S)$.
\end{definition}

\begin{theorem}\label{Theorem language generated by a formula}
    Пусть $G = (\Sigma,N,P,S)$ --- конъюнктивная грамматика. Пусть $\mathcal{A}_1,\ldots,\mathcal{A}_n,\mathcal{B}$ --- формулы, $A \in N$, $a \in \Sigma$. Тогда,
    \begin{enumerate}
        \item $L(\varepsilon) = \{\varepsilon\}$.
        \item $L(a) = \{a\}$.
        \item $L(A) = \bigcup_{A \rightarrow \alpha_1\&\ldots\&\alpha_n \in P} L((\alpha_1\&\ldots\&\alpha_m))$.
        \item $L(\mathcal{AB}) = L(\mathcal{A})*L(\mathcal{B})$
        \item $L((\mathcal{A}_1\&\ldots\&\mathcal{A}_n)) = \bigcap_{i = 1}^{n}L(\mathcal{A}_i)$.
    \end{enumerate}
\end{theorem}

Теорема~\ref{Theorem language generated by a formula} уже подразумевает интерпретацию грамматики как системы уравнений. Используем математический подход, чтобы лучше охарактеризовать конъюнктивные языки с помощью систем уравнений.

\begin{definition}[Выражение]
    Пусть $\Sigma$ конечный непустой алфавит. Пусть $X = \{X_1,\ldots,X_N\}$ вектор переменных. Выражение над алфавитом $\Sigma$, зависящее от переменных $X$, определяется индуктивно:
    \begin{itemize}
       \item $\varepsilon$ --- выражение.
       \item Любой символ $a\in\Sigma$ --- выражение.
       \item Любая переменная $X_i\in X$ --- выражение.
       \item Если $\phi_1$ и $\phi_2$ выражения, то $\phi_1\phi_2, (\phi_1\mid\phi_2), (\phi_1\&\phi_2)$ также выражения.
    \end{itemize}
    Заметим, что любая формула, в терминах определения~\ref{Definition of conjunctive formula}, является выражением, где нетерминалы формулы это переменные выражения. С другой стороны, любое выражение, не содержащее дизъюнкции, формула.
\end{definition}

Предположим, что переменные $X_i$ приняли в качестве значений слова из языка над алфавитом $\Sigma$. Определим значение всего выражения.

\begin{definition}[Значение выражения]\label{Value of conjunctive expression}
    Пусть $L = (L_1,\ldots,L_n) (L_i \subseteq \Sigma^*)$ вектор из $n$ языков над $\Sigma$, где $n \geqslant 1$. Пусть $\phi$ выражение над $\Sigma$, зависящее от переменных $X_1,\ldots,X_n$. Значение выражения $\phi$ на векторе $L$ --- это язык над тем же алфавитом $\Sigma$. Обозначим его $\phi(L)$ и определим индуктивно на структуре выражения:
    \begin{itemize}
       \item $\varepsilon(L) = \{\varepsilon\}$.
       \item $a(L) = \{a\}$ для любого $a\in\Sigma$.
       \item $X_i(L) = L_i$ для любого $X_i \in X$.
       \item $\phi_1\phi_2 = \phi_1(L) * \phi_2(L), (\phi_1\mid\phi_2)(L) = \phi_1(L) \cup \phi_2(L), (\phi_1\&\phi_2)(L) = \phi_1(L) \cap \phi_2(L)$ для любых выражений $\phi_1$ и $\phi_2$.
    \end{itemize}
\end{definition}

Обобщим определение~\ref{Value of conjunctive expression} на случай вектора выражений.

\begin{definition}[Значение вектора выражений]
    Пусть $L = (L_1,\ldots,L_n) (L_i \subseteq \Sigma^*)$ вектор из $n$ языков над $\Sigma$, где $n \geqslant 1$. Пусть $\phi_1,\ldots,\phi_m$ выражения над $\Sigma$, зависящее от переменных $X_1,\ldots,X_n$. Значение вектора выражений $P = (\phi_1,\ldots,\phi_m)$ на векторе $L$ --- это вектор языков $P(L) = (\phi_1(L),\ldots,\phi_m(L))$ над тем же алфавитом $\Sigma$. 
\end{definition}

Зададим частичный порядок относительно включения $``\preccurlyeq"$ на множестве языков и расширим его на вектора языков длины $n$: $(L_1^{'},\ldots,L_n^{'})\preccurlyeq(L_1^{''},\ldots,L_n^{''})$ если и только если $L_i^{'} \subseteq L_i^{''}$ для любого $1\leqslant i \leqslant n$

\begin{definition}\label{Definition a conjuctive system of equations}
   $X = P(X)$ система уравнений над алфавитом $\Sigma$ и $X = \{X_1,\ldots,X_n\}$, где $P = (\phi_1,\ldots,\phi_n)$ вектор выражений над алфавитом $\Sigma$, зависящий от $X$.
   
   Вектор языков $L = (L_1,\ldots,L_n)$ является решением системы уравнений если $L = P(L)$.
   
   Наименьшее решение $L$ это вектор языков, такой что для любого другого сравнимого вектора языков $L^{'}$ выполняется $L \preccurlyeq L^{'}$.
\end{definition}

Заметим, что оператор $P$ на множестве $2^{\Sigma}\times\ldots\times2^{\Sigma}$, что решение системы~\ref{Definition a conjuctive system of equations} это неподвижная точка $P$ и что наименьшее решение системы это наименьшая неподвижная точка оператора $P$.

\begin{theorem}\label{Theorem of a least fixed point solution}
    Для любой системы из определения~\ref{Definition a conjuctive system of equations} с переменными $X_1,\ldots,X_n$, оператор $P = (\phi_1,\ldots,\phi_n)$ имеет наименьшую неподвижную точку $L = (L_1,\ldots,L_n) = \lim_{i\to\infty}P^{i}\underbrace{(\varnothing,\ldots,\varnothing)}_n$.
\end{theorem}

Приведем пример конъюнктивной грамматики.

\begin{example}[Пример конъюнктивной грамматики]
    Следующая конъюнктивная грамматика $G$ порождает язык $\{a^nb^nc^n\mid n \geq 0\}$:
    
    \begin{align*}
    1.\ S   &\to A B \& D C \\
    2.\ A  &\to a A \mid \varepsilon \\ 
    3.\ B &\to b B c \mid \varepsilon \\
    4.\ C   &\to c C \mid \varepsilon \\ 
    5.\ D   &\to aDb \mid \varepsilon
    \end{align*}
    
    Легко видеть, что $L(AB) = \{a^ib^jc^k\mid j = k\}$ и $L(DC) = \{a^ib^jc^k\mid i = j\}$. Тогда $L(S) = L(AB) \cap L(DC) = \{a^nb^nc^n\mid n \geq 0\}$. 
    
    В этой грамматике строка $abc$ может быть получена следующим образом. Для начала представим грамматику в виде системы уравнений:
    \begin{align*}
    S &= A B \cap D C \\ 
    A &= \{a\}A \cup \varepsilon \\ 
    B &= \{b\}B\{c\} \cup \varepsilon \\
    C &= \{c\}C \cup \varepsilon \\ 
    D &= \{a\}D\{b\} \cup \varepsilon
    \end{align*}
    Используя теорему~\ref{Theorem of a least fixed point solution}, будем итеративно вычислять $P^{i}\underbrace{(\varnothing,\ldots,\varnothing)}_5$. На каждом шаге будем подставлять все терминальные строки из языков, порожденных нетерминалами на предыдущем шаге, в соответствующие нетерминалы правой части каждого уравнения и записывать получившиеся терминальные строки в языки нетерминалов текущего шага. Продолжаем до тех пор пока язык, порождаемый нетерминалом $S$, не будет содержать терминальную строку $``abc''$.
    \begin{enumerate}
        \item На начальном этапе имеем $P^{0}(\varnothing,\ldots,\varnothing) = (S: \varnothing, A: \varnothing, B: \varnothing, C: \varnothing, D: \varnothing)$ 
        \item Подставляем в первое уравнение терминальные строки из шага 1 в соответствующие нетерминалы, т.е. 
        \begin{align*}
            S:  \varnothing &= \varnothing\varnothing \cap \varnothing\varnothing \\ 
            A: \{\varepsilon\} &= \{a\}\varnothing \cup \{\varepsilon\} \\ 
            B: \{\varepsilon\} &= \{b\}\varnothing\{c\} \cup \{\varepsilon\} \\
            C: \{\varepsilon\} &= \{c\}\varnothing \cup \{\varepsilon\} \\ 
            D: \{\varepsilon\} &= \{a\}\varnothing\{b\} \cup \{\varepsilon\}
        \end{align*}
        В конце итерации получаем $P^{1}(\varnothing,\ldots,\varnothing) = (S: \varnothing, A: \{\varepsilon\}, B: \{\varepsilon\}, C: \{\varepsilon\}, D: \{\varepsilon\})$
        \item Делаем еще одну итерацию,
        \begin{align*}
            S:  \{\varepsilon\} &= \{\varepsilon\}\{\varepsilon\} \cap \{\varepsilon\}\{\varepsilon\} \\ 
            A: \{a, \varepsilon\} &= \{a\}\{\varepsilon\} \cup \{\varepsilon\} \\ 
            B: \{bc, \varepsilon\} &= \{b\}\{\varepsilon\}\{c\} \cup \{\varepsilon\} \\
            C: \{c, \varepsilon\} &= \{c\}\{\varepsilon\} \cup \{\varepsilon\} \\ 
            D: \{ab, \varepsilon\} &= \{a\}\{\varepsilon\}\{b\} \cup \{\varepsilon\}
        \end{align*}
        В конце итерации получаем $P^{2}(\varnothing,\ldots,\varnothing) = (S: \{\varepsilon\}, A: \{a, \varepsilon\}, B: \{bc, \varepsilon\}, C: \{c, \varepsilon\}, D: \{ab, \varepsilon\})$
        \item Еще одна итерация,
        \begin{align*}
            S:  \{\fbox{abc}, \varepsilon\} &= \{a, \varepsilon\}\{bc, \varepsilon\} \cap \{ab, \varepsilon\}\{c, \varepsilon\} \\ 
            A: \{a, aa, \varepsilon\} &= \{a\}\{a, \varepsilon\} \cup \{\varepsilon\} \\ 
            B: \{bc, bbcc, \varepsilon\} &= \{b\}\{bc, \varepsilon\}\{c\} \cup \{\varepsilon\} \\
            C: \{c, cc, \varepsilon\} &= \{c\}\{c, \varepsilon\} \cup \{\varepsilon\} \\ 
            D: \{ab, aabb, \varepsilon\} &= \{a\}\{ab, \varepsilon\}\{b\} \cup \{\varepsilon\}
        \end{align*}
        В конце итерации получили $P^{3}(\varnothing,\ldots,\varnothing) = (S: \{\fbox{abc}, \varepsilon\}, A: \{a, aa, \varepsilon\}, B: \{bc, bbcc, \varepsilon\}, C: \{c, cc, \varepsilon\}, D: \{ab, aabb, \varepsilon\})$. Заметим, что терминальная строка $``abc"$ появилась в языке, который порождает стартовый нетерминал $S$. Т.е. терминальная строка $``abc"$ выводима в грамматике $G$, что и требовалось показать.
    \end{enumerate}
    
    Заметим, что строку $``abc"$ также можно получить применением правил вывода. здесь цифра над стрелкой соответствует номеру примененного правила. 
    \begin{align*}
        S &\xRightarrow{1}(AB\&DC) \\
        &\xRightarrow{2}(aAB\&DC) \xRightarrow{2} (a\varepsilon B\&DC) \\
        &\xRightarrow{3}(abBc\&DC) \xRightarrow{3}(ab\varepsilon c\&DC) \\
        &\xRightarrow{5}(abc\&aDbC) \xRightarrow{5}(abc\&a\varepsilon bC) \\
        &\xRightarrow{4}(abc\&abcC) \xRightarrow{4}(abc\&abc\varepsilon) \\
        &\Rightarrow(abc\&abc) \Rightarrow abc
    \end{align*}
\end{example}

\begin{example}
    Конъюнктивная грамматика $G$ для языка $L = \{wcw \mid w \in \{a, b\}^*\}$:
    \begin{align*}
    S &\to C \& D \\ 
    C &\to aCa \mid aCb \mid bCa \mid bCb \mid c \\ 
    D &\to aA\&aD \mid bB\&bD \mid cE \\
    A &\to aAa \mid aAb \mid bAa \mid bAb \mid cEa \\
    B &\to aBa \mid aBb \mid bBa \mid bBb \mid cEb \\
    E &\to aE \mid bE \mid \varepsilon
    \end{align*}
\end{example}

Подробнее о конъюнктивных грамматиках можно прочитать в статьях~\cite{DBLP:journals/jalc/Okhotin01, Okhotin2002, DBLP:journals/tcs/Okhotin03a, f60a33d409364914be560cac0e54b12c}.

Дадим определение булевой грамматики.

\begin{definition}
    \textit{Булевой грамматикой} называется $G = (\Sigma,N,P,S)$, где:
    \begin{itemize}
        \item $\Sigma$ и $N$ --- дизъюнктивные конечные непустые множества терминалов и нетерминалов.
        \item $P$ --- конечное множество продукций, каждая вида
        \[
        A\rightarrow \alpha_1\&...\&\alpha_m\&\neg\beta_1\&...\&\neg\beta_n
        \]
        ,где $A \in N, m, n >=0, m+n \geq 1$ и $\alpha_i,\beta_j \in (\Sigma \cup N)^*$.
        \item $S \in N$  --- стартовый нетерминал.
    \end{itemize}
\end{definition}

Приведем пример булевой грамматики.

\begin{example}
    Следующая булева грамматика порождает язык  $\{a^mb^nc^n\mid m,n \geq 0, m \neq n \}$:
    
    \begin{align*}
    S   &\to A B \& \neg D C \\ 
    A  &\to a A \mid \varepsilon \\ 
    B &\to b B c \mid \varepsilon \\
    C   &\to c C \mid \varepsilon \\ 
    D   &\to aDb \mid \varepsilon
    \end{align*}
    
    Очевидно, что $L(AB) = \{a^mb^nc^n\mid m,n \in \mathbb{N}\}$ и $L(DC) = \{a^nb^nc^m\mid m,n \in \mathbb{N}\}$. Тогда $L(AB)\cap\overline{L(DC)} = \{a^mb^nc^n\mid m,n \geq 0, m \neq n \}$.
\end{example}

Подробнее о булевых грамматиках можно прочитать в статьях~\cite{Okhotin:2003:BG:1758089.1758123,Okhotin:2014:PMM:2565359.2565379}.

Определим бинарную нормальную форму конъюнктивной грамматики.
\begin{definition}[Бинарная нормальная форма]
    Конъюнктивная грамматика $G = (\Sigma, N, P, S)$ находится в бинарной нормальной форме, если каждое правило из P имеет вид,
    \begin{itemize}
        \item $A \rightarrow B_1 C_1 \& \ldots\& B_m C_m$, где $m \geqslant 1; A,B_i,C_i \in N$.
        \item $A \rightarrow a$, где $A \in N, a \in \Sigma$.
        \item $S \rightarrow \varepsilon$, если только $S$ не содержится в правой части всех правил.
    \end{itemize}
\end{definition}

\begin{theorem}\label{Binary normal form conjunctive grammar theorem}
    Для каждой конъюнктивной грамматики $G$ можно построить конъюнктивную грамматику в бинарной нормальной форме $G^{'}$, такую что $L(G) = L(G^{'})$.
\end{theorem}
Доказательство теоремы~\ref{Binary normal form conjunctive grammar theorem} описано в статье~\cite{DBLP:journals/jalc/Okhotin01}.



%\section{Вопросы и задачи}
%\begin{enumerate}
%  \item !!! 
%  \item !!!
%\end{enumerate}

%\input{Conclusion}

\bibliographystyle{abbrv}
\bibliography{Formal_lang_CFPQ_course_notes}


\end{document}
