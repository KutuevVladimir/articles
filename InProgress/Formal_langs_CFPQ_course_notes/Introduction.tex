\chapter*{Введение}

Теория формальных языков находит применение не только для ставших уже классическими задач синтаксического анализа кода (языков программирования, искусственных языков) и естественных языков, но и в других областях, таких как статический анализ кода, графовые базы данных, биоинформатика, машинное обучение.

Например, в машинном обучении использование формальных грамматик позволяет передать искусственной нейронной сети, предназначенной для генерации цепочек с определёнными свойствами (генеративной нейронной сети), знания о синтаксической структуре этих цепочек, что позволяет существенно упростить процесс обучения и повысить качество результата~\cite{10.5555/3305381.3305582}.
Вместе с этим, развиваются подходы, позволяющие нейронным сетям наоборот извлекать синтаксическую структуру (строить дерево вывода) для входных цепочек~\cite{kasai-etal-2017-tag,kasai-etal-2018-end}.

В биоинформатике формальные грамматики нашли широкое применение для описания особенностей вторичной структуры геномных и белковых последовательностей~\cite{Dyrka2019,WJAnderson2012,zier2013rna}.
Соответствующие алгоритмы синтаксического анализа используются при создании инструментов обработки данных.

Таким образом, теория формальных языков выступает в качестве основы для многих прикладных областей, а алгоритмы синтаксического анлиза применимы не только для обработки естественных языков или языков программирования.
Нас же в данной работе будет интересовать применение теории формальных языков и алгоритмов синтаксического анализа для анализа графовых баз данных и для статического анализа кода.

Одна из классических задач, связанных с анализом графов --- это поиск путей в графе.
Возможны различные формулировки этой задачи.
В некоторых случаях необходимо выяснить, существует ли путь с определёнными свойствами между двумя выбранными вершинами.
В других же ситуациях необходимо найти все пути в графе, удовлетворяющие некоторым свойствам или ограничениям. 
Например, в качестве ограничений можно указать, что искомый путь должен быть простым, кратчайшим, гамильтоновым и так далее.

Один из способов задавать ограничения на пути в графе основан на использовании формальных языков.
Базовое определение языка говорит нам, что язык --- это множество слов над некоторым алфавитом.
Если рассмотреть граф, рёбра которого помечены символами из алфавита, то путь в таком графе будет задавать слово: достаточно соединить последовательно символы, лежащие на рёбрах пути.
Множество же таких путей будет задавать множество слов или язык.
Таким образом, если мы хотим найти некоторое множество путей в графе, то в качестве ограничения можно описать язык, который должно задавать это множество.
Иными словами, задача поиска путей может быть сформулирована следующим образом: необходимо найти такие пути в графе, что слова, получаемые конкатенацей меток их рёбер, принадлежат заданному языку.
Такой класс задач будем называть задачами поиска путей с ограничениям в терминах формальных языков.

Подобный класс задач часто возникает в областях, связанных с анализом граф-структурированных данных и активно исследуется~\cite{doi:10.1137/S0097539798337716,axelsson2011formal,10.1007/978-3-642-22321-1_24,Ward:2010:CRL:1710158.1710234,barrett2007label,doi:10.1137/S0097539798337716}.
Исследуются как классы языков, применяемых для задания ограничений, так и различные постановки задачи.

Граф-структурированные данные встречаются не только в графовых базах данных, но и при статическом анализе кода: по программе можно построить различные графы отображающие её свойства.
Скажем, граф вызовов, граф потока данных и так далее.
Оказывается, что поиск путей в специального вида графах с использованием ограничений в терминах формальных языков позволяет исследовать некоторые свойства программы.
Например проводить межпроцедурный анализ указателей или анализ алиасов~\cite{Zheng,10.1145/2001420.2001440,10.1145/2714064.2660213}, строить срезы программ~\cite{10.1145/193173.195287}, проводить анализ типов~\cite{10.1145/373243.360208}.

В данной работе представлен ряд алгоритмов для поиска путей с ограничениями в терминах формальных языков.
Основной акцент будет сделан на контекстно-свободных языках, однако будут затронуты и другие классы: регулярные, многокомпонентные контекстно-свободные (Multiple Context-Free Languages, MCFL~\cite{SEKI1991191}) и конъюнктивные языки.
Будет показано, что теория формальных языков и алгоритмы синтаксического анализа применимы не только для анализа языков программирования или естественных языков, а также для анализа графовых баз данных и статического анализа кода, что приводит к возникновению новых задач и переосмыслению старых.


Структура данной работы такова.
В начале (в части~\ref{chpt:GraphTheoryIntro}) мы рассмотрим основные понятия из теории графов, необходимые в данной работе. Данные разделы являются подготовительными и не обязательны к прочтению, если такие понятия как \textit{ориентированный граф} и \textit{матрица смежности} уже известны читателю. Более того, они лишь вводят определения, подазумевая, что более детальное изучение соответствующих разделов науки остается за рамками этой работы и скорее всего уже проделано читателем.
Затем, в главе~\ref{chpt:FormalLanguageTheoryIntro} мы введём основные понятия из теории формальных языков.
Далее, в главе~\ref{chpt:FLPQ} рассмотрим различные варианты постановки задачи поиска путей с ограничениями в терминах формальнх языков, обсудим базовые свойства задач, её разрешимость в различных постановках и т.д..
И в итоге зафиксируем постановку, которую будем изучать далее.
После этого, в главах~\ref{chpt:CFPQ_CYK}--\ref{chpt:GLR} мы будем подробно рассматривать различные алгоритмы решения этой задачи, попутно вводя специфичные для рассматриваемого алгоритма структуры данных.
Большинство алгоритмов будут основаны на классических алгоритмах синтаксического анализа, таких как CYK или LR.
Все главы, начиная с~\ref{chpt:GraphTheoryIntro}, снабжены списком вопросов и задач для самостоятельного решения и закрепления материала.